% Options for packages loaded elsewhere
\PassOptionsToPackage{unicode}{hyperref}
\PassOptionsToPackage{hyphens}{url}
%
\documentclass[
]{book}
\usepackage{amsmath,amssymb}
\usepackage{lmodern}
\usepackage{iftex}
\ifPDFTeX
  \usepackage[T1]{fontenc}
  \usepackage[utf8]{inputenc}
  \usepackage{textcomp} % provide euro and other symbols
\else % if luatex or xetex
  \usepackage{unicode-math}
  \defaultfontfeatures{Scale=MatchLowercase}
  \defaultfontfeatures[\rmfamily]{Ligatures=TeX,Scale=1}
\fi
% Use upquote if available, for straight quotes in verbatim environments
\IfFileExists{upquote.sty}{\usepackage{upquote}}{}
\IfFileExists{microtype.sty}{% use microtype if available
  \usepackage[]{microtype}
  \UseMicrotypeSet[protrusion]{basicmath} % disable protrusion for tt fonts
}{}
\makeatletter
\@ifundefined{KOMAClassName}{% if non-KOMA class
  \IfFileExists{parskip.sty}{%
    \usepackage{parskip}
  }{% else
    \setlength{\parindent}{0pt}
    \setlength{\parskip}{6pt plus 2pt minus 1pt}}
}{% if KOMA class
  \KOMAoptions{parskip=half}}
\makeatother
\usepackage{xcolor}
\usepackage{longtable,booktabs,array}
\usepackage{calc} % for calculating minipage widths
% Correct order of tables after \paragraph or \subparagraph
\usepackage{etoolbox}
\makeatletter
\patchcmd\longtable{\par}{\if@noskipsec\mbox{}\fi\par}{}{}
\makeatother
% Allow footnotes in longtable head/foot
\IfFileExists{footnotehyper.sty}{\usepackage{footnotehyper}}{\usepackage{footnote}}
\makesavenoteenv{longtable}
\usepackage{graphicx}
\makeatletter
\def\maxwidth{\ifdim\Gin@nat@width>\linewidth\linewidth\else\Gin@nat@width\fi}
\def\maxheight{\ifdim\Gin@nat@height>\textheight\textheight\else\Gin@nat@height\fi}
\makeatother
% Scale images if necessary, so that they will not overflow the page
% margins by default, and it is still possible to overwrite the defaults
% using explicit options in \includegraphics[width, height, ...]{}
\setkeys{Gin}{width=\maxwidth,height=\maxheight,keepaspectratio}
% Set default figure placement to htbp
\makeatletter
\def\fps@figure{htbp}
\makeatother
\setlength{\emergencystretch}{3em} % prevent overfull lines
\providecommand{\tightlist}{%
  \setlength{\itemsep}{0pt}\setlength{\parskip}{0pt}}
\setcounter{secnumdepth}{5}
\usepackage[margin=1in]{geometry}
\usepackage{booktabs}
\usepackage{enumitem}

% Without these you get ! LaTeX Error: Too deeply nested.
\setlistdepth{10}
\renewlist{enumerate}{enumerate}{10}
\setlist[itemize]{labelsep=.5em}

% Correct for the way articles/sections are defined and pretty the TOC
\usepackage{fancyhdr}
\usepackage{tocbasic}

\DeclareTOCStyleEntry[%
  entryformat=\bfseries,
  pagenumberformat=\bfseries,
]{tocline}{chapter}
\DeclareTOCStyleEntries[
  pagenumberbox=\hbox,
  dynnumwidth
]{tocline}{%
  chapter,section,subsection,subsubsection,paragraph,subparagraph,%
  figure,table
}
\DeclareTOCStyleEntries[
  dynindent
]{tocline}{subsection,subsubsection,subparagraph}

\renewcommand{\chaptername}{Article}
\renewcommand{\thechapter}{\Roman{chapter}}
\renewcommand{\appendixname}{Exhibit}
\ifLuaTeX
  \usepackage{selnolig}  % disable illegal ligatures
\fi
\usepackage[]{natbib}
\bibliographystyle{plainnat}
\IfFileExists{bookmark.sty}{\usepackage{bookmark}}{\usepackage{hyperref}}
\IfFileExists{xurl.sty}{\usepackage{xurl}}{} % add URL line breaks if available
\urlstyle{same} % disable monospaced font for URLs
\hypersetup{
  pdftitle={NBA Collective Bargaining Agreement - 1999},
  pdfauthor={Robert},
  hidelinks,
  pdfcreator={LaTeX via pandoc}}

\title{NBA Collective Bargaining Agreement - 1999}
\author{Robert}
\date{2023-04-01}

\begin{document}
\maketitle

{
\setcounter{tocdepth}{1}
\tableofcontents
}
\hypertarget{preface}{%
\chapter*{Preface}\label{preface}}
\addcontentsline{toc}{chapter}{Preface}

\emph{If you are only interested in a pdf or epub of the CBA, then please click on the download icon at the top left (it is next to the ``A'') and select the format you wish to download.}

This is the 1999 NBA's Collective Bargaining Agreement (CBA) converted to markdown and disseminated in the format you are currently viewing this as.

The original version of the 1999 CBA can be found on my Github website \href{https://atlhawksfanatic.github.io/}{atlhawksfanatic.github.io} (\href{https://github.com/atlhawksfanatic/atlhawksfanatic.github.io/raw/master/research/CBA/1999-NBA-NBPA-Collective-Bargaining-Agreement.pdf}{pdf found here}) as a way to cross-reference any potential discrepancies.

The original CBA was converted to a text file with \href{https://github.com/ropensci/pdftools}{pdftools}, then broken up by Article and Exhibit through regular expressions into .Rmd files. Those .Rmd files serves as the basis for this bookdown site.

The purpose of this project is three-fold:

\begin{enumerate}
\def\labelenumi{\arabic{enumi}.}
\tightlist
\item
  to historically document collective bargaining agreements of the NBA;
\item
  to provide easier navigation of the CBA through a structured format of each Article and Section; and
\item
  for me to better understand the CBA through this exercise.
\end{enumerate}

I do not own any rights to the CBA and am simply redistributing it in a different format. This is not the official version of the CBA and I am not responsible for any errors that might be present in this document. If you believe you have found an error, please let me know and I will correct it.

Contact: \href{atlhawksfanatic@gmail.com}{via email}, \href{https://github.com/atlhawksfanatic}{through Github}, or \href{https://twitter.com/atlhawksfanatic}{on Twitter}

\hypertarget{definitions}{%
\chapter{DEFINITIONS}\label{definitions}}

\hypertarget{definitions-1}{%
\section{Definitions}\label{definitions-1}}

As used in this Agreement, the following terms shall have the following meanings:

\begin{enumerate}
\def\labelenumi{\arabic{enumi}.}
\tightlist
\item
  ``Agreement'' means this Collective Bargaining Agreement entered into January 20, 1999.
\item
  ``Audit Report'' means the audit report prepared in accordance with Article VII, Section 10.
\item
  ``Average Player Salary'' means, with respect to any Salary Cap Year, total Team Salaries for all Teams (other than Expansion Teams during their first two (2) Seasons in the NBA) as determined by the Accountants in accordance with Article VII, Section 10, divided by an amount equal to the product of the number of Teams in the NBA (other than Expansion Teams during their first two (2) Seasons) multiplied by 12.5.
\item
  ``Averaged Contract'' means a Player Contract, entered into prior to the date of this Agreement, with respect to which Salaries have been averaged in accordance with the rules set forth in any prior collective bargaining agreement between the parties.
\item
  ``Base Year Compensation'' means an amount used to calculate the Exception that results from the assignment of certain Player Contracts, as determined in accordance with Article VII, Section 6(h)(4).
\item
  ``Basketball Related Income'' or ``BRI'' means basketball related income as defined in Article VII, Section 1(a).
\item
  ``Benefits'' means the sum of all amounts paid or to be paid on an accrual basis during any Salary Cap Year by the NBA or NBA Teams, other than Expansion Teams during their first two Seasons, for the specific benefits set forth in Article IV.
\item
  ``Cash Compensation'' means the component of Compensation payable in cash or by check or electronic transfer.
\item
  ``Commissioner'' means the Commissioner of the NBA.
\item
  ``Compensation'' means the compensation in money, property, investments, or anything else of value that is or could be earned by, or is paid or payable to, an NBA player (including players whose Player Contracts have been terminated) or to a person or entity designated by a player, in accordance with a Player Contract.
\item
  ``Contract'' (see ``Uniform Player Contract'').
\item
  ``Current Cash Compensation'' means the component of Current Compensation payable in cash or by check or electronic transfer, excluding signing bonuses and Incentive Compensation.
\item
  ``Current Compensation'' means the component of Compensation other than Deferred Compensation.
\item
  ``Deferred Cash Compensation'' means the component of Deferred Compensation that is payable in cash or by check or electronic transfer.
\item
  ``Deferred Compensation'' means the component of Compensation payable to a player during the period commencing after the term covered by the Player Contract, in accordance with the rules set forth in Article VII. The determination of whether Compensation is Deferred Compensation will be based upon the time set by the Player Contract for the player to receive the Compensation, without regard to whether the obligation is funded currently or secured in any fashion.
\item
  ``Draft'' or ``NBA Draft'' means the NBA's annual draft of Rookie basketball players.
\item
  ``Early Qualifying Veteran Free Agent'' means a Veteran Free Agent who, prior to becoming a Veteran Free Agent, played under one or more Player Contracts covering some or all of each of the two preceding Seasons, and who: (i) either exclusively played with his Prior Team during such two Seasons, or, if he played for more than one Team during such period, changed Teams only (x) by means of assignment, or (y) by signing with his Prior Team during the first of the two Seasons; or (ii) became a Veteran Free Agent on July 1, 1998 and played with his Prior Team for some or all of each of the preceding two Seasons, and who did not change Teams during such two Seasons by signing with his Prior Team as a Veteran Free Agent.
\item
  ``Early Termination Option'' (or ``ETO'') means an option in favor of a player to shorten the stated term of a Player Contract in accordance with Article XII.
\item
  ``Effective Season'' means, with respect to an Early Termination Option, the first Season covered by the Early Termination Option. (For example, if a Contract were to contain an Early Termination Option exercisable following the 1999-2000 Season, the Effective Season would be the 2000-01 Season.)
\item
  ``Estimated Average Player Salary'' means, for a particular Salary Cap Year, 108\% of the prior Salary Cap Year's Average Player Salary.
\item
  ``Exception'' means an exception to the rule that a Team's Team Salary may not exceed the Salary Cap.
\item
  ``Expansion Team'' means any Team that becomes a member of the NBA through expansion following the date of this Agreement and commences play during the term of this Agreement.
\item
  ``Extension'' means an amendment to a Player Contract lengthening the term of the Contract, other than pursuant to the exercise of an Option.
\item
  ``First Round Pick'' means a player selected by a Team in the first round of the Draft.
\item
  ``Free Agent'' means: (i) a Veteran Free Agent; (ii) a Rookie Free Agent; (iii) a Veteran whose Player Contract has been terminated in accordance with the NBA waiver procedure; or (iv) a player whose last Player Contract was a 10-Day Contract and who either completed the Contract by rendering the playing services called for thereunder or was released early from such Contract.
\item
  ``Incentive Compensation'' means the component of Compensation consisting of one or more bonuses described in Article II, Sections 3(c)(iii) and (iv) and 3(d).
\item
  ``Likely Bonus'' means Incentive Compensation included in a player's Salary in accordance with Article VII, Section 3(d).
\item
  ``Member'' or ``Team'' means any team that is a member of the NBA.
\item
  ``Minimum Annual Salary'' means the minimum Salary that must be included in a Player Contract that covers the entire Regular Season in accordance with Article II, Section 6(a).
\item
  ``Minimum Player Salary'' means: (i) with respect to a Contract that covers the entire Regular Season, the Minimum Annual Salary called for under Article II, Section 6(a); (ii) with respect to a Rest-of-Season Contract, the Minimum Annual Salary called for under Article II, Section 6(a) multiplied by a fraction, the numerator of which is the number of days remaining in the NBA Regular Season as of the date such Rest-of-Season Contract is entered into, and the denominator of which is the total number of days of that NBA Regular Season; and (iii) with respect to a 10-Day Contract, the Minimum Annual Salary called for under Article II, Section 6(a) multiplied by a fraction, the numerator of which is the number of days covered by the Contract and the denominator of which is the total number of days of that NBA Regular Season.
\item
  ``Minimum Annual Salary Scale'' means the scale annexed hereto as Exhibit C.
\item
  ``Minimum Team Salary'' means the minimum amount in Salary obligations to, or on behalf of, players with respect to an NBA Season that each Team must incur or pay.
\item
  ``Negotiate'' means, with respect to a player or his representatives on the one hand, and a Team or its representatives on the other hand, to engage in any written or oral communication relating to the possible employment, or terms of employment, of such player by such Team as a basketball player, regardless of who initiates such communication.
\item
  ``Non-Cash Compensation'' means the component of Compensation that is not paid in cash or by check or electronic transfer (e.g., game tickets, automobiles).
\item
  ``Non-Qualifying Veteran Free Agent'' means a Veteran Free Agent who is not a Qualifying Veteran Free Agent or an Early Qualifying Veteran Free Agent.
\item
  ``Option'' means an option in a Player Contract in favor of a Team or player to extend such Contract beyond its stated term.
\item
  ``Option Buy-Out Amount'' means any amount payable to a player in connection with either the exercise of an Early Termination Option or the non-exercise of an Option.
\item
  ``Option Year'' means the year that would be added to a Player Contract if an Option were exercised.
\item
  ``Performance Bonus'' means any Incentive Compensation described in Article II, Section 3(c)(iii).
\item
  ``Player Contract'' (see ``Uniform Player Contract'').
\item
  ``Prior Team'' means the Team for which a player was last under Contract prior to becoming a Qualifying Veteran Free Agent, Early Qualifying Veteran Free Agent or a Non-Qualifying Veteran Free Agent.
\item
  ``Qualifying Offer'' means an offer of a Uniform Player Contract, signed by the Team, that (i) is either personally delivered to the player or his representative or sent by prepaid certified, registered or overnight mail to the last known address of the player or his representative; (ii) is for a period of one year; and (iii) provides for: (A) for First Round Picks finishing their Rookie Scale Contracts, the percentage increase over the player's fourth year Salary called for in Exhibit B hereto; and (B) for all other players subject to a right of first refusal in accordance with Article XI, the greater of 125\% of the player's prior Salary or the sum of the Minimum Annual Salary applicable to the player (for the Season covered by the Qualifying Offer) plus \$150,000. All terms and conditions in the Qualifying Offer, other than Salary and length of term, must be the same as in the last year of the player's prior Contract (provided that such terms and conditions are allowable amendments under this Agreement at the time the Qualifying Offer is made).
\item
  ``Qualifying Veteran Free Agent'' means a Veteran Free Agent who, prior to becoming a Veteran Free Agent, played under one or more Player Contracts covering some or all of each of the three preceding Seasons and either played exclusively with his Prior Team during such three Seasons, or, if he played with more than one Team during such period, changed Teams only (x) by means of assignment, or (y) by signing with his Prior Team during the first of the three Seasons.
\item
  ``Regular Salary'' means a player's Salary, less any component thereof that is a signing bonus (or deemed a signing bonus in accordance with Article VII) and any component thereof that is Incentive Compensation.
\item
  ``Regular Season'' means, with respect to any Season, the period beginning on the first day and ending on the last day of regularly scheduled (as opposed to exhibition or playoff) competition between NBA Teams.
\item
  ``Renegotiation'' means a Contract amendment that provides for changes in Salary and/or Incentive Compensation.
\item
  ``Replacement Player'' means, where appropriate, either a player who is acquired by a Team pursuant to the Assigned Player Exception, or a player who is signed or acquired by a Team pursuant to the Disabled Player Exception.
\item
  ``Required Tender'' means an offer of a Uniform Player Contract to a Draft Rookie, signed by the Team, that: (i) is either personally delivered to the player or his representative or sent by prepaid certified, registered, or overnight mail to the last known address of the player or his representative; (ii) with respect to a First Round Pick, (A) provides the player with at least until the first day of the following Regular Season to accept, (B) has a stated term of three Seasons plus an Option Year, (C) in each Season prior to the Option Year, calls for at least 80\% of the Rookie Scale Amount then applicable to the player, (D) in the Option Year, calls for the percentage increase over the player's third year Salary called for in Exhibit B hereto, (E) in each Season other than the Option Year, provides protection for lack of skill and insured or non-insured injury or illness of not less than 80\% of the Rookie Scale Amount then applicable to the player, and (F) in the Option Year, provides the same percentage of protection for lack of skill and insured or non-insured injury or illness that applies to the third Season of the Contract; and (iii) with respect to a Second Round Pick, (A) provides the player with at least thirty (30) days to accept, (B) has a stated term of one (1) Season, and (C) calls for at least the Minimum Annual Salary then applicable to the player.
\item
  ``Rookie'' means a person who has never signed a Player Contract with an NBA Team.

  \begin{enumerate}
  \def\labelenumii{\arabic{enumii}.}
  \tightlist
  \item
    ``Draft Rookie'' means a Rookie who is selected in the NBA Draft.
  \item
    ``Non-Draft Rookie'' means a Rookie who is not selected in the NBA Draft for which he is first eligible.
  \end{enumerate}
\item
  ``Restricted Free Agent'' means a Veteran Free Agent who is subject to a Team's right of first refusal in accordance with Article XI.
\item
  ``Rookie Free Agent'' means: (i) a Draft Rookie who, pursuant to the provisions of Article VIII, Section 3 or Article X, is no longer subject to the exclusive negotiating rights of any Team, and who may be signed by any Team; or (ii) a Non-Draft Rookie.
\item
  ``Rookie Scale Amounts'' means the amounts set forth in the tables annexed hereto as Exhibit B.
\item
  ``Rookie Scale Contract'' means the initial Uniform Player Contract entered into, in accordance with Article VIII, Section 1 or 2, between a First Round Pick and the Team that holds his draft rights.
\item
  ``Room'' means the extent to which: (i) a Team's then-current Team Salary is less than the Salary Cap; or (ii) a Team is entitled to use one of the Salary Cap Exceptions set forth in Article VII, Section 6(c), (d), (e) and (h) (Disabled Player, \$1 Million, Mid-Level Salary and Assigned Player Exceptions).
\item
  ``Salary'' means, with respect to a Salary Cap Year, a player's Compensation with respect to the Season covered by such Salary Cap Year, plus any other amount that is deemed to constitute Salary in accordance with the terms of this Agreement, not including Unlikely Bonuses, any benefits the player received in accordance with the terms of this Agreement (including, e.g., the benefits provided for by Article IV, per diem, and moving expenses), and any portion of the player's Compensation that is attributable to another Salary Cap Year in accordance with this Agreement. Salary also includes any consideration received by a retired player that is deemed to constitute Salary in accordance with the terms of Article XIII.
\item
  ``Salary Cap'' means the maximum allowable Team Salary for each Team for a Salary Cap Year, subject to the rules and exceptions set forth in this Agreement.
\item
  ``Salary Cap Year'' means the period from July 1 through the following June 30.
\item
  ``Season'' or ``NBA Season'' means the period beginning on the first day of training camp and ending immediately after the last game of the NBA Finals.
\item
  ``Second Round Pick'' means a player selected by a Team in the second round of the Draft.
\item
  ``Team'' or ``NBA Team'' (see ``Member'').
\item
  ``Team Affiliate'' means:

  \begin{enumerate}
  \def\labelenumii{\arabic{enumii}.}
  \tightlist
  \item
    any individual or entity who or which (directly or indirectly) holds an ownership interest in a Team (other than ownership of publicly-traded securities constituting less than 5\% of the ownership interests in a Team);
  \item
    any individual or entity who or which (directly or indirectly) controls, is controlled by or is under common control with, or who or which is an entity affiliated with or an individual related to, a Team;
  \item
    any individual or entity who or which (directly or indirectly) controls, is controlled by or is under common control with, or who or which is an entity affiliated with or an individual related to, an individual or entity described in subparagraphs (a) or (b) above; or
  \item
    any entity in which 10\% or more of the ownership interests are held (directly or indirectly) by an individual or entity who or which holds (directly or indirectly) 10\% or more of the ownership interests in a Team or in an entity described in subparagraph (b) above.
  \end{enumerate}

  For the purposes of this Section 1(iii): an individual shall only be deemed to be ``related to'' a Team or another individual or entity if such individual is an officer, director or executive employee of such Team or entity, or is a member of such individual's immediate family; and ``controls'' or ``is controlled by'' shall include (without limitation) the circumstance in which an individual or a Team or entity has or can exercise effective control.
\item
  ``Team Salary'' means, with respect to a Salary Cap Year, the sum of all Salaries attributable to a Team's active and former players plus other amounts as computed in accordance with Article VII, less applicable credit amounts as computed in accordance with Article VII.
\item
  ``Total Salaries and Benefits'' means the total Salaries included in the Team Salary of, and the total Benefits paid or payable by, all NBA Teams and/or the NBA for or with respect to a Salary Cap Year in accordance with this Agreement, other than the Salaries included in the Team Salary of, and Benefits paid or payable by, Expansion Teams during their first two Seasons, as determined in accordance with Article VII. For purposes of this definition, total Salaries shall include all Incentive Compensation excluded from Salaries in accordance with Article VII, Section 3(d) but actually earned by NBA players during such Salary Cap Year, and shall exclude all Incentive Compensation included in Salaries in accordance with Article VII, Section 3(d) but not actually earned by NBA players during such Salary Cap Year.
\item
  ``Traded Player'' means a player whose Player Contract is assigned by one Team to another Team other than by means of the NBA waiver procedure.
\item
  ``Uniform Player Contract'' or ``Player Contract'' or ``Contract'' means the standard form of written agreement between a person and a Team required for use in the NBA by Article II, pursuant to which such person is employed by such Team as a professional basketball player.
\item
  ``Unlikely Bonus'' means Incentive Compensation excluded from a player's Salary in accordance with Article VII, Section 3(d).
\item
  ``Unrestricted Free Agent'' means a Free Agent who is not subject to a Team's right of first refusal.
\item
  ``Veteran'' or ``Veteran Player'' means a person who has signed at least one Player Contract with an NBA Team.
\item
  ``Veteran Free Agent'' means a Veteran who completed his Player Contract (other than a 10-Day Contract) by rendering the playing services called for thereunder.
\item
  ``Years of Service'' means the number of years of NBA service credited to a player in accordance with the following: a player will be credited with one year of NBA service for each year that he is signed to a Uniform Player Contract to play for an NBA Team, whether or not his services are retained by that NBA Team for the start, or for any portion, of the year for which he is signed; provided, however, that if a player signs a Player Contract with a stated term of more than one year: (i) he will receive credit for the first year of such Contract; and (ii) with respect to each subsequent year of the Contract, he will receive credit for such year only if the Contract has not been terminated prior to the first day of the Season in such subsequent year. Notwithstanding the above, a player will not receive credit for a Year of Service for any year in which he: (x) withholds playing services called for by a Player Contract for more than thirty (30) days after the Season begins, (y) is signed only to one or more 10-Day Contracts, or (z) is a Restricted Free Agent, has been tendered a Qualifying Offer by his Prior Team that has been expressly left open by that Team until at least March 1, and has not signed a Player Contract with any Team by March 1. In addition, notwithstanding the above, a player will not receive credit for a Year of Service for a Player Contract that is voided due to the player's failure to pass a physical examination or that is disapproved by the Commissioner. In no event can a player be credited with more than one (1) Year of Service with respect to any one NBA Season. A Year of Service will be credited to a player on the June 30 following the Season with respect to which it is being credited. Under no circumstances shall the definition of Years of Service herein be used for purposes of determining a player's years of Credited Service under the NBA Players Pension Plan.
\end{enumerate}

\hypertarget{uniform-player-contract}{%
\chapter{UNIFORM PLAYER CONTRACT}\label{uniform-player-contract}}

\hypertarget{required-form.}{%
\section{Required Form.}\label{required-form.}}

The Player Contract to be entered into by each player and the Team by which he is employed shall be a Uniform Player Contract in the form annexed hereto as Exhibit A.

\hypertarget{limitation-on-amendments.}{%
\section{Limitation on Amendments.}\label{limitation-on-amendments.}}

\begin{enumerate}
\def\labelenumi{(\alph{enumi})}
\tightlist
\item
  Except as provided in Sections 3, 6, 7(d), 8, 9, 10 and 11 of this Article, and in Article VII, Section 7 (Extensions, Renegotiations and Other Amendments) or Article XII (Option Clauses), no amendments to the form of Uniform Player Contract provided for by Section 1 of this Article shall be permitted.
\item
  If a Team and a player enter into (i) a Uniform Player Contract containing an amendment not specifically permitted by this Agreement or (ii) a subsequent amendment to an existing Player Contract where such amendment is not specifically permitted by this Agreement, then such Contract or subsequent amendment, as the case may be, shall be disapproved by the Commissioner and, consequently, rendered null and void.
\end{enumerate}

\hypertarget{allowable-amendments.}{%
\section{Allowable Amendments.}\label{allowable-amendments.}}

In their individual contract negotiations, a player and a Team may amend the provisions of a Uniform Player Contract, but only in the following respects:

\begin{enumerate}
\def\labelenumi{(\alph{enumi})}
\tightlist
\item
  By agreeing upon provisions (to be set forth in Exhibit 1 to a Uniform Player Contract) setting forth the Cash Compensation to be paid or amounts to be loaned to the player for each Season of the Contract forrendering the services described in such Contract.
\item
  By agreeing upon provisions (to be set forth in Exhibit 1 to a Uniform Player Contract) setting forth the Non-Cash Compensation to be paid or provided to the player for rendering the services described in such Contract.
\item
  By agreeing upon provisions (to be set forth in Exhibit 1 to a Uniform Player Contract) setting forth lump sum bonuses, and the payment schedule therefore, to be paid as a result of: (i) the player's execution of a Uniform Player Contract or Extension (a ``signing bonus''), (ii) the exercise or non-exercise of an option pursuant to Articles VII and XII, (iii) the player's achievement of agreed-upon benchmarks relating to his performance as a player or the Team's performance during a particular NBA Season, subject to the limitations imposed by paragraph 3(c) of the Uniform Player Contract, or (iv) the player's achievement of agreed-upon benchmarks relating to his physical condition or academic achievement, including the player's attendance at and participation in an off-season summer league and/or an off-season skill and conditioning program designated by the Team (subject to the limitations imposed by Section 11(h) below). Any amendment agreed upon pursuant to subsections (iii) or (iv) of this subsection (c) must be structured so as to provide an incentive for positive achievement by the player and/or the Team; and any amendment agreed upon pursuant to subsection (iii) must be based upon specific numerical benchmarks or generally recognized league honors. By way of example and not limitation, an amendment agreed upon pursuant to subsection (iii) may provide for the player to receive a bonus if his free-throw percentage exceeds 80\%, but may not provide for the player to receive a bonus if his free-throw percentage improves over his previous season's percentage.
\item
  By agreeing upon provisions (to be set forth in Exhibit 1 to a Uniform Player Contract) with respect to extra promotional appearances to be performed by the player (in addition to those required by paragraph 13 of such Contract) and the Compensation therefor.
\item
  By agreeing upon a Compensation payment schedule (to be set forth in Exhibit 1 to a Uniform Player Contract) different from that provided for by paragraph 3(a) of the Uniform Player Contract; provided, however, such amendment shall comply with the provisions of Section 11(f) below and that no such amendment shall be permitted with respect to any Season in which the player's Compensation is not greater than the Minimum Player Salary called for with respect to that Season pursuant to Section 6 below.
\item
  By agreeing upon provisions (to be set forth in Exhibit 2 to a Uniform Player Contract) stating that the Cash Compensation provided for by a Uniform Player Contract (as described in Exhibit 1 to such Contract) shall be, in whole or in part, and subject to any conditions or limitations, protected or insured (as provided for by, and in accordance with the definitions set forth in, Section 4 below) in the event that such Contract is terminated by the Team by reason of the player's:

  \begin{enumerate}
  \def\labelenumii{(\roman{enumii})}
  \tightlist
  \item
    lack of skill;
  \item
    death not covered by an insurance policy procured by a Team for the player's benefit (``non- insured death'');
  \item
    death covered by an insurance policy procured by a Team for the player's benefit (``insured death'');
  \item
    disability or unfitness to play skilled basketball resulting from a basketball-related injury not covered by an insurance policy procured by a Team for the player's benefit (``non-insured basketball-related injury'');
  \item
    disability or unfitness to play skilled basketball resulting from any injury or illness not covered by an insurance policy procured by a Team for the player's benefit (``non-insured injury or illness'');
  \item
    disability or unfitness to play skilled basketball resulting from an injury or illness covered by an insurance policy procured by a Team for the player's benefit (``insured injury or illness'');
  \item
    mental disability not covered by an insurance policy procured by a Team for the player's benefit (``non-insured mental disability''); and/or(viii) mental disability covered by an insurance policy procured by a Team for the player's benefit (``insured mental disability'').
  \end{enumerate}
\item
  By agreeing upon provisions (to be set forth in Exhibit 3 to a Uniform Player Contract) limiting or eliminating the player's right to receive his Cash Compensation (in accordance with paragraphs 7(c), 16(a)(iii), and 16(b) of the Uniform Player Contract) when the player's disability or unfitness to play skilled basketball is caused by the re-injury of an injury sustained prior to, or by the aggravation of a condition that existed prior to, the execution of the Uniform Player Contract providing for such Cash Compensation.
\item
  By agreeing upon provisions (to be set forth in Exhibit 4 to a Uniform Player Contract) (i) entitling a player to earn Cash Compensation upon the assignment of such player's Uniform Player Contract, or (ii) prohibiting or limiting the Team's right to assign such player's Contract to another Team, subject, however, in either case (i) or (ii) to the provisions of Article XXIV.
\item
  By agreeing upon provisions (to be set forth in Exhibit 5 to a Uniform Player Contract) permitting the player to participate or engage in some or all of the activities otherwise prohibited by paragraph 12 of the Uniform Player Contract; provided, however, that paragraph 12 of the Uniform Player Contract may not be amended to permit a player to participate in any public game or public exhibition of basketball not approved in accordance with Article XXIII of this Agreement.
\item
  By agreeing upon provisions (to be set forth in Exhibit 6 to a Uniform Player Contract) establishing the date and time of a physical examination of the player to be performed by a physician designated by the Team within seventy-two (72) hours of the execution of the Contract, the passage of such examination by the player (in the sole discretion of the Team) to be a condition precedent to the validity of the player's Uniform Player Contract.
\item
  By agreeing to delete clauses (b)(ii) and/or (b)(iii) of paragraph 5 of the Uniform Player Contract in their entirety.
\item
  By agreeing to delete paragraph 7(b) of the Uniform Player Contract in its entirety and substituting therefor the provision set forth in Exhibit 7 to a Uniform Player Contract.
\item
  By agreeing either (i) to delete paragraph 13(b) of the Uniform Player Contract in its entirety, or (ii) to delete the last sixteen words of paragraph 13(b) of such Contract.
\item
  By agreeing upon provisions for the purpose of terminating an already-existing Uniform Player Contract prior to the expiration of its stated term, stating as follows: (i) that the Team will request waivers on the player in accordance with paragraph 16 of the Contract immediately following the Commissioner's approval of such amendment; and (ii) should the player clear waivers and his Contract thereupon be terminated, that the amount of any Cash Compensation protection or insurance contained in the Contract will immediately be reduced or eliminated. In addition to the foregoing, the parties may also agree that (x) as a result of the termination of the Contract, the payment schedule for any Compensation remaining to be paid will be accelerated over a shorter period or stretched over a longer period (subject, however, to Section 11(f) below), and/or (y) that the Team's right of set-off under Article XXVII of this Agreement will be modified or eliminated.
\item
  By agreeing upon provisions (to be set forth in Exhibit 8 to a Uniform Player Contract) stating that the Contract will be assigned to another team within forty-eight (48) hours of its execution, such assignment and the consummation of such assignment to be conditions precedent to the validity of the Contract; provided, however, that any such sign-and-trade transaction must comply with Article VII, Section 8(e).
\end{enumerate}

\hypertarget{cash-compensation-protection-or-insurance.}{%
\section{Cash Compensation Protection or Insurance.}\label{cash-compensation-protection-or-insurance.}}

\begin{enumerate}
\def\labelenumi{(\alph{enumi})}
\tightlist
\item
  \textbf{Lack of Skill.} When a Team agrees to protect, in whole or in part, the Cash Compensation provided for by a Uniform Player Contract in the event such Contract is terminated by the Team, pursuant to paragraph 16(a)(iii) thereof, by reason of the player's lack of skill, such agreement shall mean that, subject to any conditions or limitations set forth in Exhibit 2 and/or Exhibit 3 to the Uniform Player Contract, notwithstanding the provisions of paragraphs 16(a)(iii), 16(d), 16(e), and 16(g) of such Contract, the termination of such Contract by the Team on account of the player's failure to exhibit sufficient skill or competitive ability shall in no way affect the player's right to receive, in whole or in part, the Cash Compensation payable pursuant to Exhibit 1 to such Contract in the amounts and at the times called for by such Exhibit.
\item
  \textbf{Non-Insured Death.} When a Team agrees to protect, in whole or in part, the Cash Compensation provided for by a Uniform Player Contract in the event such Contract is terminated by the Team, pursuant to paragraph 16(a)(iv) thereof, by reason of the player's non-insured death, such agreement shall mean that, subject to any conditions or limitations set forth in Exhibit 2 and/or 3 to the Uniform Player Contract (in addition to the conditions and limitations set forth in this Section 4(b)), notwithstanding the provisions of paragraphs 16(a), 16(b), 16(c), 16(d), 16(e), and 16(g) of such Contract, the termination of such Contract by the Team on account of the player's failure to render his services thereunder, if such failure has been caused by the player's death, shall in no way affect the player's (or his estate's or duly appointed beneficiary's) right to receive, in whole or in part, the Cash Compensation payable pursuant to Exhibit 1 to such Contract in the amounts and at the times called for by such Exhibit; provided, however, that (i) such death does not result from the player's participation in activities prohibited by paragraph 12 of the Uniform Player Contract (as such paragraph may be modified by Exhibit 5 to the Player Contract), suicide, the abuse of alcohol, or the use of any controlled substance; (ii) at the time of the player's failure to render playing services, the player is not in material breach of such Contract; (iii) if the Team, for its own benefit, seeks to procure an insurance policy covering the player's death, the player cooperates with the Team in procuring such an insurance policy; and (iv) if the Team, for its own benefit, has procured such an insurance policy, the player's estate and/or duly appointed beneficiary cooperates with the Team and insurance company in the processing of the Team's claim under such policy.
\item
  \textbf{Insured Death.} When a Team agrees to insure, in whole or in part, the Cash Compensation provided for by a Uniform Player Contract in the event such Contract is terminated by the Team, pursuant to paragraph 16(a)(iv) thereof, by reason of the player's insured death, such agreement shall mean that, subject to any conditions set forth in Exhibit 2 and/or Exhibit 3 to the Uniform Player Contract (in addition to the conditions and limitations set forth in this Section 4(c)), the Team has procured (or will procure forthwith) an insurance policy (specifically designated in Exhibit 2 to such Contract) for the benefit of the player or his estate or beneficiary that, subject to the conditions and limitations contained in the policy, would pay a benefit in the event of the player's death in an amount equal to or less than the Cash Compensation remaining to be paid to the player under Exhibit 1 of his Player Contract at the time of his death; provided, however, that (i) such death does not result from the player's participation in activities prohibited by paragraph 12 of the Uniform Player Contract (as such paragraph may be modified by Exhibit 5 to the Player Contract), suicide, the abuse of alcohol, or the use of any controlled substance; and (ii) at the time of the player's failure to render playing services, the player is not in material breach of such Contract.(d) Non-Insured Basketball-Related Injury. When a Team agrees to protect, in whole or in part, the Cash Compensation provided for by a Uniform Player Contract in the event such Contract is terminated by the Team, pursuant to paragraphs 7(c), 16(b), and/or 16(c) thereof, by reason of the player's disability or unfitness to play skilled basketball resulting from a non-insured basketball-related injury, such agreement shall mean that, subject to any conditions or limitations set forth in Exhibit 2 and/or Exhibit 3 to the Uniform Player Contract (in addition to the conditions and limitations set forth in this Section 4(d)), notwithstanding the provisions of paragraphs 7(b), 7(c), 16(a)(iii), 16(b), 16(c), 16(d), and 16(g) of such Contract, the termination of such Contract by the Team because the player has been disabled and/or is unfit to play skilled basketball as a direct result of an injury sustained while participating in any basketball practice or game played for the Team shall in no way affect the player's right to receive, in whole or in part, the Cash Compensation payable pursuant to Exhibit 1 to such Contract in the amounts and at the times called for by such Exhibit; provided, however, that (i) such injury does not result from an attempted suicide or the use of any controlled substance; (ii) at the time of the player's termination, the player is not in material breach of such Contract; (iii) if the Team, for its own benefit, seeks to procure an insurance policy covering the player's injury, the player cooperates with the Team in procuring such an insurance policy; and (iv) if the Team, for its own benefit, has procured such an insurance policy, the player cooperates with the Team and the insurance company in the processing of the Team's claim under such policy.
\item
  \textbf{Non-Insured Injury or Illness.} When a Team agrees to protect, in whole or in part, the Cash Compensation provided for by a Uniform Player Contract in the event such contract is terminated by the Team, pursuant to paragraphs 7(c), 16(b) and/or 16(c) thereof, by reason of the player's disability or unfitness to play skilled basketball resulting from any non-insured injury or illness, such agreement shall mean that, subject to any conditions or limitations set forth in Exhibit 2 and/or Exhibit 3 to the Uniform Player Contract (in addition to the conditions and limitations set forth in this Section 4(e)), notwithstanding the provisions of paragraphs 7(b), 7(c), 16(a)(iii), 16(b), 16(c), 16(d), and 16(g) of such Contract, the termination of such Contract by the Team on account of an injury, illness, or disability suffered or sustained by the player shall in no way affect the player's right to receive, in whole or in part, the Cash Compensation payable pursuant to Exhibit 1 to such Contract in the amounts and at the times called for by such Exhibit; provided, however, that (i) such injury, illness, or disability does not result from the player's participation in activities prohibited by paragraph 12 of the Uniform Player Contract (as such paragraph may be modified in Exhibit 5 to the Player Contract), attempted suicide, the abuse of alcohol, or the use of any controlled substance; (ii) at the time of such injury, illness, or disability the player is not in material breach of such Contract; (iii) if the Team, for its own benefit, seeks to procure an insurance policy covering the player's injury and/or illness, the player cooperates with the Team in procuring such an insurance policy; and (iv) if the Team, for its own benefit, has procured such an insurance policy, the player cooperates with the Team and insurance company in the processing of the Team's claim under such policy.
\item
  \textbf{Insured Injury or Illness.} When a Team agrees to insure, in whole or in part, the Cash Compensation provided for by a Uniform Player Contract in the event such Contract is terminated by the Team, pursuant to paragraphs 7(c), 16(b) and/or 16(c) thereof, by reason of the player's disability or unfitness to play skilled basketball resulting from an insured injury or illness, such agreement shall mean that, subject to any conditions or limitations set forth in Exhibit 2 and/or Exhibit 3 to the Uniform Player Contract (in addition to the conditions and limitations set forth in this Section 4(f)), the Team has procured (or will procure forthwith) an insurance policy (specifically designated in Exhibit 2 to such Contract) for the benefit of the player or his estate or beneficiary that, subject to the conditions and limitations contained in the policy, would pay a benefit in the event of the player's disability or unfitness to play skilled basketball resulting from an injury or illness in an amount equal to or less than the Cash Compensation remaining to be paid to the player under Exhibit 1 of his Player Contract at the time of his termination; provided, however, that (i) such injury or illness does not result from the player's participation in activities prohibited by paragraph 12 of the Uniform Player Contract (as such paragraph may be modified by Exhibit 5 to the Player Contract), attempted suicide, the abuse of alcohol, or the use of any controlled substance; and (ii) at the time of the player's termination, the player is not in material breach of such Contract.
\item
  \textbf{Non-Insured Mental Disability.} When a Team agrees to protect, in whole or in part, the Cash Compensation provided for by a Uniform Player Contract in the event such Contract is terminated by the Team, pursuant to paragraph 16(a)(iii) thereof, by reason of the player's non-insured mental disability, such agreement shall mean that, subject to any conditions or limitations set forth in Exhibit 2 and/or Exhibit 3 to the Uniform Player Contract (in addition to the conditions and limitations set forth in this Section 4(g)), notwithstanding the provisions of paragraphs 16(a), 16(b), 16(c), 16(d), 16(e), and 16(g) of such Contract, the termination of such Contract by the Team on account of the player's failure to render his services thereunder, if such failure has been caused by the player's mental disability, shall in no way affect the player's (or his duly appointed legal representative's) right to receive, in whole or in part, the Cash Compensation payable pursuant to Exhibit 1 to such Contract in the amounts and at the times called for by such Exhibit; provided, however, that (i) such mental disability does not result from the player's attempted suicide or the use of any controlled substance; (ii) at the time of the player's failure to render playing services, the player is not in material breach of such Contract; (iii) if the Team, for its own benefit, seeks to procure an insurance policy covering the player's mental disability, the player (and/or his duly appointed legal representative) cooperates with the Team in procuring such an insurance policy; and (iv) if the Team, for its own benefit, has procured an insurance policy covering the player's mental disability, the player (and/or his duly appointed legal representative) cooperates with the Team and insurance company in the processing of the Team's claim under such policy.
\item
  \textbf{Insured Mental Disability.} When a Team agrees to insure, in whole or in part, the Cash Compensation provided for by a Uniform Player Contract in the event such Contract is terminated by the Team, pursuant to paragraph 16(a)(iii) thereof, by reason of the player's insured mental disability, such agreement shall mean that, subject to any conditions or limitations set forth in Exhibit 2 and/or Exhibit 3 to the Uniform Player Contract (in addition to the conditions and limitations set forth in this Section 4(h)), the Team has procured (or will procure forthwith) an insurance policy (specifi-cally designated in Exhibit 2 to such Contract) for the benefit of the player or his estate or beneficiary that, subject to the conditions and limitations contained in the policy, would pay a benefit in the event of the player's mental disability in an amount equal to or less than the Cash Compensation remaining to be paid to the player under Exhibit 1 of his Player Contract at the time of his termination; provided, however, that (i) such mental disability does not result from the player's participation in activities prohibited by paragraph 12 of the Uniform Player Contract (as such paragraph may be modified by Exhibit 4 to the Player Contract), attempted suicide or the use of any controlled substance; and (ii) at the time of the player's termination, the player is not in material breach of such Contract.
\item
  No agreement by a Team to protect, in whole or in part, the Cash Compensation provided for by a Uniform Player Contract shall require (or be construed as requiring) such Team to continue the player on the Team, Active List, or Roster; nor shall any such agreement afford the player any right to continue, or to be deemed as having continued, on such Team, Active List, or Roster for any purpose.
\item
  When a Team agrees to protect, in whole or in part, the Cash Compensation provided for by a Uniform Player Contract, and such protection is contingent on the satisfaction of a condition expressly set forth in Exhibit 2 to that Contract, such protection shall be applicable and effective only if the Player Contract has not previously been terminated at the time such condition is satisfied.
\item
  When a Team agrees to protect, in whole or in part, the Cash Compensation provided for in any Option Year (in favor of the Team or the player) included in a Uniform Player Contract, such protection shall be applicable and effective only if the option to extend the term provided for in the Contract is exercised.
\end{enumerate}

\hypertarget{conformity.}{%
\section{Conformity.}\label{conformity.}}

\begin{enumerate}
\def\labelenumi{(\alph{enumi})}
\tightlist
\item
  All currently effective Player Contracts, and all Player Contracts entered into following the execution of this Agreement that do not otherwise so provide, shall be deemed amended in such manner to require the parties to comply with all terms of this Agreement, including the terms of the Uniform Player Contract annexed hereto as Exhibit A. All Player Contracts shall be subject to the terms of this Agreement, which shall supersede the terms of any Player Contract inconsistent herewith. No Player Contract shall provide for the waiver by a player or a Team of any benefits or the sacrifice of any rights to which the player or the Team is entitled by virtue of a Uniform Player Contract or this Agreement.
\item
  Notwithstanding Section 5(a) above, neither (i) the elimination from Section 3 above of any allowable amendment that was permitted under the 1995 NBA/NBPA Collective Bargaining Agreement, nor (ii) the provisions of Article II, Sections 7, 11(d), 11(e), 11(h), or Article XII shall affect the terms of any Player Contract entered into prior to the date of this Agreement. Nor shall any such Player Contracts be affected by any provisions of this Agreement expressly indicating that they apply only to Player Contracts entered into after the date of this Agreement.
\end{enumerate}

\hypertarget{minimum-player-salary.}{%
\section{Minimum Player Salary.}\label{minimum-player-salary.}}

\begin{enumerate}
\def\labelenumi{(\alph{enumi})}
\tightlist
\item
  Except with respect to 10-Day Contracts provided for in Section 9 below, and Rest-of-Season Contracts provided for in Section 10 below, no Player Contract shall provide for a Salary of less than the applicable scale amount contained in the Minimum Annual Salary Scale set forth as Exhibit C hereto.
\item
  No 10-Day Contract or Rest-of-Season Contract (as those terms are defined in Sections 9 and 10 below) shall provide for a Salary of less than the Minimum Player Salary applicable to that player.
\item
  In determining whether a Player Contract satisfies the Minimum Player Salary applicable to that player, the allocation of signing bonuses and deemed signing bonuses (but no other bonuses) shall be considered as part of the Salary provided for by a Player Contract, provided that such Player Contract makes clear that the Salary for each Season (including the allocation of such bonuses) equals or exceeds the Minimum Player Salary for such Season.
\item
  On July 1 of each Salary Cap Year, any Player Contract (whether entered into before or after the date of this Agreement) that provides for a Salary for the upcoming Season that is less than the applicable Minimum Player Salary shall be deemed amended to provide for the applicable Minimum Player Salary.
\item
  Nothing in this Section 6 shall alter the respective rights and liabilities of a player and a Team, as provided for in the Uniform Player Contract or in this Agreement, with respect to the termination of a Player Contract.
\item
  Every Contract entered into between a player and Team that is intended to provide for only the Minimum Player Salary (with no Unlikely Bonuses) for one or more Seasons must contain the following sentence in Exhibit 1A of such Contract and shall be deemed amended in the manner described in such sentence: ``This Contract is intended to provide for a Salary for the Season(s) equal to the Minimum Player Salary for such Season(s) (with no Unlikely Bonuses) and shall be deemed amended to the extent necessary to so provide.''
\end{enumerate}

\hypertarget{maximum-annual-salary.}{%
\section{Maximum Annual Salary.}\label{maximum-annual-salary.}}

\begin{enumerate}
\def\labelenumi{(\alph{enumi})}
\tightlist
\item
  Notwithstanding any other provision of this Agreement, no Player Contract entered into after the date of this Agreement may provide for a Salary plus Unlikely Bonuses in the first Season covered by the Contract that exceeds the following amounts:

  \begin{enumerate}
  \def\labelenumii{(\roman{enumii})}
  \tightlist
  \item
    For any player who has completed fewer than seven (7) Years of Service, the greater of (x) 25\% of the Salary Cap in effect at the time the Contract is executed, (y) 105\% of the Salary for the final Season of the player's prior Contract, or (z) \$9 million.
  \item
    For any player who has completed at least seven (7) but fewer than ten (10) Years of Service, the greater of (x) 30\% of the Salary Cap in effect at the time the Contract is executed, (y) 105\% of the Salary for the final Season of the player's prior Contract, or (z) \$11 million.
  \item
    For any player who has completed ten (10) or more Years of Service, the greater of (x) 35\% of the Salary Cap in effect at the time the Contract is executed, (y) 105\% of the Salary for the final Season of the player's prior Contract, or (z) \$14 million.
  \end{enumerate}
\item
  Notwithstanding any other provision of this Agreement, no Renegotiation entered into after the date of this Agreement may provide for a Salary plus Unlikely Bonuses in the Renegotiation Season (as defined in Article VII, Section 7(c)) that exceeds the following amounts:

  \begin{enumerate}
  \def\labelenumii{(\roman{enumii})}
  \tightlist
  \item
    For any player who has completed fewer than seven (7) Years of Service, the greater of (x) 25\% of the Salary Cap in effect at the time the Renegotiation is executed, (y) 105\% of the Salary for the Season prior to the Renegotiation Season, or (z) \$9 million.
  \item
    For any player who has completed at least seven (7) but fewer than ten (10) Years of Service, the greater of (x) 30\% of the Salary Cap in effect at the time the Renegotiation is executed, (y) 105\% of the Salary for the Season prior to the Renegotiation Season, or (z) \$11 million.
  \item
    For any player who has completed ten (10) or more Years of Service, the greater of (x) 35\% of the Salary Cap in effect at the time the Renegotiation is executed, (y) 105\% of the Salary for the Season prior to the Renegotiation Season, or (z) \$14 million.
  \end{enumerate}
\item
  The parties recognize that it may not be possible to ascertain at the time an Extension is executed whether the Salary plus Unlikely Bonuses called for in the first Season of the extended term will exceed the Maximum Annual Salary set forth in this Section 7. Accordingly, and notwithstanding any other provision of this Agreement, the following rule shall apply to an Extension entered into in accordance with Article VII, Section 7(a) or a Rookie Scale Extension entered into in accordance with Article VII, Section 7(b) after the date of this Agreement: if, on the August 1 of the Salary Cap Year encompassing the first Season of the extended term of such Extension, the Salary plus Unlikely Bonuses provided for in such Season exceeds the following amounts:

  \begin{enumerate}
  \def\labelenumii{(\roman{enumii})}
  \tightlist
  \item
    For any player who has completed fewer than seven (7) Years of Service, the greater of (x) 25\% of the Salary Cap in effect on such August 1, (y) 105\% of the Salary provided for in the final Season of the original term of the Contract, or (z) \$9 million;
  \item
    For any player who has completed at least seven (7) but fewer than ten (10) Years of Service, the greater of (x) 30\% of the Salary Cap in effect on such August 1, (y) 105\% of the Salary provided for in the final Season of the original term of the Contract, or (z) \$11 million; or
  \item
    For any player who has completed ten (10) or more Years of Service, the greater of (x) 35\% of the Salary Cap in effect on such August 1, (y) 105\% of the Salary provided for in the final Season of the original term of the Contract, or (z) \$14 million; then such Salary plus Unlikely Bonuses shall immediately be deemed amended to provide for the maximum amount allowed by the applicable subsection (c)(i), (c)(ii), or (c)(iii) set forth above. In such circumstance, Salaries plus Unlikely Bonuses in subsequent Seasons of the extended term shall also immediately be deemed amended to provide for increases or decreases over the amended Salary plus Unlikely Bonuses in the first Season of the extended term in accordance with Article VII, Section 5(c).
  \end{enumerate}
\item
  A player and a Team may provide in a Rookie Scale Extension that the player's Salary (in the first Season of the extended term) will equal ``the Maximum Annual Salary applicable to such player in the first Season of the extended term,'' and that the Salaries in any subsequent Seasons of the extended term will be increased or decreased based on percentages specified by the parties that comply with Article VII, Section 5(c). Any such Rookie Scale Extension shall be deemed amended on August 1 of the Salary Cap Year covering the first Season of the extended term to provide for specific Salaries for each Season of the extended term, based on the Maximum Annual Salary applicable to such player on such August 1. A Rookie Scale Extension entered into pursuant to this subsection may not include any Incentive Compensation.
\item
  Notwithstanding any other provision of this Agreement, no assignment of a Uniform Player Contract entered into after the date of this Agreement may, by reason of an assignment bonus contained in such Contract, cause the player's Salary plus Unlikely Bonuses for the Salary Cap Year in which such assignment occurs to exceed the following amounts:

  \begin{enumerate}
  \def\labelenumii{(\roman{enumii})}
  \tightlist
  \item
    For any player who has completed fewer than seven (7) Years of Service, the greater of (x) 25\% of the Salary Cap in effect at the time the assignment bonus is earned, (y) 105\% of the player's Salary for the Season prior to the Season in which the assignment bonus is earned, or (z) \$9 million.
  \item
    For any player who has at least seven (7) but fewer than ten (10) Years of Service, the greater of (x) 30\% of the Salary Cap in effect at the time the assignment bonus is earned, (y) 105\% of the player's Salary for the Season prior to the Season in which the assignment bonus is earned, or (z) \$11 million.
  \item
    For any player who has completed ten (10) or more Years of Service, the greater of (x) 35\% of the Salary Cap in effect at the time the assignment bonus is earned, (y) 105\% of the player's Salary for the Season prior to the Season in which the assignment bonus is earned, or (z) \$14 million.
  \end{enumerate}
\item
  For purposes of calculating the Maximum Annual Salary for the 2000-01 Salary Cap Year pursuant to Sections 7(a)(i)(x), 7(a)(ii)(x), 7(a)(iii)(x), 7(b)(i)(x), 7(b)(ii)(x), 7(b)(iii)(x), 7(c)(i)(x), 7(c)(ii)(x), 7(c)(iii)(x), 7(e)(i)(x), 7(e)(ii)(x), and 7(e)(iii)(x), the Salary Cap shall be the Salary Cap without any adjustment pursuant to Article VII, Section 2(d)(1).
\end{enumerate}

\hypertarget{promotional-activities.}{%
\section{Promotional Activities.}\label{promotional-activities.}}

\begin{enumerate}
\def\labelenumi{(\alph{enumi})}
\tightlist
\item
  A player's obligation (pursuant to paragraph 13(d) of a Uniform Player Contract) to participate, upon request, in all other reasonable promotional activities of the Team and the NBA shall be deemed satisfied if, during each year of the period covered by such Contract, the Player makes five individual personal appearances and five group appearances for or on behalf of or at the request of the Team (or Team Affiliate) by which he is employed and/or the NBA. Up to two of these ten appearances may be assigned by the Team and/or the NBA in any year to NBA Properties. The Player shall be reimbursed for the actual expenses incurred in connection with any such appearance, provided that such expenses result directly from the appearance and are ordinary and reasonable. The Player shall also receive compensation from the Team by which he is employed of at least \$1,000, in accordance with paragraph 13(d) of the Uniform Player Contract, for each promotional appearance he makes for a commercial sponsor of such Team. Any personal or group appearance required under this subsection (a) must:

  \begin{enumerate}
  \def\labelenumii{(\roman{enumii})}
  \tightlist
  \item
    take place during (A) the period from the first day of a Season through the day of the NBA Draft following such Season, or (B) the off-season, provided that no player may be required to make more than one off-season appearance in any year covered by his Contract and no player may be required to make such an off-season appearance unless he resides in or is otherwise located in the area where the appearance is to take place;
  \item
    occur in the home city (or geographic vicinity thereof) of the player's Team (subject to subsection (a)(i)(B) above) or in a city (or geographic vicinity thereof) to which the player has traveled to play in a scheduled NBA game;
  \item
    not occur at a time that would interfere with a player's reasonable preparation to play on the day of a Team game;
  \item
    not occur at a time that would interfere with a player's ability to attend and participate fully in any practice session conducted by the Team, taking into account the commuting time from the practice to the appearance;
  \item
    be scheduled with the player at least fourteen (14) days in advance (by providing written notice to the player of the time, nature, location, and expected duration of the appearance) and called to his attention again seven (7) days prior to the appearance;
  \item
    not exceed a reasonable period of time; and
  \item
    not require the player to sign autographs as the pri-mary purpose of the appearance.
  \end{enumerate}
\end{enumerate}

\hypertarget{day-contracts.}{%
\section{10-Day Contracts.}\label{day-contracts.}}

\begin{enumerate}
\def\labelenumi{(\alph{enumi})}
\tightlist
\item
  Beginning on January 5 (if a business day) or the first business day following January 5 of any NBA Season, and solely for the purpose of replacing an injured player, a Team may enter into a Player Contract with a player for the longer of (i) ten (10) days, or (ii) a period encompassing three (3) games played by such Team (a ``10-Day Contract''). No Team may enter into a 10-Day Contract with the same player more than twice during the course of any one Season.
\item
  The Salary provided for by a 10-Day Contract shall not be less than the Minimum Player Salary.
\item
  Notwithstanding anything to the contrary contained in a Uniform Player Contract, a 10-Day Contract shall be terminated simply by providing written notice to the player (and not by following the waiver procedure set forth in paragraph 16 of the Uniform Player Contract) and paying only such sums as are set forth in Exhibit 1 of such Contract.
\end{enumerate}

\hypertarget{rest-of-season-contracts.}{%
\section{Rest-of-Season Contracts.}\label{rest-of-season-contracts.}}

\begin{enumerate}
\def\labelenumi{(\alph{enumi})}
\tightlist
\item
  At any time after the first day of an NBA Regular Season, a Team may enter into a Player Contract that may provide Compensation to a player only for the remainder of that Season (a ``Rest-of-Season Contract'').
\item
  The Salary provided for in a Rest-of-Season Contract shall not be less than the Minimum Player Salary.
\end{enumerate}

\hypertarget{general.}{%
\section{General.}\label{general.}}

\begin{enumerate}
\def\labelenumi{(\alph{enumi})}
\item
  \begin{enumerate}
  \def\labelenumii{(\roman{enumii})}
  \tightlist
  \item
    Subject to Section 12 below, any oral or written agreement between a player and a Team concerning terms and conditions of employment shall be reduced to writing in the form of a Uniform Player Contract or an amendment thereto as soon as practicable. Immediately upon the consummation of any such agreement, the Team shall notify the NBA by facsimile or e-mail and provide the NBA with all economic terms of such agreement. Upon its receipt of such notice, the NBA shall promptly provide the same notice to the Players Association.
  \item
    Notwithstanding subsection (a)(i) above, neither the NBA nor the Players Association shall contend that any agreement concerning terms and conditions of employment is binding upon the player or the Team until a Player Contract embodying such terms and conditions has been duly executed by the parties. Nothing herein is intended to affect (A) any authority of the Commissioner to approve or disapprove Player Contracts, or (B) the effect of the Commissioner's approval or disapproval on the validity of such Player Contracts.(iii) A violation of the first sentence of subsection (a)(i) above may be considered evidence of a violation of Article XIII.
  \end{enumerate}
\item
  No player shall attend the regular training camp of any Team, or participate in organized practices with the Team at any time, unless he is a party to a Player Contract then in effect. For purposes of this Section 11(b), a player shall be considered to be a party to a Player Contract then in effect if such Contract has been extended in accordance with an Option permitted by this Agreement.
\item
  No Team shall make any direct or indirect payment of any money, property, investments, loans, or anything else of value for fees or otherwise to an agent, attorney, or representative of a player (for or in connection with such person's representation of such player); nor shall any Player Contract provide for such payment. The foregoing shall not, however, prevent a Team from sending a player's regular paycheck to a player's agent, attorney, or representative if so instructed in writing by the player.
\item
  Notwithstanding any other provision of this Agreement, (i) no Uniform Player Contract entered into after the date of this Agreement may provide for a signing bonus that exceeds twenty-five (25) percent of the Compensation (excluding Incentive Compensation) called for by the Contract (or, in the case of an Extension, by the extended term of the Extension), and (ii) no Offer Sheet may provide for a signing bonus that exceeds twenty-five (25) percent of the Compensation (excluding Incentive Compensation) called for by the Offer Sheet.
\item
  Notwithstanding any other provision of the Agreement, no Uniform Player Contract entered into after the date of this Agreement may provide for Non-Cash Compensation in any Season that exceeds twenty-five (25) percent of the Compensation (excluding Incentive Compensation) called for with respect to such Season.
\item
  Commencing with the 2001-02 Salary Cap Year, and notwithstanding any other provision of this Agreement, every Uniform Player Contract entered into after the date of this Agreement must provide (and every Uniform Player Contract entered into before the date of this Agreement shall be deemed to provide) that for each Season of such Contract, the player will be paid at least thirty (30) percent of his Salary for such Season in Current Cash Compensation and that such portion of Current Cash Compensation will be paid in accordance with the payment schedule provided in paragraph 3 of the Contract.
\item
  No Uniform Player Contract may provide for the payment of any Compensation earned for a Season prior to the July 1 immediately preceding such Season.
\item
  \begin{enumerate}
  \def\labelenumii{(\roman{enumii})}
  \tightlist
  \item
    No Uniform Player Contract may provide for the player's attendance at and participation in an off-season skill and conditioning program that exceeds two weeks in length.
  \item
    A Uniform Player Contract that contains a bonus to be paid as a result of the player's attendance at and participation in an off-season summer league and/or an off-season skill and conditioning program designated by the Team may also contain a provision providing that such bonus will be paid if the player, in the sole discretion of a physician designated by the Team, has an injury, illness or other medical condition that renders the player unable to participate in such summer league and/or skill and conditioning program.
  \end{enumerate}
\end{enumerate}

\hypertarget{july-moratorium.}{%
\section{July Moratorium.}\label{july-moratorium.}}

Notwithstanding any other provision of this Agreement, no player and Team may enter into any oral or written agreement concerning terms and conditions of the player's employment, or reduce any such agreement to writing in the form of a Uniform Player Contract or amendment, during the month of July. The foregoing sentence shall not preclude (i) a player from accepting any Required Tender or Qualifying Offer that is outstanding during the month of July, or (ii) a player and a Team from negotiating, during the month of July, over the terms and conditions of a Player Contract or Offer Sheet that may be entered into on or after August 1.

\hypertarget{player-expenses}{%
\chapter{PLAYER EXPENSES}\label{player-expenses}}

\hypertarget{moving-expenses.}{%
\section{Moving Expenses.}\label{moving-expenses.}}

\begin{enumerate}
\def\labelenumi{(\alph{enumi})}
\tightlist
\item
  A Team's obligation to reimburse a player for ``reasonable'' expenses related to the assignment of a Player Contract from one Team to another (in accordance with paragraph 10 of a Uniform Player Contract) shall extend to the reimbursement of the actual expenses incurred by such player in moving to the home territory of his new Team, provided that such expenses result directly from the assignment and are ordinary and reasonable, and provided further that, prior to his actually incurring such expenses, the player consults with the Team to which his Contract has been assigned (furnishing a written estimate of such proposed expenses, if requested by the Team), so as to afford such assignee-Team an opportunity to make reasonably comparable alternative arrangements for the move of the player. In the event that the assignee-Team requests an estimate of such proposed expenses, the player shall furnish such estimate to the Team within a reasonable time following the notice of the assignment of the Player Contract. Upon receipt of such estimate from the player, the Team shall, within ten (10) days, either agree to reimburse the player for the expenses set forth in such estimate or make alternative arrangements (at the Team's expense) for the move of the player.
\item
  A player whose Contract is assigned from one Team to another shall be reimbursed by the assignee-Team for the cost of a hotel room in a hotel (comparable to that in which such Team's players are lodged while ``on the road'') in the assignee-Team's home city for up to thirty (30) days following the assignment.
\item
  A Player whose Contract is assigned from one Team to another shall receive from the assignee-Team a sum equal to three months' rent on his living quarters in the city from which he is assigned; provided, however, that such payment shall be made only if and to the extent that the player is legally obligated for such rent, and
  shall not exceed \$3,000 per month.
\item
  Prior to its reimbursing an assigned player as provided in this Section, an assignee-Team may require satisfactory proof that the player has paid the amounts for which he seeks reimbursement, and, in the case of rent reimbursements, satisfactory proof that the player is legally obligated to pay such rent and the amount thereof. Upon notice to the player, the assignee-Team may, as an alternative to reimbursement, pay the expenses incurred upon assignment (in accordance with the foregoing provisions of this Section) directly to the persons, firms, or corporations involved.
\item
  So as to minimize the potential liability of NBA Teams under this Section, a player who does not establish permanent or year-round residence in the home city (or geographic vicinity thereof) of the Team by which he is employed shall use his best efforts (i) to obtain a short-term lease on the living quarters he selects, and (ii) to procure lease provisions authorizing him to sublet such premises and/or granting such Team the option to take over such lease in the event the Contract of such player is assigned to another NBA Team.
\end{enumerate}

\hypertarget{meal-expense-allowance.}{%
\section{Meal Expense Allowance.}\label{meal-expense-allowance.}}

\begin{enumerate}
\def\labelenumi{(\alph{enumi})}
\tightlist
\item
  The meal expense allowance, provided for in paragraph 4 of a Uniform Player Contract, shall be as follows:
  For the 1998-99 Season: \$85 per day.
  For each subsequent Season of this Agreement: \$85 plus a cost of living adjustment (which shall be calculated by applying to \$85 the percentage increase in the national Consumer Price Index from the June 1 through the May 31 immediately preceding such Season, and which shall be rounded off to the nearest whole dollar) per day.
\item
  When a Team is ``on the road'' for less than a full day, a partial meal expense shall be paid based upon the time of departure from or time of arrival in the Team's home city, in accordance with the following:

  \begin{enumerate}
  \def\labelenumii{(\roman{enumii})}
  \tightlist
  \item
    Departure after 9:00 a.m. or arrival before 7:00 a.m., no meal expense allowance for breakfast.
  \item
    Departure after 1:00 p.m. or arrival before 11:30 a.m., no meal expense allowance for lunch.
  \item
    Departure after 7:00 p.m. or arrival before 5:30 p.m., no meal expense allowance for dinner.\\
    For purposes of this Section 2(b), the meal expense allowance for breakfast shall be deemed to be 18\% of the applicable daily meal expense allowance (rounded off to the nearest whole dollar); the meal expense allowance for lunch shall be deemed to be 28\% of the applicable daily meal expense allowance (rounded off to the nearest whole dollar); and the meal expense allowance for dinner shall be deemed to be 54\% of the applicable daily meal expense allowance (rounded off to the nearest whole dollar).
  \end{enumerate}
\item
  For purposes of this Agreement and paragraph 4 of the Uniform Player Contract, the ``home city'' of an NBA Team shall be deemed to include only the city in which the facility regularly used by the Team for home games is located and any other location at which such home games are played, provided that such other location(s) is not more than 75 miles from such city.
\end{enumerate}

\hypertarget{benefits}{%
\chapter{BENEFITS}\label{benefits}}

\hypertarget{player-benefits.}{%
\section{Player Benefits.}\label{player-benefits.}}

Except as set forth below, effective with the date of this Agreement, and continuing for the duration thereof, the NBA shall provide the following benefits to NBA players and, in the case of Section (a) below, former NBA players:

\begin{enumerate}
\def\labelenumi{(\alph{enumi})}
\item
  \begin{enumerate}
  \def\labelenumii{(\arabic{enumii})}
  \tightlist
  \item
    Subject to the provisions of Section (a)(3) below, League-wide pension benefits in accordance with the terms of the National Basketball Association Players' Pension Plan, as restated effective February 2, 1996, as amended by the First and Second Amendments thereto (the ``Plan''). In accordance with the collective bargaining agreement made as of September 18, 1995, the Plan has been amended so that the ``Normal Retirement Pension'' payable to a player under the Plan is the maximum monthly amount permitted by the applicable benefit limitations under the Internal Revenue Code of 1986, as amended (the ``Code'') to be paid to the player at his ``Normal Retirement Date'' under the Plan (the ``Maximum Monthly Benefit'').\\
    Effective only for the duration of this Agreement or as otherwise required by the Code, the Maximum Monthly Benefit shall, except as otherwise provided herein, be adjusted for increases in the cost of living in the same manner as the cost of living adjustment for the dollar limitation under Section 415(b)(1)(A) of the Code. In no event, however, shall the adjusted Maximum Monthly Benefit for a Plan Year exceed an amount that would require the actuarially determined contributions (to be made to the Plan to fund for such adjusted benefit for the Plan Year) to exceed, by more than five (5) percent, the actuarially determined contributions that would be made to the Plan for that Plan Year using the Maximum Monthly Benefit in effect for the immediately preceding Plan Year. The parties agree that the determinations described in the preceding sentence, including any actuarial assumptions and projections related thereto, shall be made by the current actuaries of the Plan on a consistent basis and any such determinations shall be binding and conclusive. Any increase in the Maximum Monthly Benefit hereunder shall be effective as of the first day of the month following the beginning of the Plan Year of the Plan to which the increase relates (the ``Benefit Increase Commencement Date''), shall apply only with respect to benefit payments to be made on or after the Benefit Increase Commencement Date, and shall not require the recalculation of benefit payments made prior to the Benefit Increase Commencement Date. Notwithstanding the foregoing:

    \begin{enumerate}
    \def\labelenumiii{(\roman{enumiii})}
    \tightlist
    \item
      The benefits payable under the Plan shall at all times be subject to the limitations on benefits under the Code.
    \item
      If all or any portion of the actuarially determined contributions to be made to the Plan will not be fully deductible under the Code when paid, the Maximum Monthly Benefit shall not exceed the amount which would result in all of such contributions being fully deductible when paid. The Players Association shall be given written notice of any such determination. The parties agree that the determinations described in this subsection (ii), including any actuarial assumptions and projections related thereto, shall be made by the current actuaries of the Plan and any such determinations shall be binding and conclusive.
    \end{enumerate}
  \item
    Notwithstanding anything else in this Agreement: (i) if any change or amendment made to the Code, or the Employee Retirement Income Security Act of 1974, as amended (``ERISA''), or to any regulations (whether final, temporary or proposed) or rulings issued thereunder; or (ii) if any interpretation, application or enforcement (or any proposed interpretation, application or enforcement), by a court of competent jurisdiction in the United States or by the Internal Revenue Service, of the Code, ERISA, or any regulations or rulings issued thereunder; or (iii) if any regulations (whether final, temporary or proposed) or rulings issued by the Internal Revenue Service under the Code or ERISA; or (iv) if any provisions of this Agreement, including any of the amendments or benefit increases to be provided under the Plan pursuant to this Section, would result in the Plan no longer being a tax-qualified Plan under Section 401(a) of the Code, or would require NBA Teams to incur costs over and above any costs required to be incurred to implement the provisions of this Agreement or any prior collective bargaining agreement in order for the Plan to maintain its tax-qualified status under Section 401(a) of the Code (provided, however, that such additional costs are incurred solely in connection with the provision of pension benefits to their non-player employees or to non-player employees of affiliates (within the meaning of Sections 414(b), (c) or (m) of the Code) of such Teams), then any obligation to maintain and/or make contributions to the Plan pursuant to this Agreement or pursuant to any prior collective bargaining agreement shall terminate; provided, however, that any such termination shall not impair the legally binding effect of any other provision of this Agreement or the legally binding effect (if any) of any other provision of any prior collective bargaining agreement, nor shall it create any right (x) to unilaterally implement during the term of this Agreement any terms concerning the provision of pension benefits to the players, (y) to lockout, or (z) to strike. In the event of such termination, the NBA Teams shall provide alternative benefits to the players, at an annual cost (as determined on an after-tax basis) to NBA Teams equal to the annual cost that such Teams would have incurred under the Plan commencing on the date of termination. The NBA and the Players Association shall agree upon the type(s) of alternative benefits to be provided.
  \item
    Players employed by Toronto and Vancouver (``Canadian Players'') or by an NBA Team located in any other country other than the United States shall receive pension benefits of comparable value. Canadian Players shall receive such benefits by means of the Plan and separate pension plans established and maintained by Toronto and Vancouver (``Separate Plans''); provided, however, that (i) if the provision of pension benefits under the Plan to the Cana-dian Players would, at any time, result in the Plan being subject to Canadian Provincial Pension Legislation and/or Canadian Federal Tax Laws (to the extent that the application of such tax laws would result in adverse tax consequences to the Plan, the NBA Teams and/or the Canadian Players), and/or (ii) if the Separate Plans would not, upon their establishment or at any future time, either satisfy U.S. tax qualification requirements or be able to be registered under Canadian Provincial Pension Legislation and/or Canadian Federal Tax Laws, then any obligation to establish, maintain and/or make contributions to both the Plan with respect to Canadian Players and the Separate Plans pursuant to this Agreement or pursuant to any prior collective bargaining agreement shall terminate. In the event of such termination, Toronto and Vancouver shall provide alternative benefits to the Canadian Players at an annual cost (as determined on an after-tax basis) to Toronto and Vancouver equal to the annual cost that Toronto and Vancouver would have incurred under the Plan and the Separate Plans commencing on the date of termination. The NBA and the Players Association shall agree upon the type(s) of alternative benefits to be provided.
  \end{enumerate}
\item
  Life insurance and accidental death and dismemberment benefits, as set forth in the Prudential Insurance Company Policy No.~NBA 16144 (the ``Prudential Policy'') (which benefits were in effect for the 1997-1998 Season).
\item
  Disability insurance benefits, as set forth in the Standard Security Life Insurance Co.~of New York, Policy No.~SSL524-16343.
\item
  Workers' compensation benefits in accordance with applicable statutes.(e) Medical and Dental insurance benefits in accordance with the terms of the Prudential Policy, which policy was modified to the following extent effective commencing with the 1996-1997 Season:

  \begin{enumerate}
  \def\labelenumii{(\arabic{enumii})}
  \tightlist
  \item
    Subject to deductibles, the Prudential Policy covers 80\% of the first \$5,000, and 100\% thereafter, of qualifying expenses (as defined in the Prudential Policy) for each player and his eligible dependents in each year, subject to a maximum co-insurance obligation per family per year of \$3,000.
  \item
    Each player pays an annual deductible of \$300 for himself and each family member; provided, however, no further deductible obligation is required for any family member in any plan year in which a deductible of \$300 has been paid for each of three family members.\\
    The Prudential Policy as applicable to medical and dental insurance benefits or any subsequent policy or plan providing medical and dental benefits shall be modified or replaced effective as of the commencement of the 1999-2000 Season or as of the commencement of any subsequent Season covered by this Agreement, as requested in writing by the Players Association (a ``Player Change''), provided such written request is delivered to the NBA on or before the July 1 preceding such Season. Any Player Change shall be subject to the approval of the NBA, which approval shall not be unreasonably withheld. Any Player Change with respect to the 1999-2000 Season shall not result in an increase in the aggregate cost of medical and dental benefits to the Teams for the players for the 1999-2000 Season, over the aggregate cost of medical and dental benefits that otherwise would have been incurred by the Teams for the players under the Prudential Policy, absent any Player Change. Any additional aggregate costs that might be incurred by the Teams for medical and dental benefits with respect to the 2000-01 or any subsequent Season as a result of a Player Change, over the aggregate cost of medical and dental benefits that otherwise would have been incurred by the Teams for the players under the Prudential Policy with respect to such Season, absent any Player Change, shall be applied against the New Benefit Amounts provided for by Section 5 below.
  \end{enumerate}
\item
  Funding for an HIV/AIDS education program through the 1998-1999 Season in accordance with the terms of the agreement between Mosaic Health Inc.~and the National Basketball Players Association dated November 7, 1995 (the ``Mosaic Agreement''), provided, however, that the Players Association shall use its best efforts to reduce the amount payable under the Mosaic Agreement for or with respect to the 1998-99 Season. For the 1999-2000 Season and subsequent Seasons, the NBA and the Players Association shall agree upon a continued or new education program(s) for players, funded in the 1999-2000 Season in an amount equal to 105\% of the average of the amounts paid under the Mosaic Agreement for the 1996-1997 through 1998-1999 Seasons, with an increase of 5\% of such average amount for each remaining Season of the Agreement so that the funding for the 2004-2005 Season would be 130\% of such average amount.
\item
  Funding for the annual Players Association High School Basketball Camp (or any substitute program mutually agreed upon by the parties) in the amount of \$232,925 for the 1998-1999 Season increasing by 10\% per Season thereafter for the term of this Agreement.
\item
  Player Playoff Pool amounts, as follows:

  \begin{longtable}[]{@{}lc@{}}
  \toprule()
  \endhead
  1998-1999 Season & \$7.5 million \\
  1999-2000 Season & \$7.5 million \\
  2000-2001 Season & \$7.5 million \\
  2001-2002 Season & \$8 million \\
  2002-2003 Season & \$8 million \\
  2003-2004 Season & \$8 million \\
  2004-2005 Season & \$8.5 million \\
  \bottomrule()
  \end{longtable}

  If the NBA increases the number of Teams participating in the playoffs, the Player Playoff Pool shall be increased by \$468,750 for each Team added with respect to the 1998-1999 through 2000-2001 Seasons; by \$500,000 for each Team added for each Season thereafter through the 2003-2004 Season; and by \$531,250 for each Team added for the 2004-2005 Season. The NBA will consult with the Players Association with respect to the method of allocation of the Player Playoff Pool.
\item
  The employer's portion of payroll taxes.
\item
  The Players Association's one-half share of the payment of fees and expenses to the Accountants in connection with any audit conducted under this Agreement.
\item
  The Players Association's share of the costs of the Anti-Drug Program as provided for by Article XXXIII.
\item
  \begin{enumerate}
  \def\labelenumii{(\arabic{enumii})}
  \tightlist
  \item
    The sum of the Compensation paid to each player with five (5) or more Years of Service who signs a one-year, 10-Day or Rest-of-Season Contract for the Minimum Player Salary during a Season, less, for each such player, (i) for the 1998-99 Season, an amount equal to 50/82 times \$500,000 (prorated with respect to 10-Day and Rest-of-Season Contracts), and (ii) for each subsequent Season, the Minimum Player Salary for a player with four (4) Years of Service.
  \item
    The Compensation paid to any player with five (5) or more Years of Service in excess of the amounts set forth in clauses (i) and (ii) in subsection (l)(1) above shall be paid by the player's Team pursuant to the terms of such player's Uniform Player Contract, and then reimbursed to the Team out of a League-wide fund created and maintained by the NBA. Such reimbursement shall be made at the conclusion of the Season covered by the Contract.
  \end{enumerate}
\item
  The benefits funded by the New Benefit Amounts set forth in Section 5 below.
\end{enumerate}

\hypertarget{insurance-carriers.}{%
\section{Insurance Carriers.}\label{insurance-carriers.}}

At any time during the term of this Agreement, the NBA may change the carrier of any of the foregoing insurance programs, subject to the Players Association's prior written approval, which approval shall not be unreasonably withheld. In no event shall any change in insurance carrier result in a change in the types or levels of any of the benefits provided for above, except as otherwise requested by the Players Association under Section 1(e) above. In the event that a type of or level of benefit is not commercially available, the NBA may substitute a type of or level of benefit of comparable value, subject to the Players Association's approval, which approval shall not be unreasonably withheld.

\hypertarget{k-plan.}{%
\section{401(k) Plan.}\label{k-plan.}}

\begin{enumerate}
\def\labelenumi{(\alph{enumi})}
\tightlist
\item
  The NBA and the Players Association shall cause to be established for the 1999-2000 Season and for each subsequent Season during the term of this Agreement, a plan qualified under Section 401(a) of the Code which will permit deferrals by players pursuant to Section 401(k) of the Code (the ``401(k) Plan''). Commencing with the 2000-2001 Season, the 401(k) Plan shall provide for Team matching contributions in respect of player deferrals, as requested in writing by the Players Association. The request for the matching contributions by the Players Association for a Season shall be made at least seventy-five (75) days prior to the commencement of that Season or under such other procedure as otherwise agreed to by the NBA and the Players Association. Team matching contributions and deferrals shall be subject to all applicable limitations under the Code. The cost of funding all such matching contributions shall be applied against the New Benefit Amounts provided for by Section 5 below. Notwithstanding the foregoing:(1) The total amount of the (i) deferrals to be made by players to the 401(k) Plan, plus (ii) the Team matching contributions to be made to the 401(k) Plan in respect of such deferral contributions, shall be limited to an amount that, after first taking into account the contributions made to the National Basketball Association Players' Pension Plan, would result in all of such deferrals and matching contributions being fully deductible under the Code when paid to the 401(k) Plan; and (2) To the extent reasonably practicable, the terms of the 401(k) Plan shall permit participation by players with respect to employment in Canada on a tax-effective basis under Canadian income tax laws. If the NBA and the Players Association determine that the 401(k) Plan cannot be provided on a tax-effective basis under Canadian income tax laws with respect to employment in Canada, an alternative arrangement relating to employment in Canada, which is acceptable to both the NBA and the Players Association, shall be established in lieu of the 401(k) Plan. The cost to the Canadian Teams of funding for any such alternative arrangement shall be applied against the New Benefit Amounts provided for by Section 5 below.
\end{enumerate}

\hypertarget{post-career-benefit-plan-and-post-career-medical-plan.}{%
\section{Post-Career Benefit Plan and Post-Career Medical Plan.}\label{post-career-benefit-plan-and-post-career-medical-plan.}}

\begin{enumerate}
\def\labelenumi{(\alph{enumi})}
\tightlist
\item
  If requested by the Players Association in writing, the NBA and the Players Association shall cause to be established for the 2000-2001 Season and for each subsequent Season during the term of this Agreement, (1) a supplemental benefit plan (the ``Post-Career Benefit Plan'') to provide a post-career income supplement for players, and/or (2) a post-career retirement medical plan (the ``Post-Career Medical Plan'') to provide post-career medical benefits for players. The Post-Career Benefit Plan and Post-Career Medical Plan shall be in such form and provide for such benefits as shall be determined by the Players Association and communicated in writing to the NBA within a reasonable time (but not less than six (6) months) prior to the beginning of the Season, and shall be subject to the approval of the NBA, which approval shall not be unreasonably withheld. The cost of funding the Post-Career Benefit Plan and Post-Career Medical Plan and the plan costs incurred that are attributable to the establishment and administration of any such plan shall be applied against the New Benefit Amounts provided for by Section 5 below. The Post-Career Benefit Plan and Post-Career Medical Plan shall be structured and maintained in a manner that will result in all contributions by the Teams being fully deductible under the Code when paid; provided, however, that if a Team is disallowed a deduction (in whole or in part) for such contributions, the Team will bear the cost of any additional taxes (and penalties and/or interest) resulting from the disallowance of such deduction. If any Team is disallowed a deduction (in whole or in part) for contributions made to either or both the Post-Career Benefit Plan and/or Post-Career Medical Plan, and unless the NBA otherwise determines, the obligation to maintain either or both such Plans (as the case may be) and to make further contributions thereto shall immediately terminate; provided, however, that any such termination shall not impair the legally binding effect of any other provision of this Agreement, nor shall it create any right (i) to unilaterally implement during the term of this Agreement any terms concerning the provision of benefits provided for by the Post-Career Benefit Plan and/or Post-Career Medical Plan (as the case may be), (ii) to lockout, or (iii) to strike. In the event of such termination, the NBA Teams shall, subject to Section 5(c) below, provide alternative benefits to the players at an annual cost (as determined on an after-tax basis) to NBA Teams equal to the annual cost that such Teams would have incurred under the Post-Career Benefit Plan and/or Post-Career Medical Plan (as the case may be) commencing on the date of termination. The NBA and the Players Association shall agree upon the type(s) of alternative benefits to be provided, and the cost of funding for any such alternative benefits shall be applied against the New Benefit Amounts provided for by Section 5 below.(b) To the extent reasonably practicable, the terms of the Post-Career Benefit Plan and/or Post-Career Medical Plan shall permit participation by players with respect to employment in Canada on a basis under Canadian income tax laws that is substantially comparable to that under U.S. income tax laws. If the NBA and the Players Association determine that the Post-Career Benefit Plan and/or Post-Career Medical Plan cannot be provided on a basis under Canadian income tax laws that is substantially comparable to that under U.S. income tax laws, an alternative arrangement relating to employment in Canada, which is acceptable to both the NBA and the Players Association, shall be established in lieu thereof. The cost to the Canadian teams of funding for any such alternative arrangement shall be applied against the New Benefit Amounts provided for by Section 5 below.
\end{enumerate}

\hypertarget{new-benefits-funding.}{%
\section{New Benefits Funding.}\label{new-benefits-funding.}}

\begin{enumerate}
\def\labelenumi{(\alph{enumi})}
\item
  Commencing with the 2000-2001 Season and for each subsequent Season during the term of this Agreement, the following aggregate amounts (the ``New Benefit Amount'') shall be provided by the Teams to fund the benefits described in subsection (b) below, unless the Players Association designates a lesser amount with respect to a Season, by notice in writing to the NBA delivered on or before the March 15 prior to the commencement of the next Salary Cap Year:

  \begin{longtable}[]{@{}
    >{\raggedright\arraybackslash}p{(\columnwidth - 2\tabcolsep) * \real{0.4737}}
    >{\centering\arraybackslash}p{(\columnwidth - 2\tabcolsep) * \real{0.5263}}@{}}
  \toprule()
  \endhead
  Season & New Benefit Amount \\
  2000-2001 & \$500,000 multiplied by the number of Teams in the NBA during such Season \\
  2001-2002 & \$666,666 multiplied by the number of Teams in the NBA during such Season \\
  2002-2003 & \$833,333 multiplied by the number of Teams in the NBA during such Season \\
  2003-2004 & \$1,000,000 multiplied by the number of Teams in the NBA during such Season \\
  2004-2005 & \$1,100,000 multiplied by the number of Teams in the NBA during such Season \\
  \bottomrule()
  \end{longtable}
\item
  Subject to subsection (c) below, the New Benefit Amount shall be utilized in the following manner for each Season, unless otherwise directed in writing by the Players Association, which direction shall be subject to the approval of the NBA, which approval shall not be unreasonably withheld:

  \begin{enumerate}
  \def\labelenumii{(\arabic{enumii})}
  \tightlist
  \item
    Subject to the provisions of Section 3 above, to fund the cost of matching contributions with respect to players under the 401(k) Plan (and, if applicable, to fund the cost of any alternative arrangement for players on Canadian Teams).
  \item
    To fund any incremental cost of changes in the medical and dental benefits made pursuant to a Player Change in accordance with the provisions of Section 1(e) above.
  \item
    Subject to the provisions of Section 4 above, to fund the Post-Career Benefit Plan and/or the Post-Career Medical Plan and to pay the costs incurred that are attributable to the establishment and administration of any such plan (and/or, if applicable, to fund the cost of any alternative benefits or arrangement as may be agreed upon pursuant to Section 4 above).
  \end{enumerate}
\item
  Notwithstanding anything to the contrary in this Article IV, in no event shall the Teams (or the NBA) be required to pay amounts for any Season with respect to the benefits described in subsection (b) above in excess of the New Benefit Amount for such Season.
\end{enumerate}

\hypertarget{projected-benefits.}{%
\section{Projected Benefits.}\label{projected-benefits.}}

\begin{enumerate}
\def\labelenumi{(\alph{enumi})}
\tightlist
\item
  For purposes of computing the Salary Cap and Minimum Team Salary in accordance with Article VII, ``Projected Benefits'' shall mean the projected amounts to be paid or accrued by the NBA or the Teams, other than Expansion Teams during their first two Seasons, for the upcoming Season with respect to the benefits to be provided for such Season. In the event that the amount of any benefit for the upcoming Season is not reasonably calculable, then, for purposes of computing Projected Benefits, such amount shall be projected to be 108\% of the amount attributable to the same benefit for the prior Season.
\item
  For purposes of computing Projected Benefits, the projected amount to be paid to players with five (5) or more Years of Service who receive the Minimum Player Salary shall be computed by assuming that the number and Years of Service of players who sign one-year, 10-Day or Rest-of-Season Contracts for the Minimum Player Salary, and the duration of such Contracts, will be identical to the immediately preceding season.
\item
  For purposes of computing Projected Benefits with respect to a Salary Cap Year, there shall be taken into account any reduction in the New Benefit Amount with respect to a Season as designated by the Players Association, by notice in writing to the NBA delivered on or before the March 15 immediately preceding the commencement of that Salary Cap Year.
\end{enumerate}

\hypertarget{compensation-and-expenses-in-connection-with-military-duty}{%
\chapter{COMPENSATION AND EXPENSES IN CONNECTION WITH MILITARY DUTY}\label{compensation-and-expenses-in-connection-with-military-duty}}

\chaptermark{COMPENSATION AND EXPENSES \ldots}

\hypertarget{salary.}{%
\section{Salary.}\label{salary.}}

A player drafted into military service during the Season, or a player serving on active duty with a reserve unit during the Season, shall be compensated for so long as the player remains on the Active List of the Team in such amount as may be negotiated between the player and the Team by which he is employed, subject to the provisions of this Agreement.

\hypertarget{travel-expenses.}{%
\section{Travel Expenses.}\label{travel-expenses.}}

\begin{enumerate}
\def\labelenumi{(\alph{enumi})}
\tightlist
\item
  A player serving on military weekend duty with a reserve unit during the Season shall be entitled to reimbursement for any net out-of-pocket expenses incurred by such player in traveling to and from his place of duty to enable him to join his Team for purposes of participating in a Regular Season game.
\item
  In the event that the Player Contract of a player who is required to serve on military weekend duty with a reserve unit is assigned to another Team, the player shall be entitled to reimbursement for any out-of-pocket expenses incurred by such player in traveling during the off-season to and from his home and his place of military weekend duty with a reserve unit; provided that (i) the player makes reasonable efforts to change his reserve unit location to one located reasonably close to his home and (ii) such obligation to reimburse the player shall cease six (6) months from the date that such player's Contract is assigned.
\end{enumerate}

\hypertarget{player-conduct}{%
\chapter{PLAYER CONDUCT}\label{player-conduct}}

\hypertarget{general.-1}{%
\section{General.}\label{general.-1}}

\begin{enumerate}
\def\labelenumi{(\alph{enumi})}
\tightlist
\item
  In addition to any other rights a Team or the NBA may have by contract (including but not limited to the rights set forth in paragraphs 9 and 16 of the Uniform Player Contract) or by law, when a player fails or refuses, without proper and reasonable cause or excuse, to render the services required by a Player Contract or this Agreement, or when a player is, for proper cause, suspended by his Team or the NBA in accordance with the terms of such Contract or this Agreement, the Compensation payable to the player for the year of the Contract during which such refusal or failure and/or suspension occurs may be reduced (or, in the case of a suspension, shall be reduced) as follows:

  \begin{enumerate}
  \def\labelenumii{(\roman{enumii})}
  \tightlist
  \item
    By \$2,500 for each of the first two practices missed by the player during any Season, and by \$5,000 per practice for each missed practice thereafter throughout the remainder of that Season;
  \item
    By 1/90th of the player's Current Cash Compensation for each missed Exhibition, Regular Season or Playoff game; and
  \item
    By \$10,000 for each missed promotional appearance required in and in accordance with Article II, Section 8 and paragraph 13(d) of the Uniform Player Contract.
  \end{enumerate}
\end{enumerate}

\hypertarget{mandatory-programs.}{%
\section{Mandatory Programs.}\label{mandatory-programs.}}

If a player fails to attend, without reasonable excuse, a program designated as mandatory by the NBA and the Players Association pursuant to Article XXIX, Section 11, he shall be fined \$10,000 by the NBA; provided, however, that if the player misses the Rookie Transition Program, he shall be suspended for five (5) games.

\hypertarget{charitable-contributions.}{%
\section{Charitable Contributions.}\label{charitable-contributions.}}

\begin{enumerate}
\def\labelenumi{(\alph{enumi})}
\tightlist
\item
  In the event that (i) a fine or suspension is imposed on a player, (ii) such fine or suspension-related Compensation amount is collected by the League, and (iii) the fine or suspension is not grieved pursuant to Article XXXI, then the NBA shall remit fifty percent (50\%) of the amount collected to the National Basketball Players Association Foundation (the ``NBPA Foundation'') or such other charitable organization selected by the Players Association that qualifies for treatment under Section 501(c)(3) of the Internal Revenue Code of 1986, as now in effect or as it may hereafter be amended (a ``Section 501(c)(3) Organization''), and that is approved by the NBA (which approval shall not be unreasonably withheld) (both hereinafter, the ``NBPA-Selected Charitable Organization''). The NBA shall remit the remaining fifty percent (50\%) of the amount collected to a Section 501(c)(3) organization selected by the NBA and approved by the Players Association, which approval shall not be unreasonably withheld.
\item
  The remittances made by the NBA pursuant to this Section 3 shall be made annually, fifteen (15) days following the end of the NBA Season during which the fine or suspension-related Compensation amount is collected by the NBA. For purposes of this Article and all other provisions of this Agreement, any money remitted or paid to the National Basketball Players Association Foundation by the NBA shall be used for charitable purposes only, and not, for example, for any salaries of Foundation employees or administrative expenses.(c) If a timely Grievance is filed under Article XXXI challenging a fine or suspension of the kind designated in Section 3(a) above, and, following the disposition of the Grievance, the Grievance Arbitrator determines that all or part of the fine or suspension-related amount (plus any accrued interest thereon) is payable by the player to the League, then the League shall remit the amount collected by the League (plus any interest) in accordance with the provisions of Sections 3(a) and (b) above.
\end{enumerate}

\hypertarget{unlawful-violence.}{%
\section{Unlawful Violence.}\label{unlawful-violence.}}

When a player is convicted of (including a plea of guilty, no contest, or nolo contendere to) a violent felony, he shall immediately be suspended by the NBA for a minimum of ten (10) games.

\hypertarget{counseling-for-violent-misconduct.}{%
\section{Counseling for Violent Misconduct.}\label{counseling-for-violent-misconduct.}}

\begin{enumerate}
\def\labelenumi{(\alph{enumi})}
\tightlist
\item
  In addition to any other rights a Team or the NBA may have by contract or law, when the NBA and the Players Association agree that there is reasonable cause to believe that a player has engaged in any type of off-court violent conduct, the player will (if the NBA and the Players Association so agree) be required to undergo a clinical evaluation by a neutral expert and, if deemed necessary by such expert, appropriate counseling, with such evaluation and counseling program to be developed and supervised by the NBA and the Players Association. For purposes of this paragraph, ``violent conduct'' shall include, but not be limited to, sexual assault and acts of domestic violence.
\item
  Any player who, after being notified in writing by the NBA that he is required to undergo the clinical evaluation and/or counseling program authorized by subsection (a) above, refuses or fails, without a reasonable explanation, to attend or participate in such evaluation and counseling program within seventy-two (72) hours following such notice, shall be fined by the NBA in the amount of \$10,000 for each day following such seventy-two (72) hours that the player refuses or fails to participate in such program.
\end{enumerate}

\hypertarget{one-penalty.}{%
\section{One Penalty.}\label{one-penalty.}}

\begin{enumerate}
\def\labelenumi{(\alph{enumi})}
\tightlist
\item
  The NBA and a Team shall not discipline a player for the same act or conduct. The NBA's disciplinary action will preclude or supersede disciplinary action by any Team for the same act or conduct.
\item
  Notwithstanding anything to the contrary contained in Section 6(a), the same act or conduct by a player may result in both a termination of the player's Uniform Player Contract by his Team and the suspension of the player by the NBA if the egregious nature of the act or conduct is so lacking in justification as to warrant such double penalty.
\end{enumerate}

\hypertarget{league-investigations.}{%
\section{League Investigations.}\label{league-investigations.}}

\begin{enumerate}
\def\labelenumi{(\alph{enumi})}
\tightlist
\item
  Players are required to cooperate with investigations of alleged player misconduct conducted by the NBA. Failure to so cooperate, in the absence of a reasonable apprehension of criminal prosecution, will subject the player to reasonable fines and/or suspensions imposed by the NBA.
\item
  Except as set forth in subsection (c) below, the NBA shall provide the Players Association with such advance notice as is reasonable in the circumstances of any interview or meeting to be held (in person or by telephone) between an NBA representative and a player under investigation by the NBA for alleged misconduct, and shall invite a representative of the Players Association to participate or attend. The failure or inability of a Players Association representative to participate in or attend the interview or meeting, however, shall not prevent the interview or meeting from proceeding as scheduled. A willful disregard by the NBA of its obligation to notify the Players Association as provided for by this Section 7(b) shall bar the NBA from using as evidence against the player in a proceeding involving such alleged misconduct any statements made by the player in the interview or meeting conducted by the NBA representative.(c) The provisions of subsection (b) above shall not apply to interviews or meetings: (i) held by the NBA as part of an investigation with respect to alleged player misconduct that occurred at the site of a game and (ii) which take place during the course of, or immediately preceding or following, such game. With respect to any such interview or meeting, the NBA's only obligation shall be to provide notice to the Players Association that the NBA will be conducting an investigation and holding an interview or meeting in connection therewith. Such notice may be given by telephone at a telephone number, pager number or message-recording number to be designated in writing by the Players Association.
\end{enumerate}

\hypertarget{on-court-conduct.}{%
\section{On-Court Conduct.}\label{on-court-conduct.}}

In addition to its authority under paragraph 5 of the Uniform Player Contract, the NBA is entitled to promulgate and enforce reasonable rules governing the conduct of players on the playing court that do not violate the provisions of this Agreement. Prior to the date on which any new rule promulgated by the NBA becomes effective, the NBA shall provide notice of such new rule to the Players Association and consult with the Players Association with respect thereto.

\hypertarget{basketball-related-income-salary-cap-minimum-team-salary-and-escrow-arrangement}{%
\chapter{BASKETBALL RELATED INCOME, SALARY CAP, MINIMUM TEAM SALARY, AND ESCROW ARRANGEMENT}\label{basketball-related-income-salary-cap-minimum-team-salary-and-escrow-arrangement}}

\chaptermark{BASKETBALL RELATED INCOME \ldots}

\hypertarget{definitions.}{%
\section{Definitions.}\label{definitions.}}

For purposes of this Agreement, the following terms shall have the meanings set forth below:

\begin{enumerate}
\def\labelenumi{(\alph{enumi})}
\tightlist
\item
  Basketball Related Income.

  \begin{enumerate}
  \def\labelenumii{(\arabic{enumii})}
  \tightlist
  \item
    ``Basketball Related Income'' (``BRI'') for a Salary Cap Year means the aggregate operating revenues (including the value of any property or services received in any barter transactions) received or to be received on an accrual basis, for or with respect to such Salary Cap Year by the NBA, NBA Properties, Inc., including any of its subsidiaries whether now in existence or created in the future (hereinafter, ``Properties''), the NBA Market Extension Partnership (``Market Extension''), any other entity which is controlled by, or in which the NBA, Properties, Market Extension, and/or a group of NBA Teams owns at least 50\% of the issued and outstanding ownership interests (hereinafter, ``League-related entity'') (but excluding the amount of such League-related entity's revenues equal to the portion of its total revenues that is proportionate to the share of the entity's profits (assuming a distribution of profits) granted to ownership interests not owned by the NBA, Properties, Market Extension and/or a group of NBA Teams), all NBA Teams other than Expansion Teams during their first two Seasons (but including the Expansion Teams' shares of national television, radio, cable and other broadcast revenues, and any other League-wide revenues shared by the Expansion Teams, provided such revenues are otherwise included in BRI) and Related Parties in accordance with subsection (a)(7)(i) below, from all sources, whether known or unknown, whether now in existence or created in the future, to the extent derived from, relating to or arising directly or indirectly out of the performance of Players in NBA basketball games or in NBA-related activities. For purposes of this definition of BRI, (i) ``operating revenues'' shall include, but not be limited to, any type of revenue included in BRI for the 1995-96 and 1996-97 Salary Cap Years (without regard to whether such type of revenue is received on a lump-sum, non-recurring or extraordinary basis, but subject to any specific rules set forth in this Article VII relating to the recognition or amortization of such amounts); and (ii) ``Player'' means a person: who is under a Player Contract to an NBA Team; who completed the playing services called for under a Player Contract with an NBA Team at the conclusion of the prior Season; or who was under a Player Contract with an NBA Team during (but not at the conclusion of) the prior Season, but only with respect to the period for which he was under such Contract. Subject to the foregoing, BRI shall include, but not be limited to, the following revenues:

    \begin{enumerate}
    \def\labelenumiii{(\roman{enumiii})}
    \tightlist
    \item
      Regular Season gate receipts, net of applicable taxes, including, without limitation, gate receipts received or to be received on an accrual basis by a Related Party in accordance with subsection (a)(7)(i) below, including: (1) the value (determined on the basis of the price of the ticket) of all tickets traded by a Team for goods or services; and (2) the value (determined on the basis of the League-wide average ticket price for non-Season tickets) of all tickets for NBA Regular Season games provided by a Team on a complimentary basis, without monetary or other compensation to a Team, provided, however, that (x) the value of 1.3 million of such complimentary tickets for all NBA Regular Season games in an NBA Season shall be excluded from BRI, and (y) in addition, tickets provided as part of sponsorships and other transactions, where the proceeds from such transactions have been included in BRI, shall not be included in determining the number of complimentary tickets in any NBA Season;
    \item
      all proceeds of any kind, net of reasonable and customary expenses related thereto, from the broadcast or exhibition of, or the sale, license or other conveyance or exploitation of the right to broadcast or exhibit, NBA preseason, Regular Season and playoff games, highlights or portions of such games, and non-game NBA programming, on any and all forms of radio, television, telephone, internet, and any other communications media, forms of reproduction and other technologies, whether presently existing or not, anywhere in the world, whether live or on any form of delay, including, without limitation, network, local, cable, direct broadcast satellite and any form of pay television, and all other means of distribution and exploitation, whether presently existing or not and whether now known or hereafter developed, including, without limitation, such proceeds received or to be received on an accrual basis by a Related Party in accordance with subsection (a)(7)(i) below, but not including the value of any broadcast, cablecast or telecast time provided as part of any such transaction that is used solely: (1) to promote or advertise the NBA, its Teams, Players, or the sport of basketball (but not the value of time used to promote or advertise the WNBA, which shall be included in BRI); (2) to promote or advertise products, programming, merchandise, services or events that produce revenues that are includable in BRI or are receivable by Properties pursuant to the Group License Agreement or Events Agreement; (3) to promote or advertise charitable, not-for-profit or governmental organizations or agencies; or (4) for public service announcements;
    \item
      all Exhibition game proceeds of any kind, net of applicable taxes and all reasonable and customary game, preseason and training camp expenses, including, without limitation, such proceeds received or to be received on an accrual basis by a Related Party in accordance with subsection (a)(7)(i) below;
    \item
      all playoff gate receipts of any kind, net of admission taxes, arena rentals to the extent reasonable and customary, and all other reasonable and customary expenses, except the player playoff pool, including, without limitation, such proceeds received or to be received on an accrual basis by a Related Party in accordance with subsection (a)(7)(i) below;
    \item
      all proceeds of any kind, net of reasonable and customary expenses related thereto, subject to the provisions of subsection (a)(6) below, from in-arena sales of novelties and concessions, sales of novelties in team-identified stores within a 75-mile radius of the arena, NBA game parking and programs, Team sponsorships (whether or not the proceeds are directly or indirectly donated to charity), Team promotions, temporary arena signage, arena club revenues, summer camps, non-NBA basketball tournaments, mascot and dance team appearances, signage affixed to the exterior of an arena (subject to subsection (a)(2)(xv) below), and the sale of the right to pour beverages, in each case, to the extent that such proceeds are related to the performance of Players in NBA basketball games or NBA-related activities, including, without limitation, such proceeds received or to be received on an accrual basis by a Related Party in accordance with subsection (a)(7)(i) below;
    \item
      forty (40) percent of the gross proceeds from fixed arena signage in the arena in which an NBA Team plays more than one-half of its Regular Season home games, including, without limitation, such proceeds received or to be received on an accrual basis by a Related Party in accordance with subsection (a)(7)(i) below;(vii) forty (40) percent of the gross proceeds of any kind, net of all applicable taxes, from the sale, lease or licensing of luxury suites calculated on the basis of the actual proceeds received or to be received on an accrual basis by the entity, including, without limitation, proceeds received or to be received on an accrual basis by a Related Party in accordance with subsection (a)(7)(i) below, that sold, leased, or licensed such luxury suites; provided, however, that, other than the additional amounts paid by luxury suite holders to the Team for tickets, if any, this amount shall be the only amount included in BRI for the sale, lease or licensing of luxury suites and that, to the extent that the sale, lease or licensing of the luxury suite grants rights to the luxury suite for a period of more than one year, for purposes of calculating the amount includable in BRI for any Salary Cap Year, the proceeds shall be determined on the basis of the annual fee or charge provided for in any such transaction and, if payments are made in addition to or in the absence of such an annual fee or charge, the value of such payments shall be amortized over the period of the sale, lease or license, unless such period exceeds twenty (20) years, in which event an amortization period of twenty (20) years shall be used;
    \item
      except as provided in subsection (a)(2) below, proceeds received by Properties, net of actual expenses that are directly attributable to the generation of such proceeds, as long as those expenses are consistent with the types and categories of expenses incurred by Properties reflected in the audited financial reports for Properties for the year ended July 31, 1994 (or, in the case of new sources of proceeds, as long as the expenses are reasonable and customary, in the opinion of the Accountants, subject to the provisions of subsection (a)(6) below), including proceeds derived from the following categories (defined in the same manner as was used in those audited financial reports): (A) international television; (B) sponsorships; (C) NBA-related revenues from NBA Entertainment; (D) the All-Star Game and McDonald's Championship; (E) other NBA special events (other than events covered by the Events Agreement); and (F) all other sources of revenue received by Properties other than those specifically excluded under subsection (a)(2) below;
    \item
      proceeds from premium seat licenses (other than licenses of luxury suites, which are governed by subsection (a)(1)(vii) above) attributable to NBA-related events amortized over the period of the license (including, without limitation, such proceeds received or to be received on an accrual basis by a Related Party in accordance with subsection (a)(7)(i) below), unless such period exceeds twenty years, in which event an amortization period of twenty years shall be used; and
    \item
      if the right to receive revenues included in BRI is sold or transferred to an entity other than an entity referred to in subsection (a)(1) above (such that those revenues would not be included in BRI pursuant to that subsection), then BRI shall be deemed to include the amount of revenues that would have been received by the seller or transferor and would have been included in BRI in such Salary Cap Year (subject to any applicable allocations provided for above) absent such sale or transfer, provided that a pledge, hypothecation, collateral assignment or other similar transaction involving such revenues shall not be considered a sale or transfer within the meaning of this subsection (x).
    \end{enumerate}
  \item
    Notwithstanding anything to the contrary in subsection (a)(1) above, it is understood that the following is a non-exclusive list of examples of revenues that are or may be received by the NBA, Properties, Market Extension, other League-related entities, NBA Teams and Related Parties (the foregoing persons or entities, beginning with ``NBA,'' collectively referred to in this subsection (a)(2) only as ``NBA-related entities'') that are not derived from, and do not relate to or arise out of, the performance of Players in NBA basketball games or in NBA-related activities or are otherwise expressly excluded from the definition of BRI:

    \begin{enumerate}
    \def\labelenumiii{(\roman{enumiii})}
    \tightlist
    \item
      proceeds from the assignment of Player Contracts;
    \item
      proceeds (A) from the sale, transfer or other disposition of any of the assets or property (excluding ordinary course sales of inventory and the revenues (if any) deemed to be included in BRI pursuant to subsection (a)(1)(x) above) of, or ownership interests in, any NBA-related entity, or (B) from loans or other financing transactions;
    \item
      proceeds from the grant of Expansion Teams;
    \item
      dues;
    \item
      capital contributions received by an NBA-related entity from one of its owners, shareholders, members or partners;
    \item
      fines;
    \item
      revenue sharing (by means of revenue transfers or otherwise) among Teams;
    \item
      interest income;
    \item
      insurance recoveries;
    \item
      proceeds from the sale or rental of real estate;
    \item
      any thing of value received in connection with the design or construction of a new or renovated arena or other team facility including, but not limited to, receipt of title to or a leasehold interest in real property or improvements, reimbursement of project-related expenses, benefits from project-related infrastructure improvements, or tax abatements, unless (and only to the extent that) such value is being provided to the Team or a Related Party in lieu of payments that the Team or Related Party would have otherwise received pursuant to an arena lease or other instrument concerning a Team's use of an arena (``lease'') and would have constituted BRI if paid to the Team or a Related Party, provided, however, that the determination of the amount, if any, to be included in BRI with respect to the value of any of the foregoing shall be made either (x) in accordance with the provisions of subsection (a)(4) below or (y) based upon direct evidence that the Team or Related Party, after proposing that it would receive certain revenues constituting arena-generated BRI, subsequently agreed specifically to forego such revenues in direct exchange for a thing of value (as described above in this subsection (xi)) with the consequence that the arena-generated BRI revenues received or to be received by the Team or Related Party were or would be (in the opinion of the Accountants) less than the fair market value of arena-generated BRI revenues received or to be received by other NBA Teams in similar transactions, or (z) based upon direct evidence that the parties to the transaction had agreed that certain revenues constituting arena-generated BRI would be paid to the Team or Related Party and that such revenues were subsequently foregone by the Team or the Related Party in direct exchange for a thing of value (as described above in this subsection (xi)); and provided further that, when a determination is made pursuant to clause (y) or clause (z) of this subsection (xi), the amount(s), if any, to be included in BRI shall be allocated (with an appropriate interest adjustment to reflect the time value of money where the thing of value received by the Team or Related Party is in the form of cash or a cash equivalent, such as a check or wire transfer) over the Salary Cap Years in which the arena-generated BRI revenues foregone would have been received by the Team or Related Party (up to a maximum of twenty (20) Salary CapYears) and not on a lump-sum basis;
    \item
      any thing of value that induces or is intended to induce a Team either to relocate to or remain in a particular geographic location, unless (and only to the extent that) such value is being provided to the Team or a Related Party in lieu of payments that the Team or Related Party would have otherwise received pursuant to an arena lease and that would have constituted BRI had they been paid to the Team or a Related Party, provided, however, that the determination of the amount, if any, to be included in BRI shall be made either (x) in accordance with the provisions of subsection (a)(4) below or (y) based upon direct evidence that the parties to the transaction had agreed that certain revenues constituting arena-generated BRI would be foregone by the Team or Related Party in direct exchange for a thing of value as described above in this subsection (xii), and provided further that, when a determination is made pursuant to clause (y) of this subsection (xii), the amount(s), if any, to be included in BRI shall be allocated (with an appropriate interest adjustment to reflect the time value of money where the thing of value received by the Team or Related Party is in the form of cash or a cash equivalent, such as a check or wire transfer) over the Salary Cap Years in which the arena-generated BRI revenues foregone would have been received by the Team or Related Party (up to a maximum of fifteen (15) Salary Cap Years) and not on a lump-sum basis;
    \item
      payments made to Teams or to the NBA pursuant to the provisions of Article VII, Section 12 (Escrow Arrangement);
    \item
      distributions, dividends or royalties paid by any NBA-related entity to owners, shareholders, members or partners;
    \item
      proceeds from the sale of arena naming rights;
    \item
      any category or source of revenue or proceeds that was expressly identified in any BRI Report or in any document or written communication (including debriefing memos) authored by the Accountants and provided to the Players Association and the NBA (but excluding any underlying work papers) in connection with the BRI audits for the 1995-96 or 1996-97 Salary Cap Years that was not included in BRI for such Salary Cap Years;
    \item
      proceeds received by Properties (1) pursuant to the Group License Agreement, (2) pursuant to the Events Agreement and (3) relating to the following categories (defined in the same manner as was used in the audited financial reports for Properties for the year ended July 31, 1997): (x) licensing; and/or (y) Properties' representation of, and services performed for, third parties. For purposes of the foregoing sentence, ``third parties'' refers to persons or entities that are not owned or controlled by persons or entities that own a majority interest in or otherwise control an NBA Team or, if such third party is a Related Party, proceeds received by NBA Properties shall not be included in BRI if representation of such Related Party does not relate either to such entity's NBA ownership or NBA Players.
    \end{enumerate}
  \item
    The parties agree that in determining whether a category or source of revenue or proceeds constitutes BRI: (i) consideration shall be given to whether such category or source is more similar in kind or nature to the included categories and sources listed in subsections (a)(1)(i) through (x) above, on the one hand, or to the excluded categories and sources listed in subsections (a)(2)(i) through (xvii) above, on the other; and, (ii) no inference may be drawn from the fact that such category or source was not included in the categories and sources listed in subsections (a)(1)(i) through (ix) above or the fact that such category or source was not included in the categories and sources listed in subsections (a)(2)(i) through (xvii) above.
  \item
    The parties agree that, with respect to any lease entered into (i) prior to the date of this Agreement (but only with respect to an arena that is opened and used by an NBA team as its home arena after the end of the 1998-99 NBA Season) or (ii) after the date of this Agreement between a Team (or a Related Party) and an arena that is not a Related Party, the Accountants may attribute to the Team (or a Related Party) for purposes of computing BRI for a Salary Cap Year portions of arena revenues received by the arena or its related entities that would be included in BRI if received by the Team (or a Related Party) to the following extent: in the event of a renewal, extension or renegotiation of a lease between the same parties, or a new lease entered into by a Team (or a Related Party) with an arena that is not a Related Party, the Team will be deemed to receive in the first Salary Cap Year covered by the new lease or by the renewal, extension or renegotiation of the existing lease (as the case may be) the greater of the amount of such revenues that the Team or the Related Party in fact receives under the lease or, if in the opinion of the Accountants, the Team (and/or the Related Party) is receiving substantially less than fair market value as determined by the Accountants (taking into account factors such as the rent paid by the Team or the Related Party, the number and identity of other major tenants in the arena, market conditions, the extent to which arena revenues are used to fund construction or renovations of the arena, and comparable lease arrangements in the League), an amount determined by the Accountants to constitute the fair market value of the revenues that a tenant, in the same circumstances as the Team or Related Party, would receive for such Salary Cap Year. In either of the preceding cases, the Accountants will also determine the amount to be included in BRI for Salary Cap Years beyond the first Salary Cap Year.
  \item
    In no event shall the same revenues be included in BRI, directly or indirectly, more than once, the purpose of this provision being to preclude the double-counting of revenues.
  \item
    Subject to Article VII, Section 11 (Players Association Audit Rights):

    \begin{enumerate}
    \def\labelenumiii{(\roman{enumiii})}
    \tightlist
    \item
      With respect to expenses incurred in connection with all proceeds coming within subsections (a)(1)(v) and (viii) above, all reported expenses shall be conclusively presumed to be reasonable and customary (other than expenses related to sources of revenues that were not reflected in the audited financial report for Properties for the year ended July 31, 1994), and such expenses shall not be the subject of the accounting procedures set forth in Article VII, Section 10 of this Agreement. Such expenses shall be disallowed, however, to the extent that they exceed the ratio of League-wide reported expenses to League-wide reported revenues (the ``Expense Ratio'') for that category of revenues set forth in Exhibit D hereto.
    \item
      With respect to the NBA Store (the ``Store'') and any other new venture undertaken by the NBA, Properties, Market Extension, or any other League-related entity requiring significant capital investment or start-up costs (``New Venture''), reasonable and customary expenses shall include, but not be limited to: cost of goods sold, sales tax, all reasonable operating expenses of the Store or New Venture (including, but not limited to, salaries and benefits directly related to the operations of the Store or New Venture, promotional and advertising costs, rent, direct overhead, general and administrative expenses of the Store or New Venture), reasonable financing costs and amortization of capital improvements and start-up costs; provided, however, that in no event shall the expenses attributable to the Store or New Venture cause the amount included in BRI for the Store or New Venture to be less than zero (0) for any Salary Cap Year.
    \item
      With respect to new categories of revenue that may be included in BRI during the term of this Agreement (other than revenues attributable to the Store or a New Venture), the NBA, Properties, Market Extension, other League related entities, NBA Teams and Related Parties shall be able to deduct all expenses that the parties agree (or, in the absence of such agreement, that the Accountants determine) are reasonable and customary, provided, however, that if a new category of revenue is substantially similar to the type of revenues described in subsections (a)(1)(i) and (iv) above, the expenses attributable to such new category of revenue shall be deductible only to the extent contemplated by such subsections.
    \end{enumerate}
  \item
    It is acknowledged by the parties hereto that for purposes of determining BRI:

    \begin{enumerate}
    \def\labelenumiii{(\roman{enumiii})}
    \tightlist
    \item
      Some NBA Teams have engaged or may engage in transactions with third parties that control, or own at least 50\% of, the NBA team or that are controlled or owned at least 50\% by the persons or entities controlling or owning at least 50\% of the NBA Team (such third parties are referred to in this Agreement as ``a Related Party''), and Related Parties themselves engage in transactions with third parties that may result in a Related Party's receipt of revenues that constitute BRI. (Any entity that was an ``entity related to an NBA team'' as defined by Article VII, Section 1(a)(4)(i) of the September 18, 1995 Collective Bargaining Agreement between the NBA and the Players Association (the ``1995 CBA'') shall be deemed a Related Party under this Agreement for so long as such entity continues to be an entity related to an NBA Team within the meaning of the 1995 CBA.) As provided in subsection (a)(1) above, the relevant proceeds received or to be received on an accrual basis by any Related Party that come within such subsection and that relate to such Related Party's Team shall be included in BRI. However, with respect to any such revenues or proceeds retained or received by a Related Party (other than arena revenues that relate to such Related Party's Team including, but not limited to, in-arena sales of novelties and concessions, NBA game parking, arena club revenues, suite and seat revenues and fixed and temporary in-arena signage, which shall be included in BRI as if received by the Team), such revenues or proceeds shall be included in BRI only to the extent that the NBA and the Players Association agree or, if they fail to agree, the Accountants shall reasonably determine the amount, if any, of such revenues or proceeds to attribute to the Team (taking into account factors such as the nature of the transaction, arrangement and/or relationship between the Team and the Related Party or between the Related Party and a third party, any amounts included in BRI with respect to other Teams (or Related Parties) that have entered into comparable transactions, arrangements and/or relationships with third parties, market conditions, the nature of any services or activities performed by the Related Party for, or in connection with, the generation of revenues or proceeds and the amount of revenues or proceeds that the Related Party would be expected to retain or receive with respect to comparable transactions, arrangements and/or relationships with third parties), and the amount so allocated shall be the only amount included in BRI. To the extent that the amount of such proceeds to be included in BRI cannot reasonably be determined with respect to any particular transaction, the Accountants shall determine a reasonable amount with respect to such transaction, which shall be included in BRI.
    \item
      With respect to the transactions listed below in this subsection (a)(7)(ii), the parties agree that, because the proceeds attributable to these transactions cannot be accurately ascertained, the following procedures shall be used for each NBA Season in which these transactions remain in effect:

      \begin{enumerate}
      \def\labelenumiv{(\Alph{enumiv})}
      \tightlist
      \item
        New York Knicks transaction with MSG Network regarding the sale of local media rights: BRI for the Knicks for each NBA Season covered by this Agreement shall include an amount equal to the net proceeds included in BRI attributable to the Los Angeles Lakers' sale, license or other conveyance of all local media rights (including, but not limited to, broadcast and cable television and radio) for such NBA season.
      \item
        New York Knicks transactions with Related Parties involving signage: BRI for the Knicks for the 1999-2000 NBA Season shall include \$3,750,000 for signage. In each subsequent Season covered by this Agreement, this amount shall be increased (or decreased, as the case may be) by the League-wide percentage increase (or decrease) in signage as determined in accordance with subsections (a)(1)(v) and (a)(1)(vi).
      \end{enumerate}
    \end{enumerate}
  \item
    Pursuant to the NBA's national broadcast, national telecast and network cable television agreements, NBA Teams might receive revenue sharing proceeds that are attributable to NBA game telecasts (referred to hereinafter as ``Revenue Sharing Proceeds''). One-third of such Revenue Sharing Proceeds, if any, shall be included in BRI in each of the three Salary Cap Years following the Salary Cap Year in which such Revenue Sharing Proceeds are actually received. Any other contingent payments received by the NBA pursuant to such agreements shall be included in BRI to the extent and in a manner agreed upon by the parties, or, if the parties cannot agree, in a reasonable manner determined by the Accountants.
  \item
    The NBA and each NBA Team shall in good faith act and use their best efforts to maximize BRI for each Salary Cap Year during the term of this Agreement. In the exercise of such best efforts, the NBA and each NBA Team shall be entitled to act in a manner consistent with their sound business judgment and shall not take any action intended to benefit, at the expense of BRI, other commercial activities (such as the WNBA) unrelated to the performance of Players in NBA basketball games or in NBA-related activities. Without limiting the generality of the foregoing, the parties agree that it is within the sound business judgment of the NBA and each NBA Team to enter into, terminate or modify commercial arrangements or transactions, in good faith, in response to market exigencies, the acts or needs of unrelated third party business partners, and/or the best interests of NBA fans.
  \item
    The parties agree that upon a finding by the System Arbitrator (which, if appealed, is affirmed by the Appeals Panel) that the NBA or an NBA Team (or a Related Party) has willfully failed to provide to the Accountants information concerning revenues or expenses material to the Accountants' preparation of an Audit Report, the System Arbitrator may award such relief (including penalties, if appropriate) as is reasonable under the circumstances.
  \item
    After the date of this Agreement, neither the NBA or a League-related entity nor a Team or a Related Party will enter into any lease or other agreement providing for the receipt of revenues includable in BRI that contains provisions that purport to limit access of the Accountants to the books and records of the NBA, such League-related entity, such Team, or such Related Party in a manner inconsistent with the terms of this Agreement or that would preclude the calculation of revenues (if any) to be included in BRI pursuant to the provisions of Section 1(a)(1)(x) of this Article VII.
  \end{enumerate}
\item
  Core Basketball Revenues.

  \begin{enumerate}
  \def\labelenumii{(\arabic{enumii})}
  \tightlist
  \item
    The parties agree to bargain in good faith with respect to a potential new definition of the revenues that form the basis for the Salary Cap system, which would replace BRI. An outline of that new definition, called Core Basketball Revenues (``CBR''), is set forth in subsection (b)(3) below. The parties further agree that either party may decide, in its sole discretion, to refuse to agree to adopt a CBR-based system and that any such failure to agree shall not create any right to lock out or to strike.
  \item
    The procedures for converting from a BRI-based Salary Cap system to a CBR-based Salary Cap system, if possible, shall be as follows:

    \begin{enumerate}
    \def\labelenumiii{(\roman{enumiii})}
    \tightlist
    \item
      The Accountants shall be directed to complete an audit report for the 1997-98 Salary Cap Year no later than May 1, 1999 based upon the definition of BRI set forth in Section 1(a) and using such procedures as may be agreed upon by the parties for the purpose of such report.
    \item
      The Accountants shall also be directed to create a separate CBR audit report for the 1995-96, 1996-97 and 1997-98 Salary Cap Years no later than October 1, 1999, based upon the definition of CBR set forth in subsection (b)(3) below (as such definition may be modified by agreement of the parties) and using such procedures as may be agreed upon by the parties for the purpose of such report.
    \item
      Upon completion of the audit reports described in subsections (i) and (ii) above, the parties will attempt to agree upon a proposed ratio for converting all BRI-based percentages stated in this Agreement to CBR-based percentages, based upon BRI and CBR amounts for each of the 1995-96, 1996-97 and 1997-98 Salary Cap Years and other factors such as projected NBA revenue growth.
    \item
      In the event the parties reach agreement on converting from BRI to CBR, then, in addition to changing each BRI percentage to an equivalent CBR percentage, this Agreement shall otherwise be amended as necessary to reflect such conversion. In addition, the parties will agree as part of any such conversion to devise procedures for preventing Teams from diverting revenues out of CBR categories for the purpose of reducing CBR.
    \end{enumerate}
  \item
    CBR shall mean the revenues stated in subsections (a)(1)(i), (ii), (iii), (iv), (vii) and (viii) above (and, if applicable, subsection (a)(1)(x) above); provided, however, that such revenues shall be included in CBR on a gross basis, without any deduction for expenses of any kind other than applicable taxes. Notwithstanding the preceding sentence, the parties acknowledge that it is in their mutual benefit for the NBA, Properties, Market Extension and the Teams to (A) enter into new businesses during the term of this Agreement and (B) modify existing commercial arrangements, and that any agreement to utilize CBR in lieu of BRI should not deter the NBA, Properties, Market Extension or the Teams from entering into such new businesses or modifying existing arrangements. Accordingly, any agreement to convert to a CBR-based Salary Cap System must include provisions satisfactory to both parties that (i) neutralize the effect on CBR of any change in a business arrangement that materially increases or decreases the expenses associated with such arrangement, and (ii) promote the entering into of new businesses by taking account of the expenses associated with such businesses in calculating their inclusion in CBR.
  \end{enumerate}
\item
  \textbf{``Projected BRI''} for a Salary Cap Year means the sum of amounts determined in accordance with the following:

  \begin{enumerate}
  \def\labelenumii{(\arabic{enumii})}
  \tightlist
  \item
    With respect to BRI sources other than national broadcast, national telecast or network cable television contracts, Projected BRI shall include BRI for the preceding Salary Cap Year, increased by 8\%.
  \item
    With respect to national broadcast, national telecast or network cable television contracts, including the NBA/NBC agreement, dated November 11, 1997 (``NBA/NBC Agreement'') (a copy of which has been provided to the Players Association) and the NBA/TBS agreement, dated November 19, 1997 (``NBA/TBS Agreement'') (a copy of which has been provided to the Players Association), and national broadcast, national telecast or network cable television contracts covering Seasons that succeed the Seasons covered by the NBA/NBC and NBA/TBS Agreements (``Successor Agreements'') (copies of which shall be provided to the Players Association within ten (10) days of execution), Projected BRI for a Salary Cap Year shall include (A) the rights fees or other non-contingent payments stated in such contracts with respect to the Season covered by such Salary Cap Year (as such rights fees or non-contingent payments may be adjusted by agreement of the parties to such contracts); (B) the amounts of Revenue Sharing Proceeds, if any, that are includable in BRI for such Salary Cap Year pursuant to subsection (a)(8) above; (C) the amounts with respect to contingent payments (other than Revenue Sharing Proceeds), if any, attributable to Salary Cap Years covered by this Agreement in Successor Agreements as such amounts are agreed upon by the parties, or if the parties do not reach agreement, by the Accountants; and (D) the amount included in BRI for the preceding Salary Cap Year with respect to the value of advertising or promotional time provided to the NBA as part of the NBA/NBC and NBA/TBS Agreements (or any Successor Agreements) that is used to promote the WNBA or for any purpose other than those listed in subsections (a)(1)(ii)(1)-(4).
  \item
    Notwithstanding the foregoing, in the event that:

    \begin{enumerate}
    \def\labelenumiii{(\roman{enumiii})}
    \tightlist
    \item
      the rights fees stated in the NBA/NBC Agreement are reduced pursuant to such agreement (or by separate agreement between the NBA and NBC), and such reduction in the rights fees for the 1999-2000, 2000-01 and 2001-02 Seasons exceeds 70\% of the total reduction in rights fees agreed upon between the NBA and NBC, the amount of such excess shall, for purposes of calculating Projected BRI and BRI, be attributed to such Seasons pro rata based on the original rights fees called for with respect to such Seasons; and/or
    \item
      the rights fees stated in the NBA/TBS Agreement are reduced pursuant to such agreement (or by separate agreement between the NBA and TBS), and such reduction in the rights fees for the 1999-2000, 2000-01 and 2001-02 Seasons exceeds 70\% of the total reduction in rights fees agreed upon between the NBA and TBS, the amount of such excess shall, for purposes of calculating Projected BRI and BRI, be attributed to such Seasons pro rata based on the original rights fees called for with respect to such Seasons.
    \end{enumerate}
  \end{enumerate}
\item
  \textbf{``Local Expansion Team BRI''} means the BRI of the Expansion Teams during their first two Seasons, but not including the Expansion Teams' share of League-wide revenues that are otherwise included in BRI (including, but not limited to, their share of national television, cable, radio and other broadcast revenues).
\item
  \textbf{``Projected Local Expansion Team BRI''} means Local Expansion Team BRI for the immediately preceding Season, increased by 8\%.
\item
  \textbf{``Interim Projected BRI''} means a projection of BRI for a Salary Cap Year using Estimated BRI in place of BRI for the previous Salary Cap Year.
\item
  \textbf{``Barter''} means to trade by exchanging one commodity, service or other non-cash item for another.
\item
  \textbf{``Estimated Total Salaries and Benefits''} means the estimate of Total Salaries and Benefits for a Salary Cap Year as set forth in the Interim Audit Report for such Salary Cap Year.
\item
  \textbf{``Estimated BRI''} means the estimate of BRI for a Salary Cap Year as set forth in the Interim Audit Report for such Salary Cap Year.
\end{enumerate}

\hypertarget{calculation-of-salary-cap-and-minimum-team-salary.}{%
\section{Calculation of Salary Cap and Minimum Team Salary.}\label{calculation-of-salary-cap-and-minimum-team-salary.}}

\begin{enumerate}
\def\labelenumi{(\alph{enumi})}
\tightlist
\item
  Salary Cap.

  \begin{enumerate}
  \def\labelenumii{(\arabic{enumii})}
  \tightlist
  \item
    For each Salary Cap Year during the term of this Agreement, there shall be a Salary Cap. Subject to the adjustments set forth in subsection (d) below, the Salary Cap for each Salary Cap Year, beginning with the 2000-01 Salary Cap Year, will equal 48.04\% of Projected BRI for such Salary Cap Year, less Projected Benefits (as defined in Article IV, Section 6) for such Salary Cap Year, divided by the number of Teams scheduled to play in the NBA during such Salary Cap Year, other than Expansion Teams during their first two Seasons in the NBA.
  \item
    Notwithstanding subsection (a)(1) above, in the event that, subject to the adjustments set forth in subsection (d) below, Projected BRI for any Salary Cap Year, beginning with the 2000-01 Salary Cap Year, in which one or more Expansion Teams is scheduled to play its second NBA Season, plus Projected Local Expansion Team BRI for such Salary Cap Year, multiplied by .4804, less Projected Benefits for such Salary Cap Year (including for the Expansion Team(s)), divided by the number of Teams scheduled to play in the NBA during such Salary Cap Year (including the Expansion Team(s)), exceeds the Salary Cap calculated in accordance with subsection (a)(1) above, the Salary Cap shall equal the amount calculated pursuant to this subsection (a)(2).
  \item
    The Salary Cap for the 1998-99 Salary Cap Year shall be \$30 million.
  \item
    The Salary Cap for the 1999-2000 Salary Cap Year shall be \$34 million.
  \item
    The Salary Cap for a Salary Cap Year will be in effect commencing on August 1 of such Salary Cap Year and shall continue through and including the subsequent July 31.
  \item
    In the event that the Audit Report for a Salary Cap Year, beginning with the 1999-2000 Salary Cap Year, has not been completed as of the July 31 immediately following the end of such Salary Cap Year, then the Salary Cap for the Salary Cap Year that commenced on the immediately preceding July 1 will be calculated pursuant to subsections (a)(1)-(2) above, except that Interim Projected BRI shall be utilized instead of Projected BRI, Estimated BRI shall be utilized instead of BRI and Estimated Total Salaries and Benefits shall be utilized instead of Total Salaries and Benefits, for all purposes under this Section 2 including, without limitation, the adjustments set forth in subsection (d) below. In the event that the Interim Audit Report for a Salary Cap Year, beginning with the 1999-2000 Salary Cap Year, has not been completed as of the July 31 immediately following the end of such Salary Cap Year, then the Salary Cap for the Salary Cap Year that commenced on the immediately preceding July 1 shall, until such Interim Audit Report is completed, be an amount that would have been the Salary Cap for the preceding Salary Cap Year had Projected BRI or Interim Projected BRI, as the case may be, for such preceding Salary Cap Year included, with respect to the NBA's national broadcast, national telecast or network cable television contracts, the rights fees or other non-contingent payments stated in such contracts for the Season following the Season covered by such preceding Salary Cap Year instead of for the Season covered by such preceding Salary Cap Year.
  \end{enumerate}
\item
  Minimum Team Salary.

  \begin{enumerate}
  \def\labelenumii{(\arabic{enumii})}
  \tightlist
  \item
    For each Salary Cap Year during the term of this Agreement, there shall be a Minimum Team Salary equal to 75\% of the Salary Cap for such Salary Cap Year.
  \item
    In the event that, by the conclusion of a Salary Cap Year, a Team has failed to make aggregate Salary payments and/or incur aggregate Salary obligations equal to or greater than the applicable Minimum Team Salary for that Salary Cap Year, the NBA shall cause such Team to make payments equal to the shortfall (to be disbursed to the players on such Team pro rata or in accordance with such other formula as may be reasonably determined by the Players Association).
  \item
    Nothing contained herein shall preclude a Team from having a Team Salary in excess of the Minimum Team Salary, provided that the Team's Team Salary does not exceed the Salary Cap plus any additional amounts authorized pursuant to the Exceptions set forth in this Article VII.
  \end{enumerate}
\item
  \textbf{Expansion Team Salary Caps and Minimum Team Salaries.} Expansion Teams shall have the same Salary Caps and Minimum Team Salaries as all other Teams, except as follows:

  \begin{enumerate}
  \def\labelenumii{(\arabic{enumii})}
  \tightlist
  \item
    During the first Salary Cap Year in which it begins play, an Expansion Team shall have a Salary Cap equal to 66 and 2/3\% of the Salary Cap applicable to all other Teams (``The First Year Expansion Team Salary Cap''); and shall have a Minimum Team Salary equal to 75\% of The First Year Expansion Team Salary Cap.
  \item
    During the second Salary Cap Year in which it begins play, an Expansion Team shall have a Salary Cap equal to 75\% of the Salary Cap applicable to all other Teams (``The Second Year Expansion Team Salary Cap''); and shall have a Minimum Team Salary equal to 75\% of The Second Year Expansion Team Salary Cap.
  \end{enumerate}
\item
  Adjustments to Salary Cap and Minimum Team Salary.

  \begin{enumerate}
  \def\labelenumii{(\arabic{enumii})}
  \tightlist
  \item
    In the event that Total Salaries and Benefits for the 1999-2000 Salary Cap Year is greater than 55\% of BRI for such Salary Cap Year, then for purposes of calculating the Salary Cap for the 2000-01 Salary Cap Year, the amount of such overage shall be deducted from 48.04\% of Projected BRI for the 2000-01 Salary Cap Year. Notwithstanding the foregoing, in no event shall the adjustment described in this subsection (d)(1) reduce the Salary Cap for the 2000-01 Salary Cap Year to an amount that is less than \$35.5 million. In the event that, but for the limitation described in the preceding sentence, the adjustment described in this subsection (d)(1) would have reduced the Salary Cap for the 2000-01 Salary Cap Year to an amount less than \$35.5 million, there shall be no carryforward of any unused adjustment amount to any future Salary Cap Year.
  \item
    In the event that 48.04\% of BRI for a Salary Cap Year, beginning with the 2000-01 Salary Cap Year, is less than 48.04\% of Projected BRI for that Salary Cap Year, then for purposes of calculating the Salary Cap for the subsequent Salary Cap Year, the difference shall be deducted from 48.04\% of Projected BRI for such subsequent Salary Cap Year.
  \item
    In the event that Total Salaries and Benefits paid with respect to any Salary Cap Year, beginning with the 1999-2000 Salary Cap Year, plus, if applicable, any shortfall amount calculated pursuant to subsection (7) below, is less than 48.04\% of BRI for such Salary Cap Year (plus, if applicable, any shortfall against 48.04\% of BRI for the prior Salary Cap Year), then for purposes of calculating the Salary Cap for the subsequent Salary Cap Year, the amount of such shortfall shall be added to 48.04\% of Projected BRI for the subsequent Salary Cap Year; provided, however, that in the event there is a shortfall with respect to the 2003-04 Salary Cap Year (or, in the alternative, if the NBA exercises its option pursuant to Article XXXIX, the 2004-05 Salary Cap Year) the amount of such shortfall shall be paid by the NBA to the Players Association for distribution to all NBA players who were on an NBA roster during the 2003-04 (or 2004-05) Season on such proportional basis as may be reasonably determined by the Players Association.
  \item
    In the event that Benefits for any Salary Cap Year, beginning with the 2000-01 Salary Cap Year, exceeds Projected Benefits for such Salary Cap Year, the difference shall be added to Projected Benefits for the subsequent Salary Cap Year.
  \item
    In the event that Benefits for any Salary Cap Year, beginning with the 2000-01 Salary Cap Year, is less than Projected Benefits for such Salary Cap Year, the difference shall be deducted from Projected Benefits for the subsequent Salary Cap Year; provided, however, that in the event there is a shortfall with respect to the 2003-04 Salary Cap Year (or, in the alternative, if the NBA exercises its option pursuant to Article XXXIX, the 2004-05 Salary Cap Year), the amount of such shortfall shall be paid by the NBA to the Players Association for distribution to all NBA players who were on an NBA roster during the 2003-04 (or 2004-05) Salary Cap Year on such proportional basis as may be reasonably determined by the Players Association.
  \item
    In the event that the Salary Cap for a Salary Cap Year, beginning with the 2000-01 Salary Cap Year, is calculated in accordance with subsection (a)(6) above (i.e., is based upon an Interim Audit Report for the prior Salary Cap Year) and BRI and Total Salaries and Benefits as set forth in the Audit Report for the prior Salary Cap Year are different from those in the Interim Audit Report such that the Salary Cap would have been different from that based upon the Interim Audit Report, any such difference in the Salary Cap shall be debited or credited, as the case may be, to the Salary Cap for the subsequent Salary Cap Year, except that, with respect to the 2003-04 Salary Cap Year (or, in the alternative, if the NBA exercises its option pursuant to Article XXXIX, the 2004-05 Salary Cap Year) any such differences shall be debited or credited, as the case may be, to the Salary Cap for the then-current Salary Cap Year, in all such cases with interest (at a rate equal to the one year Treasury Bill rate as published in The Wall Street Journal on the date of the issuance of the Interim Audit Report).
  \item
    In the event that for any of the 1998-99 through 2001-02 Salary Cap Years, Total Salaries and Benefits do not increase over the prior Salary Cap Year's Total Salaries and Benefits by at least \$50 million, then the shortfall shall be paid by the NBA to the Players Association no later than thirty (30) days following the completion of the Audit Report for such Season for distribution to all NBA players who were on an NBA roster during such Season on such proportional basis as may be reasonably determined by the Players Association.
  \end{enumerate}
\end{enumerate}

\hypertarget{determination-of-salary.}{%
\section{Determination of Salary.}\label{determination-of-salary.}}

For the purposes of determining a player's Salary with respect to an NBA Season, the following rules shall apply:

\begin{enumerate}
\def\labelenumi{(\alph{enumi})}
\tightlist
\item
  Deferred Compensation.

  \begin{enumerate}
  \def\labelenumii{(\arabic{enumii})}
  \tightlist
  \item
    General Rules:

    \begin{enumerate}
    \def\labelenumiii{(\roman{enumiii})}
    \tightlist
    \item
      All Player Contracts entered into, extended or renegotiated after the date of this Agreement shall specify the Season(s) in which any Deferred Compensation is earned. Deferred Compensation shall be included in a player's Salary in the Season in which such Deferred Compensation is earned.
    \item
      Notwithstanding subsection (a)(1)(i) above, for purposes of an annuity compensation arrangement included in a Player Contract in accordance with Article XXV, Section 3 of this Agreement only, Deferred Compensation shall include only the portion of the cost of the annuity instrument to be paid by the Team after the playing term covered by the Contract, if any, and shall not include any Compensation that the player is scheduled to receive after the term of the Contract pursuant to such annuity compensation arrangement. The portion of the cost of the annuity paid by the Team while the player is required to render playing services under the Player Contract shall be included in Salary for the Salary Cap Year in which such cost is paid.
    \end{enumerate}
  \item
    Over 36 Rule: The following provisions shall apply to any Player Contract entered into, extended, or renegotiated that, beginning with the date such Contract, Extension or Renegotiation is signed, covers four (4) or more Seasons, including one (1) or more Seasons commencing after such player will reach or has reached age thirty-six (36) (an ``Over 36 Contract''):

    \begin{enumerate}
    \def\labelenumiii{(\roman{enumiii})}
    \tightlist
    \item
      Except as provided in subsections (a)(2)(ii) - (v) below, the aggregate Salaries in an Over 36 Contract for Seasons commencing with the fourth Season of such Over 36 Contract or the first Season following the player's 36th birthday, whichever is later, shall be attributed to the prior Seasons pro rata on the basis of the Salaries for such prior Seasons.
    \item
      Except in the case of a Contract between a Qualifying Veteran Free Agent and his Prior Team, which is governed by subsection (a)(2)(iii) below, if a player who is age 33, 34 or 35 enters into an Over 36 Contract covering more than four (4) Seasons, the aggregate Salaries in such Over 36 Contract for Seasons commencing with the fifth Season shall be attributed to the prior Seasons pro rata on the basis of the Salaries for such prior Seasons.
    \item
      If a Qualifying Veteran Free Agent who is age 32, 33 or 34 enters into an Over 36 Contract with his Prior Team covering more than five (5) Seasons, the aggregate Salaries in such Over 36 Contract for Seasons commencing with the sixth Season shall be attributed to the prior Seasons pro rata on the basis of the Salaries for such prior Seasons. If a Qualifying Veteran Free Agent who is age 35 enters into an Over 36 Contract covering more than four (4) Seasons, the aggregate Salaries in such Over 36 Contract commencing with the fifth Season shall be attributed to the prior Seasons pro rata on the basis of the Salaries for such prior Seasons.
    \item
      If a player who has played for his current Team for at least ten (10) consecutive Seasons enters into an Over 36 Contract that is an Extension and that, beginning with the date the Extension is signed, covers more than five (5) Seasons, the aggregate Salaries in such Over 36 Contract for Seasons commencing with the sixth Season shall be attributed to the prior Seasons pro rata on the basis of the Salaries for such prior Seasons.
    \item
      For each Season of an Over 36 Contract beginning with the second Season prior to the First Zero Year (as defined in subsection (a)(2)(viii) below), if the player has not been placed on waivers as of the July 1 prior to such Season, then the Salaries of the player for such Season and the subsequent two (2) or fewer Seasons covered by the Contract (including any Zero Year) shall, on such July 1, be aggregated and attributed in equal shares to each of such three (3) or fewer Seasons.
    \item
      Notwithstanding subsection (a)(2)(i) above, there shall be no re-allocation of Salaries pursuant to this Section 3(a)(2) for:
      (A) any Contract or Extension (beginning with the date the Extension is signed) covering four (4) or fewer Seasons entered into by a player at age 33, 34, or 35;
      (B) any Contract between a Qualifying Veteran Free Agent and his Prior Team covering five (5) or fewer Seasons entered into by a player at age 32, 33 or 34; and
      (C) any Extension that, beginning with the date the Extension is signed, covers five (5) or fewer Seasons and is entered into by a player who has played for his current Team for at least ten (10) consecutive Seasons.
    \item
      For purposes of determining whether a Contract is an Over 36 Contract pursuant to this Section 3(a)(2) only, Seasons shall be deemed to commence on October 1 and conclude on the last day of the Salary Cap Year.
    \item
      ``Zero Year'' means, with respect to an Over 36 Contract, any Season in which the Salary called for under the Contract has been attributed, in accordance with subsection (a)(2)(i), (ii), (iii) or (iv) above, to prior Seasons of the Contract. ``First Zero Year'' means, with respect to an Over 36 Contract, the earliest Season in which the Salary called for under the Contract has been attributed, in accordance with subsections (a)(2)(i), (ii), (iii) or (iv) above, to prior Seasons of the Contract.
    \item
      For purposes of this subsection (a)(2), a player (x) whose birthday is on a date during the month of July and (y) who signs a Contract, Extension or Renegotiation on or before August 5 shall be treated as if his age, at the time of such signing, was his age on the immediately preceding June 30.
    \end{enumerate}
  \end{enumerate}
\item
  Signing Bonuses.

  \begin{enumerate}
  \def\labelenumii{(\arabic{enumii})}
  \tightlist
  \item
    Amounts Treated as Signing Bonuses: For purposes of determining a player's Salary, the term ``signing bonus'' shall include:

    \begin{enumerate}
    \def\labelenumiii{(\roman{enumiii})}
    \tightlist
    \item
      any amount provided for in a Player Contract that is earned upon the signing of such Contract;
    \item
      any Option Buy-out Amount;
    \item
      at the time of an assignment of a Player Contract, any amount that, under the terms of the Contract, is earned in the form of a bonus upon assignment of the Contract; and
    \item
      payments in excess of \$350,000 with respect to foreign players, in accordance with Section 3(e) below.
    \end{enumerate}
  \item
    Proration: Any signing bonus contained in a Player Contract shall be allocated in equal parts over the number of Seasons (or remaining Seasons in the case of a signing bonus described in subsection (b)(1)(iii) above) covered by such Contract that are fully protected for lack of skill, provided, however, that if the Player Contract provides for an Early Termination Option (``ETO''), the signing bonus shall be allocated in equal parts only over the Seasons that are fully protected for lack of skill that precede the Effective Season of such ETO. In the event that no Season covered by a Player Contract is fully protected for lack of skill, then the entire amount of the signing bonus shall be allocated to the first Season of the Contract or, in the case of a signing bonus described in subsection (b)(1)(iii) above, the Season during which the player's Contract is assigned (or the upcoming Season in the case of an assignment that occurs from July 1 through the day prior to the first day of the Season).
  \item
    Signing Bonus Credits: Upon the occurrence of an event that determines that a player shall not be entitled to receive an Option Buy-Out Amount (the ``non-payment determination''):

    \begin{enumerate}
    \def\labelenumiii{(\roman{enumiii})}
    \tightlist
    \item
      all amounts that were included in the player's Salary pursuant to subsection (b)(1)(ii) above for Salary Cap Years up to and including the Salary Cap Year in which the non-payment determination is made (the ``unpaid amounts'') shall be deducted from the calculation of Total Salaries and Benefits for the Salary Cap Year in which the non-payment determination is made;
    \item
      all amounts that were included in the player's Salary pursuant to subsection (b)(1)(ii) above for Salary Cap Years following the Salary Cap Year in which the non-payment determination is made shall be deducted from the player's Salary for such Salary Cap Years; and
    \item
      the unpaid amounts shall be deducted from the Team's Team Salary, in accordance with the following:
      (A) The total amount available to be deducted from the Team's Team Salary (the ``credit amount'') will equal the aggregate of the unpaid amounts less, for each Season in which a portion of the unpaid amounts was included in the player's Salary and in which his Team's Team Salary did not fall below the Salary Cap, the smallest amount by which his Team's Team Salary exceeded the Salary Cap during such Salary Cap Year.
      (B) The credit amount shall be allocated, in equal parts, over the same number of Salary Cap Years over which the unpaid amounts were allocated, beginning with the first Salary Cap Year following the non-payment determination, plus, for each Salary Cap Year following the first Salary Cap Year of such allocation, 10\% of the amount allocated to the first Salary Cap Year.
      (C) If, during the course of any Salary Cap Year in which a credit allocation has been made, the Team's Team Salary does not fall below the Salary Cap, the full credit allocation for such Salary Cap Year will be carried forward to a subsequent Salary Cap Year. If, during the course of a Salary Cap Year in which a credit allocation has been made, the Team's Team Salary does fall below the Salary Cap, the amount carried forward, if any, will equal the amount of the credit allocation for such Salary Cap Year less the largest amount by which the Team's Team Salary fell below the Salary Cap during such Salary Cap Year. In the event a credit allocation is carried forward pursuant to this subsection, such amount shall be deducted from Team Salary in the Salary Cap Year immediately following the last Salary Cap Year in which a portion of the credit amount is then currently being allocated, subject to the terms of this subsection (b)(3)(iii)(C).
    \end{enumerate}
  \item
    Extensions:

    \begin{enumerate}
    \def\labelenumiii{(\roman{enumiii})}
    \tightlist
    \item
      In the event that a Team with a Team Salary at or over the Salary Cap enters into an Extension that calls for or contains a signing bonus, such signing bonus shall be paid no sooner than the first day of the first Salary Cap Year covered by the extended term and shall be allocated, in equal parts, over the number of Seasons covered by the extended term that are fully protected for lack of skill. In the event that no Season in the extended term is fully protected for lack of skill, then the entire amount of the signing bonus shall be allocated to the first Season of the extended term.
    \item
      A Team with a Team Salary below the Salary Cap may enter into an Extension that calls for or contains a signing bonus to be paid at any time during the Contract's original or extended term. In the event that a Team with a Team Salary below the Salary Cap enters into an Extension that calls for or contains a signing bonus to be paid no sooner than the first day of the Salary Cap Year covered by such extended term, the bonus shall be allocated in accordance with the proration rules set forth in subsection (b)(4)(i) above. In the event a Team with a Team Salary below the Salary Cap enters into an Extension that calls for or contains a signing bonus to be paid prior to the first day of the first Salary Cap Year covered by the extended term, the following rules shall apply:

      \begin{enumerate}
      \def\labelenumiv{(\Alph{enumiv})}
      \tightlist
      \item
        The signing bonus shall be allocated in equal parts over the Seasons remaining under the original term of the Contract and the extended term that are fully protected for lack of skill; and
      \item
        The Extension shall be deemed a Renegotiation and shall be subject to the rules governing Renegotiations set forth in Section 7 below.
      \end{enumerate}
    \end{enumerate}
  \end{enumerate}
\item
  \textbf{Loans to Players.} The following rules shall apply to any loan made by any Team to a player:

  \begin{enumerate}
  \def\labelenumii{(\arabic{enumii})}
  \tightlist
  \item
    If any such loan bears no interest (or interest at an effective rate lower than 9\% per annum), then an amount equal to 9\% per annum of the outstanding balance (or an amount equal to the difference between 9\% per annum of the outstanding balance and the actual rate of interest to be paid by the player) shall be included in the player's Salary.
  \item
    No loan made to a player after July 1, 1996 may (along with other outstanding loans to the player) exceed the amount of the player's Salary for the Season covered by the then-current Salary Cap Year that is fully protected for lack of skill. All loans must be repaid through deductions from the player's remaining Current Cash Compensation over the term of the Contract that is fully protected for lack of skill (prior to the Effective Season of any Early Termination Option) in equal annual amounts (the ``annual allocable repayment amounts''). If a loan is made at a time when the remaining Current Cash Compensation due for the relevant Season that is fully protected for lack of skill is less than the annual allocable repayment amount that would be owed on a loan for the full amount of the player's Current Cash Compensation that is fully protected for lack of skill for the relevant Season (the ``maximum annual allocable repayment amount''), the maximum loan amount for that Season shall be reduced by the amount by which the maximum annual allocable repayment amount exceeds the amount of remaining Current Cash Compensation that is fully protected for lack of skill. (For example, if a Player has \$1 million in Current Cash Compensation (fully protected for lack of skill) in the first Season of a five-year Contract, and a loan is made during that Season at a time when the Player has already received his Current Cash Compensation for that Season, the loan may not exceed \$800,000.)
  \item
    In addition to the restrictions set forth in subsection (c)(2) above, no loan may be made after June 30, 2001 that would result in a violation of Article II, Section 11(f).
  \item
    Any forgiveness by a Team of a loan to a player shall be deemed a Renegotiation in the Salary Cap Year of such forgiveness and shall be subject to the rules governing Renegotiations set forth in Section 7 below.
  \end{enumerate}
\item
  Incentive Compensation.

  \begin{enumerate}
  \def\labelenumii{(\arabic{enumii})}
  \tightlist
  \item
    For purposes of determining a player's Salary each Season, except as provided in subsections (d)(2)-(4) below, any Performance Bonus (provided such Performance Bonus may be included in a Player Contract in accordance with Section 5(d) below), shall be included in Salary only if such Performance Bonus would be earned if the Team's or player's performance were identical to the performance in the immediately preceding Season.
  \item
    Notwithstanding subsection (d)(1) above, in the event that, at the time of the signing of a Contract, Renegotiation or Extension, the NBA or the Players Association believes that the performance of a player and/or his team during the immediately preceding Season does not fairly predict the likelihood of the player earning a Performance Bonus during any Season covered by the Contract, Renegotiation or extended term of the Extension (as the case may be), the NBA or the Players Association may request that a jointly selected basketball expert (``Expert'') determine whether (i) in the case of an NBA challenge, it is very likely that the bonus will be earned, or (ii) in the case of a Players Association challenge, it is very likely that the bonus will not be earned. The party initiating a proceeding before the Expert shall carry the burden of proof. The Expert shall conduct a hearing within five (5) business days after the initiation of the proceeding, and shall render a determination within five (5) business days after the hearing. Notwithstanding anything to the contrary in this subsection (d)(2), no party may, in connection with any proceeding before the Expert, refer to the facts that, absent a challenge pursuant to this subsection (d)(2), a Performance Bonus would or would not be included in a player's Salary pursuant to subsection (d)(1) above, or would be termed ``Likely'' or ``Unlikely'' pursuant to Article I, Section 1(aa) or (nnn). If, following an NBA challenge, the Expert determines that a Performance Bonus is very likely to be earned, the bonus shall be included in the player's Salary. If, following a Players Association challenge, the Expert determines that a Performance Bonus is very likely not to be earned, the bonus shall be excluded from the player's Salary. The Expert's determination that a Performance Bonus is very likely to be earned or very likely not to be earned shall be final, binding and unappealable. The fees and costs of the Expert in connection with any proceeding brought pursuant to this subsection (d)(2) shall be borne equally by the parties.
  \item
    In the case of a Rookie or a Veteran who did not play during the immediately preceding Season, a Performance Bonus will be included in Salary if it is likely to be earned. In the event that the NBA and the Players Association cannot agree as to whether a Performance Bonus is likely to be earned, such dispute will be referred to the Expert, who will determine whether the bonus is likely to be earned or not likely to be earned. The Expert shall conduct a hearing within five (5) business days after the initiation of the proceeding, and shall render a determination within five (5) business days after the hearing. The Expert's determination that a Performance Bonus is likely to be earned or not likely to be earned shall be final, binding and unappealable. The fees and costs of the Expert in connection with any proceeding brought pursuant to this subsection (d)(3) shall be borne equally by the parties.
  \item
    In the event that either party initiates a proceeding pursuant to subsection (d)(2) or (3) above, the player's Salary plus the full amount of any disputed bonuses shall be included in Team Salary during the pendency of the proceeding.
  \item
    In the event the NBA and the Players Association cannot agree on an Expert, any challenge pursuant to subsections (d)(2) and (3) above may be filed with the Grievance Arbitrator in accordance with Article XXXI, Sections 2-6 and 14.
  \item
    Notwithstanding anything to the contrary in Section 3(g) below, all Incentive Compensation described in Article II, Sections 3(c)(iv) and 3(d) shall be included in Salary.
  \end{enumerate}
\item
  Foreign Player Payments.

  \begin{enumerate}
  \def\labelenumii{(\arabic{enumii})}
  \tightlist
  \item
    Any amount in excess of \$350,000 paid or to be paid by or at the direction of any NBA Team to (i) any basketball team other than an NBA Team, or (ii) any other entity, organization, representative or person, for the purpose of inducing a foreign player (as defined in Article X, Section 6) to enter into a Player Contract or in connection with securing the right to enter into a Player Contract with a foreign player shall be deemed Salary (in the form of a signing bonus) to the player.
  \item
    Subject to Article XIII, any payment of \$350,000 or less paid by or at the direction of any NBA Team pursuant to subsection (1) above (the ``\$350,000 exclusion''), shall not be deemed Salary to the player.
  \item
    The \$350,000 exclusion may be paid in a single installment or in multiple installments. The \$350,000 exclusion, whether used in whole or in part, may be used by an NBA Team whenever it signs a foreign player to a new Player Contract, except that the \$350,000 exclusion may not be used, in whole or in part, more than once in any three-Season period with respect to the same foreign player.
  \item
    The \$350,000 exclusion, or any part of it, shall be deemed to have been used as of the date of the Player Contract to which it applies, regardless of when it is actually paid. A schedule of payments relating to the \$350,000 exclusion, or any part of it, agreed upon at the time of the signing of the Player Contract to which it applies, shall not be deemed a multiple use of the \$350,000 exclusion.
  \item
    Notwithstanding subsection (e)(1) above, no amount paid or to be paid pursuant to this subsection (e) shall be counted toward the Minimum Team Salary obligation of a Team in accordance with Section 2(b) or (c) above.
  \end{enumerate}
\item
  One-Year Minimum Contracts.\\
  Except where otherwise stated in this Agreement, the Salary of every player who, after the date of this Agreement, signs a one-year, 10-Day or Rest-of-Season Contract for the Minimum Player Salary applicable to such player shall be the lesser of (i) such Minimum Player Salary, or (ii) the portion of such Minimum Player Salary that is not reimbursed out of the league-wide benefits fund described in Article IV, Section 1(l).
\item
  Existing Contracts.\\
  A player's Salary with respect to any Salary Cap Year covered by a Contract entered into prior to the date of this Agreement shall continue to be calculated in accordance with the Salary Cap rules that were in existence at the time the Contract was entered into. In no event shall the preceding sentence apply to the calculation of Salary with respect to any Contract, Extension or Renegotiation entered into on or after the date of this Agreement.
\end{enumerate}

\hypertarget{determination-of-team-salary.}{%
\section{Determination of Team Salary.}\label{determination-of-team-salary.}}

\begin{enumerate}
\def\labelenumi{(\alph{enumi})}
\tightlist
\item
  \textbf{Computation.} For purposes of computing Team Salary under this Agreement, all of the following amounts shall be included:

  \begin{enumerate}
  \def\labelenumii{(\arabic{enumii})}
  \item
    Subject to the rules set forth in this Article VII, the aggregate Salaries of all active players (and former players to the extent provided by the terms of this Agreement) attributable to a particular Salary Cap Year, including, without limitation:

    \begin{enumerate}
    \def\labelenumiii{(\roman{enumiii})}
    \tightlist
    \item
      Salaries payable to players whose Player Contracts have been terminated pursuant to the NBA's waiver procedure (without regard to any revised payment schedule that might be provided for in the terminated Player Contracts).
    \item
      Any amount called for in a retired player's Player Contract paid or to be paid to the player.
    \item
      Amounts paid pursuant to awards or judgments for, or settlements of, disputes between a Player and a Team concerning Compensation obligations under a Player Contract except to the extent that such amounts were previously included in a player's Salary. In the event that any amounts paid to a player as described in the preceding sentence relate to a Team's Compensation obligation for only one (1) Season, such amounts shall be included in Team Salary for the Salary Cap Year during which the Team's obligation to pay such amounts is determined, unless the Team's obligation is determined during the period from November 1 through June 30 of any Salary Cap Year, in which case such amounts shall be included in Team Salary for the following Salary Cap Year. In the event that any such amounts relate to a Team's Compensation obligation for more than one (1) Season, such amounts shall be included in Team Salary in equal amounts over the same number of Salary Cap Years, with the first such Salary Cap Year being the Salary Cap Year during which the Team's obligation to pay such amounts is determined, unless the Team's obligation is determined during the period from November 1 through June 30 of any Salary Cap Year, in which case the following Salary Cap Year shall be the first Salary Cap Year in which such amounts are included in Team Salary.
    \item
      Salaries anticipated to be included in Team Salary based upon any agreement disclosed to the NBA pursuant to Article II, Section 11(a)(i), except to the extent that any such Salary is less than a player's Free Agent Amount (as defined in subsection (d) below).
    \end{enumerate}
  \item
    \begin{enumerate}
    \def\labelenumiii{(\roman{enumiii})}
    \tightlist
    \item
      With respect to each Veteran Free Agent who last played for a Team who is an Unrestricted Free Agent, the Free Agent Amount (as defined in subsection (d) below) attributable to such Veteran Free Agent.
    \item
      With respect to each Veteran Free Agent who last played for a Team who is a Restricted Free Agent, the greater of (A) the Free Agent Amount (as defined in subsection (d) below) attributable to such Veteran Free Agent, (B) the Salary called for in any outstanding Qualifying Offer tendered to such Veteran Free Agent, or (C) the Salary called for in any First Refusal Exercise Notice issued with respect to such Veteran Free Agent.
    \end{enumerate}
  \item
    The aggregate Salaries called for under all outstanding Offer Sheets.
  \item
    An amount with respect to a Team's unsigned First Round Pick, if any, as determined in accordance with subsection (e) below.
  \item
    Beginning with the 1999-2000 Salary Cap Year, an amount with respect to the number of players fewer than eleven (11) included in a Team's Team Salary, as determined in accordance with subsection (f) below.
  \item
    Value or consideration received by retired players that is determined to be includable in Team Salary in accordance with Article XIII, Section 5.
  \item
    The amount of any Salary Cap Exception that is deemed included in Team Salary in accordance with Section 6(k)(2) below.
  \end{enumerate}
\item
  Expansion.\\
  The Salary of any player selected by an Expansion Team in an expansion draft and terminated in accordance with the NBA waiver procedure before the first day of the Expansion Team's first Season shall not be included in the Expansion Team's Team Salary, except, to the extent such Salary is paid, for purposes of determining whether the Expansion Team has satisfied its Minimum Team Salary obligation for such Season.
\item
  Assigned Contracts.
  For purposes of calculating Team Salary, with respect to any Player Contract that is assigned, the assignee Team shall, upon assignment, have included in its Team Salary the entire Salary for the then-current Salary Cap Year and for all future Salary Cap Years.
\item
  Free Agents.\\
  Subject to subsection (a)(2)(ii) above, until a Team's Veteran Free Agent re-signs with his Team, signs with another NBA Team, or is renounced, he will be included in his prior Team's Team Salary at one of the following amounts (``Free Agent Amounts''):

  \begin{enumerate}
  \def\labelenumii{(\arabic{enumii})}
  \item
    \begin{enumerate}
    \def\labelenumiii{(\roman{enumiii})}
    \tightlist
    \item
      A Qualifying Veteran Free Agent, other than a Qualifying Veteran Free Agent following the last Season of his Rookie Scale Contract, will be included at 150\% of his prior Salary if it was equal to or greater than the Estimated Average Player Salary, and 200\% of his prior Salary if it was less than the Estimated Average Player Salary.
    \item
      A Qualifying Veteran Free Agent following the last Season of his Rookie Scale Contract will be included at the following amounts:

      \begin{enumerate}
      \def\labelenumiv{(\Alph{enumiv})}
      \tightlist
      \item
        For the 1998-99 Salary Cap Year: 200\% of the player's prior Salary if it was equal to or greater than the Estimated Average Player Salary, and 300\% of his prior Salary if it was less than the Estimated Average Player Salary.
      \item
        For the 1999-2000 Salary Cap Year: 225\% of the player's prior Salary if it was equal to or greater than the Estimated Average Player Salary, and 300\% of his prior Salary if it was less than the Estimated Average Player Salary.
      \item
        For the 2000-01 through 2003-04 Salary Cap Years (and for the 2004-05 Salary Cap Year if the NBA exercises its option to extend this Agreement pursuant to Article XXXIX): 250\% of the player's prior Salary if it was equal to or greater than the Estimated Average Player Salary, and 300\% of his prior Salary if it was less than the Estimated Average Player Salary.
      \end{enumerate}
    \end{enumerate}
  \item
    An Early Qualifying Veteran Free Agent will be included at 130\% of his prior Salary; provided, however, that the player's prior Team may, by written notice to the NBA, renounce its rights to sign the player pursuant to the Early Qualifying Veteran Free Agent Exception, in which case the player will be deemed a Non-Qualifying Veteran Free Agent for purposes of this Section 4(d) and Section 6(b) below.
  \item
    A Non-Qualifying Veteran Free Agent will be included at 120\% of his prior Salary.
  \item
    Notwithstanding subsections (d)(1) - (3) above, if the player's prior Salary was equal to or less than the Minimum Player Salary applicable to such player, he will be included at the portion of the then-current Minimum Annual Salary applicable to such player that would not be reimbursed out of the league-wide benefits fund described in Article IV, Section 1(l).
  \item
    Notwithstanding subsections (d)(1) - (3) above, at no time shall a player's Free Agent Amount exceed the Maximum Player Salary applicable to such player or be less than the portion of the Minimum Annual Salary applicable to such player that would not be reimbursed out of the league-wide benefits fund described in Article IV, Section 1(l).
  \item
    For purposes of this subsection (d) only, a player's ``prior Salary'' means his Regular Salary for the prior Season plus any signing bonus allocation and the amount of any Incentive Compensation actually earned for such Season.
  \item
    For purposes of this subsection (d) only, in the event that a Veteran Free Agent's prior Contract provides for an increase or decrease in Salary between the second-to-last and last Seasons covered by the Contract of greater than \$4 million, such player's prior Salary shall be deemed to be equal to the average of the Salaries for the last two Seasons of the Contract.
  \end{enumerate}
\item
  First Round Picks.

  \begin{enumerate}
  \def\labelenumii{(\arabic{enumii})}
  \tightlist
  \item
    Beginning with the 1998 Draft, a First Round Pick, immediately upon selection in the Draft, shall be included in the Team Salary of the Team that holds his draft rights at 100\% of his applicable Rookie Scale Amount, and, subject to subsection (e)(2) below, shall continue to be included in the Team Salary of any Team that holds his draft rights (including any Team to which the player's draft rights are assigned) until such time as the player signs with such Team or until the Team loses or assigns its exclusive draft rights to the player.
  \item
    In the event that a First Round Pick signs with a non-NBA team, the player's applicable Rookie Scale Amount shall be excluded from the Team Salary of the Team that holds his draft rights, beginning on the date he signs such non-NBA contract or the first day of the Regular Season, whichever is later, and shall be included again in his Team's Team Salary at the applicable Rookie Scale Amount on the following July 1 or the date the player's contract ends (or the player is released from his non-NBA contractual obligations), whichever is earlier, unless the Team renounces its exclusive rights to the player in accordance with Article X, Section 3(f). If, after such following July 1, or any subsequent July 1, the player signs another, or remains under, contract with a non-NBA team, the player's applicable Rookie Scale Amount will again be excluded from Team Salary beginning on the date of the contract signing or the first day of the Regular Season commencing after such July 1, whichever is later, and will again be included in Team Salary at the applicable Rookie Scale Amount on the following July 1 or the date the player's contract ends (or the player is released from his non-NBA contractual obligations), whichever is earlier, unless the Team renounces its exclusive rights to the player in accordance with Article X, Section 3(f).
  \item
    For purposes of this Section 4(e), in the event that a First Round Pick does not sign a Contract with the Team that holds his draft rights during the Salary Cap Year immediately following the Draft in which he was selected (or during the same Salary Cap Year in which he was drafted if the Draft occurs on or after July 1), the ``applicable Rookie Scale Amount'' for such First Round Pick means, with respect to any subsequent Salary Cap Year, the Rookie Scale Amount that would apply if the player were drafted in the Draft immediately preceding such Salary Cap Year at the same draft position at which he was actually selected.
  \end{enumerate}
\item
  Incomplete Rosters.

  \begin{enumerate}
  \def\labelenumii{(\arabic{enumii})}
  \item
    Beginning with the 1999-2000 Salary Cap Year, if at any time from July 1 through the day prior to the first day of the Regular Season a Team has fewer than eleven (11) players, determined in accordance with subsection (f)(2) below, included in its Team Salary, then the Team's Team Salary shall be increased by an amount calculated as follows:

    STEP 1: Subtract from eleven (11) the number of players included in Team Salary.

    STEP 2: If the result in Step 1 is a positive number, multiply the result in Step 1 by the Minimum Annual Salary applicable to players with zero (0) Years of Service for that Salary Cap Year.
  \item
    In determining whether a Team has fewer than eleven (11) players included in its Team Salary for purposes of subsection (1) above only, the only players who shall be counted are (i) players on the Team's Active List (including any injured players) who are included in Team Salary, (ii) Free Agents who are included in Team Salary pursuant to Section 4 (a)(2) above, and (iii) unsigned First Round Picks who are included in Team Salary pursuant to Section 4(e) above.
  \end{enumerate}
\item
  Renouncing.

  \begin{enumerate}
  \def\labelenumii{(\arabic{enumii})}
  \tightlist
  \item
    To renounce a Veteran Free Agent, a Team must provide the NBA with an express, written statement renouncing its right to re-sign the player, effective no earlier than the July 1 following the last Season covered by the player's Contract. (The NBA shall notify the Players Association of any such renunciation by fax within two (2) business days following receipt of notice of such renunciation.) From the date of such renunciation until the following July 1, the player's Prior Team will only be permitted to re-sign such player with Room (i.e., the Team cannot sign such player pursuant to Section 6(b) below) or pursuant to the Minimum Player Salary Exception. Notwithstanding the foregoing, in the event a Team renounces one or more players in order to create Room for an Offer Sheet, and the offeree-player's Prior Team subsequently matches the Offer Sheet and enters into a Contract with that player, the Team may rescind the renunciation(s) within two (2) business days of the date the Offer Sheet is matched, whereupon any such ``unrenounced'' player may again sign a Player Contract with his Prior Team as a Qualifying Veteran Free Agent, Early Qualifying Veteran Free Agent, or Non-Qualifying Veteran Free Agent, as the case may be, and will again be included in his Prior Team's Team Salary at his applicable Free Agent Amount. Notwithstanding the foregoing, a Team may not rescind the renunciation of a player if (i) at the time the player was renounced the Team's Team Salary was at or below the Salary Cap and ``unrenouncing'' the player would cause the Team's Team Salary to exceed the Salary Cap, or (ii) at the time the player was renounced the Team's Team Salary was above the Salary Cap and ``unrenouncing'' the player would cause the Team's Team Salary to exceed the Salary Cap by more than the amount by which Team Salary exceeded the Salary Cap prior to the renunciation.
  \item
    A Team cannot renounce any player to whom the Team has made a Qualifying Offer until such time as the Qualifying Offer no longer is in effect.
  \end{enumerate}
\item
  \textbf{Long-Term Injuries.} Any player who suffers a career-ending injury or illness, and whose contract is terminated by the Team in accordance with the NBA waiver procedure, will be excluded from his team's Team Salary as follows:

  \begin{enumerate}
  \def\labelenumii{(\arabic{enumii})}
  \tightlist
  \item
    If the injury or illness occurs on or after July 1, but prior to January 1 of any Season, then, beginning on the second July 1 following the injury or illness, the Team may apply to the NBA to have the player's Salary for each remaining Season of the Contract excluded from Team Salary. (For example, if the career-ending injury or illness occurs on August 1, 1999, the Team may apply to have the player's Salary excluded from Team Salary beginning on July 1, 2001.)
  \item
    If the injury or illness occurs on or after January 1 but prior to the subsequent July 1, then, beginning on the second anniversary of the injury or illness, the Team may apply to the NBA to have the player's Salary for each remaining Season of the Contract excluded from Team Salary.
  \item
    The determination of whether a player has suffered a career-ending injury or illness shall be made by a physician selected jointly by the NBA and the Players Association.
  \item
    Notwithstanding subsections (1) through (3) above, a player's Salary shall not be excluded from Team Salary if, after the date on which a career-ending injury or illness is alleged to have occurred but before his Salary is excluded from Team Salary, the player played in more than ten (10) NBA games in any one (1) Season or in a total of fifteen (15) games over two (2) Seasons.
  \item
    Notwithstanding subsections (1) through (3) above, if, after a player's Salary is excluded from Team Salary in accordance with this Section 4(h), the player plays in ten (10) NBA games in any one (1) Season, the excluded Salary for that Season and each subsequent Season shall thereupon be included in Team Salary. If, after a player's Salary is excluded from Team Salary in accordance with this Section 4(h), the player plays in fifteen (15) or more NBA games over two (2) Seasons but did not play in ten (10) games in the first of such two (2) Seasons, the excluded Salary for the second Season and each subsequent Season shall thereupon be included in Team Salary. After a player's Salary for one (1) or more Seasons has been included in Team Salary in accordance with this subsection (h)(5), the player's Team shall be permitted at the appropriate time to re-apply to have the player's Salary (for each Season remaining at the time of the re-application) excluded from Team Salary in accordance with the rules set forth in this Section 4(h).
  \end{enumerate}
\item
  Summer Contracts.

  \begin{enumerate}
  \def\labelenumii{(\arabic{enumii})}
  \tightlist
  \item
    Except as provided in subsection (i)(2) below, from August 1 until the day prior to the first day of the next Regular Season, a Team may enter into Player Contracts that will not be included in Team Salary until the first day of such Regular Season (i.e., the player will be deemed not to have any Salary until the first day of such Regular Season), provided that such Contracts satisfy the requirements of this Section 4(i) (a ``Summer Contract''). Except as set forth in the following sentence, no Summer Contract may provide for (i) Compensation of any kind that is or may be paid or earned prior to the first day of the next Regular Season, or (ii) Compensation protection or insurance of any kind pursuant to Article II, Section 3(f) or 4. The only consideration that may be provided to a player signed to a Summer Contract, prior to the start of the Regular Season, is per diem, lodging, transportation, compensation in accordance with paragraph 3(b) of the Uniform Player Contract, and a disability insurance policy covering disabilities incurred while such player participates in summer leagues or rookie camps for the Team. A Team that has entered into one or more Summer Contracts must terminate such Contracts no later than the day prior to the first day of a Regular Season, except to the extent the Team has Room for such Contracts.
  \item
    A Team may not enter into a Summer Contract with a Veteran Free Agent who last played for the Team unless the Contract is for one Season only and provides for no more than the Minimum Player Salary applicable to such player.
  \end{enumerate}
\item
  Team Salary Summaries.

  \begin{enumerate}
  \def\labelenumii{(\arabic{enumii})}
  \tightlist
  \item
    The NBA shall provide the Players Association with Team Salary summaries and a list of current Exceptions and Base Year Compensations twice a month during the Regular Season and once every week during the off-season.
  \item
    In the event that the NBA fails to provide the Players Association with any Team Salary summary or list of Exceptions or Base Year Compensations as provided for in subsection (j)(1) above, the Players Association shall notify the NBA of such failure, and the NBA, upon receipt of such notice, shall as soon as reasonably possible, but in no event later than two business days following receipt of such notice, provide the Players Association with any such summary or list that should have been provided pursuant to subsection (j)(1) above.
  \end{enumerate}
\end{enumerate}

\hypertarget{operation-of-salary-cap.}{%
\section{Operation of Salary Cap.}\label{operation-of-salary-cap.}}

\begin{enumerate}
\def\labelenumi{(\alph{enumi})}
\tightlist
\item
  Basic Rule.\\
  A Team's Team Salary may not exceed the Salary Cap at any time unless the Team is using one of the Exceptions set forth in Section 6 below.
\item
  Room.\\
  Subject to the other provisions of this Agreement, including without limitation Article II, Section 7, any Team with Room may enter into a Player Contract that calls for a Salary in the first Season of such Contract that would not exceed the Team's then-current Room.
\item
  Annual Salary Increases and Decreases.

  \begin{enumerate}
  \def\labelenumii{(\arabic{enumii})}
  \tightlist
  \item
    The following rules apply to all Player Contracts other than Contracts between Qualifying Veteran Free Agents or Early Qualifying Veteran Free Agents and their Prior Team:

    \begin{enumerate}
    \def\labelenumiii{(\roman{enumiii})}
    \tightlist
    \item
      For each Season of a Player Contract after the first Season, the player's Salary, excluding Incentive Compensation, may increase or decrease in relation to the previous Season's Salary, excluding Incentive Compensation, by no more than 10\% of the Regular Salary for the first Season of the Contract.
    \item
      In the event that the first Season of a Contract provides for Incentive Compensation, the total amount of Likely Bonuses in each subsequent Season of the Contract may increase or decrease by up to 10\% of the amount of Likely Bonuses in the first Season, and the total amount of Unlikely Bonuses in each subsequent Season may increase or decrease by up to 10\% of the amount of Unlikely Bonuses in the first Season.
    \end{enumerate}
  \item
    The following rules apply to all Players Contracts between Qualifying Veteran Free Agents or Early Qualifying Veteran Free Agents and their Prior Team:

    \begin{enumerate}
    \def\labelenumiii{(\roman{enumiii})}
    \tightlist
    \item
      For each Season of a Player Contract after the first Season, the player's Salary, excluding Incentive Compensation, may increase or decrease in relation to the previous Season's Salary, excluding Incentive Compensation, by no more than 12.5\% of the Regular Salary for the first Season of the Contract.
    \item
      In the event that the first Season of a Contract provides for Incentive Compensation, the total amount of Likely Bonuses in each subsequent Season of the Contract may increase or decrease by up to 12.5\% of the amount of Likely Bonuses in the first Season, and the total amount of Unlikely Bonuses in each subsequent Season may increase or decrease by up to 12.5\% of the amount of Unlikely Bonuses in the first Season.
    \end{enumerate}
  \item
    The following rules apply to Extensions other than Extensions of Rookie Scale Contracts:

    \begin{enumerate}
    \def\labelenumiii{(\roman{enumiii})}
    \tightlist
    \item
      For each Season of an Extension after the first Season of the extended term, the player's Salary, excluding Incentive Compensation, may increase or decrease in relation to the previous Season's Salary, excluding Incentive Compensation, by no more than 12.5\% of the Regular Salary for the last Season of the original term of the Contract.
    \item
      In the event that the last Season of the original term of the Contract provides for Incentive Compensation, the amount of Likely Bonuses and Unlikely Bonuses in each Season of the Extension after the first Season of the extended term may increase or decrease by up to 12.5\% of the amount of Likely Bonuses and Unlikely Bonuses, respectively, in the last Season of the original term.
    \end{enumerate}
  \item
    The following rules apply to Extensions of Rookie Scale Contracts:

    \begin{enumerate}
    \def\labelenumiii{(\roman{enumiii})}
    \tightlist
    \item
      For each Season of an Extension of a Rookie Scale Contract after the first Season of the extended term, the Player's Salary, excluding Incentive Compensation, may increase or decrease in relation to the previous Season's Salary, excluding Incentive Compensation, by no more than 12.5\% of the Regular Salary for the first Season of the extended term of the Contract.
    \item
      In the event that the first Season of the extended term of the Contract provides for Incentive Compensation, the amount of Likely Bonuses and Unlikely Bonuses in each Season of the Extension after the first Season of the extended term may increase or decrease by up to 12.5\% of the amount of Likely Bonuses and Unlikely Bonuses, respectively, in the first Season of the extended term.
    \end{enumerate}
  \item
    Notwithstanding anything to the contrary in subsections (1) through (4) above, the following rule applies to any Contract or Extension covering the 2000-01 Season plus one (1) or more subsequent Seasons: with respect to any Season after 2000-01, the player's Salary may not be less than the player's Salary for the 2000-01 Season.
  \item
    For purposes of this Article VII, Section 5(c) only, the amount of any bonuses that a player may receive pursuant to Article II, Sections 3(c)(iv) and 3(d) shall be added to the player's Regular Salary and excluded from his Incentive Compensation.
  \end{enumerate}
\item
  Performance Bonuses.

  \begin{enumerate}
  \def\labelenumii{(\arabic{enumii})}
  \tightlist
  \item
    Notwithstanding any other provision of this Agreement, no Player Contract may provide for Unlikely Bonuses in any Season that exceed 25\% of the player's Regular Salary for such Season.
  \item
    No Player Contract may provide for any Unlikely Bonus for the first Season of the Contract that, if included in the player's Salary for such Season, would result in the Team's Team Salary exceeding the Room under which it is signing the Contract. For the sole purpose of determining whether a Team has Room for a new Unlikely Bonus, the Team's Room shall be deemed reduced by all Unlikely Bonuses in Contracts approved by the Commissioner that may be paid to all of the Team's players that entered into Player Contracts (including Renegotiations) during that Salary Cap Year.
  \item
    The following provisions shall apply to any Averaged Contract containing a Performance Bonus:

    \begin{enumerate}
    \def\labelenumiii{(\roman{enumiii})}
    \tightlist
    \item
      In the event that at the end of any Season, a Performance Bonus that is included in a player's Salary for a subsequent Season is determined to be no longer includable in Salary for that subsequent Season, the player's Salary for such subsequent Season shall be reduced by the amount of such bonus.
    \item
      In the event that, at the end of any Season, a Performance Bonus that is not included in a player's Salary for a subsequent Season is determined to be includable in Salary for that subsequent Season, the player's Salary for such subsequent Season shall be increased by the amount of such Performance Bonus.
    \end{enumerate}
  \end{enumerate}
\item
  \textbf{No Futures Contracts.} Subject to subsection (e)(4) below, but notwithstanding any other provision in this Agreement:

  \begin{enumerate}
  \def\labelenumii{(\arabic{enumii})}
  \tightlist
  \item
    Every Player Contract must cover at least the then-current Season (or the upcoming Season in the case of a Contract entered into from July 1 through the day prior to the first day of the Season).
  \item
    No Team and player may enter into a Player Contract from the commencement of the Team's last game of the Regular Season through the following June 30. The preceding sentence shall not prohibit a Team and player from entering into an amendment to an existing Player Contract during such period if such amendment would otherwise be permitted under this Agreement.
  \item
    A Player Contract that covers more than one Season must be for a consecutive period of Seasons.
  \item
    From February 1 through May 31 of any Salary Cap Year, a First Round Pick may enter into a Rookie Scale Contract commencing with the following Season, provided that as of or at any point following the first day of the then-current Regular Season (or the preceding Regular Season in the case of a Contract signed from the day following the last day of the Regular Season through May 31) the player was a party to a player contract with a professional basketball team not in the NBA covering such Regular Season.
  \end{enumerate}
\end{enumerate}

\hypertarget{exceptions-to-the-salary-cap.}{%
\section{Exceptions to the Salary Cap.}\label{exceptions-to-the-salary-cap.}}

There shall be the following exceptions to the rule that a Team's Team Salary may not exceed the Salary Cap:

\begin{enumerate}
\def\labelenumi{(\alph{enumi})}
\tightlist
\item
  Existing Contracts. A Team may exceed the Salary Cap to the extent of its current contractual commitments, provided that such contracts satisfied the provisions of this Agreement when entered into or were entered into prior to the date of this Agreement in accordance with the rules then in effect.
\item
  Veteran Free Agent Exception. Beginning on the August 1 following the last Season covered by a Veteran Free Agent's Player Contract, such player may enter into a new Player Contract with his Prior Team (or, in the case of a player selected in an Expansion Draft that year, with the Team that selected such player in an Expansion Draft) as follows:

  \begin{enumerate}
  \def\labelenumii{(\arabic{enumii})}
  \tightlist
  \item
    If the player is a Qualifying Veteran Free Agent, the new Player Contract may provide for Salary and Unlikely Bonuses in the first Season totaling up to the maximum amount provided for in Article II, Section 7. Notwithstanding the preceding sentence, if the player is a Qualifying Veteran Free Agent whose last Contract was his Rookie Scale Contract and whose Prior Team did not exercise the Fourth Year Option to extend such Contract for a fourth Season, the new Player Contract may provide for Regular Salary, Likely Bonuses and Unlikely Bonuses in the first Season of up to the Regular Salary, Likely Bonuses and Unlikely Bonuses, respectively, that the player would have received for such Season had his Prior Team exercised its Fourth Year Option. Annual increases and decreases in Salary and Unlikely Bonuses shall be governed by Section 5(c)(2) above.
  \item
    If the player is a Non-Qualifying Veteran Free Agent, then, subject to Article II, Section 7, the new Player Contract may provide in the first Season up to the greater of: (i) 120\% of the Regular Salary for the final Season of the player's prior Contract, plus 120\% of any Likely Bonuses and Unlikely Bonuses, respectively, called for in the final Season covered by the player's prior Contract; (ii) 120\% of the then-current Minimum Annual Salary applicable to the player; or (iii) in the case of a Contract between a Team and its Restricted Free Agent, the amount required to be provided in a Qualifying Offer. Annual increases and decreases in Salary and Unlikely Bonuses shall be governed by Section 5(c)(1) above.
  \item
    If the player is an Early Qualifying Veteran Free Agent, the new Player Contract must cover at least two Seasons (not including a Season covered by an Option Year) and, subject to Article II, Section 7, may provide in the first Season up to the greater of: (i) 175\% of the Regular Salary for the final Season covered by his prior Contract, plus 175\% of any Likely Bonuses and Unlikely Bonuses, respectively, called for in the final Season covered by the player's prior Contract, or (ii) 108\% of the Average Player Salary for the prior Season (or if the prior Season's Average Player Salary has not been determined, 108\% of the Estimated Average Player Salary for the prior Season). Annual increases and decreases in Salary and Unlikely Bonuses shall be governed by Section 5(c)(2) above.
  \end{enumerate}
\item
  Disabled Player Exception.

  \begin{enumerate}
  \def\labelenumii{(\arabic{enumii})}
  \tightlist
  \item
    Subject to the rules set forth in subsection (k) below, a Team may, in accordance with the rules set forth in this subsection (c), sign or acquire one Replacement Player to replace a player who, as a result of a Disabling Injury or Illness (as defined below), is unable to render playing services (the ``Disabled Player''). Such Replacement Player's Contract may provide a Salary for the first Season of up to the lesser of (i) 50\% of the Disabled Player's Salary at the time the Disabling Injury or Illness occurred, or (ii) 108\% of the Average Player Salary for the prior Season (or, if the prior Season's Average Player Salary has not been determined, 108\% of the Estimated Average Player Salary for the prior Season). Annual increases and decreases in Salary and Unlikely Bonuses shall be governed by Section 5(c)(1) above.
  \item
    For purposes of this subsection (c), Disabling Injury or Illness means:

    \begin{enumerate}
    \def\labelenumiii{(\roman{enumiii})}
    \tightlist
    \item
      for the period July 1 through the immediately following November 30, any injury or illness that will render a player unable to play all (or the remainder) of the Season immediately following such July 1; and
    \item
      for the period December 1 through the immediately following June 30, any injury or illness that will render a player unable to play all of the following Season.
    \end{enumerate}
  \item
    The Exception for a Disabling Injury or Illness that occurs during the period July 1 through the immediately following November 30 shall arise on the earlier of (i) forty-five (45) days prior to the last day of the Regular Season immediately following such July 1, or (ii) the date the Team knew or reasonably should have known that the injury or illness would cause the player to miss the Season immediately following such July 1, and shall expire 45 days from the date the Exception arises.
  \item
    The Exception for a Disabling Injury or Illness that occurs during the period December 1 through the immediately following June 30 shall arise on the earlier of (i) forty-five (45) days prior to the October 1 immediately following the date on which the Disabling Injury or Illness occurs, or (ii) the date the Team knew or reasonably should have known that the injury or illness would cause the player to miss all of the following Season; provided, however, that if the Team knew or reasonably should have known prior to the July 1 immediately following the injury or illness that the injury or illness would cause the player to miss all of the following Season, and if the Team does not use the Exception prior to such July 1, then the Exception shall be deemed to arise on August 1. The Exception for a Disabling Injury or Illness that occurs during the period December 1 through the immediately following June 30 shall expire on the October 1 immediately following the date on which the Exception arises.
  \item
    The determination of whether a player has suffered a Disabling Injury or Illness shall be made by a physician designated by the NBA. The NBA shall advise the Players Association of the determination of its physician within one business day of such determination. In the event the Players Association disputes the NBA physician's determination, the parties will immediately refer the matter to a neutral physician (to be selected by the parties at the commencement of each Salary Cap Year) to review the relevant medical information and, if requested, examine the player. Within three business days of his receipt of such information (and examination of the player, if requested), the neutral physician shall make a final determination, which will be final, binding and unappealable. The cost of the NBA physician will be borne by the NBA. The cost of the neutral physician will be borne jointly by the NBA and the Players Association.
  \item
    If a Team requests an Exception pursuant to this subsection (c), the player with respect to whom the request is made shall cooperate in the processing of the request, including by appearing at the scheduled place and time for examination by the NBA-appointed physician and, if necessary, the neutral physician.
  \item
    Notwithstanding a determination by a physician designated by the NBA that a player has suffered a Disabling Injury or Illness, such player, upon recovering from his injury or illness, may be restored to his Team's Active List, without affecting any right the Team may have to sign a Replacement Player.
  \item
    In no event may a Team enter into a Contract with a Replacement Player pursuant to subsection (c)(4) above, unless the Disabled Player's Contract covers the Season following the Season in which the Disabling Injury or Illness occurs.
  \item
    The Disabled Player Exception is available only to the Team with which the player was under Contract at the time his Disabling Injury or Illness occurred.(10) If a Team makes a request for an Exception to replace a Disabled Player pursuant to this subsection (c) and such request is denied, the Team shall not be permitted to make any subsequent request for an Exception to replace the same player unless ninety (90) days have passed since the first request was denied and the Team establishes that the subsequent request is based on a new injury or an aggravation of the same injury.
  \end{enumerate}
\item
  \$1 Million Exception. Subject to the rules set forth in subsection (k) below:

  \begin{enumerate}
  \def\labelenumii{(\arabic{enumii})}
  \tightlist
  \item
    A Team may sign one or more Player Contracts, not to exceed two Seasons in length, that, in the aggregate, provide for first-year Salaries and Unlikely Bonuses totaling up to the amounts set forth below:

    \begin{enumerate}
    \def\labelenumiii{(\roman{enumiii})}
    \tightlist
    \item
      For the 1998-99 Season: \$1 million
    \item
      For the 1999-2000 Season: \$1.1 million
    \item
      For the 2000-01 Season: \$1.2 million
    \item
      For the 2001-02 Season: \$1.3 million
    \item
      For the 2002-03 Season: \$1.4 million
    \item
      For the 2003-04 Season: \$1.5 million
    \item
      For the 2004-05 Season \$1.6 million\\
      (the NBA exercises its option to extend this Agreement pursuant to Article XXXIX)
    \end{enumerate}
  \item
    A Team may use all or any portion of the \$1 Million Exception to sign one or more new Player Contracts during no more than three (3) separate Salary Cap Years during the term of this Agreement (or no more than four (4) separate Salary Cap Years if the NBA exercises its option to extend this Agreement pursuant to Article XXXIX); provided, however, that the \$1 Million Exception or any portion thereof may not be used in any two (2) consecutive Salary Cap Years. The prohibition in the preceding sentence against using the \$1 Million Exception or any portion thereof in any two (2) consecutive Salary Cap Years shall apply to the 1997-98 Salary Cap Year (i.e., if a Team used all or any portion of the \$1 Million Exception during the 1997-98 Salary Cap Year, that Team shall not be permitted to use all or any portion of the \$1 Million Exception during the 1998-99 Salary Cap Year).
  \item
    Player Contracts signed pursuant to the \$1 Million Exception covering two (2) Seasons may provide for an increase or decrease in Salary and Unlikely Bonuses for the second Season in accordance with Section 5(c)(1) above.
  \item
    The \$1 Million Exception, if applicable, arises on August 1 of each Salary Cap Year and expires on the last day of the Team's Regular Season during that Salary Cap Year.
  \end{enumerate}
\item
  Mid-Level Salary Exception. Subject to the rules set forth in subsection (k) below:

  \begin{enumerate}
  \def\labelenumii{(\arabic{enumii})}
  \tightlist
  \item
    A Team may sign one (1) or more Player Contracts during each of the 1998-99 through 2000-01 Salary Cap Years, not to exceed six (6) Seasons in length, that, in the aggregate, provide for first-year Salaries and Unlikely Bonuses totaling up to the amounts set forth below:

    \begin{enumerate}
    \def\labelenumiii{(\roman{enumiii})}
    \tightlist
    \item
      For the 1998-99 Season: \$1.75 million;
    \item
      For the 1999-2000 Season: \$2.0 million;
    \item
      For the 2000-01 Season: \$2.25 million;
    \end{enumerate}
  \item
    A Team may sign one (1) or more Player Contracts during each of the 2001-02 through 2003-04 Salary Cap Years (and during the 2004-05 Salary Cap Year if the NBA exercises its option to extend this Agreement pursuant to Article XXXIX), not to exceed six (6) Seasons in length, that, in the aggregate, provide for first-year Salaries and Unlikely Bonuses totaling up to 108\% of the Average Player Salary for the prior Season (or, if the prior Season's Average Player Salary has not been determined, 108\% of the Estimated Average Player Salary for the prior Season).
  \item
    Player Contracts signed pursuant to the Mid-Level Salary Exception may provide for annual increases and decreases in Salary and Unlikely Bonuses in accordance with Section 5(c)(1) above.
  \item
    The Mid-Level Salary Exception shall arise on August 1 of each Salary Cap Year and shall expire on the last day of the Team's Regular Season during that Salary Cap Year.
  \end{enumerate}
\item
  Rookie Exception. A Team may enter into a Rookie Scale Contract in accordance with Article VIII.
\item
  Minimum Player Salary Exception. A Team may sign a player to, or acquire by assignment, a Player Contract, not to exceed two (2) Seasons in length, that provides for a Salary for the first Season equal to the Minimum Player Salary applicable to that player (with no Unlikely Bonuses). A Player Contract signed pursuant to the Minimum Player Salary Exception covering two (2) Seasons shall provide for a Salary for the second Season equal to the Minimum Player Salary applicable to the player for such Season (with no Unlikely Bonuses).
\item
  Assigned Player Exception.

  \begin{enumerate}
  \def\labelenumii{(\arabic{enumii})}
  \item
    Subject to the rules set forth in subsection (k) below, a Team may, for a period of one year following the date of the assignment of a Player Contract to another Team, replace the Traded Player with one or more players acquired by assignment as follows:

    \begin{enumerate}
    \def\labelenumiii{(\roman{enumiii})}
    \tightlist
    \item
      A Team may replace a Traded Player with one or more Replacement Players whose Player Contracts are acquired simultaneously and whose post-assignment Salaries for the then-current Salary Cap Year, in the aggregate, are no more than an amount equal to 115\% of the pre-assignment Salary (or Base Year Compensation, if applicable) of the Traded Player, plus \$100,000.
    \item
      If a Team's assignment of a Traded Player and acquisition of one or more Replacement Players do not occur simultaneously, then the post-assignment Salary or aggregate Salaries of the Replacement Player(s) for the Salary Cap Year in which the Replacement Player(s) are acquired may not exceed 100\% of the pre-assignment Salary (or Base Year Compensation, if applicable) of the Traded Player at the time the Traded Player's Contract was assigned, plus \$100,000.
    \item
      A Team may aggregate the pre-assignment Salaries in two or more Player Contracts for the purpose of acquiring in a simultaneous trade one or more Replacement Players whose post-assignment Salaries, in the aggregate, are no more than an amount equal to 115\% of the pre-assignment aggregated Salaries (or Base Year Compensations, if applicable) of the Traded Players, plus \$100,000. Notwithstanding the preceding sentence, no Player Contract acquired pursuant to an Exception may give rise to an aggregated trade exception for a period of two months from the date the Player Contract is acquired.
    \end{enumerate}
  \item
    Except as provided in subsection (h)(3) below, and notwithstanding subsection (k) below, a Team with a Team Salary below the Salary Cap may acquire one or more players by assignment whose post-assignment Salaries, in the aggregate, are no more than an amount equal to the Team's Room plus \$100,000.
  \item
    In lieu of conducting a trade in accordance with subsection (h)(2) above, and notwithstanding subsection (k) below, a Team with a Team Salary below the Salary Cap may (i) replace a Traded Player with one or more Replacement Players whose Player Contracts are acquired simultaneously and whose post-assignment Salaries for the then-current Season, in the aggregate, are no more than an amount equal to 115\% of the pre-assignment Salary of the Traded Player, plus \$100,000, or (ii) aggregate the pre-assignment Salaries in two or more Player Contracts for the purpose of acquiring in a simultaneous trade one or more Replacement Players whose post-assignment Salaries, in the aggregate, are no more than an amount equal to 115\% of the pre-assignment aggregated Salaries of the Traded Players, plus \$100,000. Notwithstanding the preceding sentence, no Player Contract acquired pursuant to an Exception may be assigned by a Team in accordance with this subsection (h)(3) for a period of two months from the date the Player Contract is acquired.
  \item
    \begin{enumerate}
    \def\labelenumiii{(\roman{enumiii})}
    \tightlist
    \item
      For purposes of the Assigned Player Exception, a player shall be subject to a Base Year Compensation in the event that the Team Salary of the player's Team is at or above the Salary Cap and the player:

      \begin{enumerate}
      \def\labelenumiv{(\Alph{enumiv})}
      \tightlist
      \item
        is a Qualifying Veteran Free Agent or Early Qualifying Veteran Free Agent who, in accordance with Section 6(b) above, enters into a new Player Contract with his prior Team that provides for a Salary for the first Season of such new Contract greater than 120\% of the Salary for the last Season of the player's immediately prior Contract;
      \item
        is a First Round Pick who, in accordance with Section 7(b) below, enters into an Extension of his Rookie Scale Contract that provides for a Salary for the first Season of the extended term greater than 120\% of the Salary for the last Season of the original term of the Contract;
      \item
        is subject to a Base Year Compensation on the date of this Agreement; or
      \item
        will be subject to a Base Year Compensation on some future date based upon an Extension entered into prior to the date of this Agreement.
      \end{enumerate}
    \item
      A player's Base Year Compensation shall be computed as follows with respect to Contracts or Extensions entered into prior to July 1, 2001 (including Contracts or Extensions signed prior to the date of this Agreement):

      \begin{enumerate}
      \def\labelenumiv{(\Alph{enumiv})}
      \tightlist
      \item
        During the first 365 days from the date a player's Base Year Compensation goes into effect (``Year One''), his Base Year Compensation will equal the greater of (1) the Salary for the last Season of his preceding Contract or, in the case of an Extension, the last Season of the original term of the Contract (the preceding amount hereinafter referred to as the ``Base Year Salary''), or (2) 50\% of the Salary for Year One of his new Contract (or extended term, if applicable).
      \item
        During the second 365 days from the date the player's Base Year Compensation goes into effect (``Year Two''), his Base Year Compensation will equal the greater of (1) 120\% of his Base Year Salary, or (2) 75\% of the Salary for Year Two of his new Contract (or extended term, if applicable).
      \item
        A player's Base Year Compensation will expire and be of no further effect on the 731st day of his new Contract (or extended term, if applicable).
      \end{enumerate}
    \item
      A player's Base Year Compensation shall be computed as follows with respect to Contracts or Extensions entered into on or after July 1, 2001:
      (A) During the first 365 days from the date a player's Base Year Compensation goes into effect (``Year One''), his Base Year Compensation will equal the greater of (1) the Salary for the last Season of his preceding Contract or, in the case of an Extension, the last Season of the original term of the Contract, or (2) 50\% of the Salary for Year One of his new Contract (or extended term, if applicable).
      (B) A player's Base Year Compensation will expire and be of no further effect on the 366th day of his new Contract (or extended term, if applicable).
    \item
      In the event a player who is subject to a Base Year Compensation during the last Season of his Contract (``Prior Contract'') signs a new Player Contract or Extension with his Prior Team, the player shall continue to be subject to a Base Year Compensation in his new Contract or Extension through (A) in the case of a Prior Contract entered into prior to July 1, 2001, the 730th day from the date the player's Base Year Compensation went into effect, and (B) in the case of a Prior Contract entered into on or after July 1, 2001, the 365th day from the date the player's Base Year Compensation went into effect. For purposes of computing such Base Year Compensation during the new Contract or Extension, the player's Base Year Salary shall equal the Base Year Compensation applicable in the last Season of his prior Contract or, in the case of an Extension, the last Season of the original term of the Contract. Notwithstanding the foregoing, in the event the player is a Qualifying Veteran Free Agent or an Early Qualifying Veteran Free Agent, and the new Contract would itself subject the player to a Base Year Compensation in accordance with subsection (h)(4)(i)(A) above, then the player shall be subject to a new Base Year Compensation for (C) in the case of a new Contract entered into prior to July 1, 2001, a period of 730 days from the date the new Contract is signed, or (D) in the case of a new Contract entered into after July 1, 2001, a period of 365 days from the date the new Contract is signed, and such new Base Year Compensation shall be computed in accordance with subsection (h)(4)(ii) or (iii) above.
    \item
      A player's Base Year Compensation shall be extinguished upon any of the following:

      \begin{enumerate}
      \def\labelenumiv{(\Alph{enumiv})}
      \tightlist
      \item
        The Team Salary of the player's Team falls below the Salary Cap, unless this occurs prior to the beginning of an extended term described in subsection (h)(4)(i)(B) or (D) above;
      \item
        The player signs a Contract with a Team other than his Prior Team; or
      \item
        The player is traded, unless the trade occurs prior to the beginning of an extended term described in subsection (h)(4)(i)(B) or (D) above.
      \end{enumerate}
    \end{enumerate}
  \end{enumerate}
\item
  Reinstatement. If a player has been disqualified from further association with the NBA and subsequently reinstated pursuant to Article XXXIII (Anti-Drug Agreement), the Team for which the player last played may enter into a Player Contract with such player in accordance with the applicable rules set forth in Article XXXIII, Section 6(f) or 13(d), even if the Team has a Team Salary at or above the Salary Cap or such Player Contract causes the Team to have a Team Salary above the Salary Cap. If, in accordance with the preceding sentence, a Team and a player enter into a Player Contract and such Contract covers more than one Season, annual increases and decreases in Salary and Unlikely Bonuses shall be governed by Section 5(c)(1) above.
\item
  Non-Aggregation. Other than in accordance with subsection (h) above, a Team may not aggregate or combine any of the Exceptions set forth above in order to sign or acquire one or more players at Salaries greater than that permitted by any one of the Exceptions. If a Team has more than one Exception available at the same time, the Team shall have the right to choose which Exception it wishes to use to sign or acquire a player.
\item
  Other Rules.

  \begin{enumerate}
  \def\labelenumii{(\arabic{enumii})}
  \tightlist
  \item
    A Team shall be entitled to use the Disabled Player, \$1 Million, Mid-Level Salary, and Assigned Player Exceptions set forth in subsections (c), (d), (e) and (h) above, respectively, except as set forth in subsections (h)(2) and (3) above, only if, at the time any such Exception would arise and at all times until it is used, the Team's Team Salary, excluding the amount(s) of such Exception and any other Exception that would be included in Team Salary pursuant to subsection (k)(2) below, is (i) at or above the Salary Cap, or (ii) below the Salary Cap by less than the amount(s) of the Team's Exception(s).
  \item
    In the event that when a Disabled Player Exception, \$1 Million Exception, Mid-Level Salary Exception and/or Assigned Player Exception arises, the Team's Team Salary is below the Salary Cap (or in the event that, prior to the expiration of any such Exceptions, the Team's Team Salary falls below the Salary Cap) by less than the amount of such Exceptions, then (i) the Team's Team Salary shall include, until the Exceptions are actually used or until the Team no longer is entitled to use the Exceptions, the amount of the Exceptions (or any unused portion of the Exceptions), and (ii) the amount by which the Team's Team Salary is less than the Salary Cap shall thereby be extinguished. When the Disabled Player Exception is used to sign or acquire a player, the Replacement Player's Salary for the first Season of his Contract, instead of the amount of the Exception, shall be included in Team Salary. When a \$1 Million Exception or Mid-Level Salary Exception is used to sign a player, or when an Assigned Player Exception is used to acquire a player, the Salary for the first Season of the signed or acquired Contract plus any then-unused portion of the Exception, instead of the full amount of the Exception, shall be included in Team Salary. A Team may at any time renounce its rights to use an Exception, in which case the Exception (or any unused portion of the Exception) will no longer be included in Team Salary.
  \end{enumerate}
\end{enumerate}

\hypertarget{extensions-renegotiations-and-other-amendments.}{%
\section{Extensions, Renegotiations and Other Amendments.}\label{extensions-renegotiations-and-other-amendments.}}

\begin{enumerate}
\def\labelenumi{(\alph{enumi})}
\tightlist
\item
  Veteran Extensions. No Player Contract, other than a Rookie Scale Contract, may be extended except in accordance with the following:

  \begin{enumerate}
  \def\labelenumii{(\arabic{enumii})}
  \tightlist
  \item
    Subject to the rules set forth in subsection (2) below, a Player Contract covering a term of six (6) or seven (7) Seasons may be extended no sooner than the fourth anniversary of the signing of the Contract, and a Player Contract with a term of four (4) or five (5) Seasons may be extended no sooner than the third anniversary of the signing of the Contract.
  \item
    A Player Contract that has been extended, or that has been renegotiated to provide for an increase in Salary in any Season of the Contract of more than 10\%, may not subsequently be extended until the third anniversary of such Extension or Renegotiation.
  \item
    Subject to Article II, Section 7, a Player Contract extended in accordance with this Section 7(a) may, in the first Season of the extended term, provide for a Salary of up to 112.5\% of the Regular Salary in the last Season of the original term of the Contract. In the event that the last Season of the original term of the Contract provides for Incentive Compensation, the first Season of the extended term may provide for Likely Bonuses and Unlikely Bonuses of up to 112.5\% of the Likely Bonuses and Unlikely Bonuses, respectively, in the last Season of the original term. Annual increases and decreases in Salary and Unlikely Bonuses shall be governed by Section 5(c)(3) above.
  \item
    Notwithstanding subsection (a)(3) above:

    \begin{enumerate}
    \def\labelenumiii{(\roman{enumiii})}
    \tightlist
    \item
      Subject to Article II, Section 7, any Averaged Contract may, in the first Season of an extended term, provide for a Salary of up to the greater of (A) 112.5\% of the averaged Regular Salary in the last Season of the original term of the Contract, or (B) 112.5\% of what the Regular Salary would have been in the last Season of the original term had the Contract not been averaged. In the event that the last Season of the original term of the Contract provides for Incentive Compensation, the first Season of the extended term may provide for Likely Bonuses and Unlikely Bonuses of up to 112.5\% of the Likely Bonuses and Unlikely Bonuses, respectively, in the last Season of the original term. Annual increases and decreases in Salary and Unlikely Bonuses shall be governed by Section 5(c)(3) above, except that, for purposes of this Section 7(a)(4)(i) only, the phrase ``Regular Salary'' in Section 5(c)(3) above shall be deemed to mean the greater of (x) the averaged Regular Salary in the last Season of the original term of the Contract, or (y) what the Regular Salary would have been in the last Season of the original term had the Contract not been averaged.
    \item
      Subject to Article II, Section 7, any Player Contract of a player who has played for his current Team for at least ten (10) Seasons and whose Salary in the last Season of the original term of the Contract is less than the Salary in the second-to-last Season of such Contract may, in the first Season of an extended term, provide for a Salary equal to 112.5\% of the greater of (1) the average of the Regular Salaries for each Season covered by the original Contract beginning with the Season in which such Contract was entered into, or previously extended, as the case may be, or (2) the Regular Salary in the last Season covered by his original Contract. In the event that the last Season of the original term of the Contract provides for Incentive Compensation, the first Season of the extended term may provide for Likely Bonuses and Unlikely Bonuses of up to 112.5\% of the Likely Bonuses and Unlikely Bonuses, respectively, in the last Season of the original term. Annual increases and decreases in Salary and Unlikely Bonuses shall be governed by Section 5(c)(3) above, except that, for purposes of this Section 7(a)(4)(ii) only, the phrase ``Regular Salary'' in Section 5(c)(3) above shall be deemed to mean the greater of (x) the average of the Regular Salaries for each Season covered by the original Contract beginning with the Season in which such Contract was entered into, or previously extended, as the case may be, or (y) the Regular Salary in the last Season covered by the original Contract.
    \end{enumerate}
  \end{enumerate}
\item
  Rookie Scale Extensions.

  \begin{enumerate}
  \def\labelenumii{(\arabic{enumii})}
  \tightlist
  \item
    A First Round Pick who, as of the date of this Agreement, has completed only one (1) Season of his Rookie Scale Contract may enter into an Extension of such Rookie Scale Contract during the period August 1, 1999 through October 31, 1999.
  \item
    A First Round Pick who enters into a Rookie Scale Contract on or after the date of this Agreement may enter into an Extension of such Rookie Scale Contract during the period August 1 through October 31 of the Option Year provided for in such Contract (assuming the Team exercises such Option).
  \item
    An Extension of a Rookie Scale Contract may provide for Salary and Unlikely Bonuses in the first Season of the extended term totaling no more than the maximum amount provided for in Article II, Section 7. Annual increases and decreases in Salary and Unlikely Bonuses shall be governed by Section 5(c)(4) above.
  \end{enumerate}
\item
  Renegotiations. No Player Contract may be renegotiated except in accordance with the following:

  \begin{enumerate}
  \def\labelenumii{(\arabic{enumii})}
  \tightlist
  \item
    Subject to subsections (c)(2) and (3) below, a Player Contract covering a term of four or more Seasons may be renegotiated no sooner than the third anniversary of the signing of the Contract.
  \item
    Subject to subsection (c)(3) below, any Player Contract that has been renegotiated in accordance with subsection (c)(1) above to provide for an increase in Salary or Incentive Compensation in any Season of the Contract of more than 10\%, or extended in accordance with subsection (a) or (b) above, may not subsequently be renegotiated until the third anniversary of such Extension or Renegotiation.
  \item
    Assuming subsections (1) or (2) above are satisfied, a Team with a Team Salary below the Salary Cap may renegotiate a Player Contract in accordance with the following rules:

    \begin{enumerate}
    \def\labelenumiii{(\roman{enumiii})}
    \tightlist
    \item
      Subject to Article II, Section 7, the Renegotiation may provide for additional Regular Salary, Likely Bonuses and/or Unlikely Bonuses for the then-current Season of the Contract (or the upcoming Season in the case of a Renegotiation entered into from August 1 through the day prior to the first day of the Season) (the ``Renegotiation Season'') that, in the aggregate, would not exceed the Team's Room at the time of the Renegotiation.
    \item
      Every category (Regular Salary, Likely Bonuses and Unlikely Bonuses, respectively) that is increased for the Renegotiation Season must also be increased for each of the remaining Seasons of the Contract. For each Season of the Contract after the Renegotiation Season, the player's additional Regular Salary may increase or decrease over the previous Season's additional Regular Salary by no more than 12.5\% of the additional Regular Salary provided for in the Renegotiation Season. In the event that the Renegotiation Season provides for additional Incentive Compensation, the amount of additional Likely Bonuses and Unlikely Bonuses provided for in each Season after the Renegotiation Season may increase or decrease by up to 12.5\% of the amount of additional Likely Bonuses and Unlikely Bonuses, respectively, provided for in the Renegotiation Season.
    \item
      No Renegotiation may contain a signing bonus, unless the Renegotiation is accompanied by an Extension and the signing bonus would otherwise be permitted under the rules governing the inclusion of signing bonuses in Extensions.
    \end{enumerate}
  \item
    In no event may a Team with a Team Salary at or above the Salary Cap renegotiate a Player Contract.
  \item
    In no event may a Team and a player renegotiate a Player Contract from March 1 through June 30 of any Salary Cap Year.
  \end{enumerate}
\item
  Other.

  \begin{enumerate}
  \def\labelenumii{(\arabic{enumii})}
  \tightlist
  \item
    In no event shall a Team and player negotiate a decrease in Salary for the then-current Season (or upcoming Season in the event of a Renegotiation entered into from August 1 through the day prior to the first day of the Season), or for any remaining Season of a Player Contract.
  \item
    A Player Contract that is extended pursuant to subsection (a) above may be renegotiated simultaneously, but only if and to the extent permitted by the rules set forth in subsection (c) above.
  \item
    For the sole purpose of enabling an assignee Team to acquire a Player Contract by trade, the player and the assignor Team may agree to waive all or any portion of an assignment bonus, but only to the extent necessary to make the trade permissible in accordance with the rules set forth in Section 6(h) above, Article II, Section 7(e), or Article VIII, Section 1(d). In the event that, in connection with a trade, a player's Contract is amended in accordance with this subsection (d)(3), such Contract may not be subsequently extended or renegotiated until the later of (i) six months from the date of the assignment, or (ii) the first date on which the Contract could otherwise be extended or renegotiated pursuant to this Section 7.
  \item
    In the event that a Team and a player agree to amend a Player Contract in accordance with Article II, Section 3(n), then for purposes of calculating the player's Salary for the then-current and any remaining Season, notwithstanding any stretch or acceleration of the player's protected Compensation payment schedule, the aggregate reduction in the player's protected Compensation shall be allocated pro rata over the then-current and each remaining Season of the Contract on the basis of the Salary in each such Season.
  \item
    In no event shall a Team and player amend a Contract for the purpose of terminating or shortening the term of the Contract, except in accordance with the NBA waiver procedure or Article XII, Section 2.
  \end{enumerate}
\end{enumerate}

\hypertarget{trade-rules.}{%
\section{Trade Rules.}\label{trade-rules.}}

\begin{enumerate}
\def\labelenumi{(\alph{enumi})}
\tightlist
\item
  A Team shall not be permitted to receive in connection with any trade, directly or indirectly, more than \$3 million in cash or other compensation, including cash or other compensation received as reimbursement for Compensation obligations to players who the Team is acquiring.
\item
  A player with a one-year Contract who would be a Qualifying Veteran Free Agent or an Early Qualifying Veteran Free Agent upon completing the playing services called for under his Contract cannot be traded. Nothing in the preceding sentence shall prevent a Qualifying Veteran Free Agent or an Early Qualifying Veteran Free Agent from being traded in the last year of a multi-year Contract if such trade would otherwise be permitted under this Agreement.
\item
  A Team cannot trade any player after the NBA trade deadline occurring in the last Season of the player's Contract, or after the NBA trade deadline occurring in any Season that could be the last Season of the player's Contract based upon the exercise or non-exercise of an Option or Early Termination Option.
\item
  Except as set forth in subsection (e) below, no player who signs a Contract as a Free Agent or Draft Rookie may be traded before the later of (1) three (3) months following the date on which such Contract was signed or (2) the December 15 of the Salary Cap Year in which such Contract was signed.
\item
  A Veteran Free Agent and his Prior Team may enter into a Player Contract pursuant to an agreement between the Prior Team and another Team concerning the signing and subsequent assignment of such Contract, but only if (1) the Contract is for three (3) or more Seasons (excluding any Option Year), (2) the Contract is not signed pursuant to the Mid-Level Salary Exception or the Disabled Player Exception, (3) the first Season of the Contract is fully protected for lack of skill, and (4) the acquiring Team has Room for the player's Salary plus any Unlikely Bonuses provided for in the first Season of the Contract. Notwithstanding anything to the contrary set forth in the preceding sentence, during the 1998-99 Season only, a Veteran Free Agent and his Prior Team may enter into a Player Contract pursuant to an agreement between the Prior Team and another Team concerning the signing and subsequent assignment of such Contract if (i) the conditions set forth in clauses (1) - (3) of the preceding sentence are satisfied, (ii) the acquiring Team's Team Salary is below the Salary Cap and (iii) the acquiring Team has Room for the player's Salary provided for in the first Season of the Contract.
\item
  In the event a Rookie Scale Contract is extended pursuant to Section 7(b) above and a Team proposes to assign such Contract to another Team prior to the July 1 immediately following such extension, then, for purposes of determining whether the acquiring Team has Room for the Contract only, the Salary for the last Season of the original term of the Contract shall be deemed to equal the average of the aggregate Salaries for such Season and each Season of the extended term.
\end{enumerate}

\hypertarget{miscellaneous.}{%
\section{Miscellaneous.}\label{miscellaneous.}}

\begin{enumerate}
\def\labelenumi{(\alph{enumi})}
\tightlist
\item
  Except where this Agreement states otherwise, for purposes of any rule in this Agreement that limits, involves counting, or otherwise relates to, the number of Seasons covered by a Contract:

  \begin{enumerate}
  \def\labelenumii{(\arabic{enumii})}
  \tightlist
  \item
    If a Player Contract is signed after the beginning of a Season, the Season in which the Contract is signed shall be counted as one (1) full Season covered by the Contract.
  \item
    An Option Year shall be counted as one (1) Season covered by the Contract.
  \end{enumerate}
\item
  Except where this Agreement states otherwise, all of the rules in this Agreement that limit, affect the calculation of, or otherwise relate to, the Compensation or Salary provided for in a Player Contract shall apply to Option Years.
\end{enumerate}

\hypertarget{accounting-procedures.}{%
\section{Accounting Procedures.}\label{accounting-procedures.}}

\begin{enumerate}
\def\labelenumi{(\alph{enumi})}
\item
  \begin{enumerate}
  \def\labelenumii{(\arabic{enumii})}
  \tightlist
  \item
    The NBA and the Players Association shall jointly engage an independent auditor (the ``Accountants'') to provide the parties with an ``Audit Report'' (and a ``Draft Audit Report,'' an ``Interim Audit Report'' and, if applicable, an ``Interim Escrow Audit Report'') setting forth BRI, Team Salary and Benefits of each NBA Team for the immediately preceding Salary Cap Year and, commencing with the audit reports prepared with respect to the 2001-02 Salary Cap Year, the information called for by Section 12 of this Article VII (the ``Escrow Information''). The audit reports provided for by this subsection (a)(1) are to be prepared in accordance with the provisions and definitions contained in this Agreement. The engagement of the Accountants shall be deemed to be renewed annually unless they are discharged by either party during the period from the submission of an Audit Report up to January 1 of the following year. The parties agree to share equally the costs incurred by the Accountants in preparing the audit reports provided for by this subsection (a)(1).
  \item
    The Accountants shall submit a ``Draft Audit Report'' to the NBA and the Players Association, along with relevant supporting documentation, on or before the July 15 following the conclusion of each Salary Cap Year. The final Audit Report shall be submitted by the Accountants to the parties on or before the following July 31, or, if necessary, at a later date promptly following the final resolution of all disputes (by agreement of the parties confirmed in writing or by means of the dispute-resolution procedures provided for by this Agreement). The NBA, the Players Association and the Teams shall use their best efforts to facilitate the Accountants' timely completion of the Audit Report. In the event that, for any reason, the Accountants fail to submit to the parties a final Audit Report by July 31, the Accountants shall prepare an interim Audit Report (the ``Interim Audit Report'') by such date setting forth the Accountants' best estimate of BRI and Total Salaries and Benefits for the preceding Salary Cap Year and, commencing with such Interim Audit Report as may be prepared with respect to the 2001-02 Salary Cap Year and based upon such best estimates, the Escrow Information. Such Interim Audit Report shall include:

    \begin{enumerate}
    \def\labelenumiii{(\roman{enumiii})}
    \tightlist
    \item
      All amounts of BRI and Total Salaries and Benefits (or the portions thereof) and all Escrow Information (or the portions thereof) for such Salary Cap Year as to which the Accountants have completed their review and, by written agreement of the Players Association and the NBA (waiving their respective rights to dispute such amounts), are not in dispute.
    \item
      With respect to any amounts of BRI or Total Salaries and Benefits (or portions thereof) as to which the Accountants have not completed their review or which are the subject of a good faith dispute between the parties, the NBA's good faith proposal as to the proper amount, if any, that should be included in the Audit Report.
    \item
      With respect to any items of Escrow Information that are the subject of a good faith dispute between the parties, the Accountants' good faith determination as to such items, taking into account the provisions of subsections (i) and (ii) of this Section 10(a)(2).
    \end{enumerate}
  \end{enumerate}

  As soon as practicable after the Interim Audit Report is submitted to the parties, the Accountants shall submit the final Audit Report, including a description of the differences, if any, from the Interim Audit Report. An Audit Report shall not be deemed final until the parties have confirmed in writing their agreement with such Report or all disputes with respect to such Report have been finally resolved by means of the dispute-resolution procedures provided for by this Agreement.

  If, at the conclusion of the Audit Report Challenge Period (as defined by Section 12(b)(4) below), the Accountants have not submitted or are unable to submit a final Audit Report (because, by way of example but not limitation, there are disputes or claims that have been asserted pursuant to Article XXXII, Section 9(c) and which remain pending), the Accountants shall prepare and submit to the parties, within five (5) business days following the completion of the Audit Report Challenge Period, an Interim Escrow Audit Report that shall include the information set forth in the Interim Audit Report as adjusted or amended so as to reflect any final determinations made by the System Arbitrator or the Appeals Panel (as the case may be) in proceedings commenced pursuant to Article XXXII, Section 9(b) and involving disputes or claims with respect to such Interim Audit Report. The sole purpose for which any Interim Escrow Audit Report is to be used under this Agreement is to perform or form the basis for the calculations to be made pursuant to Article VII, Section 12.
\item
  For purposes of determining BRI, Total Salaries and Benefits and the Escrow Information, the Accountants shall perform at least such review procedures as shall be agreed upon by the parties. In connection with the preparation of Audit Reports for each Salary Cap Year, each Team and the NBA shall submit a report to the Accountants, the NBA and the Players Association setting forth BRI, Team Salaries and Benefits information for such Salary Cap Year, on forms agreed upon by the NBA, the Players Association and the Accountants (the ``BRI Reports''). The NBA and the Players Association shall agree upon such forms no later than April 1 of each
  Salary Cap Year.
\item
  The Accountants shall review the reasonableness of any estimates of revenues or expenses for a Salary Cap Year included in the Teams' and the NBA's BRI Reports for such Salary Cap Year and may make such adjustments in such estimates as they deem appropriate. To the extent the actual amounts of revenues received or expenses incurred for a Salary Cap Year differ from such estimates, adjustments shall be made in BRI for the following Salary Cap Year in accordance with the provisions of subsection (f) below.
\item
  With respect to expenses deducted by the NBA or the Teams, the NBA and the Teams shall report in BRI Reports only those expenses that are reasonable and customary in accordance with the provisions of Section 1(a)(1) of this Article VII. Subject to the terms of Section 1(a)(6) and Section 11 of this Article VII, all categories of expenses deducted in a BRI Report completed by the NBA or a Team shall be reviewed by the Accountants, but such categories shall be presumed to be reasonable and customary and the amount of the expenses deducted by the NBA or a Team that come within such expense categories shall also be presumed to be reasonable and customary, unless such categories or amounts are found by the Accountants to be either unrelated to the revenues involved or grossly excessive.
\item
  The Accountants shall notify designated representatives of the NBA and the Players Association: (i) if the Accountants have any questions concerning the amounts of revenues or expenses reported by the Teams and the NBA or any other information contained in the BRI Reports; or (ii) if the Accountants propose that any adjustments be made to any revenue or expense item or any other information contained in the BRI Reports.
\item
  The Accountants shall indicate which amounts included in BRI for a Salary Cap Year, if any, represent estimates of revenues. With respect to any such estimated revenues, the Accountants shall, in preparing the Audit Report for the immediately succeeding Salary Cap Year (``Subsequent Audit Report''), or the Audit Report for the same Salary Cap Year in the event that an Interim Audit Report was previously issued for that Salary Cap Year, determine the actual revenues received for the prior Salary Cap Year and include as a credit or debit to BRI in such Subsequent Audit Report the amount of the aggregate difference, if any, between all such estimated revenues for the prior Salary Cap Year and the actual revenues received for such Salary Cap Year (the ``Estimated Revenue Adjustment'').
\item
  All disputes with respect to any Interim Audit Report shall be resolved exclusively in accordance with the procedures set forth in Article XXXII.
\end{enumerate}

\hypertarget{players-association-audit-rights.}{%
\section{Players Association Audit Rights.}\label{players-association-audit-rights.}}

\begin{enumerate}
\def\labelenumi{(\alph{enumi})}
\tightlist
\item
  Team Audits. The Players Association shall have the right as part of the annual review of BRI Reports to retain its own accountants (the ``Players Association's Accountants''), at its own expense, after the submission of each Audit Report under this Agreement (the ``First Audit''), to audit the books and records of five (5) NBA teams (of its choosing) and shall also have the right to review the books and records of the NBA League Office, provided, however, that such review shall be limited to (i) revenue items, and (ii) expense items that appear or should have appeared in the BRI Reports. In the event that, in the opinion of the Players Association's Accountants, such audit indicates misallocations or miscategorizations of revenues or expenses (other than with respect to matters that constituted Disputed Adjustments in connection with the prior Audit Report) resulting in an understatement of BRI in excess of \$3 million, they shall submit to the NBA proposed adjustments to BRI consistent with their findings. In the event that the NBA disputes such proposed adjustments, such proposed adjustments shall be deemed to be ``Disputed Adjustments'' and shall be resolved in accordance with the procedures set forth in Article XXXII. In addition, in the event that First Audit Disputed Adjustments in excess of \$3 million are resolved in favor of the Players Association, the Players Association shall then have the right, that Season, to have the Players Association's Accountants audit an additional five NBA teams, in accordance with the foregoing procedures (the ``Second Audit''). If, as a result of the Second Audit, additional Disputed Adjustments in excess of \$3 million are resolved in favor of the Players Association, the Players Association shall then have the right, that Season, to have the Players Association's Accountants audit all remaining NBA Teams. The amount of any and all Disputed Adjustments that are ultimately resolved in favor of the Players Association in accordance with this Section 11 (a) shall be added to BRI in the Season in which such resolution is reached.
\item
  Expense Audit. The Players Association shall have the right to retain the Players Association's Accountants to conduct one audit, at its own expense, of the expenses incurred in connection with the proceeds that come within Article VII, Section 1(a)(1)(viii) regardless of whether such expenses exceed the applicable Expense Ratios set forth in Exhibit D. In the event that in the opinion of the Players Association's Accountants, such audit indicates a misallocation or miscategorization of expenses resulting in an understatement of BRI, they shall submit proposed adjustments to the NBA consistent with their findings. In the event the NBA disputes such proposed adjustments, such proposed adjustments shall be deemed to be Disputed Adjustments and resolved in accordance with the procedures set forth in Article XXXII. The amount of any and all such Disputed Adjustments that are resolved in the Players Association's favor shall be included in BRI in the Season in which such resolution is reached. In addition, in the event that any such Disputed Adjustments are resolved in the Players Association's favor, the Accountants shall be directed to correct such expense misallocations and/or miscategorizations in the remaining Seasons covered by the Agreement.
\end{enumerate}

\hypertarget{escrow-arrangement.}{%
\section{Escrow Arrangement.}\label{escrow-arrangement.}}

\begin{enumerate}
\def\labelenumi{(\alph{enumi})}
\tightlist
\item
  Commencement. The provisions set forth in this Section 12 shall take effect beginning with the 2001-02 Salary Cap Year and shall remain in effect thereafter for the duration of this Agreement.
\item
  Definitions. As used in this Agreement, the following terms shall have the following meanings:

  \begin{enumerate}
  \def\labelenumii{(\arabic{enumii})}
  \item
    ``Adjustment Player'' means, with respect to a Salary Cap Year, every current or former player included in a Team's Team Salary for such Salary Cap Year, except for players signed only to one or two 10-Day Contracts; provided, however, that in the event the Players Association proposes to the NBA an alternative definition of ``Adjustment Player'' that does not affect the NBA's ability to recover the Aggregate Compensation Adjustment Amount for any Salary Cap Year, then, subject to the NBA's approval, which shall not be unreasonably withheld, such alternative definition shall be used in lieu of the preceding definition.
  \item
    ``Aggregate Compensation Adjustment Amount'' means, with respect to a Salary Cap Year, the lesser of (i) the Overage for such Salary Cap Year, and (ii) 10\% of Total Salaries and Benefits for such Salary Cap Year.
  \item
    ``Aggregate Compensation Adjustment Amount Shortfall'' means the amount by which the amount received by the NBA from the Escrow Agent with respect to a Salary Cap Year pursuant to subsection (e)(1) below is less than the Aggregate Compensation Adjustment Amount for such Salary Cap Year.
  \item
    ``Audit Report Challenge Period'' means the period beginning with the date on which an Interim Audit Report is issued by the Accountants and ending on the last date by which all challenges thereto brought pursuant to Article XXXII, Section 9(b) are resolved.
  \item
    ``Deduction Date'' means each of the seven (7) semi-monthly payment dates from February 1 through May 1 provided for under paragraph 3 of the Uniform Player Contract.
  \item
    ``Designated Percentage'' means, with respect to a Salary Cap Year, the percentage set forth in subsection (c)(3) below.
  \item
    ``Escrow Agent'' means, for purposes of this Article VII, Section 12, the escrow agent identified in the Salary Escrow Agreement.
  \item
    ``Escrow Amount'' means for an Adjustment Player, with respect to a Salary Cap Year, the amount calculated by multiplying the Projected Aggregate Compensation Adjustment Amount for such Salary Cap Year by a fraction, the numerator of which is the Adjustment Player's Salary for such Salary Cap Year, and the denominator of which is the sum of all Adjustment Players' Salaries for such Salary Cap Year as of November 30 of such Salary Cap Year; provided, however, that in the event the Players Association proposes to the NBA an alternative method for calculating an Adjustment Player's Escrow Amount that results in the sum of all Adjustment Players' Escrow Amounts equaling or exceeding the Projected Aggregate Compensation Adjustment Amount, then, subject to the NBA's approval, which shall not be unreasonably withheld, such alternative formula shall be used in lieu of the preceding formula. For purposes of calculating the fraction described in the preceding sentence, the Salary of a player under a one-year Contract making the Minimum Player Salary shall include the portion of such Minimum Player Salary that is reimbursed out of the league-wide benefits fund described in Article IV, Section 1(l).
  \item
    ``Escrow Schedules'' means the schedules prepared by the NBA with respect to a Salary Cap Year setting forth: (A) Projected Total Salaries and Benefits for such Salary Cap Year; (B) the Projected Overage for such Salary Cap Year; (C) the Projected Aggregate Compensation Adjustment Amount for such Salary Cap Year; (D) the Escrow Amount to be deducted from the Cash Compensation of each Adjustment Player on each Team; (E) the amount that each Team must deduct with respect to each Adjustment Player on each Deduction Date.
  \item
    ``Individual Compensation Adjustment Amount'' means for an Adjustment Player, with respect to a Salary Cap Year, the amount calculated following the conclusion of the Salary Cap Year by multiplying the Aggregate Compensation Adjustment Amount for such Salary Cap Year by a fraction, the numerator of which is the Adjustment Player's Salary for such Salary Cap Year and the denominator of which is the sum of all Adjustment Players' Salaries for such Salary Cap Year; provided, however, that in the event the Players Association proposes to the NBA an alternative method for calculating an Adjustment Player's Individual Compensation Adjustment Amount that results in the sum of all Adjustment Players' Individual Compensation Adjustment Amounts equaling the Aggregate Compensation Adjustment Amount, then, subject to the NBA's approval, which shall not be unreasonably withheld, such alternative formula shall be used in lieu of the preceding formula. For purposes of calculating the fraction described in the preceding sentence: (i) a player's Salary shall include all Performance Bonuses excluded from Salary under Article VII, Section 3(d) but actually earned by the player during such Salary Cap Year, and shall exclude all Performance Bonuses included in Salary under Article VII, Section 3(d) but not actually earned by the player during such Salary Cap Year; and (ii) the Salary of a player under a one-year Contract making the Minimum Player Salary shall include the portion of such Minimum Player Salary that is reimbursed out of the league-wide benefits fund described in Article IV, Section 1(l).
  \item
    ``Individual Shortfall Adjustment Amount'' means, with respect to each Contract that is amended pursuant to subsection (f)(1) below, the amount that the Cash Compensation otherwise payable in accordance with that Contract shall be reduced pursuant to subsection (f)(2) below.
  \item
    ``Overage'' means the amount, if any, by which Total Salaries and Benefits for a Salary Cap Year exceed an amount equal to the Designated Percentage of BRI for such Salary Cap Year.
  \item
    ``Projected Aggregate Compensation Adjustment Amount'' means: (i) with respect to each of the 2001-02 and 2002-03 Salary Cap Years (and the 2003-04 Salary Cap Year in the event the NBA exercises its option to extend this Agreement in accordance with Article XXXIX), the lesser of (A) the Projected Overage for such Salary Cap Year, and (B) 10\% of Projected Total Salaries and Benefits for such Salary Cap Year; and (ii) with respect to the 2003-04 Salary Cap Year (or the 2004-05 Salary Cap Year in the event the NBA exercises its option to extend this Agreement in accordance with Article XXXIX), 10\% of Projected Total Salaries and Benefits for such Salary Cap Year.
  \item
    ``Projected Overage'' means the amount, if any, by which Projected Total Salaries and Benefits for a Salary Cap Year exceeds an amount equal to the Designated Percentage of Projected BRI for such Salary Cap Year, or Interim Projected BRI in the event that the Audit Report for the prior Salary Cap Year has not been completed as of November 15.
  \item
    ``Projected Total Salaries and Benefits'' means, with respect to a Salary Cap Year, 110\% of the sum of the following amounts: (i) the aggregate Salaries of all active players (and former players to the extent provided by the terms of this Agreement) as of November 30 of such Salary Cap Year, including, without limitation, the amounts set forth in Article VII, Section 4(a)(1)(i)-(iii); (ii) with respect to each Team whose Team Salary as of November 30 of such Salary Cap Year is below the Minimum Team Salary, the aggregate amount by which such Teams' Team Salaries are below the Minimum Team Salary as of November 30; and (iii) Projected Benefits for such Salary Cap Year (as defined in Article IV, Section 6).
  \item
    ``Salary Escrow Agreement'' means the escrow agreement in the form agreed upon by the parties (or such other form to which the parties may agree) to be entered into with the Escrow Agent.
  \item
    ``Team Escrow Limit'' means, with respect to a Salary Cap Year, an amount determined by the following calculation:

    Step 1: Divide an amount equal to the Designated Percentage of BRI for such Salary Cap Year by 0.9;

    Step 2: Subtract Benefits for such Salary Cap Year from the result in Step 1.

    Step 3: Divide the result in Step 2 by the number of Teams in the NBA during such Salary Cap Year.
  \end{enumerate}
\item
  Compensation Adjustment Rules.

  \begin{enumerate}
  \def\labelenumii{(\arabic{enumii})}
  \tightlist
  \item
    In the event that there is an Overage in any Salary Cap Year, the Contracts of all Adjustment Players shall be amended by operation of this Agreement, such that the aggregate Cash Compensation otherwise payable to all such Adjustment Players with respect to such Salary Cap Year shall be reduced by the Aggregate Compensation Adjustment Amount.
  \item
    Subject to subsections (e) and (f) below, to effectuate the aggregate reduction provided for in subsection (c)(1) above, the Cash Compensation otherwise payable in accordance with each Adjustment Player's Contract shall be reduced by the player's respective Individual Compensation Adjustment Amount.
  \item
    The Designated Percentages for the 2001-02 Salary Cap Year and each subsequent Salary Cap Year during the term of this Agreement are as follows:
  \end{enumerate}

  \begin{longtable}[]{@{}cc@{}}
  \toprule()
  Salary Cap Year & Designated Percentage \\
  \midrule()
  \endhead
  2001-02: & 55\% \\
  2002-03: & 55\% \\
  2003-04: & 55\% \\
  2004-05 & 57\% \\
  \bottomrule()
  \end{longtable}

  (only if the NBA exercises its option to extend this Agreement in accordance with Article XXXIX):
\item
  Escrow Procedure.

  \begin{enumerate}
  \def\labelenumii{(\arabic{enumii})}
  \tightlist
  \item
    In the event that there is a Projected Overage for a Salary Cap Year, the following shall apply (subject to subsection (e) below regarding final reconciliation):

    \begin{enumerate}
    \def\labelenumiii{(\roman{enumiii})}
    \tightlist
    \item
      The Cash Compensation otherwise payable to each Adjustment Player shall be reduced by the Escrow Amount applicable to such Adjustment Player; and
    \item
      Each Team shall deposit the Escrow Amount with respect to each of its Adjustment Players with the Escrow Agent.
    \end{enumerate}
  \item
    Except as set forth in subsection (d)(4) below, the Escrow Amount for each Adjustment Player shall be collected through seven (7) equal installments from each of the player's semi-monthly Cash Compensation payments on each Deduction Date.
  \item
    The procedure for deducting and depositing Escrow Amounts shall be as follows:

    \begin{enumerate}
    \def\labelenumiii{(\roman{enumiii})}
    \tightlist
    \item
      The NBA will prepare and send to the Players Association the Escrow Schedules on or before December 15 of each Salary Cap Year in which there is a Projected Overage, and periodically thereafter to reflect any new or adjusted Escrow Amounts calculated in accordance with subsection (d)(4) below.
    \item
      Within three (3) business days after each Deduction Date, each Team shall deliver to the Escrow Agent, in accordance with the Salary Escrow Agreement, the aggregate amount that the Team is obligated to deduct with respect to such Deduction Date for all of its Adjustment Players. All amounts received by the Escrow Agent shall be invested and disbursed in accordance with the provisions of the Salary Escrow Agreement.
    \end{enumerate}
  \item
    After December 15, the NBA shall periodically update the Escrow Schedules to add Escrow Amounts for players who enter into Player Contracts on or after December 1 and to make such adjustments as may be necessary to previously-listed Escrow Amounts (such as adjustments resulting from Renegotiations). Any portion of an Escrow Amount that has not been deducted as of the date any such updated Schedules are prepared shall be deducted in equal installments from each of the remaining semi-monthly Cash Compensation payments to be made to the player from February 1 through May 1.
  \item
    Within seven (7) days after receiving any set of Escrow Schedules from the NBA, or within seven (7) days after any event that the Players Association believes warrants a change in any previously-issued Schedules, the Players Association may bring a proceeding before the System Arbitrator, in accordance with Article XXXII, Section 10, contesting the NBA's calculation of the Projected Aggregate Compensation Adjustment Amount for such Salary Cap Year and/or any player's Escrow Amount for such Salary Cap Year. Notwithstanding the commencement of any such proceeding, each Team shall commence and continue remitting to the Escrow Agent the total deductions due with respect to each Deduction Date as set forth in the Schedules, and in no event shall any Team be prohibited from remitting to the Escrow Agent any such deduction prior to a final determination in any such proceeding.
  \item
    In the event that the NBA makes a determination in accordance with subsection (d)(4) above, or a final determination is made in a proceeding in accordance with subsection (d)(5) above, that an Escrow Amount was erroneously calculated by the NBA, the sole remedy with respect to any amounts erroneously deducted from the player's Salary shall be to modify, as soon as practicable, the deduction schedule applicable to such player so as to reduce, in equal amounts, all scheduled future deductions from post-determination payments of Cash Compensation until the amount of any prior over-deduction is fully off-set; provided, however, that to the extent that reducing the player's future deductions would not fully offset the prior over-deductions, the parties shall instruct the Escrow Agent to pay the player as soon as practicable, with interest, such additional amounts as are necessary to fully offset such over-deductions.
  \end{enumerate}
\item
  Reconciliation Procedures.

  \begin{enumerate}
  \def\labelenumii{(\arabic{enumii})}
  \tightlist
  \item
    In the event of an Overage: (i) the NBA shall be entitled to receive from the Escrow Agent, with respect to each Adjustment Player, such player's Individual Compensation Adjustment Amount (or, in the event that the player's Escrow Amount is less than his Individual Compensation Adjustment Amount, a portion of his Individual Compensation Adjustment Amount equal to his Escrow Amount); and (ii) each Adjustment Player shall be entitled to receive from the Escrow Agent the amount, if any, by which the player's Escrow Amount exceeds his Individual Compensation Adjustment Amount. In the event that there is no Overage, each Adjustment Player shall be entitled to receive from the Escrow Agent his entire Escrow Amount.
  \item
    Any interest earned on Escrow Amounts remitted to the Escrow Agent shall be allocated among the Adjustment Players, collectively, and the NBA in proportion to the percentage of the aggregate Escrow Amounts that the Adjustment Players, collectively, and the NBA are to receive from the Escrow Agent in accordance with subsection (e)(1) above. The Adjustment Players' collective share of interest shall be allocated among the individual players in proportion to each player's Escrow Amount.
  \item
    The parties shall cause the Accountants to include in the Interim Audit Report and the Audit Report (or, if no final Audit Report has been submitted at the conclusion of the Audit Report Challenge Period, in the Interim Escrow Audit Report) for each Salary Cap Year schedules setting forth, with respect to such Salary Cap Year:

    \begin{enumerate}
    \def\labelenumiii{(\roman{enumiii})}
    \tightlist
    \item
      the amount of any Overage;
    \item
      the Aggregate Compensation Adjustment Amount, if any;
    \item
      each Adjustment Player's Individual Compensation Adjustment Amount, if any;
    \item
      each Adjustment Player's Escrow Amount, if any, as set forth in the Escrow Schedules;
    \item
      a list of all Adjustment Players whose Individual Compensation Adjustment Amounts exceed their Escrow Amounts, which list shall also include (A) each such player's Individual Compensation Adjustment Amount, (B) each such player's Escrow Amount, (C) the amount by which each such player's Individual Compensation Adjustment Amount exceeds his Escrow Amount, (D) the sum of all such players' Escrow Amounts, (E) the sum of all such players' Individual Compensation Adjustment Amounts, and (F) the aggregate amount by which all such players' Individual Compensation Adjustment Amounts exceed their Escrow Amounts;
    \item
      a list of all Adjustment Players whose Individual Compensation Adjustment Amounts are equal to or less than their Escrow Amounts, which list shall also include (A) each such player's Individual Compensation Adjustment Amount, (B) each such player's Escrow Amount, (C) the amount, if any, by which each such player's Escrow Amount exceeds his Individual Compensation Adjustment Amount, (D) the sum of all such players' Escrow Amounts, (E) the sum of all such players' Individual Compensation Adjustment Amounts, and (F) the aggregate amount by which all such players' Escrow Amounts exceed their Individual Compensation Adjustment Amounts;
    \item
      in accordance with the provisions of subsections (e)(1) and (e)(2) above, (A) the percentage of the interest earned on the Escrow Amounts to be allocated to the NBA, (B) the percentage of the interest earned on the Escrow Amounts to be allocated to the Adjustment Players collectively, and (C) the percentage of the interest earned on the Escrow Amounts to be allocated to each individual Adjustment Player;
    \item
      the Team Escrow Limit; and
    \item
      the amount, if any, by which each Team's Team Salary as computed in subsection (g)(3) below exceeds the Team Escrow Limit.
    \end{enumerate}
  \item
    In addition to the information described in subsection (e)(3) above, the parties shall cause the Accountants to include in the Audit Report (or, if no final Audit Report has been submitted at the conclusion of the Audit Report Challenge Period, in the Interim Escrow Audit Report) a separate Notice to Escrow Agent, in the form attached to the Salary Escrow Agreement, setting forth:

    \begin{enumerate}
    \def\labelenumiii{(\roman{enumiii})}
    \tightlist
    \item
      in the space designated in paragraph 1 of the Notice, the sum of the amounts described in subsections (e)(3)(v)(D) and (e)(3)(vi)(E) above, which sum is to be disbursed by the Escrow Agent to the NBA;
    \item
      in the space designated in paragraph 2 of the Notice, the amounts described in subsection (e)(3)(vi)(C) above, which amounts are to be disbursed by the Escrow Agent to each respective Adjustment Player described in subsection (e)(3)(vi) above;
    \item
      in the space designated in paragraph 3 of the Notice, the information described in subsection (e)(3)(vii)(A) above, which information shall be the basis for the Escrow Agent's calculation of interest earned on the Escrow Amounts, which interest is to be disbursed by the Escrow Agent to the NBA; and
    \item
      in the space designated in paragraph (4) of the Notice, the information described in subsection (e)(3)(vii)(C) above, which information shall be the basis for the Escrow Agent's calculation of interest earned on the Escrow Amounts, which interest is to be disbursed by the Escrow Agent to each Adjustment Player.
    \end{enumerate}
  \item
    No later than seven (7) business days after the earlier of (i) the completion of the Audit Report for the prior Salary Cap Year or (ii) the completion of the Audit Report Challenge Period, the parties shall cause the Accountants to deliver to the Escrow Agent the completed Notice to Escrow Agent. As soon as practicable following receipt of such notice, the Escrow Agent shall disburse the specified sums to the specified payees.
  \item
    Any amounts that the Escrow Agent is obligated to disburse to a player pursuant to this Section 12, including, if the Players Association so elects, the amounts described in subsection (e)(4)(iv) above, shall be reduced by all amounts required to be withheld by federal, state, and local authorities, which withholdings shall be disbursed by the Escrow Agent to the Player's Team for remittance to the appropriate authorities. To assist the Escrow Agent in disbursing the appropriate amounts to each Adjustment Player and his respective Team, each Team, based on the information set forth in paragraph 2 (and, if applicable, paragraph 4) of the Notice to Escrow Agent, shall promptly provide the Escrow Agent with a schedule for each of its Adjustment Players showing the exact withholding amount to be disbursed to the Team for remittance to the appropriate federal, state and local authorities. In no circumstance shall the employer's share of FICA, FUTA, or any other employer taxes be paid out of the amounts deposited in escrow or any interest or earnings thereon. Any such obligations shall remain with each player's individual employer.
  \end{enumerate}
\item
  Aggregate Compensation Adjustment Amount Shortfalls.

  \begin{enumerate}
  \def\labelenumii{(\arabic{enumii})}
  \tightlist
  \item
    If, with respect to any Salary Cap Year, there is an Aggregate Compensation Adjustment Amount Shortfall, then the Contract of each of the following Salary Cap Year's Adjustment Players shall be amended by operation of this Agreement, in accordance with subsection (f)(2) below, such that the aggregate Cash Compensation paid to all such players with respect to the Season covered by such following Salary Cap Year shall be reduced by the Aggregate Compensation Adjustment Amount Shortfall, which reduction shall be in addition to the full amount of any reduction for such following Salary Cap Year called for in subsections (c)(1)-(2) above.
  \item
    The Individual Shortfall Adjustment Amount for each Adjustment Player whose Contract is amended in accordance with subsection (f)(1) above shall be calculated by multiplying the Aggregate Compensation Adjustment Amount Shortfall for the prior Salary Cap Year by a fraction, the numerator of which is the player's then-current Salary, and the denominator of which is the sum of all such players' then-current Salaries. For purposes of calculating the fraction described in the preceding sentence, the Salary of a player making the Minimum Player Salary shall include the portion of such Minimum Player Salary that is reimbursed out of the league-wide benefits fund described in Article IV, Section 1(l).
  \item
    The Individual Shortfall Adjustment Amount for each Adjustment Player shall be deducted by the player's Team in four (4) equal installments from each of the player's first four (4) semi-monthly Cash Compensation payments following delivery to the Escrow Agent of the completed Notice to Escrow Agent. All such deductions shall be promptly remitted by the Teams to the NBA.
  \end{enumerate}
\item
  Team Payments.

  \begin{enumerate}
  \def\labelenumii{(\arabic{enumii})}
  \item
    In the event that the Overage in any Salary Cap Year exceeds 10\% of Total Salaries and Benefits for such Salary Cap Year, each Team whose Team Salary exceeded the Team Escrow Limit for such Salary Cap Year shall be required to pay a tax to the NBA equal to the amount by which the Team's Team Salary as of the date of the Team's last Regular Season game exceeds the Team Escrow Limit.

    Example: (Numbers rounded for purposes of illustration).

    Assumptions:

    BRI = \$2.8 billion

    Benefits = \$70 million

    \begin{longtable}[]{@{}
      >{\centering\arraybackslash}p{(\columnwidth - 2\tabcolsep) * \real{0.5283}}
      >{\centering\arraybackslash}p{(\columnwidth - 2\tabcolsep) * \real{0.4717}}@{}}
    \toprule()
    \begin{minipage}[b]{\linewidth}\centering
    Total Salaries and Benefits as a Percentage of BRI
    \end{minipage} & \begin{minipage}[b]{\linewidth}\centering
    Corresponding Average Team Salary
    \end{minipage} \\
    \midrule()
    \endhead
    48.04\% & \$44 million (Salary Cap) \\
    55\% (Designated Percentage) & \$50.7 million \\
    61.1\% & \$56.6 million (Team Escrow Limit) \\
    \bottomrule()
    \end{longtable}

    If Total Salaries and Benefits exceed 61.1\% of BRI, then a Team with a Team Salary of \$60 million would pay a \$3.4 million tax to the NBA.
  \item
    Each Team that owes a tax shall make the required tax payment to the NBA no later than ten (10) business days following the earlier of (i) the completion of the Audit Report for the prior Salary Cap Year or (ii) the completion of the Audit Report Challenge Period.
  \item
    For purposes of this subsection (g), Team Salary shall be calculated by the Accountants in the same manner as Team Salary is calculated by the Accountants for purposes of computing Total Salaries and Benefits in the Audit Report.
  \end{enumerate}
\item
  Miscellaneous.

  \begin{enumerate}
  \def\labelenumii{(\arabic{enumii})}
  \tightlist
  \item
    All amounts remitted to the NBA by the Escrow Agent or NBA Teams pursuant to this Section 12 shall be the exclusive property of the NBA, and the use and/or disposition of all such amounts, including the allocation or distribution of such amounts to one or more NBA Teams, if any, shall be within the NBA's sole discretion.
  \item
    Notwithstanding any other provision of this Agreement, the computation of an Adjustment Player's Salary shall for purposes of the rules set forth in this Agreement be made without regard to any reduction in such player's Compensation made pursuant to this Section 12.
  \item
    The NBA shall be permitted to assign to such designee, as the NBA may determine, any rights the NBA has to receive amounts from the Escrow Agent or NBA Teams pursuant to this Article VII, Section 12.
  \item
    Consistent with Article VII, Section 3(f) (One-Year Minimum Contracts), except for purposes of calculating the fractions referred to in the definitions of Escrow Amount, Individual Compensation Adjustment Amount, and Individual Shortfall Adjustment Amount (set forth in subsections (b)(8), (b)(10), and (f)(2) above), the Salary of every player who signs a one-year Contract after the date of this Agreement for the Minimum Player Salary applicable to such player shall, for all other purposes in this Section 12, be the lesser of (i) such Minimum Player Salary, or (ii) the portion of such Minimum Player Salary that is not reimbursed out of the league-wide benefits fund described in Article IV, Section 1(l).
  \item
    In the event that the Overage for any Salary Cap Year exceeds the Aggregate Compensation Adjustment Amount for such Salary Cap Year, the NBA shall not be entitled to reduce player Compensation in such Salary Cap Year or any subsequent Salary Cap Year so as to recover any amounts in excess of the Aggregate Compensation Adjustment Amount.
  \end{enumerate}
\end{enumerate}

\hypertarget{rookie-scale}{%
\chapter{ROOKIE SCALE}\label{rookie-scale}}

\hypertarget{rookie-scale-contracts-for-first-round-picks.}{%
\section{Rookie Scale Contracts for First Round Picks.}\label{rookie-scale-contracts-for-first-round-picks.}}

\begin{enumerate}
\def\labelenumi{(\alph{enumi})}
\tightlist
\item
  Players who: (i) were drafted in the 1995-1997 NBA Drafts; and (ii) signed a Rookie Scale Contract prior to the date of this Agreement, remain subject to the Rookie Scale amounts that were applicable to such players when they were drafted, and such players are not subject to any provisions of this Article VIII.
\item
  Except as provided in Sections 2 and 3 below, for players who: (i) are drafted in the 1998-2004 NBA Drafts (and players drafted in the 2005 NBA Draft, if the NBA exercises its option to extend this Agreement pursuant to Article XXXIX); or (ii) were drafted in the 1995-1997 NBA Drafts and who did not sign a Rookie Scale Contract prior to the date of this Agreement, the following rules shall apply to every Rookie Scale Contract for such players:
\item
  Each Rookie Scale Contract shall cover a period of three (3) Seasons, but shall have an Option in favor of the Team for the player's fourth Season, as set forth in Article XI, Section 4 below.
\item
  A Rookie Scale Contract shall provide in each of the three (3) Seasons covered by the Contract at least 80\% of the applicable Rookie Scale Amount in Current Cash Compensation. Components of Salary in excess of 80\%, if any, are subject to individual negotiation, except that (i) in no event may Salary plus Unlikely Bonuses plus (if earned) any assignment bonus in any Season exceed 120\% of the applicable Rookie Scale Amount, and (ii) a Rookie Scale Contract may not provide for a signing bonus (except for an assignment bonus or ``foreign player payments'' in excess of \$350,000 made in accordance with Article VII, Section 3(e)) or a loan. A Rookie Scale Contract may provide for a payment schedule in any Season that is more favorable to the player than that called for under paragraph 3 of the Uniform Player Contract, subject to the other provisions of this Agreement.
\item
  A First Round Pick who does not sign with the Team that holds his draft rights for any portion of the Season immediately following the Draft in which he was selected shall be treated, for purposes of determining the applicable Rookie Scale Amounts at such time as he enters into a Rookie Scale Contract, as if he were drafted in the Draft immediately preceding the first Season of such Contract at the same draft position at which he was actually selected.
\item
  A Rookie Scale Contract must provide for Cash Compensation protection for skill and non-insured injury or illness in each Season to the extent of not less than 80\% of the applicable Rookie Scale Amounts.
\end{enumerate}

\hypertarget{rookie-contracts-for-later-signed-first-round-picks.}{%
\section{Rookie Contracts for Later-Signed First Round Picks.}\label{rookie-contracts-for-later-signed-first-round-picks.}}

Except as provided in Section 3 below, a First Round Pick who does not sign with the Team that holds his draft rights for any portion of the three (3) Seasons following the NBA Draft in which he was selected (and who did not play intercollegiate basketball during such period) may enter into either (i) a Rookie Scale Contract in accordance with Section 1 above, or (ii) if the Team has Room in excess of the applicable first-year Rookie Scale Amount, a Contract covering no fewer than three Seasons that provides for Salary plus Unlikely Bonuses in the first Season up to the amount of the Team's Room and increases or decreases in Salary and Unlikely Bonuses in subsequent Seasons in accordance with Article VII, Section 5(c)(1).

\hypertarget{loss-of-draft-rights.}{%
\section{Loss of Draft Rights.}\label{loss-of-draft-rights.}}

If for any reason a Team fails to make a Required Tender to a First Round Pick in accordance with Article X, withdraws a Required Tender in accordance with Article X, or renounces a First Round Pick in accordance with Article X, or if a First Round Pick selected in a Subsequent Draft does not sign a Contract for a period of one (1) year following such Subsequent Draft in accordance with Article X, then the rules set forth in Sections 1 and 2 above shall not apply, and such First Round Pick shall become a Rookie Free Agent. In addition, any Team that fails to make a Required Tender to a First Round Pick, withdraws a Required Tender, renounces a First Round Pick, or fails to sign within one (1) year a First Round Pick selected in a Subsequent Draft shall be prohibited from signing such player until after he has signed a Player Contract with another NBA Team, and either (i) the player completes the playing services called for under the Contract, or (ii) the Contract is terminated in accordance with the NBA waiver procedure.

\hypertarget{length-of-player-contracts}{%
\chapter{LENGTH OF PLAYER CONTRACTS}\label{length-of-player-contracts}}

\hypertarget{maximum-term.}{%
\section{Maximum Term.}\label{maximum-term.}}

Except where a shorter term is expressly provided for elsewhere in this Agreement, a Player Contract entered into after the date of this Agreement may cover, in the aggregate, up to but no more than six (6) Seasons (including any Season covered by an Option) from the date such Contract is signed; provided, however, that (a) a Player Contract between a Qualifying Veteran Free Agent and his Prior Team may cover, in the aggregate, up to but no more than seven (7) Seasons (including any Season covered by an Option) from the date such Contract is signed, and (b) an Extension of a Rookie Scale Contract may cover, in the aggregate, up to but no more than seven (7) Seasons (including any Season covered by an Option) from the date such extension is signed.

\hypertarget{computation-of-time.}{%
\section{Computation of Time.}\label{computation-of-time.}}

For purposes of Section 1 above, if a Player Contract or Extension is signed after the beginning of a Season, the Season in which the Contract or Extension is signed shall be counted as one(1) full Season covered by the Contract or Extension.

\hypertarget{nba-draft}{%
\chapter{NBA DRAFT}\label{nba-draft}}

\hypertarget{term-and-timing-of-draft-provisions.}{%
\section{Term and Timing of Draft Provisions.}\label{term-and-timing-of-draft-provisions.}}

An NBA Draft will be held prior to the commencement of each NBA Season covered by the term of this Agreement and, despite the expiration of the other terms of this Agreement pursuant to Article XXXIX, prior to the commencement of the 2004-05 NBA Season (or, if the NBA exercises its option to extend the Agreement pursuant to Article XXXIX, prior to the commencement of the 2005-06 NBA Season). Each such Draft will be held prior to the July 15 preceding the commencement of the NBA Season on a date to be designated by the Commissioner.

\hypertarget{number-of-choices.}{%
\section{Number of Choices.}\label{number-of-choices.}}

The NBA Draft shall consist of two (2) rounds, with each round consisting of the same number of selections as there will be Teams in the NBA the following Season.

\hypertarget{negotiating-rights-to-draft-rookies.}{%
\section{Negotiating Rights to Draft Rookies.}\label{negotiating-rights-to-draft-rookies.}}

\begin{enumerate}
\def\labelenumi{(\alph{enumi})}
\tightlist
\item
  A Team that drafts a player shall, during the period from the date of such NBA Draft (hereinafter the ``Initial Draft'') to the date of the next Draft (hereinafter the ``Subsequent Draft''), be the only Team with which such player may negotiate or sign a Player Contract, provided that, on or before the July 15 immediately following the Initial Draft (for a First Round Pick), or on or in the two (2) weeks before the September 5 immediately following the Initial Draft (for a Second Round Pick), such Team has made a Required Tender to such player. If a Team has made a Required Tender to such a player and the player has not signed a Player Contract within the period between the Initial Draft and the Subsequent Draft, the Team that drafts the player shall lose its exclusive right to negotiate with the player and the player will then be eligible for selection in the Subsequent Draft.
\item
  A Team that, in the Subsequent Draft, drafts a player who (i) was drafted in the Initial Draft, (ii) received a Required Tender from the Team that drafted him in the Initial Draft, and (iii) did not sign a Player Contract with such first Team prior to the Subsequent Draft, shall, during the period from the date of the Subsequent Draft to the date of the next NBA Draft, be the only Team with which such player may negotiate or sign a Player Contract, provided such Team has made a Required Tender. If such player has not signed a Player Contract within the period between the Subsequent Draft and the next NBA Draft with the Team that drafted him in the Subsequent Draft, that Team shall lose its exclusive right, which it obtained in the Subsequent Draft, to negotiate with the player, and the player will become a Rookie Free Agent as of the date of the next NBA Draft.
\item
  If a player is drafted in an Initial Draft and (i) receives a Required Tender, (ii) does not sign a Player Contract with a Team prior to the Subsequent Draft, and (iii) is not drafted by any Team in such Subsequent Draft, the player will become a Rookie Free Agent immediately upon the conclusion of the Subsequent Draft.
\item
  If a player is drafted by a Team in either an Initial or Subsequent Draft and that Team does not make a Required Tender to such player, the player will become a Rookie Free Agent on the July 16 following such Draft (for a First Round Pick) or on the September 6 following such Draft (for a Second Round Pick).
\item
  A Team may at any time withdraw a Required Tender it has made to a player, provided that the player agrees in writing to the withdrawal. In the event that a Required Tender is withdrawn, the player shall thereupon become a Rookie Free Agent.(f) A Team that holds the exclusive rights to negotiate with and sign a drafted player may at any time renounce such exclusive rights, except that, if the Team has made a Required Tender to the player, a renunciation shall not be permitted during the time the player has been given to accept the Required Tender. In order to renounce its exclusive rights with respect to a drafted player, a Team shall provide the NBA with an express, written statement renouncing such exclusive rights. The NBA shall provide a copy of such statement to the Players Association within three (3) business days following its receipt thereof.
\end{enumerate}

\hypertarget{effect-of-contracts-with-other-professional-teams.}{%
\section{Effect of Contracts with Other Professional Teams.}\label{effect-of-contracts-with-other-professional-teams.}}

If a player is drafted by a Team in either an Initial or Subsequent Draft and, during a period in which he may negotiate and sign a Player Contract with only the Team that drafted him, either (i) is a party to a previously existing player contract with a professional basketball team not in the NBA that covers all or any part of the NBA Season immediately following said Initial or Subsequent Draft, or (ii) signs such a player contract, then the following rules will apply:

\begin{enumerate}
\def\labelenumi{(\alph{enumi})}
\tightlist
\item
  Subject to subsection (b) below, the Team that drafts the player shall retain the exclusive NBA rights to negotiate with and sign him for the period ending one year from the earlier of the following two dates: (i) the date the player notifies such Team that he is available to sign a Player Contract with such Team immediately, provided that such notice will not be effective until the player is under no contractual or other legal impediment to sign with such Team; or (ii) the date of the NBA Draft occurring in the twelve-month period from September 1 to August 30 in which the player notifies such Team of his availability and intention to play in the NBA during the Season immediately following said twelve-month period, provided that such notice will not be effective until the player is under no contractual or other legal impediment to play with such Team for said Season.
\item
  If, by July 1 of any year, the player notifies the Team that has drafted him that by September 1 of such year he will, immediately thereafter or for any future season, be under no contractual or other legal impediment to sign and play with such Team, and provided that on such September 1 the player is in fact under no such contractual or other legal impediment, then, in order to retain the exclusive NBA rights to negotiate with and sign the player as provided in subsection (a), such Team must make a Required Tender to the player by September 5 of such year.
\item
  If the player gives the required notice by July 1 of any year, and the Team that drafted him fails to make a Required Tender by September 5 of such year, the player shall thereupon become a Rookie Free Agent.
\item
  If, during the one-year period of exclusive NBA negotiating rights set forth in subsection (a) above, the player signs a player contract with a professional basketball team not in the NBA and (i) the player has not made a bona fide effort to negotiate a Player Contract with the Team possessing his exclusive NBA rights or (ii) such bona fide effort is made and such Team makes a Required Tender to such player in accordance with subsection (b) above, then such Team shall retain the exclusive NBA rights to negotiate with and sign the player for additional one-year periods as measured in and in accordance with the provisions of subsection (a).
\item
  If, during the one-year period of exclusive NBA negotiating rights set forth in subsection (a) above, the player signs a player contract with a professional basketball team not in the NBA and (i) the player has made a bona fide effort to negotiate a Player Contract with the Team possessing his exclusive NBA rights, and (ii) such Team fails to make a Required Tender to such player in accordance with subsection (b) above, then in no event shall said exclusive NBA rights be retained.
\item
  If, during the one-year period of exclusive NBA negotiating rights set forth in subsection (a) above, the Team makes or has made a Required Tender to the player and the player does not sign a player contract with any professional basketball team, then (i) in the case of a player who was previously drafted in an Initial Draft, the next NBA Draft following such one-year period shall be deemed the Subsequent Draft as to such player, and the rules applicable to a player who is subject to a Subsequent Draft will apply, or (ii) in the case of a player who was previously drafted in a Subsequent Draft, such player shall become a Rookie Free Agent at the end of such one-year period.
\item
  Notice under this Section 4 shall be provided in writing by personal delivery or prepaid certified, registered, or overnight mail sent to the Team's principal address or principal office (as then listed in the NBA's records), to the attention of the Team's general manager. For purposes of this Section 4, a ``professional basketball team'' shall mean any team in any country that pays money or compensation of any kind (in excess of a stipend for living expenses) to a basketball player for rendering services for such team.
\end{enumerate}

\hypertarget{application-to-players-with-remaining-intercollegiate-eligibility.}{%
\section{Application to Players with Remaining Intercollegiate Eligibility.}\label{application-to-players-with-remaining-intercollegiate-eligibility.}}

\begin{enumerate}
\def\labelenumi{(\alph{enumi})}
\tightlist
\item
  A person residing within the United States whose high school class has graduated shall become eligible to be selected in an NBA Draft if he renounces his intercollegiate basketball eligibility by written notice to the NBA at least forty-five (45) days prior to such Draft. If such person is selected in such Draft by a Team, the following rules apply:

  \begin{enumerate}
  \def\labelenumii{(\roman{enumii})}
  \tightlist
  \item
    Subject to Section 5(b) below, if the player does not thereafter play intercollegiate basketball, then the Team that drafted him shall, during the period from the date of such Draft to the date of the Draft in which the player would, absent renunciation of intercollegiate eligibility, first have been eligible to be selected, be the only Team with which the player may negotiate or sign a Player Contract, provided that such Team makes a Required Tender to the player each year. For purposes hereof, the Draft in which such player would, absent renunciation of such intercollegiate eligibility, first have been eligible to be selected, will be deemed the ``Subsequent Draft'' as to that player, and the rules applicable to a player who has been drafted in a Subsequent Draft will apply. If the player, having been selected in a Draft for which he was eligible by virtue of renunciation of intercollegiate eligibility, has not signed a Player Contract with the Team that drafted him in such Draft following a Required Tender by that Team and is not drafted in the Subsequent Draft (as defined in the previous sentence), he shall become a Rookie Free Agent.
  \item
    If the player does thereafter play intercollegiate basketball, then the Team that drafted him shall retain the exclusive NBA rights to negotiate with and sign the player for the period ending one year from the date of the Draft in which the player would, absent renunciation of intercollegiate eligibility, first have been eligible to be selected, provided that such Team makes a Required Tender to the player each year. For purposes hereof, the Draft in which such player would, absent renunciation of intercollegiate eligibility, first have been eligible to be selected, will be deemed the ``Initial Draft'' as to that player. The next NBA Draft shall be deemed the ``Subsequent Draft'' as to that player, and the rules applicable to a player who has been drafted in a Subsequent Draft will apply.
  \end{enumerate}
\item
  A person residing within the United States whose high school class has graduated, who is not yet eligible to be selected in an NBA Draft, and who signs a player contract with a professional basketball team not in the NBA, shall thereupon become eligible to be selected in the next NBA Draft, and if so selected, shall be treated as though he were a player referred to in Section 4 above. For purposes of this subsection, a ``professional basketball team'' shall mean any team in any country that pays money or compensation of any kind (in excess of a stipend for living expenses) to a basketball player for rendering services to such team.
\end{enumerate}

\hypertarget{application-to-foreign-players.}{%
\section{Application to Foreign Players.}\label{application-to-foreign-players.}}

\begin{enumerate}
\def\labelenumi{(\alph{enumi})}
\tightlist
\item
  For purposes of this Section, a ``foreign player'' shall mean any person residing outside of the United States who participates in the game of basketball as an amateur or as a professional.
\item
  A foreign player is eligible to be selected in an NBA Draft held during the calendar year in which such player has his twenty-second (22nd) birthday. Any foreign player who is older than twenty-two (22), and who was not selected in the NBA Draft held during the calendar year of his twenty-second (22nd) birthday, is a Rookie Free Agent.
\item
  Notwithstanding subsection (b) above, a foreign player who is at least eighteen (18) years old and who has not exercised intercollegiate basketball eligibility in the United States shall become eligible to be selected in an NBA Draft held prior to the calendar year in which he has his twenty-second (22nd) birthday if he expresses his desire to become eligible to be selected in the next NBA Draft by written notice to the NBA at least forty-five (45) days prior to such Draft.
\item
  A foreign player who exercises intercollegiate basketball eligibility in the United States during the season prior to an NBA Draft shall be subject to the rules regarding completion or renunciation of collegiate eligibility, as set forth in Section 5 above.
\end{enumerate}

\hypertarget{assignment-of-draft-rights.}{%
\section{Assignment of Draft Rights.}\label{assignment-of-draft-rights.}}

In the event that the exclusive right to negotiate with a player obtained in any NBA Draft is assigned by a Team to another Team, in accordance with NBA procedures, the Team to which such right has been assigned shall have the same, but no greater, right to negotiate with and sign such player as possessed by the Team assigning such right, and such player shall have the same, but no greater, obligation to the Team to which such right has been assigned as he had to the Team assigning such right.

\hypertarget{general.-2}{%
\section{General.}\label{general.-2}}

\begin{enumerate}
\def\labelenumi{(\alph{enumi})}
\tightlist
\item
  The placement of a Rookie on the Armed Services List, or on any of the other lists described in the NBA By-Laws, or on any other list created by the NBA, shall not extend the period of exclusive negotiating rights which a Team has to any Draft Rookie beyond the period specified in this Agreement.
\item
  Nothing contained herein shall prevent the NBA, in accordance with the applicable provisions of the NBA Constitution and By-Laws, from prohibiting or otherwise responding to violations by Teams of the exclusive NBA rights obtained in any NBA Draft, as set forth or referred to in this Article. Other than as specifically agreed to herein, nothing contained in this Agreement shall be deemed to be an agreement by the Players Association to any provision of the NBA Constitution and By-Laws.
\item
  A person who has renounced his intercollegiate eligibility and expressed his desire to become eligible to be selected in the next NBA Draft pursuant to Section 5 or Section 6 above shall be entitled to withdraw from such Draft by providing written notice that is received by the NBA seven days prior to such Draft.
\item
  Any claim by a player that a Contract offered as a Required Tender pursuant to this Article X fails to meet one or more of the criteria for a Required Tender shall be made by written notice to the Team (with copies sent to the NBA and the Players Association), no later than ten (10) days after the receipt of such Contract by the Players Association. Such notice must set forth the specific changes that the player asserts must be made to the offered Contract in order for it to constitute a Required Tender. Upon receipt of such notice, if the requested changes are necessary to satisfy the requirements of a Required Tender, the Team may within five (5) business days offer the player an amended Contract incorporating the requested changes. If the Team offers such an amended Contract, the player shall be precluded from asserting that such Contract does not constitute a timely
  and valid Required Tender.
\item
  For purposes of this Article X, any rights afforded to ``a Team that drafts a player'' shall also be afforded to any Team to which such rights are subsequently assigned.
\end{enumerate}

\hypertarget{free-agency}{%
\chapter{FREE AGENCY}\label{free-agency}}

\hypertarget{general-rules.}{%
\section{General Rules.}\label{general-rules.}}

\begin{enumerate}
\def\labelenumi{(\alph{enumi})}
\tightlist
\item
  Subject to the provisions of Article VII, including, but not limited to, Article VII, Section 6(b): (i) an Unrestricted Free Agent is free at any time after July 1 to negotiate, and free at any time after August 1 to enter into, a Player Contract with any Team; and (ii) a Restricted Free Agent is free at any time after July 1 to negotiate a Player Contract with his Prior Team, to accept a Qualifying Offer from his Prior Team, and to negotiate an Offer Sheet with any Team other than his Prior Team, and is free at any time after August 1 to enter into a Player Contract with his Prior Team or an Offer Sheet with any Team other than his Prior Team.
\item
  No compensation obligation of any kind to another Team shall be applicable to any Free Agent. No right of first refusal of any kind shall be applicable to any Free Agent other than a Restricted Free Agent.
\end{enumerate}

\hypertarget{no-individually-negotiated-right-of-first-refusal.}{%
\section{No Individually-Negotiated Right of First Refusal.}\label{no-individually-negotiated-right-of-first-refusal.}}

\begin{enumerate}
\def\labelenumi{(\alph{enumi})}
\tightlist
\item
  No Player Contract, or any Renegotiation, Extension, or other amendment of a Player Contract, executed after the date of this Agreement, may include any individually negotiated right of first refusal or other limitation on player movement following the last Salary Cap Year covered by such Player Contract.
\item
  No right of first refusal rule, practice, policy, regulation or agreement providing for a right of first refusal shall be applied to any player as a result of that player's entry into a player contract with or the playing with any team in any professional basketball league other than the NBA.
\end{enumerate}

\hypertarget{withholding-services.}{%
\section{Withholding Services.}\label{withholding-services.}}

A player who withholds playing services called for by a Player Contract for more than thirty (30) days after the start of the last Season covered by his Player Contract shall be deemed not to have ``complet{[}ed{]} his Player Contract by rendering the playing services called for thereunder.'' Accordingly, such a player shall not be a Veteran Free Agent and shall not be entitled to negotiate or sign a Player Contract with any other professional basketball team unless and until the Team for which the player last played expressly agrees otherwise.

\hypertarget{fourth-year-option-for-first-round-picks.}{%
\section{Fourth Year Option for First Round Picks.}\label{fourth-year-option-for-first-round-picks.}}

\begin{enumerate}
\def\labelenumi{(\alph{enumi})}
\tightlist
\item
  For First Round Picks who (i) are drafted in the 1998-2004 NBA Drafts (or are drafted in the 2005 NBA Draft, if the NBA exercises its option to extend this Agreement pursuant to Article XXXIX); or (ii) were drafted in the 1995-1997 NBA Drafts and who do not sign a Rookie Scale Contract until after the date of this Agreement: if the Team that drafted the player (or a Team to which the player has been assigned) has exercised its Fourth Year Option (as described in subsection (b) below) to such player on or before the October 31 following the second Season of such player's Rookie Scale Contract with the Team, then such Team shall be deemed to have exercised the Option Year of the player's Rookie Scale Contract, to cover the Season immediately following the first three Seasons of such player's Rookie Scale Contract, on the terms and conditions contained therein in accordance with subsection (b) below.
\item
  The Fourth Year Option Notice shall be a notice to the player that is either personally delivered to the player or his representative or sent by prepaid certified, registered, or overnight mail to the last known address of the player or his representative, signed by the Team, informing the player that the Team has exercised its Option for the player's fourth NBA Season (``Fourth Year Option''). The terms and conditions that apply to the Option Year shall be unchanged from all terms and conditions that applied to the player's third NBA Season (including but not limited to Cash Compensation protection), except that the Salary for the Option Year shall be increased over the Salary for his third Season by the applicable percentage specified in Exhibit B hereto.
\item
  If a Team has not delivered a Fourth Year Option Notice by the specified deadline, the player shall, following his third NBA Season, become an Unrestricted Free Agent.
\end{enumerate}

\hypertarget{qualifying-offers-to-make-certain-players-restricted-free-agents.}{%
\section{Qualifying Offers to Make Certain Players Restricted Free Agents.}\label{qualifying-offers-to-make-certain-players-restricted-free-agents.}}

\begin{enumerate}
\def\labelenumi{(\alph{enumi})}
\tightlist
\item
  From the day following a Season covered by a Fourth Year Option through the immediately following June 30, the player's Team may make a Qualifying Offer to the player covered by such Option. If such a Qualifying Offer is made, then, on the July 1 following the Season covered by the player's Fourth Year Option, the player shall become a Restricted Free Agent, subject to a Right of First Refusal in favor of the Team (``ROFR Team''), as set forth in Section 6 below. If such a Qualifying Offer is not made, then the player shall become an Unrestricted Free Agent on such July 1.
\item
  Beginning with the 1999-2000 Salary Cap Year, any Veteran Free Agent whose first Season in the NBA was the 1998-99 Season or later (other than a Veteran Free Agent whose Fourth Year Option was not exercised), and who will have three (3) or fewer Years of Service as of the June 30 following the end of the last Season covered by his Player Contract, will be a Restricted Free Agent if his Prior Team makes a Qualifying Offer to the player at any time from the day following such Season through the immediately following June 30. If such a Qualifying Offer is made, then, on the July 1 following the last Season covered by the player's Player Contract, the player shall become a Restricted Free Agent, subject to a Right of First Refusal in favor of the Team (``ROFR Team''), as set forth in Section 6 below. If such a Qualifying Offer is not made, then the player shall become an Unrestricted Free Agent on such July 1.
\item
  A Qualifying Offer made to a Restricted Free Agent may be withdrawn at any time up to the following August 7. If the Qualifying Offer is not withdrawn by August 7, it must thereafter remain open until the following October 1. If a Qualifying Offer is withdrawn, the player shall immediately become an Unrestricted Free Agent. However, a player who knows that he has a physical disability that would render him physically unable to perform the playing services required under a Player Contract the following Season may not validly accept a Qualifying Offer received under this Section 5 or Section 6 below, unless the ROFR Team consents after disclosure of such physical disability. Notwithstanding the immediately preceding sentence, a player who knows that he has a physical disability that would render him physically unable to perform the playing services required under a Player Contract the following Season remains subject to the ROFR Team's Right of First Refusal.
\item
  Any claim that a Contract offered as a Qualifying Offer fails to meet one or more of the criteria for a Qualifying Offer shall be made by notice to the Team, in writing, no later than ten (10) days after a copy of the Qualifying Offer was given by the Team or the NBA to the Players Association. Such notice must set forth the specific changes that allegedly must be made to the offered Contract in order for it to constitute a Qualifying Offer. Upon receipt of such notice, if the requested changes are necessary to satisfy the requirements of a Qualifying Offer, the Team may, within five (5) business days, offer the player an amended Contract incorporating the requested changes. If the Team offers such an amended Contract, the player and the Players Association shall be precluded from asserting that such Contract does not constitute a timely and valid Qualifying Offer.
\end{enumerate}

\hypertarget{restricted-free-agency.}{%
\section{Restricted Free Agency.}\label{restricted-free-agency.}}

\begin{enumerate}
\def\labelenumi{(\alph{enumi})}
\item
  If a Restricted Free Agent does not sign an Offer Sheet with any Team by March 1 of the NBA Season for which the Qualifying Offer is made, and does not sign a Player Contract with the ROFR Team before that Season ends, then his ROFR Team may reassert its Right of First Refusal for the following NBA Season by extending another Qualifying Offer (on the same terms as the prior Qualifying Offer) by the next July 1. A ROFR Team may continue to reassert its Right of First Refusal by following the foregoing procedure in each subsequent year in which that Restricted Free Agent does not sign an Offer Sheet with any Team by March 1 of the NBA Season for which the Qualifying Offer is made, and does not sign a Player Contract with the ROFR Team before that Season ends.
\item
  When a Restricted Free Agent receives an offer to sign a Player Contract from a Team (the ``New Team'') other than the ROFR Team, which he desires to accept, he shall give to the ROFR Team a completed certificate substantially in the form of Exhibit G annexed hereto (the ``Offer Sheet''), signed by the Restricted Free Agent and the New Team, which shall have attached to it a Uniform Player Contract separately specifying: (i) the ``Principal Terms'' (as defined in Subsection 6(c) below) of the New Team's offer; and (ii) any non-Principal Terms of the New Team's offer that the ROFR Team is not required to match (as specified in Subsection 6(c) below) but which would be included in the player's Player Contract with the New Team if the ROFR Team does not exercise its Right of First Refusal. The Offer Sheet must be for a Player Contract with a term of more than two NBA seasons (not including any Option Year). In order to extend an Offer Sheet, the New Team must have Room for the player's Player Contract at the time the Offer Sheet is signed. The ROFR Team, upon receipt of the Offer Sheet, may exercise its Right of First Refusal, which shall have the consequences hereinafter set forth below in this Section 6.
\item
  The Principal Terms of an Offer Sheet shall not include any Non-Cash Compensation. In addition, the Principal Terms of an Offer Sheet are only:

  \begin{enumerate}
  \def\labelenumii{(\roman{enumii})}
  \tightlist
  \item
    the fixed and specified Cash Compensation that the New Team will pay or lend to the Restricted Free Agent and/or his designees as a signing bonus, Current Cash Compensation, and/or Deferred Cash Compensation in specified installments on specified dates;
  \item
    Incentive Compensation payable in cash; provided, however, that the only elements of such Incentive Compensation that shall be included in the Principal Terms are the following: (A) bonuses that qualify as Likely Bonuses based upon the performance of the Team extending the Offer Sheet and the ROFR Team; and (B) generally recognized league honors to be agreed upon by the Players Association and the NBA; and
  \item
    Any allowable amendments to the terms contained in the Uniform Player Contract (e.g., Cash Compensation protection, Early Termination Options, assignment bonuses).
  \end{enumerate}
\item
  If, within fifteen (15) days from the date it receives an Offer Sheet, the ROFR Team gives to the Restricted Free Agent a ``First Refusal Exercise Notice'' substantially in the form of Exhibit H annexed hereto, such Restricted Free Agent and the ROFR Team shall be deemed to have entered into a Player Contract containing all the Principal Terms included in the Uniform Player Contract attached to the Offer Sheet.
\item
  If the ROFR Team does not give the First Refusal Exercise Notice within the aforementioned fifteen (15) day period, the player and the New Team shall be deemed to have entered into a Player Contract containing all of the terms and conditions included in the Uniform Player Contract attached to the Offer Sheet.
\item
  After exercising its Right of First Refusal as described in this Section 6, the ROFR Team may not trade the Restricted Free Agent for one (1) year, without the player's consent. Even with the player's consent, for one (1) year, neither the ROFR Team exercising its Right of First Refusal nor any other Team may trade the player to the Team whose Offer Sheet was matched.
\item
  Any Team that exercises its Right of First Refusal may do so subject to the player's passing a physical examination to be conducted by the Team within five (5) days from its exercise of the Right of First Refusal. In the event the player does not pass the physical examination, the ROFR Team may withdraw its First Refusal Exercise Notice within five (5) days of such examination; however, the New Team may not withdraw the previously submitted Offer Sheet. In the event the player, after being given reasonable advance notice, does not submit to a requested physical examination within five (5) days of the exercise of the Right of First Refusal then, until such time as the player submits to the requested physical examination, the ROFR Team may withdraw its First Refusal Exercise Notice, which shall have the effect of invalidating the Offer Sheet and causing the Team that issued the Offer Sheet to be prohibited from signing or acquiring the player for a period of one (1) year from the date the First Refusal Exercise Notice was withdrawn.
\item
  There may be only one Offer Sheet signed by a Restricted Free Agent outstanding at any one time, provided that the Offer Sheet has also been signed by a Team. An Offer Sheet, both before and after it is given to the ROFR Team, may be revoked or withdrawn only upon the written consent of the ROFR Team, the New Team and the Restricted Free Agent. In such event, a Restricted Free Agent shall again be free to negotiate and sign an Offer Sheet with any Team, and any Team shall again be free to negotiate and sign an Offer Sheet with such Restricted Free Agent, subject only to the ROFR Team's renewed Right of First Refusal.
\item
  An expedited arbitration before the System Arbitrator, whose decision shall be final and binding upon all parties, shall be the exclusive method for resolving any disputes concerning this Section. If a dispute arises between the player and either the ROFR Team or the New Team, as the case may be, relating to the contents of an Offer Sheet, and/or whether the binding agreement is between the Restricted Free Agent and the New Team or the Restricted Free Agent and the ROFR Team, such dispute shall immediately be submitted to the System Arbitrator, who shall resolve such dispute within five (5) days.
\item
  A Restricted Free Agent may not give an Offer Sheet to the ROFR Team at any time after the March 1 of the Season for which he has been made a Qualifying Offer.
\item
  On the same day as the giving of an Offer Sheet to the ROFR Team, the ROFR Team shall cause a copy thereof to be given to the NBA, which shall cause a copy thereof to be promptly given to the Players Association. On the same day as the giving of a First Refusal Exercise Notice to the Restricted Free Agent, the Restricted Free Agent shall cause a copy thereof to be given to the New Team, which shall cause a copy thereof to be promptly given to the NBA, which shall cause a copy thereof to be promptly given to the Players Association.
\item
  There may be no consideration of any kind given by one Team to another Team in exchange for a Team's decision to exercise or not to exercise its Right of First Refusal, or in exchange for a Team's decision to submit or not to submit an Offer Sheet to a Restricted Free Agent.
\item
  Any Offer Sheet, First Refusal Exercise Notice or other writing required or permitted to be given under this Section 6, shall be either by personal delivery or by prepaid certified, registered or overnight mail addressed as follows:

  To any NBA Team: addressed to that Team at the principal address of such Team as then listed on the records of the NBA or at the Team's principal office, to the attention of the Team's general manager;

  To the NBA: National Basketball Association, Olympic Tower, 645 Fifth Avenue, New York, NY 10022, Att: General Counsel;

  To the Players Association: National Basketball Players Association, 1700 Broadway, Suite 1400, New York, NY 10019, Att: Counsel.

  To a Restricted Free Agent: to his address listed on the Offer Sheet and, if the Restricted Free Agent designates a representative on the Offer Sheet and lists such representative's address thereof, a copy shall be sent to such representative at such address.
\item
  An Offer Sheet shall be deemed given only when actually received by the ROFR Team. A First Refusal Exercise Notice shall be deemed given when sent by the ROFR Team. A Qualifying Offer shall be deemed given when sent by the ROFR Team. Other writings required or permitted to be given under this Section 6 shall be deemed given only when actually received by the party to whom addressed.
\end{enumerate}

\hypertarget{option-clauses}{%
\chapter{OPTION CLAUSES}\label{option-clauses}}

\hypertarget{team-options.}{%
\section{Team Options.}\label{team-options.}}

Except as provided by Article VIII, Section 1, a Player Contract shall not contain any option in favor of the Team, except an Option (as defined in Article I, Section 1(jj)) that: (i) is specifically negotiated between a Veteran or a Rookie (other than a First Round Pick) and a Team; (ii) authorizes the extension of such Contract for no more than one (1) year beyond the stated term; (iii) is exercisable only once; and (iv) provides that the Salary payable with respect to the option year is no less than 100\% of the Salary payable with respect to the last year of the stated term of such Contract and that all other non-monetary terms applicable in the last year of the stated term of such Contract shall be applicable in the option year.

\hypertarget{player-options.}{%
\section{Player Options.}\label{player-options.}}

A Player Contract shall not contain any option in favor of the player, except:

\begin{enumerate}
\def\labelenumi{(\alph{enumi})}
\tightlist
\item
  an Option that: (i) is specifically negotiated between a Veteran or a Rookie (other than a First Round Pick) and a Team; (ii) authorizes the extension of such Contract for no more than one (1) year beyond the stated term; (iii) is exercisable only once; and (iv) provides that the Salary payable with respect to the option year is no less than 100\% of the Salary payable with respect to the last year of the stated term of such Contract and that all other non-monetary terms applicable in the last year of the stated term of such Contract shall be applicable in the option year; and/or
\item
  an Early Termination Option (or ``ETO'') (as defined in Article I, Section 1(r)), provided that such ETO is exercisable only once and takes effect no earlier than the end of the fifth Season of the Contract. A Contract that does not provide for an ETO when signed may not be amended to provide for an ETO during the original term of the Contract. A Team and a player may enter into an Extension that contains an ETO, provided that such ETO takes effect no earlier than the fifth Season following (i) the Season during which the Extension is signed, or (ii) if the Extension is signed between Seasons, the date on which the Extension is signed. Notwithstanding the foregoing, an ETO contained in an Extension of a Rookie Scale Contract may take effect no earlier than the end of the fifth season of the extended term of the Contract. A Contract (including an Extension) that contains an ETO must specify either the Effective Season of the ETO or that the Effective Season is contingent; provided, however, that the only allowable contingency shall be whether the player or Team meets performance benchmarks designated at the time the Contract is signed.
\end{enumerate}

\hypertarget{exercise-period.}{%
\section{Exercise Period.}\label{exercise-period.}}

Any Option or ETO must be exercised prior to the July 1 immediately prior to the Effective Season covered by such Option.

\hypertarget{option-buy-outs.}{%
\section{Option Buy-Outs.}\label{option-buy-outs.}}

Subject to the rules set forth in Article VII, a Player Contract that contains an Option or an ETO may provide for an Option Buy-Out Amount; provided, however, that in no event may an Option Buy-Out Amount exceed, in the case of an Option, 50\% of the Salary called for in the option year or, in the case of an ETO, 50\% of the Salary in the first Effective Season of the ETO.

\hypertarget{inapplicability-to-prior-player-contracts.}{%
\section{Inapplicability to Prior Player Contracts.}\label{inapplicability-to-prior-player-contracts.}}

The provisions of this Article XII shall not apply to any Player Contract entered into prior to the date of this Agreement.

\hypertarget{circumvention}{%
\chapter{CIRCUMVENTION}\label{circumvention}}

\hypertarget{general-prohibitions.}{%
\section{General Prohibitions.}\label{general-prohibitions.}}

\begin{enumerate}
\def\labelenumi{(\alph{enumi})}
\tightlist
\item
  It is the intention of the parties that the provisions agreed to herein, including, without limitation, those relating to the Salary Cap, the Exceptions to the Salary Cap, the scope of Basketball Related Income (or Core Basketball Revenues), the Escrow System, the Rookie Scale, the Right of First Refusal, the Maximum Player Salary, and free agency, be interpreted so as to preserve the essential benefits achieved by both parties to this Agreement. Neither the Players Association, the NBA, nor any Team (or Team Affiliate) or player (or person or entity acting with authority or apparent authority on behalf of such player), shall enter into any agreement, including, without limitation, any Player Contract (including any Renegotiation, Extension, or amendment of a Player Contract), or undertake any action or transaction, including, without limitation, the assignment or termination of a Player Contract, which is, or which includes any term that is, designed to serve the purpose of defeating or circumventing the intention of the parties as reflected by all of the provisions of this Agreement.
\item
  It shall constitute a violation of subsection (a) above for a Team (or Team Affiliate) to enter into an agreement or understanding with any sponsor or business partner or third party under which such sponsor, business partner or third party pays or agrees to pay compensation for basketball services (even if such compensation is ostensibly designated as being for non-basketball services) to a player under Contract to the Team. Such an agreement with a sponsor or business partner or third party may be inferred where: (i) such compensation from the sponsor or business partner or third party is substantially in excess of the fair market value of any services to be rendered by the player for such sponsor or business partner or third party; and (ii) the Compensation in the Player Contract between the player and the Team is substantially below the fair market value of such Contract.
\end{enumerate}

\hypertarget{no-undisclosed-agreements.}{%
\section{\texorpdfstring{\textbf{No Undisclosed Agreements.}}{No Undisclosed Agreements.}}\label{no-undisclosed-agreements.}}

\begin{enumerate}
\def\labelenumi{(\alph{enumi})}
\tightlist
\item
  At no time shall there be any undisclosed agreements (i.e., undisclosed to the NBA) of any kind, express or implied, oral or written, or promises, undertakings, representations, commitments, inducements, assurances of intent, or understandings of any kind, between a player (or any person or entity acting with authority or apparent authority on behalf of such player) and any Team (or Team Affiliate):

  \begin{enumerate}
  \def\labelenumii{(\roman{enumii})}
  \tightlist
  \item
    involving consideration of any kind to be paid, furnished or made available to the player, or any person or entity controlled by or related to the player, by the Team or Team Affiliate either during the term of the Player Contract or thereafter; or
  \item
    concerning any future Renegotiation, Extension, or amendment of an existing Player Contract, or entry into a new Player Contract.
  \end{enumerate}
\item
  In addition to the foregoing, it shall be a violation of this Section 2 for any Team (or Team Affiliate) or any player (or person or entity acting with authority or apparent authority on behalf of such player) to attempt to enter into or to intentionally solicit any agreement, promise, undertaking, representation, commitment, inducement, assurance of intent or understanding that would be prohibited by Section 2(a) above.
\item
  A violation of Section 2(a) may be proven by direct or circumstantial evidence, including, but not limited to, evidence that a Player Contract or any term or provision thereof cannot rationally be explained in the absence of conduct violative of Section 2(a).
\item
  In any proceeding brought before the System Arbitrator pursuant to this Section 2, no adverse inference shall be drawn against the party initiating such proceeding because that party, when it first suspected or believed that a violation of Section 2 may have occurred, deferred the initiation of such proceeding until it had further reason to believe that such a violation had occurred.
\end{enumerate}

\hypertarget{penalties.}{%
\section{Penalties.}\label{penalties.}}

\begin{enumerate}
\def\labelenumi{(\alph{enumi})}
\tightlist
\item
  Upon a finding of a violation of Section 1 above by the System Arbitrator, but only following the conclusion of any appeal to the Appeals Panel, the Commissioner shall be authorized to:

  \begin{enumerate}
  \def\labelenumii{(\roman{enumii})}
  \tightlist
  \item
    impose a fine of up to \$2,500,000 (50\% of which shall be payable to the NBA, and 50\% of which shall be payable to the NBPA-Selected Charitable Organization) on any Team found to have committed such violation for the first time;
  \item
    impose a fine of up to \$3,000,000 (50\% of which shall be payable to the NBA, and 50\% of which shall be payable to the NBPA-Selected Charitable Organization) on any Team found to have committed such violation for at least the second time;
  \item
    direct the forfeiture of one first round draft pick; and/or
  \item
    void any Player Contract, or any Renegotiation, Extension, or amendment of a Player Contract, between any player and any Team when both the player (or any person or entity acting with authority or apparent authority on behalf of such player) and the Team (or Team Affiliate) are found to have committed such violation.
  \end{enumerate}
\item
  Upon a finding of a violation of Section 2 above by the System Arbitrator, but only following the conclusion of any appeal to the Appeals Panel, the Commissioner shall be authorized to:

  \begin{enumerate}
  \def\labelenumii{(\roman{enumii})}
  \tightlist
  \item
    impose a fine of up to \$3,500,000 (50\% of which shall be payable to the NBA, and 50\% of which shall be payable to the NBPA-Selected Charitable Organization) on any Team found to have committed such violation;
  \item
    direct the forfeiture of draft picks;
  \item
    void any Player Contract, or any Renegotiation, Extension, or amendment of a Player Contract, between any player and any Team when both the player (or any person or entity acting with authority or apparent authority on behalf of such player) and the Team (or Team Affiliate) are found to have committed such violation; and/or
  \item
    suspend for up to one year any Team personnel found to have willfully engaged in such violation.
  \end{enumerate}
\item
  In any proceeding before the System Arbitrator in which it is alleged that a player agent or other person or entity acting with authority or apparent authority on behalf of a player has violated Section 2 of this Article, the System Arbitrator shall make a specific determination with respect to such allegation. If the System Arbitrator finds such violation and such finding, if appealed, is affirmed by the Appeals Panel, the System Arbitrator shall refer such finding to the Players Association's Committee on Agent Regulation. The Committee shall accept as binding and conclusive the finding(s) of the System Arbitrator (or, in the case of an appeal, the Appeals Panel) that a violation of Section 2(a) or 2(b) has occurred and shall consider such finding(s) as establishing a violation of the Players Association's regulations applicable to such person or entity. The Players Association represents that it will impose such discipline as is appropriate under the circumstances on the person or entity found to have violated Section 2 of this Article.
\end{enumerate}

\hypertarget{production-of-tax-materials.}{%
\section{Production of Tax Materials.}\label{production-of-tax-materials.}}

In any proceeding to enforce Section 1 or 2 above, the System Arbitrator shall have the authority, upon good cause shown, to direct any Team, Team Affiliate, or player to produce any tax returns or other relevant tax materials disclosing income figures for the player (non-income figures may be redacted), or disclosing expense figures by the Team or Team Affiliate (non-expense figures may be redacted), which materials shall not be released to the general public or the media and shall be treated as strictly confidential by all parties.

\hypertarget{other-transactions.}{%
\section{Other Transactions.}\label{other-transactions.}}

\begin{enumerate}
\def\labelenumi{(\alph{enumi})}
\item
  After the date of this Agreement, no player may enter into any transaction or arrangement with a Team (other than as expressly set forth in a Uniform Player Contract subject to, and in accordance with, the terms of this Agreement), or, except as provided for in subsection (b) below, with a Team Affiliate, if, pursuant to such transaction or arrangement, the player is to receive compensation or to be provided with an investment opportunity.
\item
  Subject to the provisions of subsections (c), (d), and (e) below, nothing in subsection (a) above shall prevent a player from entering into a transaction or arrangement with a Team Affiliate who or which holds an ownership interest in a Team, or who or which is owned or controlled by a person or entity who or which holds an ownership interest in a Team, provided that (in either or both such cases) such ownership interest does not exceed 10\%, and provided further that such Team Affiliate does not control the Team and is not controlled by the person or entity who or which controls the Team.
\item
  The Terms of any transaction or arrangement permitted by subsection (b) above shall be disclosed in writing to the League Office and the Players Association prior to or within five (5) business days after the entering into of the transaction or arrangement. The NBA shall have ten (10) days after such disclosure in which to challenge the transaction or arrangement, pursuant to the procedures set forth in subsection (e) below, on the ground that (x) the compensation or the investment opportunity provided to the player is greater than a reasonable approximation of the fair market value of the services or other consideration provided by the player in the transaction or arrangement, or that (y) the consideration paid to the player for performing basketball services represents less than a reasonable approximation of the fair market value of such player's basketball services under the Salary Cap system.
\item
  If (i) a Team or Team Affiliate enters into a transaction after the date of this Agreement with a retired player who played for the Team within the five-year (5) period preceding such transaction and the terms of such transaction provide for the retired player to be compensated in excess of \$10,000 or to be provided with an investment opportunity, and if (ii) the compensation the retired player received from the Team when he was a player was substantially below the then fair market value for his services, then the NBA may challenge the transaction, pursuant to the procedures set forth in subsection (e) below, on the ground that: (x) the compensation to the player substantially exceeds the fair market value of the services or other consideration provided by the retired player in the business transaction; or that (y) the amount of the player's investment is not commercially reasonable, given the relative risks and rewards of such investment.
\item
  \begin{enumerate}
  \def\labelenumii{(\roman{enumii})}
  \tightlist
  \item
    Any challenge under this Section 5 shall be filed in writing with a business valuation expert jointly selected by the NBA and the Players Association. The business valuation expert shall conduct a hearing in which the player or retired player, the Team and/or Team Affiliate, the Players Association, and the NBA are afforded the opportunity to appear and participate. The NBA shall have the burden of proof in the proceeding. The business valuation expert may permit discovery of relevant documents necessary to undertake the valuation, and shall render a decision within fifteen (15) days following the conclusion of the hearing. Within ten (10) days of any decision by the business valuation expert, any of the parties may file an appeal with the System Arbitrator, who shall conduct a hearing and render a decision within twenty (20) days of the filing of the appeal. There shall be no right of further appeal to the Appeals Panel.
  \item
    If the NBA prevails in its challenge under this Section 5, the player or retired player and the Team and/or Team Affiliate shall have fifteen (15) days after the date of such determination (or the date of the conclusion of any appeal) in which to renegotiate or terminate: (x) the business transaction, if all parties to the transaction so agree; and (y) in the case of a challenge under Section 5(c), any Player Contract entered into contemporaneously with such transaction, without regard to any time limitations in this Agreement applicable to Renegotiations. If the player or retired player and the Team and/or Team Affiliate do not renegotiate or terminate the business transaction or Player Contract by the conclusion of such fifteen-day (15) period, then, at that time: (xx) in the case of a challenge under Section 5(c), the consideration received by the player, or the value of the investment opportunity (net of any contribution by the player), in each case as determined by the business valuation expert or the System Arbitrator, as the case may be, shall be included in the player's Salary, subject to the Team's Room and other Salary Cap rules, and further subject to any allocation over time that the business valuation expert or System Arbitrator determines is appropriate; and (yy) in the case of a challenge under Section 5(d), the difference between (A) the compensation received by the retired player, or the value of the investment opportunity received by the retired player (net of any contribution by the retired player), and (B) a reasonable estimate of the fair market value of the services or other consideration provided by the retired player, or a reasonable estimate of the fair market value of the investment opportunity, in each case as determined by the business valuation expert or the System Arbitrator, as the case may be, shall be included in the Team's Team Salary, subject to the Team's Room and other Salary Cap rules, and further subject to any allocation over time that the business valuation expert or System Arbitrator determines is appropriate.
  \item
    If the NBA prevails in its challenge under this Section 5, and the player or retired player and the Team and/or Team Affiliate renegotiate or terminate the business transaction or Player Contract, any revised terms of the transaction or Player Contract shall be promptly disclosed to the NBA and the Players Association, and may, at the request of the NBA, be re-subjected to the procedures of this subsection (e).
  \end{enumerate}
\item
  Any information disclosed to the League Office and the Players Association pursuant to the procedures of this Section 5 shall be treated strictly confidential, and shall not be released to the general public or the media.
\end{enumerate}

\hypertarget{other-undertakings.}{%
\section{Other Undertakings.}\label{other-undertakings.}}

\begin{enumerate}
\def\labelenumi{(\alph{enumi})}
\tightlist
\item
  No Team or Team Affiliate shall have a financial arrangement with or offer a financial inducement to any player (not including retired players) not signed to a current Player Contract.
\item
  Prior to the assignment of any Player Contract, the Team from which such Player Contract is to be assigned and the player whose Player Contract is to be assigned shall be required to divest themselves, on terms mutually agreeable to the player and the Team, of any pre-existing financial arrangements between such Team and such player. The foregoing shall not apply to Compensation earned by the player prior to the assignment or to loans.
\item
  Nothing contained in subsections (a) and (b) above shall interfere with a Team's obligation to pay a player Deferred Compensation earned under a prior Player Contract.
\end{enumerate}

\hypertarget{anti-collusion-provisions}{%
\chapter{ANTI-COLLUSION PROVISIONS}\label{anti-collusion-provisions}}

\hypertarget{no-collusion.}{%
\section{\texorpdfstring{\textbf{No Collusion.}}{No Collusion.}}\label{no-collusion.}}

Subject to Section 2 below, no NBA Team, its employees or agents, will enter into any contracts, combinations or conspiracies, express or implied, with the NBA or any other NBA Team, their employees or agents: (a) to negotiate or not to negotiate with any Veteran or Rookie; (b) to submit or not to submit an Offer Sheet to any Restricted Free Agent; (c) to offer or not to offer a Player Contract to any Free Agent; (d) to exercise or not to exercise a Right of First Refusal; or (e) concerning the terms or conditions of employment offered to any Veteran or Rookie.

\hypertarget{non-collusive-conduct.}{%
\section{\texorpdfstring{\textbf{Non-Collusive Conduct.}}{Non-Collusive Conduct.}}\label{non-collusive-conduct.}}

The following conduct shall not be a violation of Section 1 above:

\begin{enumerate}
\def\labelenumi{(\alph{enumi})}
\tightlist
\item
  the formulation and negotiation of collective bargaining proposals;
\item
  agreements between NBA Teams necessary to the assignment of a Player Contract of a Veteran or the assignment of the exclusive negotiating rights to a Draft Rookie, where such assignment is contingent upon (i) the signing by the Veteran of an amendment to an existing Player Contract (including, for example, an Extension) or (ii) the signing by the Draft Rookie of a new Player Contract; provided, however, that if such contingency is fulfilled by the Veteran entering into an amended Player Contract (including, for example, an Extension) or the Draft Rookie entering into a new Player Contract, this subsection shall only apply if the assignment is actually consummated;
\item
  an agreement between NBA Teams concerning the signing of a new Player Contract by a Veteran Free Agent with his Prior Team, where such agreement is necessary for the subsequent assignment of the new Player Contract between the agreeing Teams; provided, however, that this subsection (c) shall apply only if the subsequent assignment is consummated, and only if the agreement and the new Player Contract comply with the provisions of Article VII, Section 8(e).
\item
  the conduct authorized by the terms and conditions of the NBA Draft (as set forth in Article X above); and
\item
  any action taken by the NBA League Office to exclude from the League, suspend or discipline any player for reasons involving gambling, drugs, or the commission of a crime. (This subsection, however, shall not affect any other rights of any player or the Players Association to contest such action.)
\end{enumerate}

\hypertarget{individual-negotiations.}{%
\section{Individual Negotiations.}\label{individual-negotiations.}}

No NBA Team shall fail or refuse to negotiate with, or enter into a Player Contract with, any player who is free to negotiate and sign a Player Contract with any NBA Team, on any of the following grounds:

\begin{enumerate}
\def\labelenumi{(\alph{enumi})}
\tightlist
\item
  that the player has previously been subject to the exclusive negotiating rights obtained by another NBA Team in an NBA Draft; or
\item
  that the player has previously refused or failed to enter into a Player Contract containing an Option; or
\item
  that the player has become a Restricted Free Agent or an Unrestricted Free Agent; or
\item
  that the Player is or has been subject to a Right of First Refusal.The fact that a Team has not negotiated with, made any offers to, or entered into any Player Contracts with players who are free to negotiate and sign Player Contracts with any Team, shall not, by itself, be deemed proof that such Team failed or refused to negotiate with, make any offers to, or enter into any Player Contracts with any players on any of the prohibited grounds referred to in this Section 3.
\end{enumerate}

\hypertarget{league-disclosures.}{%
\section{League Disclosures.}\label{league-disclosures.}}

The NBA League Office shall not knowingly communicate or disclose, directly or indirectly, to any NBA Team that another NBA Team has negotiated with or is negotiating with any Restricted Free Agent, unless and until an Offer Sheet shall have been given to the ROFR Team, or any Free Agent prior to the execution of a Player Contract with that player.

\hypertarget{meet-and-confer.}{%
\section{Meet and Confer.}\label{meet-and-confer.}}

\begin{enumerate}
\def\labelenumi{(\alph{enumi})}
\tightlist
\item
  During the period from July 1 through the first day of the Regular Season of each year covered by this Agreement, the General Counsel of the NBA (or his designee) shall meet once a week with the General or Chief Counsel of the Players Association (or his designee) for the purpose of reviewing (i) each Team's Team Salary summary and the list of Exceptions then currently available to each Team and (ii) any advice rendered during the previous week by either the NBA or the Players Association regarding the interpretation of the Salary Cap.
\item
  During the period from the first day of the Regular Season through June 30 of each year covered by this Agreement:

  \begin{enumerate}
  \def\labelenumii{(\roman{enumii})}
  \tightlist
  \item
    the General Counsel of the NBA (or his designee) and the General or Chief Counsel of the Players Association (or his designee) shall meet at reasonable intervals upon the request of the Players Association for the purpose of reviewing (x) each Team's Team Salary summary and the list of Exceptions then currently available to each Team and (y) any advice regarding the interpretation of the Salary Cap rendered since the last such meeting; and
  \item
    if the NBA informs a Team that a specifically proposed assignment or other player transaction would be inconsistent with or in violation of the terms of this Agreement and/or the limitations of the Salary Cap as interpreted by the NBA, the NBA shall promptly notify the Players Association that such an interpretation has been communicated and the basis for such interpretation. The NBA shall provide such notice to the Players Association within two business days following the communication of such an interpretation to a Team.
  \end{enumerate}
\item
  The substance of any communications under this Section 5 may be referred to or used by the NBA or the Players Association in any proceeding.
\item
  By agreeing to the meeting and notification requirement of subsections (a) and (b) of this Section 5, neither the NBA nor the Players Association intends to waive nor shall be deemed to have waived any attorney-client or other privilege with respect to any communications.
\item
  The provisions of this Agreement are not intended to create any substantive rights in any party, other than as provided for herein. This Agreement may be enforced, and any alleged violations may be remedied, only as provided for herein.
\end{enumerate}

\hypertarget{enforcement-of-anti-collusion-provisions.}{%
\section{Enforcement of Anti-Collusion Provisions.}\label{enforcement-of-anti-collusion-provisions.}}

Any individual player, or the Players Association acting in that player's or any number of players' behalf, may bring an action before the System Arbitrator alleging a violation of Article XIV, Section 1 of this Agreement. Issues of relief and liability shall be determined in the same proceeding (including the amount of damages, pursuant to Section 10 below, if any). The complaining party will bear the burden of demonstrating by a clear preponderance of the evidence that the challenged conduct was in violation of Article XIV, Section 1 of this Agreement and caused economic injury to such player(s).

\hypertarget{satisfaction-of-burden-of-proof.}{%
\section{Satisfaction of Burden of Proof.}\label{satisfaction-of-burden-of-proof.}}

The failure by a Team or Teams to submit Offer Sheets to Restricted Free Agents, or to make offers or sign Contracts for the playing services of Free Agents shall not, by itself or in combination only with evidence about the playing skills of the player(s) not receiving such offers or contracts, satisfy the burden of proof set forth in Section 6 above. However, such evidence may support a finding of a violation of Section 1 above, but only in combination with other evidence that either by itself or in combination with the evidence referred to in the immediately preceding sentence indicates that the challenged conduct was in violation of Section 1 and caused economic injury to such player(s).

\hypertarget{summary-judgment.}{%
\section{Summary Judgment.}\label{summary-judgment.}}

The System Arbitrator may, at any time following the conclusion of any permitted discovery, determine whether or not the complainant's evidence is sufficient to raise a genuine issue of material fact capable of satisfying the standards imposed by Sections 6 and 7 above. If the System Arbitrator determines that complainant's evidence is not so sufficient, he shall dismiss the action.

\hypertarget{remedies.}{%
\section{Remedies.}\label{remedies.}}

In the event that an individual player or players, or the Players Association acting on his or their behalf, successfully proves a violation of Section 1 above, the player or players determined by the System Arbitrator to have suffered economic injury as a result of the violation will have the right:

\begin{enumerate}
\def\labelenumi{(\alph{enumi})}
\tightlist
\item
  to terminate his (or their) existing Player Contract(s) at his (or their) option (however, such termination shall not take effect until the conclusion of a then ongoing NBA Season, if any). Such right of termination shall not arise until the recommendation of the System Arbitrator finding a violation is no longer subject to further appeal and must be exercised by the player within thirty days therefrom. If, at the time the Player Contract is terminated, such player would have been an Unrestricted Free Agent pursuant to the provisions of this Agreement, he shall immediately become an Unrestricted Free Agent. If, at the time the Player Contract is terminated, such player would have been a Restricted Free Agent pursuant to the provisions of this Agreement, such player shall immediately become a Restricted Free Agent upon such termination; however, any such player may choose to reinstate his Player Contract at any time up until September 15 of that year; and
\item
  to recover damages as described in Section 10 below. However, if the player terminates his Player Contract under subsection (a) above and does not reinstate it pursuant thereto, he may not recover damages for the period after such termination takes effect. A player who does not terminate his contract, or who reinstates it pursuant to subsection (a) above, may recover damages for the entire period of his injury.
\end{enumerate}

\hypertarget{calculation-of-damages.}{%
\section{Calculation of Damages.}\label{calculation-of-damages.}}

Upon any finding of a violation of Section 1 above, compensatory damages (i.e., the amount by which any player has been injured as a result of such violation) and non-compensatory damages (i.e., the amount exceeding compensatory damages) shall be awarded as follows:

\begin{enumerate}
\def\labelenumi{(\alph{enumi})}
\tightlist
\item
  Two (2) times the amount of compensatory damages, in the event that all of the Teams found to have violated Section 1 have committed such a violation for the first time. Any Team found to have committed such a violation for the first time shall be jointly and severally liable for two (2) times the amount of compensatory damages.
\item
  Three (3) times the amount of compensatory damages, in the event that any of the Teams found to have violated Section 1 have committed such a violation for the second time. In the event that damages are awarded pursuant to this subsection (b): (i) any Team found to have committed such a violation for the first time shall be jointly and severally liable for two (2) times the amount of compensatory damages; and (ii) any Team found to have committed such a violation for the second time shall be jointly and severally liable for three (3) times the amount of compensatory damages.
\item
  Three (3) times the amount of compensatory damages, plus, for each Team found to have violated Section 1 for at least the third time, three million dollars (\$3,000,000), in the event that any of the Teams found to have violated Section 1 have committed such violation for at least the third time. In the event that damages are awarded pursuant to this subsection (c): (i) any Team found to have committed such a violation for the first time shall be jointly and severally liable for two (2) times the amount of compensatory damages; (ii) any Team found to have committed such a violation for at least the second time shall be jointly and severally liable for three (3) times the amount of compensatory damages; and (iii) any Team found to have committed such a violation for at least the third time shall, in addition, pay a fine of three million dollars (\$3,000,000).
\end{enumerate}

\hypertarget{payment-of-damages.}{%
\section{Payment of Damages.}\label{payment-of-damages.}}

In the event damages are awarded pursuant to Section 10 above, the amount of compensatory damages shall be paid to the injured player or players. The amount of non-compensatory damages, including any fines, shall be paid to the Players Association, which may use it for any purpose other than to pay it to any player who has received compensatory damages, except that any such player may receive some portion of a non-compensatory damage award as part of a proportional distribution to Players Association members.

\hypertarget{effect-of-damages-on-salary-cap.}{%
\section{Effect of Damages on Salary Cap.}\label{effect-of-damages-on-salary-cap.}}

In the event damages are awarded pursuant to Section 10 above, the amount of non-compensatory damages, including any fines, will not be included in any of the computations described in Article VII above. The amount of compensatory damages awarded will be included in such computations.

\hypertarget{contribution.}{%
\section{Contribution.}\label{contribution.}}

Any Team found liable under Section 1 above shall have the right to seek contribution from any other Team found liable for the same violation in a proceeding before the Commissioner who shall determine what contribution, if any, is fair and equitable. The Commissioner's determination with regard to contribution shall be final and binding upon and unappealable by any Team. A contribution determination by the Commissioner may be appealed by the Players Association to the System Arbitrator, except that if such a determination involves fewer than four (4) Teams found to have committed a violation of Section 1 above and allocates damages equally among the Teams found liable, there shall be no appeal to the System Arbitrator. In the event of a contribution determination by the Commissioner, the NBA shall provide the Players Association with the data and information that the Commissioner used or relied upon in making his determination. Any contribution determination appealed by the Players Association to the System Arbitrator shall be upheld unless it is clearly erroneous.

\hypertarget{no-reimbursement.}{%
\section{No Reimbursement.}\label{no-reimbursement.}}

Any damages awarded pursuant to Section 10 above must be paid by the individual Teams found liable and those Teams may not be reimbursed or indemnified by any other Team or the NBA, except to the extent of any award of contribution made pursuant to Section 13 above.

\hypertarget{costs.}{%
\section{Costs.}\label{costs.}}

In any action brought for an alleged violation of Section 1 above, the System Arbitrator shall order the payment of reasonable attorneys' fees by any party found to have brought such an action or to have asserted a defense to such an action without any reasonable basis for asserting such a claim or defense.

\hypertarget{termination-of-agreement.}{%
\section{Termination of Agreement.}\label{termination-of-agreement.}}

The Players Association shall have the right to terminate this Agreement (pursuant to the procedure set forth in Article XXXIX, Section 3 of this Agreement), under the following circumstances:

\begin{enumerate}
\def\labelenumi{(\alph{enumi})}
\tightlist
\item
  Where there has been a finding or findings of one (1) or more instances of a violation of Section 1 above with respect to any one NBA Season which, either individually or in total, involved five (5) or more Teams and caused injury to five (5) or more players; or
\item
  Where there has been a finding or findings of one (1) or more instances of a violation of Section 1 above with respect to any two (2) consecutive NBA Seasons which, either individually or in total, involved seven (7) or more Teams and caused economic injury to seven (7) or more players. For purposes of this Section 16(b), a player found to have been injured by a violation of Section 1 in each of two (2) consecutive Seasons shall be counted as an additional player injured by such a violation for each such NBA Season; or
\item
  Where, in a proceeding brought by the Players Association, it is shown by clear and convincing evidence that ten (10) or more Teams have engaged in a violation or violations of Section 1 above, causing economic injury to one or more NBA players. In order to terminate this Agreement pursuant to this subsection (c) and Article XXXIX, Section 3 of this Agreement:

  \begin{enumerate}
  \def\labelenumii{(\roman{enumii})}
  \tightlist
  \item
    the proceeding must be brought by the Players Association; and
  \item
    the NBA and the System Arbitrator must be informed at the outset of any such proceeding that the Players Association is proceeding under this subsection (c) for the purpose of establishing its entitlement to terminate this Agreement.
  \end{enumerate}
\end{enumerate}

\hypertarget{discovery.}{%
\section{Discovery.}\label{discovery.}}

\begin{enumerate}
\def\labelenumi{(\alph{enumi})}
\tightlist
\item
  In any of the actions described in this Article XIV, the System Arbitrator shall grant reasonable and expedited discovery upon the application of any party where, and to the extent, he or she determines it is reasonable to do so. Such discovery may include the production of documents and the taking of depositions.
\item
  Notwithstanding Section 17(a) above, the Players Association and the NBA shall each have the right to obtain discovery upon request in any three (3) proceedings brought during the term of this Agreement. The scope and extent of such discovery shall be determined by the System Arbitrator.
\end{enumerate}

\hypertarget{time-limits.}{%
\section{Time Limits.}\label{time-limits.}}

Any action under Section 1 above must be brought within 90 days of the time when the player knows or reasonably should have known that he had a claim, or within 90 days of the start of the NBA Season in which a violation of Section 1 is claimed, whichever is later. In the absence of a System Arbitrator, the complaining party shall file such claim for breach of this Agreement pursuant to Section 301 of the Labor Management Relations Act in either the U.S. District Court for the Southern District of New York or the U.S. District Court for the District of New Jersey. Any party alleged to have violated Section 1 shall have the right, prior to any proceedings on the merits, to make an initial motion to dismiss any complaint that does not comply with the timeliness requirement of this Section 18.

\hypertarget{certifications}{%
\chapter{CERTIFICATIONS}\label{certifications}}

\hypertarget{contract-certification.}{%
\section{Contract Certification.}\label{contract-certification.}}

\begin{enumerate}
\def\labelenumi{(\alph{enumi})}
\tightlist
\item
  Every Player Contract (other than a 10-Day Contract), or any Renegotiation, Extension, or amendment of a Player Contract, entered into during the term of this Agreement, shall be accompanied by a certification, sworn to separately by (i) the person who executed the Player Contract on behalf of the Team, (ii) the player, and (iii) any player agent who negotiated the Contract on behalf of the player, under penalties of perjury, that the Player Contract, Renegotiation, Extension, or amendment sets forth all components of a player's Compensation from the Team or any Team Affiliate, and that there are no undisclosed agreements of any kind, express or implied, oral or written, and that there are no promises, undertakings, representations, commitments, inducements, assurances of intent, or understandings of any kind that have not been disclosed to the NBA:

  \begin{enumerate}
  \def\labelenumii{(\roman{enumii})}
  \tightlist
  \item
    involving consideration to be paid, furnished or made available to the player, or any person or entity controlled by or related to the player, by the Team or any Team Affiliate, either during the term of the Player Contract or thereafter; or
  \item
    concerning any future Renegotiation, Extension, or other amendment of the Player Contract or the entry into any new Player Contract.
  \end{enumerate}
\item
  Prior to the assignment of any Player Contract of a player who is in the last Salary Cap Year of the Contract (or the last Salary Cap Year before the player has the right to terminate the Contract), the player, the player's agent, and the Team to which such Contract is to be assigned shall each submit to the NBA a certification, sworn to under penalties of perjury, that there are no undisclosed agreements of any kind, express or implied, oral or written, and that there are no promises, undertakings, representations, commitments, inducements, assurances of intent, or understandings of any kind that have not been disclosed to the NBA concerning any new Player Contract, any future Renegotiation, Extension, or other amendment of the Player Contract that has been assigned, or any matters of compensation (including future compensation) between the player (or the player's agent) and the Team to which the Player Contract has been assigned.
\item
  If a player, within two (2) years after the assignment of such player's Player Contract, enters into a new Player Contract, or any Renegotiation, Extension, or other amendment of the Player Contract that had been assigned, the Team, the player, and the player's agent shall each submit to the NBA a certification, sworn to under penalties of perjury, that, at the time of the assignment, there were no agreements of any kind, express or implied, oral or written, or promises, undertakings, representations, commitments, inducements, assurances of intent, or understandings of any kind, between the player (or the player's agent) and the Team to which the Contract has been assigned that have not been disclosed to the NBA concerning any new Player Contract, or any future Renegotiation, Extension, or other amendment of the Player Contract that had been assigned.
\end{enumerate}

\hypertarget{end-of-season-certification.}{%
\section{End of Season Certification.}\label{end-of-season-certification.}}

\begin{enumerate}
\def\labelenumi{(\alph{enumi})}
\tightlist
\item
  At the conclusion of each NBA Season, a Governor (or Alternate Governor) and the executive primarily responsible for basketball operations on behalf of the Team shall each submit to the NBA a certification, sworn to under penalties of perjury, that the Team has not, to the extent of their knowledge after reasonable inquiry, violated the terms of Article XIV, Section 1, nor received from the NBA League Office any communication disclosing that an NBA Team has negotiated with any Free Agent prior to the execution of a Player Contract with that player. Upon receipt of each such certification, the NBA shall forward a copy of the certification to the Players Association.
\item
  A violation of this Section 2 may be deemed evidence of a violation of Article XIV, Section 1.
\end{enumerate}

\hypertarget{false-certification.}{%
\section{False Certification.}\label{false-certification.}}

Any criminal complaint of perjury filed by the NBA or any Team based upon a certification required pursuant to Section 1 above shall be against the player, the player's agent, and the Team official making such certification.

\hypertarget{mutual-reservation-of-rights}{%
\chapter{MUTUAL RESERVATION OF RIGHTS}\label{mutual-reservation-of-rights}}

Upon the expiration or termination of this Agreement, no person shall be deemed to have waived, by reason of the entry into or effectuation of this Agreement, any other collective bargaining agreement, or any Player Contract, or any of the terms of any of them, or by reason of any practice or course of dealing, their respective rights under law with respect to any issue or their ability to advance any legal argument.

\hypertarget{procedure-with-respect-to-playing-conditions-at-various-facilities}{%
\chapter{PROCEDURE WITH RESPECT TO PLAYING CONDITIONS AT VARIOUS FACILITIES}\label{procedure-with-respect-to-playing-conditions-at-various-facilities}}

\chaptermark{PROCEDURE WITH RESPECT \ldots}

When a new franchise is granted or when an existing franchise moves to another city, the Players Association shall, upon request and within a reasonable period of time, have the right to inspect the facility to be used by such franchise. Similarly, the Players Association shall, upon reasonable notice to the Team(s) involved and the NBA, have the right to inspect the training camp and practice facilities used by such Team(s). If, following such inspection, the Players Association is of the opinion that the playing conditions at such facility will endanger the health and safety of NBA players, it shall promptly notify the Commissioner in writing. Promptly following such notice, representatives of the Players Association and of the Team(s) involved, and the Commissioner or his designee shall meet in an effort to resolve the matter. It is agreed that the failure of the parties to resolve the matter shall not impair the legally binding effect of this Agreement or create any right to (a) unilaterally implement during the term of this Agreement any provision concerning such unresolved matter, (b) lock out, or (c) strike. If no resolution satisfactory to the Players Association, the Team(s) involved and the Commissioner is reached, the issue whether the playing conditions at the facility in question will endanger the health and safety of NBA players will, without interruption of the schedule or training game or practice activities, immediately be submitted to and determined by the Grievance Arbitrator in accordance with the provisions of Article XXXI; provided, however, that the Grievance Arbitrator need not render an award within 24 hours of the conclusion of the hearing, but shall issue his award as expeditiously as possible under the circumstances.

\hypertarget{travel-accommodations-locker-room-facilities-and-parking}{%
\chapter{TRAVEL ACCOMMODATIONS, LOCKER ROOM FACILITIES AND PARKING}\label{travel-accommodations-locker-room-facilities-and-parking}}

\chaptermark{TRAVEL ACCOMMODATIONS, LOCKER ROOM \ldots}

\hypertarget{hotel-arrangements.}{%
\section{Hotel Arrangements.}\label{hotel-arrangements.}}

\begin{enumerate}
\def\labelenumi{(\alph{enumi})}
\tightlist
\item
  Each Team agrees to use its best efforts to make the following arrangements for its players while they are ``on the road'':

  \begin{enumerate}
  \def\labelenumii{(\arabic{enumii})}
  \tightlist
  \item
    to have their baggage picked up by porters;
  \item
    to have them stay in first class hotels; and
  \item
    to have extra-long beds available to them in each hotel.
  \end{enumerate}

  If there is a finding that a Team has committed a willful violation of this subsection (a), the NBA shall impose a \$1,000 fine on such Team.
\item
  When its players are ``on the road,'' each Team shall provide an individual hotel room for each player, except when such rooms are not available.
\end{enumerate}

\hypertarget{first-class-travel.}{%
\section{First Class Travel.}\label{first-class-travel.}}

\begin{enumerate}
\def\labelenumi{(\alph{enumi})}
\tightlist
\item
  Each Team shall provide first class travel accommodations on all trips in excess of one hour, except when such accommodations are not available; provided, however, that a Team's head coach may fly first class in place of a player when eight or more first class seats are provided to players. In the event a Team's head coach flies first class in place of a player, one player, designated by the Players Association, shall be paid the difference between the amount paid by such Team for a first class seat on the flight involved and the cost of the seat purchased for such designated player on that flight.
\item
  If there is a finding that a Team has committed a willful violation of this Section, the NBA shall impose a \$1,000 fine on such Team.
\end{enumerate}

\hypertarget{locker-room-facilities.}{%
\section{Locker Room Facilities.}\label{locker-room-facilities.}}

Each Team agrees to use its best efforts to provide suitable locker room facilities and to stabilize the temperature in locker rooms to make it consistent with the temperature on playing courts.

\hypertarget{parking-facilities.}{%
\section{Parking Facilities.}\label{parking-facilities.}}

Each Team agrees to make parking facilities available to its players without charge in connection with games and practices conducted at the facility regularly used by such Team for home games and/or practices.

\hypertarget{union-security-dues-and-check-off}{%
\chapter{UNION SECURITY, DUES AND CHECK-OFF}\label{union-security-dues-and-check-off}}

\hypertarget{membership.}{%
\section{Membership.}\label{membership.}}

As a condition of employment commencing with the execution of this Agreement, for the duration of this Agreement only, and wherever legal: (a) any active player who is or later becomes a member in good standing of the Players Association must maintain his membership in good standing in the Players Association; and (b) any active player (including a player in the future) who is not a member in good standing of the Players Association must, on the 30th day following the beginning of his employment or the 30th day following the execution of this Agreement, whichever is later, pay, pursuant to Section 2 below or otherwise, to the Players Association an annual service fee in the same amount as any initiation fee and dues required of members of the Players Association.

\hypertarget{check-off.}{%
\section{Check-off.}\label{check-off.}}

Commencing with the execution of this Agreement and for the duration of this Agreement only, each Team, following its receipt of the requisite authorization form, will check-off the initiation fee and annual dues, assessments or service fees, as the case may be, in equal installments from the first four payments made thereafter to the player pursuant to paragraph 3(a) of the Uniform Player Contract or from such lesser number of payments made thereafter as provided for by Exhibit 1 to such Contract, for each player for whom a current check-off authorization has been provided to the Team. The Team will forward the check-off monies to the Players Association within fourteen (14) days of each check-off. If the Team fails to do so, interest at 7\% per annum, payable to the Players Association, shall begin to accrue on such check-off monies upon the conclusion of such 14-day period.

\hypertarget{enforcement.}{%
\section{Enforcement.}\label{enforcement.}}

\begin{enumerate}
\def\labelenumi{(\alph{enumi})}
\tightlist
\item
  Upon written notification to the NBA by the Players Association that a player has not paid any initiation fee, dues or the equivalent service fee in violation of Section 1 above, the NBA will raise the matter for discussion with the player and his Team. If there is no resolution of the matter within seven (7) days, then the Team will, upon the written request of the Players Association, suspend the player without pay, wherever legal. Such suspension will continue until the Players Association has notified the Team in writing that the suspended player has satisfied his obligation as contained in Section 1 above. The parties hereby agree that suspension without pay is adopted as a substitute for and in lieu of discharge as the penalty for a violation of the union security clause of this Agreement and that no player will be discharged for a violation of that clause. A copy of all notices required by this Section 3(a) will be simultaneously mailed to the player involved and the NBA.
\item
  The term ``member in good standing'' as used in this Article XIX applies only to the payment of dues or any initiation fee and not to any other factors involved in union discipline.
\item
  Other than pursuant to Section 2 above, no Team shall pay any initiation fees, dues, or equivalent service fee on behalf of any player.
\end{enumerate}

\hypertarget{no-liability.}{%
\section{No Liability.}\label{no-liability.}}

Neither the NBA nor any Team shall be liable for any salary, bonus, or other monetary or non-monetary claims that result from a player being suspended pursuant to the terms of Section 3 above.

\hypertarget{scheduling}{%
\chapter{SCHEDULING}\label{scheduling}}

\hypertarget{number-of-regular-season-games.}{%
\section{Number of Regular Season Games.}\label{number-of-regular-season-games.}}

Each Team agrees that in no event will it play more than 82 Regular Season games.

\hypertarget{location-of-games.}{%
\section{Location of Games.}\label{location-of-games.}}

During the Regular Season, games between NBA Teams may be played at any location, within or outside the United States and Canada. The NBA shall supervise the arrangements made with respect to games played outside the United States and Canada and the accommodations provided to players participating in such games.

\hypertarget{holidays.}{%
\section{Holidays.}\label{holidays.}}

\begin{enumerate}
\def\labelenumi{(\alph{enumi})}
\tightlist
\item
  No Team will be required to play a game on December 25, unless such game is to be telecast or cablecast nationally.
\item
  Games scheduled to be played on January 1 and Good Friday shall not commence prior to 6 p.m. (local time), unless the Players Association consents thereto, which consent shall not be unreasonably withheld. The Players Association will, upon request, consent to the earlier commencement of two games on each of such dates if such games are to be broadcast or cablecast nationally, and provided that the Teams involved are in the same time zone or otherwise in close geographic proximity.
\item
  Teams at home on December 25 and January 1 (each, a ``Holiday'') may, but shall not be required to, conduct a practice on either (or both) of such Holidays, provided: (i) the Team's players have requested that they practice on the Holiday, as communicated to the Team by the Team's player representative; and (ii) within seven (7) days before or after the Holiday, the Team's players are provided with a ``day off'' --- i.e., the Team will not conduct any practice, including any optional practice, on such date, and the Team will not have a scheduled game on such date.
\item
  Teams shall not depart for an away game or series of away games prior to 3 p.m. (local time) on December 25 or January 1, unless reasonable transportation arrangements for such game or games cannot be made at or after 3 p.m. (local time).
\end{enumerate}

\hypertarget{travel.}{%
\section{Travel.}\label{travel.}}

The NBA and its Teams shall use their best efforts to devise reasonable travel schedules when Regular Season games are played outside the United States and Canada. No Team shall be required to play a scheduled game on the same day that such Team has traveled across two time zones, except in unusual circumstances and unless the Players Association consents thereto, which consent shall not be unreasonably withheld.

\hypertarget{training-camp.}{%
\section{Training Camp.}\label{training-camp.}}

\begin{enumerate}
\def\labelenumi{(\alph{enumi})}
\tightlist
\item
  When players (other than Rookies) are required to report to training camp pursuant to paragraph 2 of the Uniform Player Contract on the twenty-ninth day prior to the first game of an NBA Regular Season, they may only be required to attend a team dinner and team meetings, to participate in photograph sessions, and to submit to a physical examination on that day.
\item
  Notwithstanding the foregoing, if a Veteran Player is under contract to a Team that is scheduled during a particular NBA Season to participate outside North America in a Pre-Season Exhibition game or a Regular Season game during the first week of the Regular Season, the player will not be required to attend training camp during that particular NBA Season earlier than 2 p.m. (local time) on the thirty-second day prior to the first game of an NBA Regular Season. Rookies may be required to attend training camp at an earlier date, but no earlier than ten (10) days prior to the date that Veterans are required to attend.
\end{enumerate}

\hypertarget{nba-all-star-game}{%
\chapter{NBA ALL-STAR GAME}\label{nba-all-star-game}}

\hypertarget{awards.}{%
\section{Awards.}\label{awards.}}

\begin{enumerate}
\def\labelenumi{(\alph{enumi})}
\tightlist
\item
  For their participation in the All-Star Game, during the 1999-2000 and 2000-01 Seasons, players on the winning team shall each receive \$15,000 and players on the losing team shall each receive \$7,500. For their participation in the All-Star Game during the 2001-02, 2002-03, 2003-04, and (if applicable) 2004-05 Seasons, players on the winning team shall each receive \$20,000 and players on the losing team shall each receive \$10,000.
\item
  For their participation in the Rookie All-Star Game, during the 1999-2000 and 2000-01 Seasons, players on the winning team shall each receive \$4,000 and players on the losing team shall each receive \$3,000. For their participation in the Rookie All-Star Game during the 2001-02, 2002-03, 2003-04, and (if applicable) 2004-05 Seasons, players on the winning team shall each receive \$7,500 and players on the losing team shall each receive \$5,000.
\end{enumerate}

\hypertarget{player-guests.}{%
\section{Player Guests.}\label{player-guests.}}

Each player who participates in the All-Star Game, Rookie All-Star Game, or any All-Star Skills Competition may invite a guest, who shall be reimbursed for the cost of round-trip first-class air transportation between the home city of the Team by which such player is employed and the site of the All- Star Game, Rookie All-Star Game or All-Star Skills Competition.

\hypertarget{players-not-participating-in-all-star-activities.}{%
\section{Players Not Participating in All-Star Activities.}\label{players-not-participating-in-all-star-activities.}}

Players not invited to participate in the All-Star Game or the Rookie All-Star Game shall have three days off during the All-Star Game break. Players who do not participate in any All-Star Skills Competition shall have three days off during the All-Star Game break.

\hypertarget{all-star-skills-competitions.}{%
\section{All-Star Skills Competitions.}\label{all-star-skills-competitions.}}

The All-Star Skills Competitions that take place during any All-Star Weekend shall be selected by the NBA; provided, however, that before adding any new event to the All-Star Skills Competitions that take place during any All-Star Weekend (i.e., an event different from any conducted by the NBA during any All-Star Weekend held prior to the 1998-99 Season), the NBA shall obtain the consent of the Players Association, which consent shall not be unreasonably withheld.

\hypertarget{medical-treatment-of-players-and-release-of-medical-information}{%
\chapter{MEDICAL TREATMENT OF PLAYERS AND RELEASE OF MEDICAL INFORMATION}\label{medical-treatment-of-players-and-release-of-medical-information}}

\chaptermark{MEDICAL TREATMENT OF \ldots}

\hypertarget{one-surgeon.}{%
\section{One Surgeon.}\label{one-surgeon.}}

Each Team agrees that a player requiring the care and treatment of an orthopedic surgeon will, so far as practicable, be referred to and treated by one orthopedic surgeon (rather than several).

\hypertarget{committee-of-team-physicians.}{%
\section{Committee of Team Physicians.}\label{committee-of-team-physicians.}}

Representatives designated by the Players Association shall participate in meetings of the committee of Team physicians appointed by the NBA for the purpose of discussing matters related to the medical care and treatment of players.

\hypertarget{public-release-of-medical-information.}{%
\section{Public Release of Medical Information.}\label{public-release-of-medical-information.}}

\begin{enumerate}
\def\labelenumi{(\alph{enumi})}
\tightlist
\item
  Subject to subsection (b) below, each Team may make public medical information relating to the players in its employ, provided that such information relates solely to the reasons why any such player has not been or is not rendering services as a player.
\item
  A player or his immediate family (where appropriate) shall have the right to approve the terms and timing of any public release of medical information relating to any injuries or illnesses suffered by that player that are potentially life- or career-threatening, or that do not arise from the player's participation in NBA games or practices.
\end{enumerate}

\hypertarget{non-team-physicians.}{%
\section{Non-Team Physicians.}\label{non-team-physicians.}}

A player who consults a physician other than such player's Team physician shall give notice of such consultation to his Team's physician and shall authorize and direct such other physician to provide his Team with all information it may request concerning any condition that, in the judgment of the Team's physician, may affect such player's ability to play skilled basketball.

\hypertarget{arbitration-proceedings.}{%
\section{Arbitration Proceedings.}\label{arbitration-proceedings.}}

During the course of any arbitration proceeding, the Grievance Arbitrator may, by appropriate process, require any person (including, but not limited to, a Team and a Team physician, and a player and any physician consulted by such player) to provide to the player or that player's Team, as the case may be, all medical information in the possession of any person relating to the subject matter of the arbitration.

\hypertarget{draftees.}{%
\section{Draftees.}\label{draftees.}}

Prior to any NBA Draft, the NBA and/or its Teams acting jointly may request that persons eligible for such Draft voluntarily submit to the administration of standardized medical or laboratory tests (other than tests for controlled substances), the results of which shall be made available to any Team upon request, but which shall be kept confidential from the public and the media. Any person who submits to the administration of such tests may, prior to such Draft, be requested to submit voluntarily to an examination by the physician(s) for an NBA Team(s), but shall not be requested to undergo any medical or laboratory test administered at the request of the NBA and/or its Teams.

\hypertarget{exhibition-games-and-off-season-games-and-events}{%
\chapter{EXHIBITION GAMES AND OFF-SEASON GAMES AND EVENTS}\label{exhibition-games-and-off-season-games-and-events}}

\chaptermark{EXHIBITION GAMES AND \ldots}

\hypertarget{exhibition-games.}{%
\section{Exhibition Games.}\label{exhibition-games.}}

Subject to the provisions of paragraph 2 of the Uniform Player Contract, players shall be required to participate in Exhibition games between an NBA Team and a non-member of the NBA at any location, within or outside the United States, subject to the following conditions:

\begin{enumerate}
\def\labelenumi{(\alph{enumi})}
\tightlist
\item
  The NBA shall supervise the arrangements made with respect to tournaments or series conducted outside the United States and the accommodations provided to NBA players participating in such foreign tournaments or series.
\item
  The NBA shall use its best efforts to establish an Exhibition game schedule pursuant to which excessive travel will be avoided and reasonable periods of time between games will be allotted.
\item
  In any year in which it is played, the annual Basketball Hall of Fame Exhibition game shall be considered as one of the eight exhibition games prior to the Regular Season referred to in paragraph 2 of the Uniform Player Contract.
\end{enumerate}

\hypertarget{inter-squad-scrimmage.}{%
\section{Inter-squad Scrimmage.}\label{inter-squad-scrimmage.}}

In addition to the Exhibition games provided for by paragraph 2 of the Uniform Player Contract, and during each of the playoff series conducted during the term of this Agreement, any Team that qualifies for the playoffs but is not required to participate in the first round thereof, may arrange and require its players to participate in one inter-squad game or scrimmage with another similarly situated Team, provided that such game or scrimmage is not open to members of the general public.

\hypertarget{off-season-basketball-events.}{%
\section{Off-Season Basketball Events.}\label{off-season-basketball-events.}}

No player may play in any public off-season basketball game, summer league (e.g., Southern California Pro League or the Rocky Mountain Revue Summer League), or public exhibition or competition of basketball skills (e.g., a slam dunk contest or a ``tour'' organized by an NBA business partner) (each a ``Basketball Event'') unless such Basketball Event is approved in writing by the NBA and complies with the terms and conditions of this Section 3, or is conducted by, or with the written authorization of, NBA Properties pursuant to the rights granted to NBA Properties in the Events Agreement. The NBA will consider an off-season Basketball Event for approval only if a request for such approval is submitted in writing to the NBA prior to the April 1 immediately preceding the off-season in which such Basketball Event is to be conducted, and only if the arrangements made with respect to any such off-season Basketball Event are confirmed in writing by the NBA and satisfy the following requirements, in addition to such other reasonable requirements as the NBA may impose:

\begin{enumerate}
\def\labelenumi{(\alph{enumi})}
\tightlist
\item
  General Requirements.

  \begin{enumerate}
  \def\labelenumii{(\roman{enumii})}
  \tightlist
  \item
    The Basketball Event takes place on or after July 1, but in no event later than September 15 (or, in the case of a summer league, September 1);
  \item
    Prior to the Basketball Event, each participating player receives the express written consent of his Team to participate in the Basketball Event;
  \item
    The person(s) organizing the Basketball Event obtains disability insurance for the benefit of each participating player's Team, in an amount acceptable to the NBA (provided, however, that this requirement shall not apply to summer leagues);
  \item
    The Basketball Event or any portion thereof is not televised, broadcast on radio, or otherwise exploited in any media, live or on tape; and
  \item
    The names and logos of the NBA and/or any NBA Team are not used or referred to in connection with the Basketball Event.
  \end{enumerate}
\item
  Additional Charitable Game Requirements. The NBA will consider an off-season charitable game for approval only if, in addition to the general requirements set forth in subsection (a) above, the arrangements made with respect to such charitable game also satisfy the following:

  \begin{enumerate}
  \def\labelenumii{(\roman{enumii})}
  \tightlist
  \item
    The Players Association approves the game in writing (which approval shall not be unreasonably withheld);
  \item
    All proceeds from the sale of tickets to the game are used for charitable purposes;
  \item
    The game is officiated by NBA referees;
  \item
    There is at least one (1) NBA Team trainer and at least one (1) physician present at the game;
  \item
    The name or likeness of an NBA player is not used, or referred to, in advertisements or promotions for or related to the game, except that if the organizer of the game is an NBA player, such organizer-player's name or likeness may be used, or referred to, in such advertisements or promotions;
  \item
    Only current or former professional basketball players participate in the game;
  \item
    The game is not accompanied by an exhibition or competition of basketball skills (such as a slam dunk contest), unless such exhibition or competition has been separately approved in writing by the NBA and the Players Association;
  \item
    Participating players are not paid or compensated (in excess of per diem and actual reasonable expenses incurred in traveling to and participating in the game);
  \item
    The organizer guarantees that the game will produce at least \$100,000 for charity, and, if directed by the NBA and the Players Association, the organizer (or a third party if the organizer itself is a charity) posts security for such amount in a form satisfactory to the NBA and the Players Association which grants the NBA and/or the Players Association the right to sue to recover such amount for the benefit of the charity;
  \item
    The game is played in the United States or Canada; and
  \item
    The organizer agrees to provide the NBA and the Players Association with an audited statement of revenues and expenses, in a form acceptable to the NBA and the Players Association, within sixty (60) days following the game.
  \end{enumerate}
\item
  Additional Summer League Requirements. The NBA will consider an off-season summer league for approval only if, in addition to the general requirements set forth in subsection (a) above, the arrangements made with respect to each league game in which an NBA player participates also satisfy the following:

  \begin{enumerate}
  \def\labelenumii{(\roman{enumii})}
  \tightlist
  \item
    The game is officiated by an official approved by the NBA;
  \item
    Participating players are not paid or compensated (except as provided under Section 4(c) below);
  \item
    NBA players do not participate in an exhibition or competition of basketball skills (such as a slam dunk contest), unless such exhibition or competition has been separately approved in writing by the NBA;
  \item
    There is at least one (1) trainer and at least one (1) physician or other emergency medical personnel present at the game; and
  \item
    The game is played in the United States or Canada.
  \end{enumerate}
\item
  Notwithstanding any other terms of this Section 3, and without limiting the right of the NBA to approve all arrangements of a proposed Basketball Event, the NBA may, in its sole discretion, require, as a condition of its approval of a Basketball Event (other than a charitable game or summer league), that the Basketball Event organizer pay an appropriate fee to the NBA prior to the commencement of the Basketball Event. In addition, nothing in this Section 3 shall limit any rights granted to NBA Properties in the Events Agreement.
\item
  For purposes of this Section 3, any off-season game or competition in which an NBA player participates on behalf of any national basketball federation will not be considered a Basketball Event.
\end{enumerate}

\hypertarget{summer-leagues.}{%
\section{Summer Leagues.}\label{summer-leagues.}}

\begin{enumerate}
\def\labelenumi{(\alph{enumi})}
\tightlist
\item
  No NBA Team may simultaneously enroll more than three (3) Veterans in any summer basketball league during an off-season. For purposes of this subsection, the following players are not considered Veterans:

  \begin{enumerate}
  \def\labelenumii{(\roman{enumii})}
  \tightlist
  \item
    a player who has never signed a Player Contract or whose first Player Contract begins with the Season immediately following the off-season in which such summer league is to be conducted;
  \item
    a player not under contract to an NBA Team at the time he enrolls in such summer league;
  \item
    a player under contract to an NBA Team but who missed twenty-five (25) or more of the Team's games during the Regular Season immediately preceding such off-season; and
  \item
    a player who played for a team in the Continental Basketball Association during all, or any portion, of the Regular Season immediately preceding such off-season.
  \end{enumerate}
\item
  Prior to playing in a summer basketball league, each player who is under contract with a Team for the following Season shall be provided by his Team, and requested to sign, a ``Notice to Veteran Players Concerning Summer Leagues'' in the form attached hereto as Exhibit E, and/or a ``Form Regarding Summer League Participation'' as attached hereto as Exhibit F. Each Team shall promptly forward to the NBA a copy of each signed form, a copy of which shall promptly be forwarded by the NBA to the Players Association.
\item
  The only compensation that may be paid by a Team or any person or entity affiliated with a Team to a player participating in a summer basketball league is a reasonable expense allowance for: (i) meals, but no greater than that set forth in Article III, Section 2; (ii) lodging; and (iii) transportation to and from the player's home to the site of the summer league, and to and from the site of the player's lodging during the summer league to the site of summer-league-related activities. In addition, the Team may purchase a disability insurance policy for the player covering the term of the applicable summer league.
\item
  No Team shall schedule, and no player shall participate in, a summer basketball league that is scheduled to extend, or does in fact extend, past September 1 of any calendar year.
\end{enumerate}

\hypertarget{prohibition-of-no-trade-contracts}{%
\chapter{PROHIBITION OF NO-TRADE CONTRACTS}\label{prohibition-of-no-trade-contracts}}

\hypertarget{general-limitation.}{%
\section{General Limitation.}\label{general-limitation.}}

No Player Contract may contain any prohibition or limitation of an NBA Team's right to assign such Contract to another NBA Team.

\hypertarget{exceptions-to-general-limitation.}{%
\section{Exceptions to General Limitation.}\label{exceptions-to-general-limitation.}}

Notwithstanding the provisions of Section 1 of this Article XXIV:

\begin{enumerate}
\def\labelenumi{(\alph{enumi})}
\tightlist
\item
  A Player Contract may contain (in Exhibit 4 to such Player Contract) a provision entitling a Player to earn, on a one-time basis, upon the first assignment of that Player Contract, a specific sum of money that does not exceed 15\% of the Compensation remaining to be earned by the Player pursuant to that Player Contract at the time of such first assignment; provided, however, that any such payment must be made within thirty (30) days of the date of the first assignment of the Contract.
\item
  A Player Contract entered into by a player who has eight (8) or more Years of Service in the NBA and who has rendered four (4) or more Years of Service for the Team entering into such Contract may contain a prohibition or limitation of such Team's right to assign such Contract to another NBA Team.
\end{enumerate}

\hypertarget{limitation-on-deferred-compensation}{%
\chapter{LIMITATION ON DEFERRED COMPENSATION}\label{limitation-on-deferred-compensation}}

\hypertarget{general-limitation.-1}{%
\section{General Limitation.}\label{general-limitation.-1}}

No NBA Team may sign a Player Contract with any player under which more than 30\% of Compensation is Deferred Compensation. For purposes of this provision only, Deferred Compensation shall mean Deferred Compensation during the period commencing more than two (2) years after the playing term covered by a Player Contract.

\hypertarget{attribution.}{%
\section{Attribution.}\label{attribution.}}

All Player Contracts shall specify the Season(s) to which any Deferred Compensation is attributable.

\hypertarget{rabbi-trusts.}{%
\section{Rabbi Trusts.}\label{rabbi-trusts.}}

\begin{enumerate}
\def\labelenumi{(\alph{enumi})}
\tightlist
\item
  Notwithstanding Section 1, a Player Contract may provide for an annuity to be purchased by the Team that will pay the Player (or his designees) an amount of Deferred Compensation in excess of 30\% of
  Compensation, provided that:

  \begin{enumerate}
  \def\labelenumii{(\roman{enumii})}
  \tightlist
  \item
    The Team and the Player agree with respect to the form and terms of the annuity instrument and the institution from which it is purchased;
  \item
    Ownership of the annuity and all related aspects are structured in a manner that qualifies the arrangement as a tax deferred (``rabbi'') trust, in the opinion of the NBA's tax advisor; and
  \item
    The total cost of the annuity and the schedule of payment of such costs are specified in the Player Contract.
  \end{enumerate}
\item
  Notwithstanding anything to the contrary contained in subsection (a) above:

  \begin{enumerate}
  \def\labelenumii{(\roman{enumii})}
  \tightlist
  \item
    If the institution obligated to make payment under the annuity fails to do so for any reason (other than non-compliance by the Team with the provisions of the annuity contract), the Team shall thereupon become obligated to pay to the Player as Deferred Compensation an amount, if any, equal to the unpaid portion of the purchase price of the annuity for which the Team remains obligated; and
  \item
    If the creditors of the Team and not the Player receive payments under the annuity, the Team shall thereupon become obligated to pay to the Player as Deferred Compensation an amount equal to the full purchase price of the annuity.
  \end{enumerate}
\end{enumerate}

\hypertarget{team-rules}{%
\chapter{TEAM RULES}\label{team-rules}}

Each Team may maintain or establish rules with which its players shall comply at all times, whether on or off the playing floor; provided, however, that such rules are in writing, are reasonable, and do not violate the provisions of this Agreement or the Uniform Player Contract.

\hypertarget{right-of-set-off}{%
\chapter{RIGHT OF SET-OFF}\label{right-of-set-off}}

\hypertarget{set-off-calculation.}{%
\section{Set-Off Calculation.}\label{set-off-calculation.}}

\begin{enumerate}
\def\labelenumi{(\alph{enumi})}
\tightlist
\item
  When a Team terminates a Player Contract in circumstances where such Team, following the termination, continues to be liable for the Compensation called for by such Contract (including any Deferred Compensation), the Team's liability for such Compensation shall be reduced pro rata by any compensation earned by the player (for services as a player) from any professional basketball team or teams during the period covered by the terminated Contract (including, but not limited to, compensation earned but not paid during such period); provided, however, that such reduction in liability shall not be more than 50\% of that portion of the player's total compensation from the new team or teams which is in excess of the lesser of (i) the Minimum Player Salary applicable to such player at the time of termination, or (ii) the Minimum Player Salary for a player with two (2) Years of Service at the time of termination.
\item
  For the purposes of this Article, a ``professional basketball team'' shall mean any team in any country that pays money or compensation of any kind to a basketball player for rendering services to such team (other than a reasonable stipend limited to basic living expenses). For purposes of this Article, ``compensation'' earned by a player shall include all forms of Cash Compensation and Non-Cash Compensation other than benefits comparable to the type of benefits (e.g., medical and dental insurance) provided to an NBA player in accordance with Article IV above, travel and moving expenses, and any car and housing provided temporarily by a professional basketball team to the player during the period of time for which the player renders services to such team.
\end{enumerate}

\hypertarget{successive-terminations.}{%
\section{Successive Terminations.}\label{successive-terminations.}}

In the event of successive terminations by NBA Teams of Player Contracts involving the same player, the Team first to terminate shall be entitled to the right of set-off provided for by this Article until its compensation liability has been eliminated in its entirety, and the right of set-off shall then pass in order to the Team(s) terminating any subsequent Contract(s).

\hypertarget{deferred-compensation.}{%
\section{Deferred Compensation.}\label{deferred-compensation.}}

In calculating the amount of set-off to which a Team may be entitled pursuant to this Article, Deferred Compensation payable to a player for or with respect to a period covered by the terminated Contract shall be discounted on an annual basis by a percentage equal to the prime rate as set by Citibank, N.A. and in effect at the time the agreement providing for such Deferred Compensation was made.

\hypertarget{waiver-of-set-off-right.}{%
\section{Waiver of Set-Off Right.}\label{waiver-of-set-off-right.}}

A Team and a player may agree in an amendment to an already-existing Player Contract to modify or eliminate the set-off right provided in this Article, but only pursuant to and to the extent allowed by Article II, Section 3(n).

\hypertarget{broadcast-or-telecast-rights}{%
\chapter{BROADCAST OR TELECAST RIGHTS}\label{broadcast-or-telecast-rights}}

\hypertarget{league-rights.}{%
\section{League Rights.}\label{league-rights.}}

During the term of this Agreement, the Players Association agrees that the NBA, Properties, Market Extension, and NBA Teams have the right to use, distribute, or license any performance by the players, under this Agreement or the Uniform Player Contract, for any form of broadcast or telecast, including over-the-air television, cable television, pay television, direct broadcast satellite television, and any form of cassette, cartridge, or disk system, or other means of distribution known or unknown.

\hypertarget{no-suit.}{%
\section{No Suit.}\label{no-suit.}}

The Players Association, for itself and present and future NBA players, covenants not to sue (or finance any suit against) the NBA, Properties, Market Extension, and any NBA Team, or, any of their respective past, present and future owners (direct and indirect) acting in their capacity as owners of any of the foregoing entities, officers, directors, trustees, employees, agents, attorneys, licensees, successors, heirs, administrators, executors and assigns, with respect to the use, distribution, or license, for any form of broadcast or telecast, including over-the-air television, cable television, pay television, or direct broadcast satellite television, and any form of cassette, cartridge, or disk system, or other means of distribution known or unknown, of any performances by any player rendered under this Agreement or prior collective bargaining agreements, or under Player Contracts made pursuant thereto, during any period up to and including the day following the last Playoff game of the 2003-04 NBA Season (or, if the NBA exercises its option to extend this Agreement, up to and including the day following the last Playoff game of the 2004-05 NBA Season).

\hypertarget{reservation-of-rights.}{%
\section{Reservation of Rights.}\label{reservation-of-rights.}}

The Players Association expressly reserves its rights to bargain collectively on the subject described in Section 1 above at the expiration of this Agreement. Such reservation shall not, however, preclude the NBA from contending that the subject described in Section 1 above is not a mandatory subject of collective bargaining.

\hypertarget{miscellaneous}{%
\chapter{MISCELLANEOUS}\label{miscellaneous}}

\hypertarget{active-roster-size.}{%
\section{Active Roster Size.}\label{active-roster-size.}}

Each Team agrees to have twelve (12) players on its Active List and to have a minimum of eight (8) players on the bench for all Regular Season games. Notwithstanding the foregoing, any Team may from time to time as appropriate, but for no more than two (2) consecutive weeks at a time during the Regular Season, have eleven (11) players on its Active List.

\hypertarget{playing-rules-and-officiating.}{%
\section{Playing Rules and Officiating.}\label{playing-rules-and-officiating.}}

\begin{enumerate}
\def\labelenumi{(\alph{enumi})}
\tightlist
\item
  One representative of the Players Association shall be permitted to attend the meetings of and have a vote on the NBA Competition Committee with respect to issues relating to the Official Playing Rules and Officiating.
\item
  The Players Association may on behalf of the players annually submit to the Commissioner one (1) written critique of referees, without reference to any individual referee.
\end{enumerate}

\hypertarget{playoffs.}{%
\section{Playoffs.}\label{playoffs.}}

The number of Teams participating in the playoffs shall equal sixteen (16). Notwithstanding the foregoing, the NBA shall have the right to increase the number of Teams participating in the playoffs.

\hypertarget{implementation-of-agreement.}{%
\section{Implementation of Agreement.}\label{implementation-of-agreement.}}

\begin{enumerate}
\def\labelenumi{(\alph{enumi})}
\tightlist
\item
  The NBA and the Players Association will use their respective best efforts to have NBA Teams and NBA players comply with the terms and provisions of this Agreement.
\item
  The NBA and the Players Association shall use their respective best efforts and take all reasonable steps to cooperate to defend the enforceability of this Agreement against any challenge thereto.
\end{enumerate}

\hypertarget{release-for-fighting.}{%
\section{Release for Fighting.}\label{release-for-fighting.}}

Each NBA Team (hereinafter ``such Team'') hereby releases and waives every claim it may have against any player employed by other NBA Teams for injuries sustained by any player in the employ of such Team which arise out of or in connection with any fighting or other form of violent and/or unsportsmanlike conduct during the course of any Exhibition, Regular Season, and/or Playoff game.

\hypertarget{game-tickets-for-retired-players.}{%
\section{Game Tickets for Retired Players.}\label{game-tickets-for-retired-players.}}

Each Team agrees to provide retired players with three (3) or more years of NBA service with the opportunity to purchase two (2) tickets at box office prices to its NBA home games, and to hold such tickets for such players, provided tickets are available and the retired players provide the Team with forty-eight (48) hours advance notice of their desire for such tickets.

\hypertarget{limitation-on-player-ownership.}{%
\section{Limitation on Player Ownership.}\label{limitation-on-player-ownership.}}

During the term of this Agreement, no NBA player may acquire or hold a direct or indirect interest in the ownership of any NBA Team, provided, however, that any player may own shares of any publicly traded company that directly or indirectly owns an NBA Team.

\hypertarget{nondisclosure.}{%
\section{Nondisclosure.}\label{nondisclosure.}}

The parties agree that (a) the economic terms of any individual Uniform Player Contract entered into by a Team and a player and (b) any information contained in or disclosed to the Players Association in connection with the Audit Reports shall not be disclosed to the media by (i) the NBA, its teams, or their respective employees, or (ii) the Players Association, NBA players, or their respective employees, agents, or representatives.

\hypertarget{game-tickets.}{%
\section{Game Tickets.}\label{game-tickets.}}

\begin{enumerate}
\def\labelenumi{(\alph{enumi})}
\tightlist
\item
  In the event that a Team provides home-game tickets to its players, seat locations must be allocated to players based on seniority, with the most senior players (based on years of NBA service) receiving the most favorable seat locations.
\item
  NBA Teams shall provide four (4) tickets to authorized representatives of the Players Association to any home game at box office prices, provided notice of such request is given at least forty-eight (48) hours before the game.
\end{enumerate}

\hypertarget{additional-canadian-provisions.}{%
\section{Additional Canadian Provisions.}\label{additional-canadian-provisions.}}

\begin{enumerate}
\def\labelenumi{(\alph{enumi})}
\item
  The bases upon which a player may be disciplined or discharged or a Player Contract terminated, as set forth in this Agreement and/or in the Uniform Player Contract, shall constitute just and reasonable cause within the meaning of any applicable Canadian statute (federal or provincial).
\item
  During the term of this Agreement, the NBA and Players Association shall consult regularly about issues relating to the workplace which affect the parties or any player bound by this Agreement.
\item
  \begin{enumerate}
  \def\labelenumii{(\roman{enumii})}
  \tightlist
  \item
    If and to the extent Sections 48 and 49 of the Ontario Labour Relations Act are or may be found applicable to this Agreement, the parties agree that the provisions thereof shall apply only to disputes between the Toronto Raptors and players for the Toronto Raptors.
  \item
    If and to the extent Section 84(2) of the British Columbia Labour Relations Code is or may be found applicable to this Agreement, the parties agree that the provisions thereof shall apply only to disputes between the Vancouver Grizzlies and players for the Vancouver Grizzlies.
  \end{enumerate}
\item
  The parties acknowledge and agree that a player employed by an NBA Team pursuant to the provisions of a Uniform Player Contract, a 10-Day Contract, or a Rest-of-Season Contract is and/or shall be deemed to be an ``employee employed for a definite term or task'' within the meaning of Section 57(10)(a) of the Ontario Employment Standards Act and an ``employee employed for a definite term'' within the meaning of Section 65(1)(b) of the British Columbia Employment Standards Act, so as to render inapplicable to NBA players the provisions of Section 57 of the Ontario Employment Standards Act and Sections 63 and 64 of the British Columbia Employment Standards Act.
\item
  The parties acknowledge and agree that the severance benefits provided to players pursuant to this Agreement (including the provisions of Player Contracts that provide, in certain circumstances, for the continued payment of Salary to a player following the termination of a Player Contract) constitute and/or shall be deemed to constitute a ``settlement of all severance pay claims'' within the meaning of Section 58(18) of the Ontario Employment Standards Act and/or ``a contractual severance pay scheme under which payments for loss of employment based upon length of service are provided'' within the meaning of Section 58(7)(b) of the Ontario Employment Standards Act, so as to render inapplicable to NBA players the provisions of such Section 58 of such Act.
\item
  Upon the NBA's request, the Players Association shall cooperate with the NBA in a reasonable manner in connection with any effort the NBA may make to seek an exemption from any Canadian (federal or provincial) law or regulation affecting the employment relationship that is inconsistent with the provisions of this Agreement or any other agreement between the Players Association and the NBA (or NBA Properties) or between any player and any NBA Team.
\item
  The parties hereby specifically exclude the operation of subsections (2) and (3) of Section 50 of the British Columbia Labour Relations Code.
\item
  All players employed by NBA Teams shall be paid in U.S. dollars, regardless of where such Teams are located.
\end{enumerate}

\hypertarget{mandatory-programs.-1}{%
\section{Mandatory Programs.}\label{mandatory-programs.-1}}

NBA players shall be required to attend and participate in educational programs designated as ``Mandatory Programs'' by the NBA and the Players Association. Such Mandatory Programs, which shall be jointly administered by the NBA and the Players Association, shall include a rookie transition program (for rookies only), a substance abuse awareness program, a gambling awareness program, an HIV awareness program, and such other programs as the NBA and the Players Association shall jointly designate as mandatory.

\hypertarget{no-strike-and-no-lockout-provisions-and-other-undertakings}{%
\chapter{NO-STRIKE AND NO-LOCKOUT PROVISIONS AND OTHER UNDERTAKINGS}\label{no-strike-and-no-lockout-provisions-and-other-undertakings}}

\chaptermark{NO-STRIKE AND NO-LOCKOUT \ldots}

\hypertarget{no-strike.}{%
\section{No Strike.}\label{no-strike.}}

During the term of this Agreement, neither the Players Association nor its members shall engage in any strikes, cessations or stoppages of work, or any other similar interference with the operations of the NBA or any of its Teams.

\hypertarget{no-lockout.}{%
\section{No Lockout.}\label{no-lockout.}}

During the term of this Agreement, neither the NBA nor its Teams shall engage in any lockouts, cessations or stoppages of work or any other similar interference with the employment of NBA players by NBA Teams.

\hypertarget{no-breach-of-player-contracts.}{%
\section{No Breach of Player Contracts.}\label{no-breach-of-player-contracts.}}

The Players Association agrees that it will not engage in any concerted activities to breach, induce the breach of, or threaten to breach or induce the breach of, any Player Contract.

\hypertarget{best-efforts-of-players-association.}{%
\section{Best Efforts of Players Association.}\label{best-efforts-of-players-association.}}

The Players Association will use its best efforts to prevent each player from rendering, or threatening to render, services as a professional basketball player for another professional basketball team during the term of a Player Contract between such player and the Team for which he plays (except as said Player Contract may be assigned, sold, or transferred in accordance with the provisions thereof); to prevent each player from refusing, or threatening to refuse, to participate in any scheduled Exhibition game, Regular Season game, All-Star Game, Rookie All-Star Game, All-Star Skills Competition, or Playoff game; to prevent each player from otherwise breaching, or threatening to breach, such Player Contract; and to prevent each player from making any demand upon the NBA or any of its Teams, including, but not limited to, a demand (accompanied by threats that the player will render services as a professional basketball player for another professional basketball team during the term of such Player Contract) that such Player Contract be renegotiated during the term thereof; provided, however, that this provision is not intended to prevent any player from entering into negotiations with a Team, in accordance with Article VII, with respect to the compensation to be paid to said player for the Season(s) following the last playing Season covered by any Player Contract, or renewal or extension thereof.

\hypertarget{players-threat-to-withhold-services.}{%
\section{Player's Threat to Withhold Services.}\label{players-threat-to-withhold-services.}}

The NBA and the Players Association agree that a player who publicly demands a renegotiation of his Player Contract, and who threatens to withhold the services he has agreed to render under such Player Contract or to perform at a level below his full capabilities unless such renegotiation takes place, shall be considered to have engaged in conduct impairing the faithful and thorough discharge of the duties incumbent upon the player within the meaning of paragraph 5 of the Uniform Player Contract.

\hypertarget{no-negotiations-with-other-teams-while-under-contract.}{%
\section{No Negotiations with Other Teams While Under Contract.}\label{no-negotiations-with-other-teams-while-under-contract.}}

Except as permitted in accordance with Article XI (Free Agency), no player who is a party to a Player Contract with a Team shall, during the term of such Contract (including any permissible option year), enter into negotiations with another Team.

\hypertarget{grievance-and-arbitration-procedure}{%
\chapter{GRIEVANCE AND ARBITRATION PROCEDURE}\label{grievance-and-arbitration-procedure}}

\hypertarget{scope.}{%
\section{Scope.}\label{scope.}}

\begin{enumerate}
\def\labelenumi{(\alph{enumi})}
\tightlist
\item
  Any dispute (such dispute hereinafter being referred to as a ``Grievance'') involving the interpretation or application of, or compliance with, the provisions of this Agreement or the provisions of a Player Contract (except as provided in paragraph 9 of a Uniform Player Contract), including a dispute concerning the validity of a Player Contract, shall be resolved exclusively by the Grievance Arbitrator in accordance with the procedures set forth in this Article; provided, however, that disputes arising under Articles VII, VIII, X, XI, XII, XIII, XIV, XV, XVI, XXXVII, XXXIX, and XL shall (except as otherwise specifically provided by Article VII, Section 3(d)(5) above) be determined by the System Arbitrator provided for in Article XXXII.
\item
  The Grievance Arbitrator shall also have jurisdiction over disputes involving player discipline to the extent set forth in Section 8 below and over disputes concerning the disposition of funds deposited in accordance with Section 9 below to the extent set forth in that Section.
\end{enumerate}

\hypertarget{initiation.}{%
\section{Initiation.}\label{initiation.}}

\begin{enumerate}
\def\labelenumi{(\alph{enumi})}
\tightlist
\item
  Grievances may be initiated, as set forth below, by a player, a Team, the NBA, or the Players Association, except that the Players Association may not initiate a Grievance involving player discipline without the approval of the player(s) concerned.
\item
  No party may initiate a Grievance until and unless it has first discussed the matter with the party or parties against whom the Grievance is to be initiated in an attempt to settle it.
\item
  A Grievance must be initiated within thirty (30) days from the date of the occurrence upon which the Grievance is based, or within thirty (30) days from the date upon which the facts of the matter became known or reasonably should have become known to the party initiating the Grievance, whichever is later.
\item
  Subject to the provisions of Sections 2(a)-(c) above, (i) a player or the Players Association may initiate a Grievance by filing written notice thereof with a Team and furnishing a copy of such notice to the NBA; (ii) a Team may initiate a Grievance by filing written notice thereof with the Players Association and furnishing copies of such notice to the player(s) involved and to the NBA; and (iii) the NBA may initiate a Grievance by filing written notice thereof with the Players Association and furnishing copies of such notice to the player(s) and Team(s) involved.
\end{enumerate}

\hypertarget{hearings.}{%
\section{Hearings.}\label{hearings.}}

\begin{enumerate}
\def\labelenumi{(\alph{enumi})}
\tightlist
\item
  Upon at least thirty (30) days' written notice to the other side, the NBA and the Players Association may arrange to have a hearing scheduled on a date that is mutually convenient to the parties to the dispute, the NBA, the Players Association, and the Grievance Arbitrator; provided, however, that if the NBA and the Players Association cannot agree on a hearing date, the Grievance Arbitrator shall set a reasonable hearing date that follows the expiration of the 30-day notice period. Only the NBA and the Players Association may schedule or postpone hearings with the Grievance Arbitrator.
\item
  Notwithstanding the provisions of Section 3(a), during each Salary Cap Year covered by this Agreement, the Players Association and the NBA shall each have the right, upon a showing of need, to have two (2) Grievances scheduled for hearing on or after the tenth day following service of the notice provided for by Section 3(a). The provisions of this subsection (b) shall not limit or otherwise affect the rights of the NBA or the Players Association pursuant to Section 12 of this Article XXXI.
\item
  If a Grievance is scheduled for hearing under Section 3(a) or 3(b) above and the hearing date is thereafter postponed by either the NBA or the Players Association, the postponement fee (if any) of the Arbitrator will be borne by the postponing party unless that party objects and the Arbitrator finds that the postponement was for good cause. Should good cause be found, the parties will share any postponement costs equally.
\item
  In any Grievance matter, neither the NBA nor the Players Association may request or be granted more than one postponement of a hearing previously scheduled under Sections 3(a) or (b) above. If, after postponing such a hearing, a party fails to attend a second scheduled hearing in the same Grievance matter, the Grievance shall be resolved against that party.
\item
  If a hearing on a Grievance initiated after the execution of this Agreement is not scheduled to take place within one (1) year of the filing of the Grievance, or, in the case of a postponement of the first hearing date by the opposing party, if a second hearing in that Grievance is not scheduled to take place within two (2) years of the filing of the Grievance, the Grievance shall be dismissed with prejudice.
\item
  Any Grievance filed prior to the execution of this Agreement shall be dismissed with prejudice unless the NBA or the Players Association, as the case may be, schedules a hearing under Sections 3(a) or (b) above to take place within six (6) months of the execution of this Agreement.
\item
  For purposes of computing time under this Section 3, the time shall be tolled during any period when there is no Grievance Arbitrator or when the grieving party has been unable to schedule a hearing (after making efforts to do so) because the Grievance Arbitrator is unavailable.
\item
  Hearings before the Grievance Arbitrator shall be held in New York (alternating between the NBA and Players Association offices). All such hearings shall be conducted in accordance with the Labor Arbitration Rules of the American Arbitration Association; provided, however, that in the event of any conflict between such Rules and the provisions of this Agreement, the provisions of this Agreement shall control.
\end{enumerate}

\hypertarget{procedure.}{%
\section{Procedure.}\label{procedure.}}

\begin{enumerate}
\def\labelenumi{(\alph{enumi})}
\tightlist
\item
  Not later than seven (7) days prior to the hearing, the parties shall submit to the Grievance Arbitrator a joint statement of the issue(s) in dispute. If the parties cannot agree on such a joint statement, they shall each submit separate statements setting forth the disputed issue(s).
\item
  Not later than three (3) business days prior to the hearing, the parties shall exchange witness lists, relevant documents and other evidentiary materials, and citations of legal authorities that each side intends to rely on in its affirmative case. Absent a showing of good cause, no party may proffer, refer to, or rely on the testimony of any witness, any document or other evidentiary material in its affirmative case that has not been identified to the other side as required by this subsection.
\item
  The Grievance Arbitrator shall grant the request of any party to file a pre-hearing and/or post-hearing brief, unless the opposing party demonstrates that the filing of briefs is unreasonable in the circumstances.
\end{enumerate}

\hypertarget{arbitrators-decision-and-award.}{%
\section{Arbitrator's Decision and Award.}\label{arbitrators-decision-and-award.}}

\begin{enumerate}
\def\labelenumi{(\alph{enumi})}
\tightlist
\item
  Except as set forth in Section 12 below, the Grievance Arbitrator shall render an Award as soon as practicable. That Award may be accompanied by a written opinion, or the written opinion may follow within a reasonable time thereafter. In no event shall the Award or any written opinion be issued more than thirty (30) days following the conclusion of a Grievance hearing (or the submission of post-hearing briefs where applicable). The Award shall constitute full, final and complete disposition of the Grievance, and shall be binding upon the player(s) and Team(s) involved and the parties to this Agreement.
\item
  The Grievance Arbitrator shall have jurisdiction and authority only to: (i) interpret, apply, or determine compliance with the provisions of this Agreement; (ii) interpret, apply or determine compliance with the provisions of Player Contracts; (iii) determine the validity of Player Contracts pursuant to Section 1 of this Article; (iv) award damages in connection with a proceeding provided for in Section 11 below; (v) award declaratory relief in connection with a proceeding initiated by a Team to determine whether such Team may properly terminate a Player Contract pursuant to paragraph 16(a) of such Contract, and what, if any, liability such Team would incur as a result of such termination; and (vi) resolve disputes arising under Article VII, Section 3(d)(5), Article XXII, Section 5, Article XXVI, and Article XXXIII in the manner set forth therein. The Grievance Arbitrator shall not have jurisdiction or authority to add to, detract from, or alter in any way the provisions of this Agreement (including the provisions of this subsection) or any Player Contract. Nor, in the absence of agreement by the NBA and the Players Association, shall the Grievance Arbitrator have jurisdiction or authority to resolve questions of substantive arbitrability.
\item
  In any Grievance that involves an action taken by the Commissioner (or his designee) concerning (i) the preservation of the integrity of, or the maintenance of public confidence in, the game of basketball, and (ii) a fine and/or suspension that results in a financial impact to the player of more than \$25,000, the Grievance Arbitrator shall apply an ``arbitrary and capricious'' standard of review.
\end{enumerate}

\hypertarget{grievance-arbitrator.}{%
\section{Grievance Arbitrator.}\label{grievance-arbitrator.}}

The parties to this Agreement shall agree upon the appointment of a new Grievance Arbitrator, who shall serve for the duration of this Agreement; provided, however, that as of September 1, 1999, and as of each successive September 1, either of the parties to this Agreement may discharge the Grievance Arbitrator by serving thirty (30) days' prior written notice upon him and upon the other party to this Agreement; and provided, further, that as of the April 30 of the last Season covered by this Agreement, either of the parties may discharge the Grievance Arbitrator by serving thirty (30) days' prior written notice upon him and upon the other party to this Agreement. If the Grievance Arbitrator is discharged, the parties shall thereupon either agree upon a successor Grievance Arbitrator or select a successor from an American Arbitration Association list of prominent professional arbitrators, alternately striking names from such list until only one remains; provided, however, that if the Grievance Arbitrator is discharged as of the April 30 of the last Season covered by this Agreement, the parties shall, if a successor Grievance Arbitrator is not agreed upon by June 1 of such Season, select such successor from an American Arbitration Association list (by alternatively striking names) no later than June 20. A Grievance Arbitrator who is discharged shall continue to serve until his successor is agreed upon or selected. The same procedures shall be followed if the parties cannot agree upon the initial Grievance Arbitrator within thirty (30) days from the date of this Agreement.

\hypertarget{injury-grievances.}{%
\section{Injury Grievances.}\label{injury-grievances.}}

\begin{enumerate}
\def\labelenumi{(\alph{enumi})}
\tightlist
\item
  Disputes arising under paragraphs 7, 16(a)(iii), 16(b), or 16(c) of a Uniform Player Contract as to (i) whether a player was in sufficiently good condition to play skilled basketball, (ii) whether the player was injured as a direct result of participating in any basketball practice or game played for the Team, (iii) whether such injury disabled the player and/or rendered him unfit to play skilled basketball, (iv) whether, at the time of termination pursuant to paragraph 16(a)(iii), the player was unfit to play skilled basketball as the result of an injury resulting directly from his playing for the Team and, if so, the duration of such unfitness, and/or (v) whether the player is being asked to submit to appropriate medical treatment shall be processed and determined in the same manner as a Grievance under Sections 2-6 above, except that if a party to such Grievance so elects, the NBA and the Players Association shall agree upon a neutral physician or (in the absence of such agreement) jointly request that the President of the American College of Orthopedic Surgeons (or such other similar organization as the parties agree may be most appropriate to the issues in dispute) designate a physician who has no relationship with any party covered by this Agreement to conduct a physical examination of the player and to perform the functions of the Grievance Arbitrator. The physician so selected shall render a written decision which shall constitute full, final and complete disposition of the dispute, and shall be binding upon the player(s) and Team(s) involved and the parties to this Agreement. Any fees or costs associated with the physician's determination will be borne equally by both sides.
\item
  All other disputes arising under paragraphs 7, 16(a)(iii), 16(b), or 16(c) of a Uniform Player Contract (including, but not limited to, a dispute as to whether the suspension of a player or the termination of a Player Contract was by reason of a disability resulting from a re-injury to or aggravation of a previous injury or preexisting condition) shall not be subject to the special procedure set forth in Section 7(a) above, but rather shall be processed and determined in the same manner as any other Grievance under Sections 2-6 above.
\end{enumerate}

\hypertarget{special-procedure-with-respect-to-player-discipline.}{%
\section{Special Procedure with Respect to Player Discipline.}\label{special-procedure-with-respect-to-player-discipline.}}

\begin{enumerate}
\def\labelenumi{(\alph{enumi})}
\tightlist
\item
  Any dispute involving (i) a fine or suspension imposed upon a player by the Commissioner (or his designee) for conduct on the playing court (regardless of its financial impact on the player), or (ii) action taken by the Commissioner (or his designee) concerning the preservation of the integrity of, or the maintenance of public confidence in, the game of basketball resulting in a financial impact to the player of \$25,000 or less, shall be processed exclusively as follows:

  \begin{enumerate}
  \def\labelenumii{(\roman{enumii})}
  \tightlist
  \item
    Within twenty (20) days following written notification of the action taken by the Commissioner (or his designee), a player affected thereby or the Players Association may appeal in writing to the Commissioner.
  \item
    Upon the written request of the Players Association, the Commissioner shall designate a time and place for hearing as soon as is reasonably practicable following his receipt of the notice of appeal.
  \item
    As soon as reasonably practicable, but not later than twenty (20) days, following the conclusion of such hearing, the Commissioner shall render a written decision, which decision shall constitute full, final and complete disposition of the dispute, and shall be binding upon the player(s) and Team(s) involved and the parties to this Agreement.
  \item
    In the event such appeal involves a fine or suspension imposed by the Commissioner's designee, the Commissioner, as a consequence of such appeal and hearing, shall have authority only to affirm or reduce such fine or suspension, and shall not have authority to increase such fine or suspension.
  \end{enumerate}
\item
  In the event a matter filed as a Grievance in accordance with the provisions of Section 2(d) gives rise to issues involving the integrity of, or public confidence in, the game of basketball, and the financial impact on the player of the action being grieved is \$25,000 or less, the Commissioner may, at any stage of its processing, order that the matter be withdrawn from such processing and thereafter be processed in accordance with the procedure provided in Section 8(a).
\end{enumerate}

\hypertarget{procedure-with-respect-to-fine-and-suspension-amounts.}{%
\section{Procedure with Respect to Fine and Suspension Amounts.}\label{procedure-with-respect-to-fine-and-suspension-amounts.}}

In the event that a Grievance challenging a Commissioner or Team-imposed fine and/or suspension is filed in accordance with this Article, the amount of any fine or salary lost by virtue of the suspension shall be deposited in a separate interest-bearing account maintained for such fines or suspension-related amounts. The NBA shall provide written notice to the Players Association of the date and amount of each deposit made pursuant to this Section 9, and the custodian of the account shall deliver monthly statements reflecting the investment activity in such account to the NBA and the Players Association. In the absence of agreement between the NBA and the Players Association, the Grievance Arbitrator (in a manner consistent with his determination of a Grievance subject to the provisions of this Section) shall determine the amount of the deposited funds to be payable to the player, the Team, or the NBA, and any interest earned on such deposit shall be allocated to the parties in proportion thereto.

\hypertarget{disputes-with-respect-to-the-terms-of-a-player-contract.}{%
\section{Disputes with Respect to the Terms of a Player Contract.}\label{disputes-with-respect-to-the-terms-of-a-player-contract.}}

\begin{enumerate}
\def\labelenumi{(\alph{enumi})}
\tightlist
\item
  If either the NBA or the Players Association asserts that a term or provision of a Player Contract is not permitted by this Agreement, either may have the dispute involving such Contract term or provision resolved by initiating a Grievance. If such a Grievance is initiated by the NBA, the 30-day time period referred to in Section 2(c) of this Article XXXI shall commence with the date upon which the NBA received the Player Contract (or amendment thereto) containing the disputed term or provision. If such a Grievance is initiated by the Players Association, the 30-day time period referred to in Section 2(c) of this Article XXXI shall commence with the date upon which the Player Contract (or amendment thereto) containing the disputed term or provision was first made available for inspection by the Players Association.
\item
  If, as a result of the Grievance and Arbitration procedure, a Player Contract is found to contain a term or provision that is not permitted by this Agreement, then (i) such term or provision shall be deleted from the Player Contract and have no force or effect, and the Player Contract shall in all other respects remain valid and binding upon the parties thereto, and (ii) if the Team and the Player agree to reform or revise the Player Contract within 30 days of the Grievance Arbitrator's decision, such reformation or revision shall be exempted from the rules governing Renegotiations contained in Article VII, Section 7(c).
\item
  Nothing set forth above shall affect in any manner the Commissioner's authority with respect to the approval or disapproval of Player Contracts pursuant to paragraph 11 of the Uniform Player Contract; and the fact that the Commissioner has approved or not disapproved a Player Contract containing a term or provision not permitted by this Agreement shall not be referred to in the course of the Grievance and Arbitration procedure and shall not be considered in any manner or for any purpose by the Grievance Arbitrator in connection with a dispute concerning that Player Contract.
\end{enumerate}

\hypertarget{disputes-with-respect-to-players-under-contract-who-withhold-playing-services.}{%
\section{Disputes with Respect to Players Under Contract Who Withhold Playing Services.}\label{disputes-with-respect-to-players-under-contract-who-withhold-playing-services.}}

In addition to any other rights a Team may have under contract or law, including those under paragraph 9 of a Uniform Player Contract, a Team may recover damages in a proceeding before the Grievance Arbitrator when a player who is party to a currently effective Player Contract fails or refuses to render the services called for under the Player Contract. In any such proceeding, where the Grievance Arbitrator determines that damages are continuing to accrue at the time of the hearing, the Arbitrator shall award such damages (if any) as the Team has by then sustained, and the hearing shall remain open to enable the submission of proof on the issue of continuing damages.

\hypertarget{expedited-procedure.}{%
\section{Expedited Procedure.}\label{expedited-procedure.}}

\begin{enumerate}
\def\labelenumi{(\alph{enumi})}
\tightlist
\item
  Notwithstanding the foregoing, in the event of a dispute arising under Article XVII, Article XXX, or Article XXXI, Section 11 of this Agreement, or under paragraph 15 of a Uniform Player Contract (but only insofar as such paragraph provides), or in the event of an alleged breach by a player of paragraph 9 of a Uniform Player Contract, the NBA or the Players Association may request that such dispute or alleged breach be referred immediately to the Grievance Arbitrator. In any such case, the dispute or alleged breach shall be asserted by notice in writing or by facsimile given to the other party or parties, the NBA, the Players Association, and the Grievance Arbitrator.
\item
  The Grievance Arbitrator shall convene a hearing with respect to such dispute or alleged breach at the earliest possible time, but in no event later than 24 hours following his receipt of such notice. If the Grievance Arbitrator is not immediately available and the parties are unable to agree upon another grievance arbitrator, the American Arbitration Association shall appoint such other grievance arbitrator.
\item
  The award, which shall be issued not later than twenty-four (24) hours after the conclusion of the hearing, shall be in writing and may be issued with or without opinion. If any party desires an opinion, one shall be issued but its issuance shall not delay compliance with or enforcement of the award. The award shall constitute full, final and complete disposition of the dispute or alleged breach, and shall be binding upon the player(s) and Team(s) involved and the parties to this Agreement.
\item
  The failure of any party to attend the hearing as scheduled shall not delay the hearing, and the Grievance Arbitrator shall be authorized to proceed to take evidence and issue an award as though such party were present.
\end{enumerate}

\hypertarget{threshold-amounts-for-certain-grievances.}{%
\section{Threshold Amounts for Certain Grievances.}\label{threshold-amounts-for-certain-grievances.}}

A fine or suspension imposed by a Team shall be appealable to the Grievance Arbitrator only if it results in a financial impact on the player of more than \$2,000. A fine or suspension imposed by the Commissioner shall be appealable to the Grievance Arbitrator only if it results in a financial impact on the player of more than \$25,000.

\hypertarget{miscellaneous.-1}{%
\section{Miscellaneous.}\label{miscellaneous.-1}}

\begin{enumerate}
\def\labelenumi{(\alph{enumi})}
\tightlist
\item
  Each of the time limits set forth herein may be extended by mutual agreement of the parties involved.
\item
  In any meeting or hearing provided for by this Article, a player may be accompanied by a representative of the Players Association who may participate in such meeting or hearing and represent the player. In any such meeting or hearing, the NBA and any Team involved may attend and be accompanied by a representative who may participate in such meeting or hearing and represent the NBA and any such Team.
\item
  The parties recognize that a player may be subjected to disciplinary action for just cause by his Team or by the Commissioner (or his designee). Therefore, in Grievances regarding discipline, the issue to be resolved shall be whether there has been just cause for the penalty imposed.
\item
  Nothing contained herein shall excuse a player from prompt compliance with any discipline imposed upon him. If discipline imposed upon a player is determined to be improper by a final disposition under this Article XXXI, the player shall promptly be made whole.
\item
  Nothing contained in this Article XXXI shall be deemed to limit or impair the right of the NBA or any Team to impose discipline upon a player(s) or to take any other action not inconsistent with the provisions of a Player Contract or this Agreement.
\item
  Subject to Section 3(c) above, all costs of arbitration, including the fees and expenses of the Grievance Arbitrator, shall be borne equally by the parties thereto; but each party shall bear the cost of its own witnesses, counsel, and the like.
\item
  A Team shall not be required to terminate a Player Contract under the NBA waiver procedure as a condition precedent to the filing of a Grievance with respect to such Player Contract. To the extent that the decision of the Impartial Arbitrator in In re Otis Birdsong, Dec.~No.~87-2, May 14, 1987, is inconsistent with the foregoing, it is hereby
  overruled.
\item
  In a proceeding involving the interpretation of a Player Contract, no Uniform Player Contract (whether signed during the term of this Agreement or during the term of any prior collective bargaining agreement between the parties), or amendment thereto, other than the Player Contract or amendment that is the subject of dispute shall be admissible as evidence of the meaning of, or of the parties' intentions with respect to, any individually negotiated terms or provisions in the Player Contract or amendment that is the subject of dispute.
\end{enumerate}

\hypertarget{system-arbitration}{%
\chapter{SYSTEM ARBITRATION}\label{system-arbitration}}

\hypertarget{jurisdiction-and-authority.}{%
\section{Jurisdiction and Authority.}\label{jurisdiction-and-authority.}}

The NBA and the Players Association shall agree upon a System Arbitrator, who shall (except as otherwise specifically provided by Article VII, Section 3(d)(5) above) have exclusive jurisdiction to determine any and all disputes arising under Articles VII, VIII, X, XI, XII, XIII, XIV, XV, XVI, XXXVII, XXXIX, and XL of this Agreement, and those made subject to his jurisdiction by Sections 9 and 10 of this Article.

\hypertarget{initiation.-1}{%
\section{Initiation.}\label{initiation.-1}}

\begin{enumerate}
\def\labelenumi{(\alph{enumi})}
\tightlist
\item
  Subject to Article XIV, Section 6, System Arbitrations may be initiated, as set forth below, only by the NBA or the Players Association.
\item
  No party may initiate a System Arbitration until and unless it has first discussed the matter with the other party in an attempt to settle it.
\item
  A System Arbitration must be initiated within three (3) years from the date of the act or omission upon which the System Arbitration is based, or within three (3) years from the date upon which such act or omission became known or reasonably should have become known to the party initiating the System Arbitration, whichever is later.
\item
  Either the NBA or the Players Association may initiate a System Arbitration by filing written notice thereof with the System Arbitrator and serving a copy of such notice on the other party.
\end{enumerate}

\hypertarget{hearings.-1}{%
\section{Hearings.}\label{hearings.-1}}

The System Arbitrator shall hold hearings on alleged violations of the Articles set forth in Section 1 above. Except as otherwise provided in Article XI, Section 6(i) and Sections 9 and 10 below, awards issued by the System Arbitrator shall be subject to review by the Appeals Panel, in the manner and in accordance with the procedures set forth in Sections 3 and 5 of this Article XXXII.

\begin{enumerate}
\def\labelenumi{(\alph{enumi})}
\tightlist
\item
  The System Arbitrator shall make findings of fact and award appropriate relief including, without limitation, damages and specific performance. The System Arbitrator shall render an award as soon as practicable, and shall set forth the basis for such award in a written opinion that either accompanies the award or is issued within a reasonable time thereafter. In no event shall either the award or the written decision be issued more than thirty (30) days following the date upon which the record of a System Arbitration proceeding is closed (or, where applicable, the date designated by the System Arbitrator for the submission of post-hearing briefs).
\item
  The System Arbitrator shall have authority to order the production of documents, the conduct of pre-hearing depositions, and the attendance of witnesses at the hearing with respect to the NBA and the Players Association, and/or any player or Team. The System Arbitrator shall have the authority to compel the attendance of witnesses and the production of documents at any hearing within the jurisdiction of the System Arbitrator in accordance with the New York C.P.L.R.
\item
  Awards of the System Arbitrator shall upon their issuance constitute full, final and complete disposition of the dispute, shall be binding upon the parties to this Agreement and upon any player(s) or Team(s) involved, and shall be followed by them unless (in cases where this Agreement provides for an appeal to the Appeals Panel) a notice of appeal is served by the appealing party upon the responding party and filed with the System Arbitrator within ten (10) days of the date of the award of the System Arbitrator appealed from. If and when an award of the System Arbitrator is reversed or modified by the Appeals Panel, the effect of such reversal or modification shall be deemed by the parties to be retroactive to the time of issuance of the award of the System Arbitrator. The parties may seek appropriate relief to effectuate and enforce this provision.
\item
  The System Arbitrator shall not have jurisdiction or authority to add to, detract from, or alter in any way the provisions of this Agreement or any Player Contract.
\end{enumerate}

\hypertarget{costs-relating-to-system-arbitration.}{%
\section{Costs Relating to System Arbitration.}\label{costs-relating-to-system-arbitration.}}

The compensation of the System Arbitrator and the costs and expenses incurred in connection with any proceeding brought before the System Arbitrator shall be borne equally by the parties to this Agreement; provided, however, that each participant in such proceeding shall bear its own attorneys' fees and litigation costs.

\hypertarget{procedure-for-system-arbitration.}{%
\section{Procedure for System Arbitration.}\label{procedure-for-system-arbitration.}}

All matters before the System Arbitrator shall be heard and determined in an expedited manner, provided that such expedition is reasonable under the circumstances. A proceeding may be commenced upon seventy-two (72) hours' written notice (or upon shorter notice if ordered by the System Arbitrator) served upon the party against whom the proceeding is brought and filed with the System Arbitrator. All such notices and all orders and notices issued and directed by the System Arbitrator shall be served on the NBA, counsel for the NBA, the Players Association, counsel for the Players Association, and any counsel appearing for individual NBA players or individual NBA Teams. In any proceeding commenced pursuant to Article XIV, Section 6, the Players Association (on its own behalf and/or on behalf of a player) and the NBA (on its own behalf and/or on behalf of a Team) shall have the right to participate.

\hypertarget{selection-of-system-arbitrator.}{%
\section{Selection of System Arbitrator.}\label{selection-of-system-arbitrator.}}

In the event that the Players Association and the NBA cannot agree on the identity of a System Arbitrator, the parties shall jointly request the Center for Public Resources (or such other organization(s) as the parties may have agreed upon) to submit to the parties a list of eleven (11) attorneys (none of whom shall have, nor whose firm shall have, represented within the past five years players; player agents; labor organizations representing athletes; sports leagues, governing bodies, or their affiliates; sports teams or their affiliates; or owners in any professional sport). If the parties cannot within seven (7) days from the receipt of such list agree to the identity of the System Arbitrator from among the names on such list, they shall return said list, with up to five names deleted therefrom by each party, to the Center for Public Resources (or such other organization as the parties may have agreed upon), which shall choose from the remaining names on the list the identity of the System Arbitrator. The first System Arbitrator selected under the provisions of this Agreement shall serve until December 1, 1999. Thereafter the System Arbitrator shall serve for continually renewing two-year terms unless notice of termination is given either by the NBA or by the Players Association. Notice of termination of the System Arbitrator shall be given to the other party, and to the System Arbitrator at least forty-five (45) days preceding the end of any term. Following the giving of such notice, a new System Arbitrator shall be selected in accordance with the procedures set forth in this paragraph. The System Arbitrator whose term has ended shall continue to hear all disputes filed prior to the date of the appointment of a new System Arbitrator.

\hypertarget{selection-of-appeals-panel.}{%
\section{Selection of Appeals Panel.}\label{selection-of-appeals-panel.}}

There shall be a three-member Appeals Panel for each appeal noticed from an award of the System Arbitrator. In the event the Players Association and the NBA cannot agree upon the members of such a panel, the parties will jointly request the Center for Public Resources (or such other organization(s) as the parties may agree) to submit to the parties a list of fifteen (15) attorneys (none of whom shall have, nor whose firm shall have, represented within the past five (5) years players; player agents; labor organizations representing athletes; sports leagues, governing bodies, or their affiliates; sports teams or their affiliates; or owners in any professional sport). If the parties cannot within seven (7) days from the receipt of such list agree to the identity of the Appeals Panel from among the names on such list, they shall meet and alternate striking one (1) name at a time from the list until three (3) names on the list remain. The three (3) remaining names on the list shall comprise the Appeals Panel for that particular appeal. The compensation of the members of the Appeals Panel and the costs of proceedings before the Appeals Panel shall be borne equally by the parties to this Agreement; provided, however, that each participant in an Appeals Panel proceeding shall bear its own attorneys' fees and litigation costs.

\hypertarget{procedure-relating-to-appeals-of-determination-by-the-system-arbitrator.}{%
\section{Procedure Relating to Appeals of Determination by the System Arbitrator.}\label{procedure-relating-to-appeals-of-determination-by-the-system-arbitrator.}}

\begin{enumerate}
\def\labelenumi{(\alph{enumi})}
\tightlist
\item
  Any party seeking to appeal (in whole or in part) an award of the System Arbitrator must serve on the other party and file with the System Arbitrator a notice of appeal, within ten (10) days of the date of the award appealed from. The timely service and filing of a notice of appeal shall automatically stay the award of the System Arbitrator pending resolution by the Appeals Panel.
\item
  Following the timely service and filing of a notice of appeal, the NBA and the Players Association shall attempt to agree upon a briefing schedule. In the absence of such agreement, the briefing schedule shall be set by the Appeals Panel; provided, however, that any party seeking to appeal (in whole or in part) from an award of the System Arbitrator shall be afforded no less than fifteen (15) and no more than twenty-five (25) days from the date of the issuance of such award, or the date of the issuance of the System Arbitrator's written opinion, or the date upon which the members of the Appeals Panel have been selected in accordance with the provisions of Section 7, whichever is latest, to serve on the opposing party and file with the Appeals Panel its brief in support thereof; and provided further that the responding party or parties shall be afforded the same aggregate amount of time to serve and file its or their responding brief(s). The Appeals Panel shall schedule oral argument on the appeal(s) no less than five (5) and no more than ten (10) days following the service and filing of the responding brief(s), and shall issue a written decision within thirty (30) days from the date of argument.
\item
  The Appeals Panel shall review the findings of fact and conclusions of law made by the System Arbitrator using the standards of review employed by the U.S. Court of Appeals for the Second Circuit. The decision of the Appeals Panel shall constitute full, final, and complete disposition of the dispute, and shall be binding upon the parties to this Agreement and upon any player(s) or Team(s) involved.
\end{enumerate}

\hypertarget{special-procedure-for-disputes-with-respect-to-interim-audit-reports.}{%
\section{Special Procedure for Disputes with Respect to Interim Audit Reports.}\label{special-procedure-for-disputes-with-respect-to-interim-audit-reports.}}

\begin{enumerate}
\def\labelenumi{(\alph{enumi})}
\tightlist
\item
  Notwithstanding any of the other provisions of this Agreement, at the request of either the NBA or the Players Association, and irrespective of which party may commence the proceeding, the procedures set forth in this Section 9 shall apply to the resolution of any disputes with respect to an Interim Audit Report, including but not limited to disputes concerning any Escrow Information set forth in an Interim Audit Report. If in connection with such disputes, there is any conflict between the procedures set forth in this Section 9 and those set forth elsewhere in this Agreement, the procedures set forth in this Section shall control.
\item
  A proceeding before the System Arbitrator shall be commenced, in the manner provided for by Sections 2(d) and 5 of this Article XXXII, no more than thirty (30) days following the delivery by the Accountants of the Interim Audit Report with respect to any dispute or claim concerning (i) the amount(s) of BRI or Total Salaries (or portions thereof) as to which the Accountants have completed their review and which is the subject of a good faith dispute between the parties, (ii) the amount(s) of BRI or Total Salaries (or portions thereof) as to which the Accountants have not completed their review and with respect to which the parties have a good faith disagreement, (iii) such Escrow Information as is included in the Interim Audit Report as to which the parties have a good faith disagreement and/or (iv) all other disputes (including but not limited to disputes over the amounts and includability of any revenues or expenses included or excluded from the Interim Audit Report) of which the parties were aware or reasonably should have been aware, at the time the proceeding was commenced, based upon the contents of the BRI Reports, the Draft Audit Report or Interim Audit Report or other documents or writings provided to the parties by the Accountants in connection with their BRI audit.
\item
  A party's failure to commence a proceeding before the System Arbitrator within the thirty-day (30) period provided for by subsection (b) above with respect to the disputes or claims enumerated in that subsection shall forever bar that party from asserting or seeking relief of any kind for any such dispute or claim; provided, however, that the provisions of subsection (b) above and this subsection (c) shall not bar a party from commencing a proceeding before the System Arbitrator and seeking appropriate relief, subject to the limitations imposed by Section 2 of this Article XXXII:

  \begin{enumerate}
  \def\labelenumii{(\roman{enumii})}
  \tightlist
  \item
    With respect to a dispute or claim concerning an Interim Audit Report as to which such party was not aware or reasonably should not have been aware, based upon the materials referred to in subsection (b) above, during the thirty-day (30) period following the delivery of such Interim Audit Report; or
  \item
    With respect to any dispute or claim relating to a subsequent Salary Cap Year, including but not limited to any dispute concerning the includability or non-includability in BRI of a category or type of revenue or the allowance or disallowance of a category or type of expense, without regard to whether, based upon the materials referred to in subsection (b) above (other than a BRI Report, Draft Audit Report or Interim Audit Report), the party was or reasonably should have been aware of such dispute or claim during the thirty-day (30) period following the delivery of such Interim Audit Report.
  \item
    Subject to subsection (c)(ii) above, no determination made by the System Arbitrator or the Appeals Panel (as the case may be) in a proceeding commenced pursuant to subsections (c)(i) or (ii) above shall affect any calculations made pursuant to Article VII, Section 12.
  \end{enumerate}
\item
  Where a hearing before the System Arbitrator is provided for by this Section 9, such hearing shall be conducted within fifteen (15) days from the commencement of the proceeding, and the System Arbitrator shall render an award and issue a written decision as soon as possible, but in no event later than fifteen (15) days following the close of the hearing. Where a right to appeal from the System Arbitrator's award is provided for by this Section 9, any party seeking to appeal (in whole or in part) from such an award shall serve and file a notice of appeal therefrom within five (5) days from the date of such award and shall serve and file its brief in support of such appeal within fifteen (15) days from the date of the System Arbitrator's award or within five (5) days from the date upon which the members of the Appeals Panel have been selected, whichever is later. The party opposing such appeal shall serve and file its brief in opposition within ten (10) days following its receipt of the brief in support of the appeal. The Appeals Panel shall schedule oral argument at its discretion, but shall issue a written decision within twenty (20) days following its receipt of the brief from the party opposing the appeal.
\item
  Any dispute concerning the amounts (as opposed to the includability) of any revenues or expenses to be included in an Interim Audit Report (hereinafter referred to as ``Disputed Adjustments'') shall, whenever such Disputed Adjustments for all Teams are adverse to the party asserting the dispute in an aggregate amount of less than \$5 million for any Season covered by this Agreement, be resolved by the Accountants; and the determination of the Accountants shall constitute full, final and complete disposition of the dispute and shall be binding upon the parties to this Agreement. Notwithstanding the foregoing, any Disputed Adjustments that involve the interpretation, validity or application of this Agreement shall be resolved by the System Arbitrator and shall be appealable to the Appeals Panel in accordance with the provisions of subsection (d) above.
\item
  If the Disputed Adjustments for all Teams are adverse to the party asserting the dispute in an aggregate amount of \$5 million or more but less than \$10 million for any Season covered by this Agreement, the determination of the System Arbitrator shall constitute full, final and complete disposition of the dispute and shall be binding upon the parties to this Agreement, and there shall be no appeal to the Appeals Panel. Notwithstanding the foregoing, any Disputed Adjustments that involve the interpretation, validity or application of this Agreement shall be resolved by the System Arbitrator and shall be appealable to the Appeals Panel in accordance with the provisions of subsection (d) above.
\item
  If the Disputed Adjustments for all Teams are adverse to the party asserting the dispute in an aggregate amount of \$5 million or more but less than \$10 million for any Season covered by this Agreement, and if the party asserting such dispute does not prevail before the System Arbitrator, then that party shall pay all of the fees and expenses of the System Arbitrator and the reasonable costs and expenses, including attorneys' fees, of the other party for its defense of the proceeding, provided, however, that if each party has asserted a dispute upon which it has not prevailed, all such fees, expenses and costs shall be borne in the manner provided for by Section 4 of this Article.
\item
  All other disputes involving an Interim Audit Report (including but not limited to disputes over the amounts and includability of any revenues or expenses to be included in such Reports) and the Escrow Information shall be resolved by the System Arbitrator and shall be appealable to the Appeals Panel in accordance with the provisions of subsection (d) above.
\end{enumerate}

\hypertarget{special-procedure-for-disputes-with-respect-to-the-escrow-schedules.}{%
\section{Special Procedure for Disputes with Respect to the Escrow Schedules.}\label{special-procedure-for-disputes-with-respect-to-the-escrow-schedules.}}

\begin{enumerate}
\def\labelenumi{(\alph{enumi})}
\tightlist
\item
  Notwithstanding any of the other provisions of this Agreement, the procedures set forth in this Section 10 shall apply to the resolution of any disputes with respect to the Escrow Schedules described in Article VII, Section 12. If in connection with such disputes, there is any conflict between the procedures set forth in this Section 10 and those set forth elsewhere in this Agreement, the procedures set forth in this Section shall control.
\item
  In the event of any dispute with respect to the Escrow Schedules, the proceeding before the System Arbitrator shall be commenced, in the manner provided for by Sections 2(d) and 5 of this Article XXXII no more than seven (7) days following the transmittal to the Players Association of any of such schedules.
\item
  The hearing before the System Arbitrator with respect to a dispute concerning the Escrow Schedules shall be conducted within ten (10) days following the commencement of the proceeding and the briefs of the parties, if any, shall be filed before the opening of the hearing on a date or dates set by the System Arbitrator. The hearing shall be conducted on an expedited basis and, unless the parties otherwise agree or a party demonstrates that such limitation will result in undue prejudice, will not last longer than two (2) full days.
\item
  If in connection with the Escrow Schedules, there is no dispute between the NBA and the Players Association as to the Projected Aggregate Compensation Adjustment Amount, but there is a claim asserted by the Players Association concerning the calculation made with respect to an individual player's Escrow Amount, the Players Association shall provide the NBA with a written proposal as to how the Escrow Amount of such individual player and of other players should be adjusted. If the NBA rejects that proposal, such claim shall be resolved by the System Arbitrator. The determination of the System Arbitrator shall constitute full, final and complete disposition of the dispute and shall be binding upon the parties to this Agreement and any player(s) involved, and there shall be no appeal to the Appeals Panel.
\item
  If in connection with the Escrow Schedules, there is a dispute between the NBA and the Players Association as to the Projected Aggregate Compensation Adjustment Amount and the amount in controversy is \$2,500,000 or less, the determination of the System Arbitrator shall constitute full, final and complete disposition of the dispute and shall be binding upon the parties to this Agreement, and there shall be no appeal to the Appeals Panel. If with respect to such dispute the amount in controversy is more than \$2,500,000, either party may appeal a determination of the System Arbitrator to the Appeals Panel.
\item
  If in connection with the Escrow Schedules, there is a dispute between the NBA and the Players Association with respect to any matter other than as described in subsections (d) and (e) above, the determination of the System Arbitrator shall constitute full, final and complete disposition of the dispute and shall be binding upon the parties to this Agreement, and any player(s) involved, and there shall be no appeal to the Appeals Panel.
\item
  In connection with any dispute concerning the Escrow Schedules, the System Arbitrator shall render an award and issue a written decision as soon as possible, but in no event later than ten (10) days following the close of the hearing. When the award is issued, the System Arbitrator shall set forth the basis therefore either in a written opinion or orally at a conference with the parties (which conference may be conducted by telephone) of which a stenographic record shall be made. Any party seeking to appeal (in whole or in part) from an award of the System Arbitrator rendered pursuant to subsection (e) above shall serve and file a notice of appeal therefrom within two (2) business days from the date of such award. The party seeking to appeal shall serve and file its brief in support of such appeal within ten (10) days from the date of the System Arbitrator's award or within three (3) days from the date upon which the members of the Appeals Panel have been selected, whichever is later. The party opposing such appeal shall serve and file its brief in opposition within ten (10) days following its receipt of the brief in support of the appeal. The Appeals Panel shall schedule oral argument at its discretion, but shall issue a written decision within twenty (20) days following its receipt of the brief from the party opposing the appeal.
\end{enumerate}

\hypertarget{anti-drug-program}{%
\chapter{ANTI-DRUG PROGRAM}\label{anti-drug-program}}

\hypertarget{definitions.-1}{%
\section{Definitions.}\label{definitions.-1}}

As used in this Article XXXIII, the following terms shall have the following meanings:

\begin{enumerate}
\def\labelenumi{(\alph{enumi})}
\tightlist
\item
  ``Authorization for Testing'' shall mean a notice issued by the Independent Expert pursuant to the provisions of Section 5 in the form annexed hereto as Exhibit I-1 to this Agreement.
\item
  ``Come Forward Voluntarily'' shall mean that a player has directly communicated to the NBA, the Players Association, or the Medical Director his desire to enter the Program and seek treatment for a problem involving the use of a Prohibited Substance.
\item
  ``Counselors'' or ``Anti-Drug Counselors'' shall mean the persons selected by the Medical Director to provide counseling and other treatment to players in the Program.
\item
  ``Drugs of Abuse'' shall mean any of the substances listed as drugs of abuse on Exhibit I-2 to this Agreement.
\item
  ``Drugs of Abuse Program'' shall mean the education, treatment, and counseling program for Drugs of Abuse established by the Medical Director (after consultation with the NBA and the Players Association), which program may contain such elements---including, but not limited to, urine, blood, breath, or other testing for Prohibited Substances other than Steroids---as may be determined by the Medical Director in his or her professional judgment.
\item
  ``First-Year Player'' shall mean a player under Contract to an NBA Team who, prior to the then-current Season, has not been on the roster of an NBA Team following the first game of a Regular Season.
\item
  ``In-Patient Facility'' shall mean such treatment center or other facility as may be selected by the Medical Director and agreed upon by the NBA and the Players Association.
\item
  ``Independent Expert'' or ``Expert'' shall mean the person selected by the NBA and the Players Association in accordance with Section 2(b) below.
\item
  ``Marijuana Program'' shall mean the education, treatment, and counseling program for marijuana established by the Medical Director (after consultation with the NBA and the Players Association), which program may contain such elements---including, but not limited to, urine, blood, breath, or other testing for Prohibited Substances other than Steroids---as may be determined by the Medical Director in his or her professional judgment.
\item
  ``Medical Director'' shall mean the person selected by the NBA and the Players Association in accordance with Section 2(a).
\item
  ``Prohibited Substance'' shall mean any of the substances listed on Exhibit I-2 to this Agreement and any other substance added to such Exhibit under the provisions of Section 16 below.
\item
  ``Program'' shall mean this Anti-Drug Program, and shall include the Drugs of Abuse Program, the Marijuana Program, and the Steroids Program.
\item
  ``Prohibited Substances Committee'' shall mean the committee selected by the NBA and the Players Association in accordance with Section 2(d) below.
\item
  ``Steroids'' shall mean the performance-enhancing substances listed on Exhibit I-2 to this Agreement.
\item
  ``Steroids Program'' shall mean the education, treatment, and counseling program for Steroids established by the Medical Director (after consultation with the NBA and the Players Association), which program may contain such elements---including, but not limited to, urine, blood, breath or other testing for Steroids (but not for any other Prohibited Substance)---as may be determined by the Medical Director in his or her professional judgment.
\item
  ``Tender'' shall mean an offer of a Uniform Player Contract, signed by the Team, that is either personally delivered to the player or his representative or sent by prepaid certified, registered, or overnight mail to the last known address of the player or his representative.
\item
  ``Veteran Player'' shall mean any player who is not a First-Year Player.
\end{enumerate}

\hypertarget{administration.}{%
\section{Administration.}\label{administration.}}

\begin{enumerate}
\def\labelenumi{(\alph{enumi})}
\tightlist
\item
  The NBA and the Players Association shall jointly select a Medical Director who shall be a person experienced in the field of testing and treatment for substance abuse. The Medical Director shall have the responsibility, among other duties, for selecting and supervising the Counselors and other personnel necessary for the effective implementation of the Program, for evaluating and treating players subject to the Program, and for otherwise managing and overseeing the Program, subject to the control of the NBA and the Players Association. To the extent practicable, the Medical Director shall select qualified retired NBA players to serve as Counselors.
\item
  The NBA and the Players Association shall jointly select an Independent Expert who shall be a person experienced in the field of substance abuse detection and enforcement and shall have the responsibility for issuing Authorizations for Testing in accordance with Section 5 below.
\item
  The Medical Director and the Independent Expert shall each serve for the duration of this Agreement, unless either the NBA or the Players Association has, by September 1 of any year covered by this Agreement, served written notice of discharge upon the other party and, as appropriate, the Medical Director and/or the Independent Expert. Such notice of discharge shall be effective as of the immediately following September 30; provided, however, that if the parties do not reach agreement by such September 30 as to who shall serve thereafter as the Medical Director and/or the Independent Expert, as the case may be, each party shall, by the immediately following October 10, appoint a person who shall have no relationship to or affiliation with that party. Such persons shall then have until the immediately following November 1 to agree on the appointment of a new Medical Director and/or Independent Expert. Until a new Medical Director and/or Independent Expert has been appointed, the previous Medical Director and/or Independent Expert shall continue to serve.
\item
  The NBA and the Players Association shall form a Prohibited Substances Committee, which shall be comprised of one representative from the NBA, one representative from the Players Association, and three individuals jointly selected by the NBA and the Players Association who shall be experts in the field of testing and treatment for drugs of abuse and performance-enhancing substances. The members of this Committee shall serve for the duration of the Agreement.
\item
  Unless specifically stated otherwise in this Article XXXIII, all costs of the Program in excess of those covered by the NBA Players Group Health Plan, including the fees and expenses of the Medical Director, the Independent Expert, and the Prohibited Substances Committee, but not including fees and expenses incurred by the NBA in conducting testing pursuant to Sections 5, 6, or 7 below, shall be shared equally by the NBA and Players Association. The Players Association's share shall be paid by the NBA and included in Player Benefits under Article IV, Section 1(k) of this Agreement.
\item
  Any and all disputes arising under this Article XXXIII shall be resolved in accordance with Article XXXI, Sections 2-6 and 14 of this Agreement, provided, however, that in any challenge to a decision, recommendation, or other conduct of the Medical Director or Independent Expert, or in any challenge to an action or process over which the Medical Director has supervision, the Grievance Arbitrator shall apply an ``arbitrary and capricious'' standard of review; provided further that nothing in this Section 2(f) shall limit or otherwise affect paragraph 19 of the Uniform Player Contract.
\end{enumerate}

\hypertarget{confidentiality.}{%
\section{Confidentiality.}\label{confidentiality.}}

\begin{enumerate}
\def\labelenumi{(\alph{enumi})}
\tightlist
\item
  Other than as reasonably required in connection with the suspension or disqualification of a player, the NBA, the Teams, and the Players Association, and all of their members, affiliates, agents, consultants, and employees, are prohibited from publicly disclosing information about the diagnosis, treatment, prognosis, test results, compliance, or the fact of participation of a player in the Program (``Program Information'').
\item
  The Medical Director and the Counselors, and all of their affiliates, agents, consultants, and employees, are prohibited from publicly disclosing Program Information; provided, however, that the Medical Director shall not be prohibited from disclosing such information to the NBA and the Players Association.
\item
  The Independent Expert is prohibited from publicly disclosing any information supplied to him by the NBA or the Players Association pursuant to Section 5 below.
\item
  Any Program Information that is publicly disclosed (i) under subsection (a) above, (ii) by the player, (iii) with the player's authorization, or (iv) through release by sources other than the NBA, NBA Teams, the Players Association, the Medical Director, the Counselors, or the Independent Expert, or any of their members, affiliates, agents, consultants, and employees, will, after such disclosure, no longer be subject to the confidentiality provisions of this Section.
\item
  Other than as reasonably required by the Reasonable Cause Testing procedure set forth in Section 5 below, neither the NBA nor the Players Association shall divulge to any other person or entity (including their respective members, affiliates, agents, consultants, employees, and the player and Team involved):

  \begin{enumerate}
  \def\labelenumii{(\roman{enumii})}
  \tightlist
  \item
    that it has received information regarding the use, possession, or distribution of a Prohibited Substance by a player;
  \item
    that it is considering requesting, has requested, or has had, a conference with the Independent Expert concerning the suspected use, possession, or distribution of a Prohibited Substance by a player;
  \item
    any information disclosed to the Independent Expert; or
  \item
    the results of any conference with the Independent Expert.
  \end{enumerate}
\item
  Notwithstanding anything to the contrary contained in subsections (a)-(e) above, the NBA and the Players Association shall promptly advise and make available to each other all information either of them may have in their possession, custody, or control that provides cause to believe that a player is engaged in the use, possession, or distribution of a Prohibited Substance.
\item
  Nothing contained in this Section 3 shall prohibit a Team from providing to the NBA information concerning whether a player is engaged in the use, possession, or distribution of a Prohibited Substance.
\end{enumerate}

\hypertarget{testing.}{%
\section{Testing.}\label{testing.}}

\begin{enumerate}
\def\labelenumi{(\alph{enumi})}
\tightlist
\item
  Testing conducted pursuant to this Article XXXIII, whether by the NBA or the Medical Director, shall be conducted in compliance with the analytical techniques described in Exhibit I-3 to this Agreement. Such testing shall also comply with the collection procedures described in Exhibit I-4 to this Agreement and such additional procedures and protocols as may be established by the NBA (after consultation with the Players Association) or the Medical Director (after consultation with the NBA and the Players Association). The NBA (after consultation with the Players Association) and the Medical Director (after consultation with the NBA and the Players Association) are authorized to retain such consultants and support services as are necessary and appropriate to administer and conduct such testing.
\item
  All tests conducted pursuant to this Article XXXIII shall be analyzed by laboratories selected by the Medical Director, approved by the NBA and the Players Association, and certified by the Substance Abuse and Mental Health Services Administration (in the case of testing for Prohibited Substances other than Steroids) or the International Olympic Committee and/or the College of American Pathologists (in the case of testing for Steroids).
\item
  Any test conducted pursuant to this Article XXXIII will be considered ``positive'' for a Prohibited Substance under the following circumstances:

  \begin{enumerate}
  \def\labelenumii{(\roman{enumii})}
  \tightlist
  \item
    If the test is for a Prohibited Substance other than Steroids and it is confirmed by laboratory analysis at the levels established at the time of the test by the National Institute for Drug Abuse (NIDA); provided, however, if there is no confirmatory level established by NIDA for one or more of such Prohibited Substances at the time of the test, then the level for such Prohibited Substance shall be: amphetamines and their analogs---500 ng/ml; cocaine metabolites---150 ng/ml; LSD---200 pg/ml; marijuana metabolites---15 ng/ml; MDMA---500 ng/ml; opiate metabolites---300 ng/ml; phencyclidine---25 ng/ml.
  \item
    If the test is for Steroids, and it is confirmed by laboratory analysis at levels to be established by the Prohibited Substances Committee.
  \item
    If the player fails or refuses to submit to a scheduled test, or refuses to cooperate fully with the testing process, without a reasonable explanation satisfactory to the Medical Director.
  \item
    If the player attempts to substitute, dilute, mask, or adulterate a specimen sample or in any other manner alter a test result.
  \end{enumerate}
\item
  The NBA shall promptly notify the Players Association of any positive test conducted by the NBA, and shall thereafter notify the player. The Medical Director shall promptly notify the NBA and the Players Association of any positive test conducted by the Medical Director, and (i) if the positive test will result in a penalty to be imposed on the player, the NBA shall thereafter notify the player of such test result and such penalty, or (ii) if the positive test will not result in a penalty to be imposed on the player, the Medical Director shall thereafter notify the player of such test result.
\item
  Any player who is notified of a positive test pursuant to Section 4(d) above may, within two (2) business days of such notification, inform the NBA and the Players Association that he requests testing of the split or ``B'' sample of his specimen. Any such test shall be subject to the provisions of this Section 4 and shall be performed within ten (10) business days of the player's request. The test of the ``B'' sample will be performed at a laboratory other than the laboratory that performed the test on the original or ``A'' sample.
\end{enumerate}

\hypertarget{reasonable-cause-testing-or-hearing.}{%
\section{Reasonable Cause Testing or Hearing.}\label{reasonable-cause-testing-or-hearing.}}

\begin{enumerate}
\def\labelenumi{(\alph{enumi})}
\tightlist
\item
  In the event that either the NBA or the Players Association has information that gives it reasonable cause to believe that a player is engaged in the use, possession, or distribution of a Prohibited Substance, including information that a First-Year Player may have been engaged in such conduct during the period beginning three (3) months prior to his entry into the NBA, such party shall request a conference with the other party and the Independent Expert, which shall be held within twenty-four (24) hours or as soon thereafter as the Expert is available. Upon hearing the information presented, the Independent Expert shall immediately decide whether there is reasonable cause to believe that the player in question has been engaged in the use, possession, or distribution of a Prohibited Substance. If the Independent Expert decides that such reasonable cause exists, the Expert shall thereupon issue an Authorization for Testing with respect to such player.
\item
  In evaluating the information presented to him, the Independent Expert shall use his independent judgment based upon his experience in substance abuse detection and enforcement. The parties acknowledge that the type of information to be presented to the Independent Expert is likely to consist of reports of conversations with third parties of the type generally considered by law enforcement authorities to be reliable sources, and that such sources might not otherwise come forward if their identities were to become known. Accordingly, neither the NBA nor the Players Association shall be required to divulge to each other or to the Independent Expert the names (or other identifying characteristics) of their sources of information regarding the use, possession, or distribution of a Prohibited Substance, and the absence of such identification of sources, standing alone, shall not constitute a basis for the Expert to refuse to issue an Authorization for Testing with respect to a player. In conferences with the Independent Expert, the player involved shall not be identified by name until such time as the Expert has determined to issue an Authorization for Testing with respect to such player.
\item
  Immediately upon the Independent Expert's issuance of an Authorization for Testing with respect to a particular player, the NBA shall arrange for such player to undergo testing for Drugs of Abuse (if the Authorization for Testing was based on information regarding the use, possession, or distribution of a Drug of Abuse), for marijuana (if the authorization for Testing was based on information regarding the player's use, possession, or distribution of marijuana), or for Steroids (if the Authorization for Testing was based on information regarding the player's use, possession, or distribution of Steroids) no more than four (4) times during the six-week period commencing with the issuance of the Authorization for Testing. Such testing may be administered at any time, in the discretion of the NBA, without prior notice to the player.
\item
  In the event that the player tests positive for a Drug of Abuse pursuant to this Section 5, he shall immediately be dismissed and disqualified from any association with the NBA or any of its Teams in accordance with the provisions of Section 12(a). If the player tests positive for marijuana or Steroids pursuant to this Section 5, he shall enter the Program and suffer the applicable consequences set forth in Sections 9 or 10 below, as the case may be.
\item
  In the event that either the NBA or the Players Association determines that there is sufficient evidence to demonstrate that, within the previous year, a player has engaged in the use, possession, or distribution of a Prohibited Substance, or has received treatment for use of a Prohibited Substance other than in accordance with the terms of this Article XXXIII, it may, in lieu of requesting the testing procedure set forth in subsections (a)-(d) above, request a hearing on the matter before the Grievance Arbitrator. If the Grievance Arbitrator concludes that, within the previous year, the player has used, possessed, or distributed a Prohibited Substance, or has received treatment other than in accordance with the terms of this Article XXXIII, the player shall immediately be dismissed and disqualified from any association with the NBA or any of its Teams in accordance with the provisions of Section 12(a) below, notwithstanding the fact that the player has not undergone the testing procedure set forth in this Section 5; provided, however, that if the Grievance Arbitrator concludes that the player has used or possessed only marijuana or Steroids, he shall enter the Program and only suffer the applicable consequences set forth in Sections 9 or 10 below, as the case may be.
\end{enumerate}

\hypertarget{testing-of-first-year-players.}{%
\section{Testing of First-Year Players.}\label{testing-of-first-year-players.}}

\begin{enumerate}
\def\labelenumi{(\alph{enumi})}
\item
  In addition to the testing procedures set forth in Section 5 above, a First-Year Player may be required to undergo testing for Prohibited Substances at any time, in the sole discretion of the NBA and without prior notice to the player, during the following periods and subject to the following restrictions:

  \begin{enumerate}
  \def\labelenumii{(\roman{enumii})}
  \tightlist
  \item
    No more than one (1) time during regular training camp, or, in the case of a First-Year Player who joins a Team with fewer than fifteen (15) days remaining in regular training camp or who joins a Team during the Regular Season, no more than one (1) time during the first fifteen (15) days after such player reports to his Team; and
  \item
    No more than three (3) times during the then-current Regular Season, or, in the case of a First-Year Player who signs a Player Contract after March 1 of an NBA Season, no more than three (3) times during such Season and/or the Season immediately following such Season.
  \end{enumerate}
\item
  Any First-Year Player who tests positive for a Drug of Abuse pursuant to this Section 6 shall immediately be dismissed and disqualified from any association with the NBA or its Teams for a period of one (1) year, his Player Contract shall be rendered null and void and of no further force or effect (subject to the provisions of Paragraph 8 of the Uniform Player Contract), and he shall enter Stage 1 of the Drugs of Abuse Program. Such dismissal and disqualification shall be mandatory and may not be rescinded or reduced by the Player's Team or the NBA. If the player tests positive for marijuana or Steroids pursuant to this Section 6, he shall suffer the applicable consequences set forth in Sections 9 or 10 below, as the case may be.
\item
  During the period while such First-Year Player is dismissed and disqualified from the NBA and in compliance with his in-patient or aftercare obligations under the Program (as determined by the Medical Director), he shall receive from his Team a reasonable and necessary living expense stipend to be agreed upon by the NBA and the Players Association which (i) shall not exceed twenty-five percent (25\%) of the Salary that the player would otherwise have been entitled to earn for the period of his dismissal and disqualification and (ii) shall not be payable for more than one (1) year from the date of such dismissal and disqualification.
\item
  After a period of no less than one (1) year from the date of a First-Year Player's dismissal and disqualification pursuant to Section 6(b), such player may apply for reinstatement as a player in the NBA. However, such player shall have no right to reinstatement under any circumstance and the reinstatement shall be granted only with the prior approval of both the NBA and the Players Association. The NBA and the Players Association will consider applications for reinstatement only if the player has, in the opinion of the Medical Director, successfully completed any in-patient treatment and/or aftercare prescribed by the Medical Director. The approval of the NBA and the Players Association shall rest in their absolute and sole discretion, and their decision shall be final, binding, and unappealable. The granting of an application for reinstatement may be conditioned upon random testing of the player or such other terms as may be agreed upon by the NBA and the Players Association, whether or not such terms are contemplated by this Article XXXIII.
\item
  In the event that the application for reinstatement of a First-Year Player dismissed and disqualified pursuant to Section 6(b) above is approved, such player, by reason of his Player Contract having been rendered null and void pursuant to Section 6(b) above, shall be deemed not to have completed his Player Contract by rendering the playing services called for thereunder. Accordingly, such player shall not be a Free Agent and shall not be entitled to negotiate or sign a Player Contract with any NBA Team, except as specifically provided in Section 6(f).
\item
  \begin{enumerate}
  \def\labelenumii{(\roman{enumii})}
  \tightlist
  \item
    A First-Year Player who has been reinstated pursuant to this Section 6 shall, immediately upon such reinstatement, notify the Team to which he was under contract at the time of his dismissal and disqualification (the ``previous Team''). Upon receipt of such notification, and subject to Section 6(f)(ii) below, the previous Team shall then have thirty (30) days in which to make a Tender to the player with a stated term of at least one (1) full NBA Season (or, in the event that the Tender is made during a Season, of at least the remainder of that Season) and calling for at least the Minimum Player Salary then applicable to that player but not more than the Salary provided for in Section 6(f)(iii) below. If the previous Team makes such a Tender, it shall, for a period of one (1) year from the date of the Tender, be the only NBA Team with which the player may negotiate and sign a Player Contract. If the player does not sign a Player Contract with the previous Team within the year following such Tender, the player shall thereupon be deemed a Restricted Free Agent, subject to a Right of First Refusal. If the previous Team fails to make a Tender, the player shall become an Unrestricted Free Agent.
  \item
    Notwithstanding anything to the contrary in Section 6(f)(i) above, the 30-day period for the previous Team to make a Tender shall be tolled if (x) on the date the player serves the notice required by Section 6(f)(i), he is under contract to a professional basketball team not in the NBA, or (y) the player signs a contract with a professional basketball team not in the NBA at any point after the date on which the player serves the notice required by Section 6(f)(i) and before the date on which the previous Team makes a Tender. If the 30-day period for making a Tender is tolled pursuant to the preceding sentence, the period shall remain tolled until the date on which the player notifies the Team that he is immediately available to sign and begin rendering playing services under a Player Contract with such Team, provided that such notice will not be effective until the player is under no contractual or other legal impediment to sign with and begin rendering playing services for such Team.
  \item
    A player who is reinstated pursuant to this Section 6 may enter into a Player Contract with his previous Team that provides for a Salary and Unlikely Bonuses for the first Season of up to the player's Salary and Unlikely Bonuses, respectively, for the Salary Cap Year in which he was dismissed and disqualified (reduced on a pro rata basis if the first Season of the new Contract is a partial Season), even if the Team has a Team Salary at or above the Salary Cap or such Player Contract causes the Team to have a Team Salary above the Salary Cap. If the player and the previous Team enter into such Player Contract and such Contract covers more than one Season, increases and decreases in Salary for Seasons following the first Season shall be governed by Article VII, Section 5(c)(1); provided, however, that if the player who is reinstated was dismissed and disqualified during the term of his Rookie Scale Contract, then (x) the number of Seasons in the player's new Contract may not exceed three (3) Seasons plus a Team Option Year and the Salary and Unlikely Bonuses called for in any Season of the player's new Contract, including the Option Year, may not exceed the Salary and Unlikely Bonuses called for during the corresponding Season of his Rookie Scale Contract, and (y) if the new Contract contains terms identical to those contained in the remaining Seasons of the player's Rookie Scale Contract at the time he was dismissed and disqualified, and the Team ultimately exercises the Option following the third Season of the new Contract, then the player's Team shall retain the same rights with respect to such new Contract as it would have retained under Article XI following the completion of the player's Rookie Scale Contract.
  \end{enumerate}
\end{enumerate}

\hypertarget{testing-of-veteran-players.}{%
\section{Testing of Veteran Players.}\label{testing-of-veteran-players.}}

\begin{enumerate}
\def\labelenumi{(\alph{enumi})}
\tightlist
\item
  In addition to the testing procedures set forth in Section 5 above, a Veteran Player may be required to undergo testing for Prohibited Substances by the NBA no more than one (1) time each Season (i) during regular training camp, or, (ii) in the case of a Veteran Player who joins a Team with fewer than fifteen (15) days remaining in regular training camp or who joins a Team during the Regular Season (and provided such Veteran Player had not previously been tested during such training camp or Season), during the first fifteen (15) days after such player reports to his Team.
\item
  In the event that a Veteran Player tests positive for a Drug of Abuse pursuant to this Section 7, he shall immediately be dismissed and disqualified from any association with the NBA or any of its Teams in accordance with the provisions of Section 12(a). If the player tests positive for marijuana or Steroids pursuant to this Section 7, he shall enter the Program and suffer the applicable consequences set forth in Sections 9 or 10 below, as the case may be.
\end{enumerate}

\hypertarget{drugs-of-abuse-program.}{%
\section{Drugs of Abuse Program.}\label{drugs-of-abuse-program.}}

\begin{enumerate}
\def\labelenumi{(\alph{enumi})}
\item
  Voluntary Entry.

  \begin{enumerate}
  \def\labelenumii{(\roman{enumii})}
  \tightlist
  \item
    A player may enter the Drugs of Abuse Program voluntarily at any time by Coming Forward Voluntarily for a problem involving the use of a Drug of Abuse; provided, however, that a player may not Come Forward Voluntarily (A) until he has been selected in an NBA Draft or invited to an NBA training camp; (B) during any period in which an Authorization for Testing as to that player remains in effect pursuant to Section 5 above; (C) during any period in which he remains subject to in-patient or aftercare treatment in Stage 1 of the Drugs of Abuse Program; or (D) after he has reached Stage 2 of the Drugs of Abuse Program.
  \item
    If a player who has not previously entered the Drugs of Abuse Program Comes Forward Voluntarily for a problem involving the use of a Drug of Abuse, he shall enter Stage 1 of the Drugs of Abuse Program.
  \item
    If a player who has not previously entered Stage 2 of the Drugs of Abuse Program, but who has been notified by the Medical Director that he has successfully completed Stage 1 of that Program, Comes Forward Voluntarily for a problem involving the use of a Drug of Abuse, he shall enter Stage 2 of the Drugs of Abuse Program.
  \item
    No penalty of any kind will be imposed on a player as a result of having Come Forward Voluntarily for a problem involving the use of a Drug of Abuse. The foregoing sentence shall not preclude the imposition of a penalty under Section 8(c)(iv) below as a result of the player's entering Stage 2 of the Drugs of Abuse Program, or any penalty called for by this Article XXXIII as a result of conduct by the player that occurs after he has Come Forward Voluntarily.
  \end{enumerate}
\item
  Stage 1.

  \begin{enumerate}
  \def\labelenumii{(\roman{enumii})}
  \tightlist
  \item
    Any player who has entered Stage 1 of the Drugs of Abuse Program shall be required to submit to an evaluation by the Medical Director, provide (or cause to be provided) to the Medical Director such relevant medical and treatment records as the Medical Director may request, and commence the treatment and testing program prescribed by the Medical Director.
  \item
    If a player, within ten (10) days of the date on which he was notified that he had entered Stage 1 of the Drugs of Abuse Program and without a reasonable excuse, fails to comply (in the professional judgment of the Medical Director) with any of the obligations set forth in Section 8(b)(i) above, he shall be suspended until such time as the Medical Director determines that he has fully complied with Section 8(b)(i). If such noncompliance continues without a reasonable excuse (in the professional judgment of the Medical Director) for thirty (30) days from the date on which the player was notified that he had entered Stage 1 of the Drugs of Abuse Program, the player shall (A) advance to Stage 2 of the Drugs of Abuse Program, or (B) the player's Team may, notwithstanding any term or provision in or amendment to the player's Uniform Player Contract, terminate such Contract without any further obligation to pay Compensation, except to pay the Compensation (either Current or Deferred) that may have been earned by the player to the date of termination.
  \item
    Except as provided in this Article XXXIII, no penalty of any kind will be imposed on a player while he is in Stage 1 of the Drugs of Abuse Program and, provided he complies with the terms of his prescribed treatment, he will continue to receive his Compensation during the term of his treatment for a period of up to three (3) months of care in an In-Patient Facility and such aftercare as may be required by the Medical Director.
  \end{enumerate}
\item
  Stage 2.

  \begin{enumerate}
  \def\labelenumii{(\roman{enumii})}
  \tightlist
  \item
    Any player who has entered Stage 2 of the Drugs of Abuse Program shall be required to submit to an evaluation by the Medical Director, provide (or cause to be provided) to the Medical Director such relevant medical and treatment records as the Medical Director may request, and commence the treatment and testing program prescribed by the Medical Director.
  \item
    If a player, within thirty (30) days of the date on which he was notified that he had entered Stage 2 of the Drugs of Abuse Program and without a reasonable excuse, fails to comply (in the professional judgment of the Medical Director) with any of the obligations set forth in Section 8(c)(i) above, he shall immediately be dismissed and disqualified from any association with the NBA or any of its Teams in accordance with the provisions of Section 12(a) below.
  \item
    A player in Stage 2 of the Drugs of Abuse Program shall be suspended during the period of his in-patient treatment and for at least the first six (6) months of his aftercare treatment. The player shall remain suspended during any subsequent period in which he is undergoing treatment that, in the professional judgment of the Medical Director, prevents him from rendering the playing services called for by his Uniform Player Contract.
  \item
    Any subsequent use, possession, or distribution of a Drug of Abuse by a player in Stage 2, even if voluntarily disclosed, or any conduct by a player in Stage 2 that results in his advancing one Stage in the Drugs of Abuse Program, shall result in the player being immediately dismissed and disqualified from any association with the NBA or any of its Teams in accordance with the provisions of Section 12(a) below.
  \end{enumerate}
\item
  Treatment and Testing Program.

  A player who enters the Drugs of Abuse Program shall be required to comply with such in-patient and aftercare program as may be prescribed and supplemented from time to time by the Medical Director. Such program may include random testing for Prohibited Substances other than Steroids, and for alcohol, and such non-testing elements as may be determined in the professional judgment of the Medical Director.
\end{enumerate}

\hypertarget{marijuana-program.}{%
\section{Marijuana Program.}\label{marijuana-program.}}

\begin{enumerate}
\def\labelenumi{(\alph{enumi})}
\item
  Voluntary Entry.

  \begin{enumerate}
  \def\labelenumii{(\roman{enumii})}
  \tightlist
  \item
    A player may enter the Marijuana Program voluntarily at any time by Coming Forward Voluntarily; provided, however, that a player may not Come Forward Voluntarily for a problem involving the use of marijuana (A) until he has been selected in an NBA Draft or invited to an NBA training camp; (B) during any period in which an Authorization for Testing as to that player remains in effect pursuant to Section 5 above; or (C) during any period in which he remains subject to in-patient or aftercare treatment in the Marijuana Program.
  \item
    If a player who has not previously entered the Marijuana Program, or a player who has been notified by the Medical Director that he has successfully completed that Program, Comes Forward Voluntarily for a problem involving the use of marijuana, he shall enter the Marijuana Program.
  \item
    No penalty of any kind will be imposed on a player as a result of having Come Forward Voluntarily for a problem involving the use of marijuana. The foregoing sentence shall not preclude the imposition of any penalty called for by this Article XXXIII as a result of conduct by the player that occurs after he has Come Forward Voluntarily.
  \end{enumerate}
\item
  Treatment.

  \begin{enumerate}
  \def\labelenumii{(\roman{enumii})}
  \tightlist
  \item
    A player who enters the Marijuana Program shall be required to submit to an evaluation by the Medical Director, provide (or cause to be provided) to the Medical Director such relevant medical and treatment records as the Medical Director may request, and commence the treatment and testing program prescribed by the Medical Director. Such program may include random testing for Prohibited Substances other than Steroids, and for alcohol, and such non-testing elements as may be determined in the professional judgment of the Medical Director.
  \item
    If a player, within five (5) days of the date on which he was notified that he had entered the Marijuana Program and without a reasonable excuse, fails to comply (in the professional judgment of the Medical Director) with any of the obligations set forth in the first sentence of Section 9(b)(i) above, he shall be fined \$10,000; if the player thereafter fails to comply, without a reasonable excuse, with such obligations (in the professional judgment of the Medical Director) within eight (8) days of such notification, he shall be fined an additional \$10,000; and for each additional day beyond the 8th day that the player, without a reasonable excuse, fails to comply with such obligations (in the professional judgment of the Medical Director), he shall be fined an additional \$10,000. The total amount of such fines may not exceed the player's total Compensation.
  \end{enumerate}
\item
  Penalties.

  Any player who (i) tests positive for marijuana pursuant to Section 5 (Reasonable Cause Testing), Section 6 (Testing of First-Year Players), or Section 7 (Testing of Veteran Players), (ii) is adjudged by the Grievance Arbitrator pursuant to Section 6(e) to have used or possessed marijuana, or (iii) has been convicted of (including a plea of guilty, no contest or nolo contendere to) the use or possession of marijuana in violation of the law, shall suffer the following penalties:

  \begin{enumerate}
  \def\labelenumii{(\Alph{enumii})}
  \tightlist
  \item
    For the first such violation, the player shall be required to enter the Marijuana Program;
  \item
    For the second such violation, the player shall be fined \$15,000 and required to enter the Marijuana Program;
  \item
    For the third and any subsequent such violation, the player shall be suspended for five (5) games and required to enter the Marijuana Program.
  \end{enumerate}
\end{enumerate}

\hypertarget{use-or-possession-of-steroids.}{%
\section{Use or Possession of Steroids.}\label{use-or-possession-of-steroids.}}

\begin{enumerate}
\def\labelenumi{(\alph{enumi})}
\item
  Voluntary Entry.

  \begin{enumerate}
  \def\labelenumii{(\roman{enumii})}
  \tightlist
  \item
    A player may enter the Steroids Program voluntarily at any time by Coming Forward Voluntarily; provided, however, that a player may not Come Forward Voluntarily for a problem involving the use of Steroids (i) until he has been selected in an NBA Draft or invited to an NBA training camp; (ii) during any period in which an Authorization for testing as to that player remains in effect pursuant to Section 5 above; or (iii) during any period in which he remains subject to in-patient or aftercare treatment in the Steroids Program.
  \item
    If a player who has not previously entered the Steroids Program Comes Forward Voluntarily for a problem involving the use of Steroids, he shall enter the Steroids Program.
  \item
    No penalty of any kind will be imposed on a player as a result of having Come Forward Voluntarily for a problem involving the use of Steroids. The foregoing sentence shall not preclude the imposition of any penalty called for by this Article XXXIII as a result of conduct by the player that occurs after he has Come Forward Voluntarily.
  \end{enumerate}
\item
  Treatment.

  \begin{enumerate}
  \def\labelenumii{(\roman{enumii})}
  \tightlist
  \item
    A player who enters the Steroids Program shall be required to submit to an evaluation by the Medical Director, provide (or cause to be provided) to the Medical Director such relevant medical and treatment records as the Medical Director may request, and commence the treatment and testing program prescribed by the Medical Director. Such program may include random testing for Steroids and such non-testing elements as may be determined in the professional judgment of the Medical Director.
  \item
    If a player, within five (5) days of the date on which he was notified that he had entered the Steroids Program and without a reasonable excuse, fails to comply (in the professional judgment of the Medical Director) with any of the obligations set forth in the first sentence of Section 10(b)(i) above, he shall be fined \$10,000; if the player, without a reasonable excuse, thereafter fails to comply with such obligations (in the professional judgment of the Medical Director) within eight (8) days of such notification, he shall be fined an additional \$10,000; and for each additional day beyond the 8th day that the player, without a reasonable excuse, fails to comply with such obligations (in the professional judgment of the Medical Director), he shall be fined an additional \$10,000. The total amount of such fines shall not exceed the player's total Compensation.
  \end{enumerate}
\item
  Penalties.

  Any player who (i) tests positive for Steroids pursuant to Section 5 (Reasonable Cause Testing), Section 6 (Testing of First-Year Players), or Section 7 (Testing of Veteran Players), or (ii) is adjudged by the Grievance Arbitrator pursuant to Section 5(e) to have used or possessed Steroids, shall suffer the following penalties:

  \begin{enumerate}
  \def\labelenumii{(\Alph{enumii})}
  \tightlist
  \item
    for the first such violation, the player shall be suspended for five (5) games and required to enter the Steroids Program;
  \item
    for the second such violation, the player shall be suspended for ten (10) games and required to enter the Steroids Program;
  \item
    for the third and any subsequent violation, the player shall be suspended for twenty-five (25) games and required to enter the Steroids Program.
  \end{enumerate}
\end{enumerate}

\hypertarget{noncompliance-with-treatment.}{%
\section{Noncompliance with Treatment.}\label{noncompliance-with-treatment.}}

\begin{enumerate}
\def\labelenumi{(\alph{enumi})}
\item
  Drugs of Abuse.

  \begin{enumerate}
  \def\labelenumii{(\roman{enumii})}
  \tightlist
  \item
    Any player who, after entering Stage 1 or Stage 2 of the Drugs of Abuse Program, fails to comply with his treatment or his aftercare program as prescribed and determined by the Medical Director, shall be suspended. Such suspension shall continue until the player has, in the professional judgment of the Medical Director, resumed full compliance with his treatment program.
  \item
    Notwithstanding Section 11(a) above, any player who in the professional judgment of the Medical Director, after entering Stage 1 or Stage 2 of the Drugs of Abuse Program, fails to comply with his treatment program through (A) a pattern of behavior that demonstrates a mindful disregard for his treatment responsibilities, or (B) a positive test for a Prohibited Substance other than Steroids that is not clinically expected by the Medical Director, shall suffer the following penalties:

    \begin{enumerate}
    \def\labelenumiii{(\arabic{enumiii})}
    \tightlist
    \item
      if the player is in Stage 1 of the Drugs of Abuse Program, he shall advance to Stage 2 and be suspended until, in the professional judgment of the Medical Director, he has resumed full compliance with his treatment program; or
    \item
      if the player already is in Stage 2 of the Drugs of Abuse Program, he shall immediately be dismissed and disqualified from any association with the NBA or any of its Teams in accordance with the provisions of Section 12(a) below.
    \end{enumerate}
  \end{enumerate}
\item
  Marijuana.

  \begin{enumerate}
  \def\labelenumii{(\roman{enumii})}
  \tightlist
  \item
    Any player who, after entering the Marijuana Program, fails to comply (without a reasonable excuse) with his treatment program as prescribed and determined by the Medical Director, shall be fined \$1,000 for each day that he fails to comply. Such fines shall continue until the player has, in the professional judgment of the Medical Director, resumed full compliance with his treatment program. The total amount of such fines shall not exceed the player's total Compensation.
  \item
    Notwithstanding Section 11(b)(i) above, any player who, after entering the Marijuana Program, fails to comply with his treatment program through (A) a pattern of behavior that demonstrates a mindful disregard for his treatment responsibilities, or (B) a positive test for marijuana that is not clinically expected by the Medical Director, shall suffer the following penalties:

    \begin{enumerate}
    \def\labelenumiii{(\arabic{enumiii})}
    \tightlist
    \item
      if the player has not previously been fined \$15,000 under Section 9(c) above or this Section 11(b)(ii), a fine of \$15,000;
    \item
      if the player has previously been fined \$15,000 under Section 9(c) above or this Section 11(b)(ii), a suspension of five (5) games; or
    \item
      if the player has previously been suspended for five (5) games under Section 9(c) above or this Section 11(b)(ii), an indefinite number of five-game (5) suspensions that shall continue until, in the professional judgment of the Medical Director, the player resumes full compliance with his treatment program.
    \end{enumerate}
  \item
    In addition to any consequence to the player under Section 11(b)(ii) above, any player who has entered the Marijuana Program but not the Drugs of Abuse Program, and tests positive for a Drug of Abuse in any test conducted by the Medical Director, shall enter Stage 1 of the Drugs of Abuse Program.
  \end{enumerate}
\item
  Steroids.

  \begin{enumerate}
  \def\labelenumii{(\roman{enumii})}
  \tightlist
  \item
    Any player who, after entering the Steroids Program, fails to comply (without a reasonable excuse) with his treatment program as prescribed and determined by the Medical Director, shall be fined \$5,000 per day for each day that he fails to comply. Such fines shall continue until the player has, in the professional judgment of the Medical Director, resumed full compliance with his treatment program. The total amount of such fines shall not exceed the player's total Compensation.
  \item
    Notwithstanding Section 11(c)(i) above, any player who, after entering the Steroids Program, fails to comply with his treatment program as prescribed and determined by the Medical Director through (A) a pattern of behavior that demonstrates a mindful disregard of his treatment responsibilities, or (B) a positive test for Steroids that was not clinically expected by the Medical Director, shall suffer the following penalties:

    \begin{enumerate}
    \def\labelenumiii{(\arabic{enumiii})}
    \tightlist
    \item
      if the player has not previously been suspended for five (5) games under Section 10(c) above or this Section 11(c)(ii), a suspension of five (5) games;
    \item
      if the player has previously been suspended for five (5) games under Section 10(c) above or this Section 11(c)(ii), a suspension of ten (10) games;
    \item
      if the player has previously been suspended for ten (10) games under Section 10(c) above or this Section 11(c)(ii), a suspension of twenty-five (25) games; or
    \item
      if the player has previously been suspended for twenty-five (25) games under Section 10(c) above or this Section 11(c)(ii), an indefinite suspension that shall continue until, in the professional judgment of the Medical Director, the player resumes full compliance with his treatment program.
    \end{enumerate}
  \end{enumerate}
\item
  Directed Testing.

  Any player who, after entering the Program, and without a reasonable explanation satisfactory to the Medical Director, (i) fails to appear for any of his Team's scheduled games, or (ii) misses, during any consecutive seven-day (7) period, any two (2) airplane flights on which his team is scheduled to travel, any two (2) Team practices, or a combination of any one (1) practice and any one (1) Team flight, shall appear at his Team's office and submit to a urine test, to be conducted by the NBA, within twenty-four (24) hours of the game for which the player failed to appear or within twenty-four (24) hours of the second missed flight or practice or combination of one missed flight and one (1) missed practice, as the case may be. If such player is ``on the road'' with his Team, the player shall be required to telephone the Medical Director within the required 24-hour period, advise the Medical Director of his location, and comply with the Medical Director's instructions as to where and when he will be tested. If any test conducted pursuant to this Section 11(d) is positive: (x) for a Drug of Abuse (for a player in the Drugs of Abuse Program), then the player shall suffer the applicable consequences set forth in Section 11(a)(ii) above; (y) for marijuana (for a player in the Marijuana Program), then the player shall suffer the applicable consequences set forth in Section 11(b)(ii) above; (z) for Steroids (for a player in the Steroids Program), then the player will suffer the applicable consequences set forth in Section 11(c)(ii) above.
\end{enumerate}

\hypertarget{dismissal-and-disqualification.}{%
\section{Dismissal and Disqualification.}\label{dismissal-and-disqualification.}}

\begin{enumerate}
\def\labelenumi{(\alph{enumi})}
\tightlist
\item
  A player who, under the terms of this Agreement, is ``dismissed and disqualified from any association with the NBA or any of its Teams in accordance with the provisions of Section 12(a)'' shall, without exception, immediately be so dismissed and disqualified for a period of not less than two (2) years, and such player's Player Contract shall be rendered null and void and of no further force or effect (subject to the provisions of paragraph 8 of the Uniform Player Contract). Such dismissal and disqualification shall be mandatory and may not be rescinded or reduced by the player's Team or the NBA.
\item
  In addition to any other provision of this Agreement requiring that a player be dismissed and disqualified from any association with the NBA or any of its Teams in accordance with the provisions of Section 12(a) above, a player will also be dismissed and disqualified under Section 12(a) if he is convicted of (including a plea of guilty, no contest, or nolo contendere to) a crime involving the use or possession of a Prohibited Substance other than marijuana.
\end{enumerate}

\hypertarget{reinstatement.}{%
\section{Reinstatement.}\label{reinstatement.}}

\begin{enumerate}
\def\labelenumi{(\alph{enumi})}
\item
  After a period of at least two (2) years from the time of a player's dismissal and disqualification under Section 12(a) above, such player may apply for reinstatement as a player in the NBA. However, such player shall have no right to reinstatement under any circumstance and the reinstatement shall be granted only with the prior approval of both the NBA and the Players Association. The approval of the NBA and the Players Association shall rest in their absolute and sole discretion, and their decision shall be final, binding, and unappealable. Among the factors that may be considered by the NBA and the Players Association in determining whether to grant reinstatement are (without limitation): the circumstances surrounding the player's dismissal and disqualification; whether the player has satisfactorily completed a treatment and rehabilitation program; the player's conduct since his dismissal, including the extent to which the player has since comported himself as a suitable role model for youth; and whether the player is judged to possess the requisite qualities of good character and morality.
\item
  A player will not be reinstated unless he can demonstrate, by proof of random urine testing acceptable to the Medical Director (conducted on at least a weekly basis), that he has not tested positive (i) for a Prohibited Substance within the twelve (12) months prior to the submission of his application for reinstatement and during any period while his application is being reviewed, and (ii) if the Medical Director deems it necessary in his or her professional judgment, for alcohol for the six (6) months prior to the submission of his application for reinstatement and during any period while his application is being reviewed.
\item
  The granting of an application for reinstatement may be conditioned upon random testing of the player or such other terms as may be agreed upon by the NBA and the Players Association, whether or not such terms are contemplated by the terms of this Article XXXIII.
\item
  \begin{enumerate}
  \def\labelenumii{(\roman{enumii})}
  \tightlist
  \item
    A player who has been reinstated pursuant to this Section 13 shall, immediately upon such reinstatement, notify the Team to which he was under contract at the time of his dismissal and disqualification (the ``previous Team''). Upon receipt of such notification, and subject to Section 13(d)(ii) below, the previous Team shall then have thirty (30) days in which to make a Tender to the player with a stated term of at least one (1) full NBA Season (or, in the event the Tender is made during a Season, of at least the rest of that Season) and calling for a Salary in the first Season covered by the Tender at least equal to the lesser of (x) the player's Salary for the Salary Cap Year in which he was dismissed and disqualified, or (y) the Estimated Average Player Salary during the then-current Season, in either case reduced on a pro rata basis if the first Season covered by the Tender is a partial Season, but not greater than the Salary provided in Section 13(d)(iii) below. If the previous Team makes such a Tender, it shall, for a period of one (1) year from the date of the Tender, be the only NBA Team with which the player may negotiate and sign a Player Contract. If the player does not sign a Player Contract with the previous Team within the year following such Tender, then the player shall thereupon be deemed a Restricted or an Unrestricted Free Agent, in accordance with the provisions of Article XI. If the previous Team fails to make a Required Tender, the player shall become an Unrestricted Free Agent.
  \item
    Notwithstanding anything to the contrary in Section 13(d)(i) above, the 30-day period for the previous Team to make a Tender shall be tolled if (x) on the date the player serves the notice required by Section 13(d)(i), he is under contract to a professional basketball team not in the NBA, or (y) the player signs a contract with a professional basketball team not in the NBA at any point after the date on which he serves the notice required by Section 13(d)(i) and before the date on which the previous Team makes a Tender. If the 30-day period for making a Tender is tolled pursuant to the preceding sentence, the period shall remain tolled until the date on which the player notifies the Team that he is available to sign a Player Contract with and begin rendering playing services for such Team immediately, provided that such notice will not be effective until the player is under no contractual or other legal impediment to sign with and begin rendering playing services for such Team.
  \item
    A player who is reinstated pursuant to this Section 13 may enter into a Player Contract with his previous Team that provides for a Salary and Unlikely Bonuses for the first Season of up to the player's Salary and Unlikely Bonuses, respectively, for the Salary Cap Year in which he was dismissed and disqualified (reduced on a pro rata basis if the first Season of the new Contract is a partial Season), even if the Team has a Team Salary at or above the Salary Cap or such Player Contract causes the Team to have a Team Salary above the Salary Cap. If the player and the previous Team enter into such Player Contract and such Contract covers more than one Season, increases and decreases in Salary for Seasons following the first Season shall be governed by Article VII, Section 5(c)(i); provided, however, that if the player who is reinstated was dismissed and disqualified during the term of his Rookie Scale Contract, then (x) the number of Seasons in the player's new Contract may not exceed the number of Seasons (including the Team Option Year) that remained under the player's Rookie Scale Contract at the time he was dismissed and disqualified, and the Salary called for in any Season of the player's new Contract (including any Option Year), may not exceed the Salary called for during the corresponding Season of his Rookie Scale Contract, and (y) if the new Contract contains terms identical to those contained in the remaining Seasons of the player's Rookie Scale Contract at the time he was dismissed and disqualified, and the player's Team ultimately exercises its Option, then such Team shall retain the same rights with respect to such new Contract as it would have retained under Article XI following the completion of the player's Rookie Scale Contract.
  \end{enumerate}
\end{enumerate}

\hypertarget{exclusivity-of-the-program.}{%
\section{Exclusivity of the Program.}\label{exclusivity-of-the-program.}}

\begin{enumerate}
\def\labelenumi{(\alph{enumi})}
\tightlist
\item
  Except as expressly provided in this Article XXXIII, there shall be no other screening or testing for Prohibited Substances conducted by the NBA or any Team, and no player shall be required to undergo such screening or testing. If any Team is found to have tested a player surreptitiously, the NBA will impose a substantial fine not to exceed \$500,000 upon such Team pursuant to the NBA's Constitution and By-Laws.
\item
  The penalties set forth in this Article XXXIII shall be the exclusive penalties to be imposed upon a player for the use, possession or distribution of a Prohibited Substance.
\item
  No Uniform Player Contract entered into after the date hereof shall include any term or provision that modifies, contradicts, changes, or is inconsistent with paragraph 8 of such Contract or provides for the testing of a player for illegal substances. Any term or provision of a currently effective Uniform Player Contract that is inconsistent with paragraph 8 of such Contract shall be deemed null and void only to the extent of the inconsistency.
\end{enumerate}

\hypertarget{additional-bases-for-testing.}{%
\section{Additional Bases for Testing.}\label{additional-bases-for-testing.}}

\begin{enumerate}
\def\labelenumi{(\alph{enumi})}
\tightlist
\item
  Any player who seeks treatment outside the Program for a problem involving a Prohibited Substance shall, as directed by the NBA (after notice to the Players Association), submit himself to an evaluation by the Medical Director and provide (or cause to be provided) to the Medical Director such medical and treatment records as the Medical Director may request. The Medical Director may, in his or her professional judgment, also require such a player, without prior notice, to submit to testing for Prohibited Substances, provided that the frequency of such testing shall not exceed three (3) times per week and the duration of such testing shall not exceed one (1) year from the date of the player's initial evaluation by the Medical Director.
\item
  If, pursuant to Section 15(a) above, a player (i) tests positive for a Drug of Abuse; (ii) refuses or fails to submit to an evaluation or provide (or cause to be provided) the information requested by the Medical Director; or (iii) submits to treatment outside the Program for a substance abuse problem involving a Prohibited Substance, but does not Come Forward Voluntarily within 60 days of being requested to do so by the NBA (with notice to the Players Association), the player shall advance two stages in the Drugs of Abuse Program---i.e., the player shall enter Stage 2 of the Drugs of Abuse Program (if the player had not previously entered Stage 1 of such Program), and the player shall be dismissed and disqualified from any association with the NBA or any of its Teams in accordance with the provisions of Section 12(a) above (if the player had previously entered Stage 1 or Stage 2 of such Program).
\item
  If, pursuant to Section 15(a) above, a player tests positive for marijuana, he shall suffer the consequences set forth in Section 9(c)(B) above or, if the player had previously been penalized under Section 9(c)(B), the consequences set forth in Section 9(c)(C) above.
\item
  If, pursuant to Section 15(a) above, a player tests positive for Steroids, he shall suffer the consequences set forth in Section 10(c)(B) above or, if the player had previously been penalized under Section 10(c)(B), the consequences set forth in Section 10(c)(C) above.
\item
  Nothing in this Section 15 shall limit or otherwise affect any of the provisions of Section 5 (Reasonable Cause Testing).
\end{enumerate}

\hypertarget{additional-prohibited-substances.}{%
\section{Additional Prohibited Substances.}\label{additional-prohibited-substances.}}

\begin{enumerate}
\def\labelenumi{(\alph{enumi})}
\tightlist
\item
  At any time during the term of this Agreement, either the NBA or the Players Association may convene a meeting of the Prohibited Substances Committee to request that a substance or substances be added to the list of Prohibited Substances set forth on Exhibit I-2 to this Agreement. Any such addition may only include (i) a substance that is illegal or (ii) a substance that is physically harmful to players and improperly performance-enhancing. The determination of the Committee to add to the list of Prohibited Substances shall be made by a majority vote of all five Committee members, and shall be final, binding, and unappealable. Players will receive notice of any addition to the list of Prohibited Substances six (6) months prior to the date on which such addition becomes effective under this Article XXXIII.
\item
  Prior to the commencement of the 1999-2000 NBA Season, the Prohibited Substances Committee shall establish a list of Steroids that will be considered Prohibited Substances under this Article XXXIII. The determination of this list by the Committee shall be made by a majority vote of all five Committee members, and shall be final, binding, and unappealable.
\end{enumerate}

\hypertarget{recognition-clause}{%
\chapter{RECOGNITION CLAUSE}\label{recognition-clause}}

The NBA recognizes the Players Association as the exclusive collective bargaining representative of persons who are employed by NBA Teams as professional basketball players (and/or who may become so employed during the term of this Agreement or any extension thereof); and the Players Association warrants that it is duly empowered to enter into this Agreement for and on behalf of such persons. The NBA and the Players Association agree that, notwithstanding the foregoing, such persons and NBA Teams may, on an individual basis, bargain with respect to and agree upon the provisions of Player Contracts, but only as and to the extent permitted by this Agreement.

\hypertarget{savings-clause}{%
\chapter{SAVINGS CLAUSE}\label{savings-clause}}

In the event that any provision hereof is found to be inconsistent with the Internal Revenue Code (or the rules and regulations issued thereunder), the National Labor Relations Act, any other federal, state, provincial, or local statute or ordinance, or the rules and regulations of any other government agency, or is determined to have an adverse effect upon the right of the NBA (or any successor entity) to a tax exemption under Section 501(c)(6) of the Internal Revenue Code of 1954 (or any successor section of like import), then the parties hereto agree to make such changes as are necessary to avoid such inconsistency or to obtain or maintain such exemption retaining, to the extent possible, the intention of such provision.

\hypertarget{player-agents}{%
\chapter{PLAYER AGENTS}\label{player-agents}}

\hypertarget{approval-of-player-contracts.}{%
\section{Approval of Player Contracts.}\label{approval-of-player-contracts.}}

The NBA shall not approve any Player Contract between a player and a Team unless such player: (i) is represented in the negotiations with respect to such Player Contract by an agent or representative duly certified by the Players Association in accordance with the Players Association's Agent Regulation Program and authorized to represent him; or (ii) acts on his own behalf in negotiating such Player Contract.

\hypertarget{fines.}{%
\section{Fines.}\label{fines.}}

The NBA shall impose a fine of \$20,000 upon any Team that negotiates a Player Contract with an agent or representative not certified by the Players Association in accordance with the Players Association's Agent Regulation Program if, at the time of such negotiations, such Team either (i) knows that such agent or representative has not been so certified or (ii) fails to make reasonable inquiry of the NBA as to whether such agent or representative has been so certified. Notwithstanding the preceding sentence, in no event shall any Team be subject to a fine if the Team negotiates a Player Contract with the agent designated as the player's authorized agent on the then-current agent list provided by the Players Association to the NBA in accordance with Section 4 below.

\hypertarget{indemnity.}{%
\section{Indemnity.}\label{indemnity.}}

The Players Association agrees to indemnify and hold harmless the NBA, its Teams and each of its and their respective past, present and future owners (direct and indirect) acting in their capacity as Team owners, officers, directors, trustees, employees, successors, agents, attorneys, heirs, administrators, executors and assigns, from any and all claims of any kind arising from or relating to (i) the Players Association's Agent Regulation Program, and (ii) the provisions of this Article, including, without limitation, any judgments, costs and settlements, provided that the Players Association is immediately notified of such claim in writing (and, in no event later than five (5) days from the receipt thereof), is given the opportunity to assume the defense thereof, and the NBA and/or its Teams (whichever is sued) use their best efforts to defend such claim, and do not admit liability with respect to and do not settle such claim without the prior written consent of the Players Association.

\hypertarget{agent-lists.}{%
\section{Agent Lists.}\label{agent-lists.}}

The Players Association agrees to provide the NBA League Office with a list of (i) all agents certified under the Players Association's Agent Regulation Program, and (ii) the players represented by each such agent. Such list shall be updated once every two (2) weeks from the day after the NBA Finals to the first day of the next succeeding Regular Season and shall be updated once every month at all other times.

\hypertarget{confirmation-by-the-players-association.}{%
\section{Confirmation by the Players Association.}\label{confirmation-by-the-players-association.}}

If the NBA has reason to believe that the agent representing a player in Contract negotiations is not a certified agent or is not the agent authorized to represent the player, the NBA may, at its election, request in writing from the Players Association confirmation as to whether the agent that represented the player in the Contract negotiations is in fact the player's certified representative. If within three (3) business days of the date the Players Association receives such written request, the NBA does not receive a written response from the Players Association stating that the agent that represented the player is not the player's certified representative, then the NBA shall be free to act as if the agent is the Player's confirmed certified representative.

\hypertarget{group-licensing-rights}{%
\chapter{GROUP LICENSING RIGHTS}\label{group-licensing-rights}}

\hypertarget{rights-granted.}{%
\section{Rights Granted.}\label{rights-granted.}}

The Players Association, on behalf of present and future NBA players, agrees that NBA Properties, Inc., has the exclusive right to use the ``Player's Attributes'' of each NBA player as such term is defined and for such group licensing purposes as are set forth in the Agreement between NBA Properties, Inc., and the National Basketball Players Association, dated as of September 18, 1995, as amended January 20, 1999 (the ``Group License Agreement'').

\hypertarget{player-appearances.}{%
\section{Player Appearances.}\label{player-appearances.}}

A player may, during each Contract Year covered by a Player Contract to which he is a party, be required (a) to make up to four (4) appearances at the request of and in connection with licensing arrangements made by NBA Properties, Inc., in accordance with the terms of the Group License Agreement, and (b) to make up to two (2) additional appearance at the request of NBA Properties in accordance with paragraph 13(d) of a Uniform Player Contract and Article II, Section 8. Any appearance that a player is required to make shall comply with the terms of Article II, Section 8, and when a player makes an appearance in accordance with that Section, he shall be paid at least \$1,000. When a player fails, without reasonable excuse, to appear or reasonably to cooperate during an appearance at any of the licensing appearances referred to in this Section, he may be fined for each failure in an amount up to \$10,000.

\hypertarget{uniform.}{%
\section{Uniform.}\label{uniform.}}

During any NBA game or practice, including warm-up periods and going to and from the locker room to the playing floor, a player shall wear only the Uniform as supplied by his Team. For purposes of the preceding sentence only, ``Uniform'' means all clothing and other items (such as kneepads, wristbands and headbands, but not including Sneakers) worn by a player during an NBA game or practice. ``Sneakers'' means athletic shoes of the type worn by players while playing an NBA game.

\hypertarget{integration-entire-agreement-interpretation-choice-of-law}{%
\chapter{INTEGRATION, ENTIRE AGREEMENT, INTERPRETATION, CHOICE OF LAW}\label{integration-entire-agreement-interpretation-choice-of-law}}

\chaptermark{INTEGRATION, ENTIRE AGREEMENT \ldots}

\hypertarget{integration-entire-agreement.}{%
\section{Integration, Entire Agreement.}\label{integration-entire-agreement.}}

This Agreement, together with the exhibits hereto, constitutes the entire understanding between the parties and all understandings, conversations and communications, proposals, and counterproposals, oral and written (including any draft of this Agreement) between the Members of the NBA and the Players Association, or on behalf of them, are merged into and superseded by this Agreement and shall be of no force or effect, except as expressly provided herein. No such understandings, conversations, communications, proposals, counterproposals or drafts shall be referred to in any proceeding by the parties. Further, no understanding contained in this Agreement shall be modified, altered or amended, except by a writing signed by the party against whom enforcement is sought.

\hypertarget{interpretation.}{%
\section{Interpretation.}\label{interpretation.}}

The NBA and Players Association recognize that this Agreement is separate and distinct from a collective bargaining agreement entered into between WNBA, LLC (``WNBA'') and the Women's National Basketball Players Association (``WNBPA''), and intend for this Agreement to be interpreted without reference to the WNBA/WNBPA collective bargaining agreement (or to any other current or prior agreement between the WNBA or WNBAEnterprises, LLC, on the one hand, and the WNBPA on the other), or to any of its or their provisions, or to the fact that a subject was not or is not covered by or included in agreements between the WNBA (or WNBA Enterprises, LLC) and the WNBPA. Accordingly, the parties agree that they will make no reference to any agreement between the WNBA (or WNBA Enterprises, LLC) and the WNBPA or to any provisions thereof (or to the fact that a particular provision was not or is not included in any such agreement), or to any practice or policy of the WNBA (or WNBA Enterprises, LLC) or the WNBPA, in any arbitral, judicial, administrative, or other proceeding, including, without limitation, a proceeding brought under Articles XXXI or XXXII of this Agreement. The parties further agree that no such agreement, provisions (or absence of provisions), practices, or policies may be relied upon by any decisionmaker in such proceedings.

\hypertarget{choice-of-law.}{%
\section{Choice of Law.}\label{choice-of-law.}}

This Agreement is made under and shall be governed by the internal law of the State of New York, except where federal law may govern.

\hypertarget{term-of-agreement}{%
\chapter{TERM OF AGREEMENT}\label{term-of-agreement}}

\hypertarget{expiration-date.}{%
\section{Expiration Date.}\label{expiration-date.}}

Except as provided in Section 2 of this Article XXXIX or as otherwise expressly provided herein, this Agreement shall be effective from January 20, 1999 and shall continue in full force and effect through June 30, 2004.

\hypertarget{nba-option-to-extend.}{%
\section{NBA Option to Extend.}\label{nba-option-to-extend.}}

The NBA shall have the option to extend this Agreement for one (1) year (i.e., through June 30, 2005) by serving written notice of its exercise of such option on the Players Association on or before December 15, 2003.

\hypertarget{termination-by-players-associationanti-collusion.}{%
\section{Termination by Players Association/Anti-Collusion.}\label{termination-by-players-associationanti-collusion.}}

In the event the conditions of Article XIV, Section 16 are satisfied, the Players Association shall have the right to terminate this Agreement. To execute such a termination, the Players Association may serve upon the NBA written notice of termination within thirty (30) days after the System Arbitrator's report finding the requisite conditions (pursuant to Article XIV, Section 16) becomes final and any appeals therefrom have been exhausted. In the absence of a System Arbitrator, the Players Association shall have the option to execute such a termination by serving upon the NBA written notice of such termination within thirty (30) days after any decision by a court finding the requisite conditions (pursuant to Article XIV, Section 16). In the latter situation, if the finding of the court is reversed on appeal, the Agreement shall be immediately reinstated and both parties reserve their rights with respect to any conduct by the other party during the period from the termination notice to the date upon which the Agreement was reinstated.

\hypertarget{termination-by-players-associationaverage-player-salary}{%
\section{Termination by Players Association/Average Player Salary}\label{termination-by-players-associationaverage-player-salary}}

\begin{enumerate}
\def\labelenumi{(\alph{enumi})}
\tightlist
\item
  If with respect to the 1999-2000 NBA Season, or any Season thereafter during the term of this Agreement, the Players Association establishes in a proceeding before the System Arbitrator that the Average Player Salary for such Season was less than the average player salary for the last fully-completed season in Major League Baseball (``MLB''), the National Football League (``NFL'') or the National Hockey League (``NHL''), it shall have the right the terminate this Agreement by serving written notice of termination on the NBA within thirty (30) days after the System Arbitrator's determination becomes final and any appeals therefrom have been exhausted. A proceeding under this Section 4(a) shall be commenced on or before the September 1 immediately following the Season to which it pertains, and any termination shall be effective as of the June 30 following the NBA's receipt of the required notice.
\item
  In any proceeding brought pursuant to this Section 4:

  \begin{enumerate}
  \def\labelenumii{(\roman{enumii})}
  \tightlist
  \item
    Calculation of the Average Player Salary shall be based upon Total Salaries and Benefits less Benefits as set forth in the Audit Report for the NBA Season immediately preceding the commencement of such proceeding (or, if that Audit Report has not been submitted to the parties by the time the proceeding is commenced, on Estimated Total Salaries and Benefits less Benefits (as estimated) as set forth in the Interim Audit Report for the NBA Season immediately preceding the commencement of the proceeding); and
  \item
    Calculation of the average player salary in the other professional sports leagues shall be based upon the total team salaries for the last fully-completed season in each such league, with those salaries computed in the manner that Salary is computed in accordance with the provisions of this Agreement (except that bonuses shall be included only to the extent actually earned), and with such total team salaries being divided by (x) for MLB, the number of teams in MLB multiplied by 26, (y) for the NFL, the number of teams in the NFL multiplied by 56, and (z) for the NHL, the number of teams in the NHL multiplied by 26.
  \end{enumerate}
\end{enumerate}

\hypertarget{mutual-right-of-termination.}{%
\section{Mutual Right of Termination.}\label{mutual-right-of-termination.}}

If at any time during the term of this Agreement any provision contained in Article VII, X, XI and XIV of this Agreement is enjoined, vacated, declared null and void or is rendered unenforceable by any court of competent jurisdiction, then either the NBA or the Players Association shall have the right to terminate this Agreement by serving upon the other party written notice of termination within thirty (30) days.

\hypertarget{no-obligation-to-terminate-no-waiver.}{%
\section{No Obligation to Terminate; No Waiver.}\label{no-obligation-to-terminate-no-waiver.}}

The grant to either party of a right or option to terminate pursuant to the provisions of this Article shall not carry with it the obligation to exercise that right or option; and the failure of the NBA or the Players Association to exercise any right or option to terminate this Agreement with respect to any playing season in accordance with this Article shall not be deemed a waiver of or in any way impair or prejudice the NBA or the Players Association's right or option, if any, to terminate this Agreement in accordance with this Article with respect to any succeeding season.

\hypertarget{expansion}{%
\chapter{EXPANSION}\label{expansion}}

The NBA may determine during the term of this Agreement to expand the number of Teams and to have existing Teams make available for assignment to any such Expansion Teams the Player Contracts of a certain number of Veterans under the same terms and in the same manner that Player Contracts were made available to the Vancouver and Toronto expansion teams pursuant to the 1995 NBA/NBPA Collective Bargaining Agreement.

\hypertarget{player-team-cooperation}{%
\chapter{PLAYER-TEAM COOPERATION}\label{player-team-cooperation}}

\hypertarget{player-team-planning-committee.}{%
\section{Player-Team Planning Committee.}\label{player-team-planning-committee.}}

\begin{enumerate}
\def\labelenumi{(\alph{enumi})}
\tightlist
\item
  A Player-Team Planning Committee (hereinafter ``Planning Committee'') shall be established for the purpose of examining and discussing issues arising with respect to the implementation of this Agreement, improving relations between players and Teams, and fostering the growth and success of the NBA.
\item
  The Planning Committee shall consist of eight (8) members: the Commissioner of the NBA and three (3) NBA Team owners, and the Executive Director of the Players Association and three (3) NBA players. The respective owners and players who shall be members of the Planning Committee will be selected, and the lengths of their terms fixed, under such rules as the NBA and the Players Association separately establish; the original members of the Committee will be selected within thirty (30) days following the execution of this Agreement. The Planning Committee will hold regular face-to-face meetings at least twice each year on dates and at sites mutually agreeable to the Committee members. The meetings may be attended by staff members and advisors of the NBA and the Players Association.
\end{enumerate}

\hypertarget{other}{%
\chapter{OTHER}\label{other}}

\hypertarget{headings-and-organization.}{%
\section{Headings and Organization.}\label{headings-and-organization.}}

The headings and organization of this Agreement are solely for the convenience of the parties, and shall not be deemed part of, or considered in construing or interpreting, this Agreement.

\hypertarget{time-periods.}{%
\section{Time Periods.}\label{time-periods.}}

Unless specifically stated otherwise, the specification of any time period in this Agreement shall include any non-business days within such period, except that any deadline falling on a Saturday, Sunday, or Federal Holiday shall be deemed to fall on the following business day.

\hypertarget{exhibits.}{%
\section{Exhibits.}\label{exhibits.}}

All of the Exhibits hereto are an integral part of this Agreement and of the agreement of the parties thereto.

NATIONAL BASKETBALL ASSOCIATION\\
By: /s/ DAVID J. STERN\\
David J. Stern\\
Commissioner

NATIONAL BASKETBALL PLAYERS ASSOCIATION\\
By: /s/ PATRICK EWING\\
Patrick Ewing\\
President

\hypertarget{appendix-appendix}{%
\appendix}


\hypertarget{national-basketball-association-uniform-player-contract}{%
\chapter{NATIONAL BASKETBALL ASSOCIATION UNIFORM PLAYER CONTRACT}\label{national-basketball-association-uniform-player-contract}}

\hypertarget{section}{%
\section{}\label{section}}

THIS AGREEMENT made this \_\_\_ day of \_\_\_\_\_\_\_\_, \_ is by and between \_\_\_\_\_\_\_\_\_\_\_ (hereinafter called the ``Team''), a member of the National Basketball Association (hereinafter called the ``NBA'' or ``League'') and \_\_\_\_\_\_\_\_\_\_\_, an individual whose address is shown below (hereinafter called the ``Player''). In consideration of the mutual promises hereinafter contained, the parties hereto promise and agree as follows:

\begin{enumerate}
\def\labelenumi{\arabic{enumi}.}
\item
  TERM.

  The Team hereby employs the Player as a skilled basketball player for a term of \_\_\_\_\_\_\_\_\_\_\_ year(s) from the 1st day of \_\_\_\_\_\_\_\_\_\_\_ September.
\item
  SERVICES.
\end{enumerate}

\begin{enumerate}
\def\labelenumi{(\alph{enumi})}
\tightlist
\item
  The services to be rendered by the Player pursuant to this Contract shall include: (i) training camp, (ii) practices, meetings, and conditioning sessions conducted by the Team during the Season, (iii) games scheduled for the Team during any Regular Season, (iv) Exhibition games scheduled by the Team or the League during and prior to any Regular Season, (v) the NBA's All-Star Game (including the Rookie Game) and every event conducted in association with such All-Star Game (including, but not limited to, a reasonable number of media sessions and any event that is part of an All-Star Skills Competition if the Player had previously agreed to participate in that Competition), if the Player is invited to participate therein, (vi) Playoff games scheduled by the League subsequent to any Regular Season, and (vii) promotional activities of the Team and the League as set forth in paragraph 13 herein.
\item
  If the Player is a Veteran, the Player will not be required to attend training camp earlier than 2 p.m. (local time) on the twenty-ninth (29th) day prior to the first game of any Regular Season. Notwithstanding the foregoing, if the Team is scheduled during a particular NBA Season to participate outside of North America in an Exhibition game or a Regular Season game during the first week of the Regular Season, such Veteran Player may be required to attend the training camp conducted in advance of that Regular Season by 2 p.m. (local time) on the thirty-second (32nd) day prior to the first game of the Regular Season. Rookies may be required to attend training camp at an earlier date, but no earlier than ten (10) days prior to the date that Veterans are required to attend.
\item
  Exhibition games shall not be played on the three (3) days prior to the opening of the Team's Regular Season schedule, nor on the day prior to a Regular Season game, nor on the day prior to and the day following the All-Star Game. Exhibition games prior to any Regular Season shall not exceed eight (including intra-squad games for which admission is charged), and Exhibition games during any Regular Season shall not exceed three.
\end{enumerate}

\begin{enumerate}
\def\labelenumi{\arabic{enumi}.}
\setcounter{enumi}{2}
\tightlist
\item
  COMPENSATION.
\end{enumerate}

\begin{enumerate}
\def\labelenumi{(\alph{enumi})}
\tightlist
\item
  Subject to paragraph 3(b) below, the Team agrees to pay the Player for rendering the services described herein the Compensation described in Exhibit 1 or Exhibit 1A hereto (less all amounts required to be withheld by federal, state, and local authorities, and exclusive of any amount(s) which the Player shall be entitled to receive from the Player Playoff Pool). Unless otherwise provided in Exhibit 1, such Compensation shall be paid in twelve (12) equal semi-monthly payments beginning with the first of said payments on November 15th of each year covered by the Contract and continuing with such payments on the first and fifteenth of each month until said Compensation is paid in full.
\item
  The Team agrees to pay the Player \$1,500 per week, pro rata, less all amounts required to be withheld by federal, state, and local authorities, for each week (up to a maximum of four (4) weeks for veterans and up to a maximum of five (5) weeks for Rookies) prior to the Team's first Regular Season game that the Player is in attendance at training camp or Exhibition games; provided, however, that no such payments shall be made if, prior to the date on which he is required to attend training camp, the Player has been paid \$10,000 or more in compensation with respect to the NBA Season scheduled to commence immediately following such training camp. Any Compensation paid by the Team pursuant to this subparagraph shall be considered an advance against any Compensation owed to the Player pursuant to paragraph 3(a) above, and the first scheduled payment of such Compensation (or such subsequent payments, if the first scheduled payment is not sufficient) shall be reduced by the amount of such advance.
\item
  The Team will not pay and the Player will not accept any bonus or anything of value on account of the Team's winning any particular NBA game or series of games or attaining a certain position in the standings of the League as of a certain date, other than the final standing of the Team.
\end{enumerate}

\begin{enumerate}
\def\labelenumi{\arabic{enumi}.}
\setcounter{enumi}{3}
\item
  EXPENSES.

  The Team agrees to pay all proper and necessary expenses of the Player, including the reasonable lodging expenses of the Player while playing for the Team ``on the road'' and during the training camp period (defined for this paragraph only to mean the period from the first day of training camp through the day of the Team's first Exhibition game) for as long as the Player is not then living at home. The Player, while ``on the road'' (and during the training camp period, only if the player is not then living at home and the Team does not pay for meals directly), shall be paid a meal expense allowance as set forth in the Collective Bargaining Agreement currently in effect between the NBA and the National Basketball Players Association (hereinafter ``the NBA/NBPA Collective Bargaining Agreement''). No deductions from such meal expense allowance shall be made for meals served on an airplane. During the training camp period (and only if the player is not then living at home and the Team does not pay for meals directly), the meal expense allowance shall be paid in weekly installments commencing with the first week of training camp. For the purposes of this paragraph, the Player shall be considered to be ``on the road'' from the time the Team leaves its home city until the time the Team arrives back at its home city.
\item
  CONDUCT.
\end{enumerate}

\begin{enumerate}
\def\labelenumi{(\alph{enumi})}
\tightlist
\item
  The Player agrees to observe and comply with all Team rules, as maintained or promulgated in accordance with the NBA/NBPA Collective Bargaining Agreement, at all times whether on or off the playing floor. Subject to the provisions of the NBA/NBPA Collective Bargaining Agreement, such rules shall be part of this Contract as fully as if herein written and shall be binding upon the Player.
\item
  The Player agrees (i) to give his best services, as well as his loyalty, to the Team, and to play basketball only for the Team and its assignees; (ii) to be neatly and fully attired in public; (iii) to conduct himself on and off the court according to the highest standards of honesty, citizenship, and sportsmanship; and (iv) not to do anything that is materially detrimental or materially prejudicial to the best interests of the Team or the League.
\item
  For any violation of Team rules, any breach of any provision of this Contract, or for any conduct impairing the faithful and thorough discharge of the duties incumbent upon the Player, the Team may reasonably impose fines and/or suspensions on the Player in accordance with the terms of the NBA/NBPA Collective Bargaining Agreement.
\item
  The Player agrees to be bound by Article 35 of the NBA Constitution, a copy of which, as in effect on the date of this Contract, is attached hereto. The Player acknowledges that the Commissioner is empowered to impose fines upon and/or suspend the Player for causes and in the manner provided in such Article, provided that such fines and/or suspensions are consistent with the terms of the NBA/NBPA Collective Bargaining Agreement.
\item
  The Player agrees that if the Commissioner, in his sole judgment, shall find that the Player has bet, or has offered or attempted to bet, money or anything of value on the outcome of any game participated in by any team which is a member of the NBA, the Commissioner shall have the power in his sole discretion to suspend the Player indefinitely or to expel him as a player for any member of the NBA, and the Commissioner's finding and decision shall be final, binding, conclusive, and unappealable.
\item
  The Player agrees that he will not, during the term of this Contract, directly or indirectly, entice, induce, or persuade, or attempt to entice, induce, or persuade, any player or coach who is under contract to any NBA team to enter into negotiations for or relating to his services as a basketball player or coach, nor shall he negotiate for or contract for such services, except with the prior written consent of such team. Breach of this subparagraph, in addition to the remedies available to the Team, shall be punishable by fine and/or suspension to be imposed by the Commissioner.
\item
  When the Player is fined and/or suspended by the Team or the NBA, he shall be given notice in writing (with a copy to the Players Association), stating the amount of the fine or the duration of the suspension and the reasons therefor.
\end{enumerate}

\begin{enumerate}
\def\labelenumi{\arabic{enumi}.}
\setcounter{enumi}{5}
\tightlist
\item
  WITHHOLDING.
\end{enumerate}

\begin{enumerate}
\def\labelenumi{(\alph{enumi})}
\tightlist
\item
  In the event the Player is fined and/or suspended by the Team or the NBA, the Team shall withhold the amount of the fine or, in the case of a suspension, the amount provided in Article VI of the NBA/NBPA Collective Bargaining Agreement from any Current Cash Compensation due or to become due to the Player with respect to the contract year in which the conduct resulting in the fine and/or the suspension occurred (or a subsequent contract year if the Player has received all Current Cash Compensation due to him for the then current contract year). If, at the time the Player is fined and/or suspended, the Current Cash Compensation remaining to be paid to the Player under this Contract is not sufficient to cover such fine and/or suspension, then the Player agrees promptly to pay the amount directly to the Team. In no case shall the Player permit any such fine and/or suspension to be paid on his behalf by anyone other than himself.
\item
  Any Current Cash Compensation withheld from or paid by the Player pursuant to this paragraph 6 shall be retained by the Team or the League, as the case may be, unless the Player contests the fine and/or suspension by initiating a timely Grievance in accordance with the provisions of the NBA/NBPA Collective Bargaining Agreement. If such Grievance is initiated and it satisfies Article XXXI, Section 13 of the NBA/NBPA Collective Bargaining Agreement, the amount withheld from the Player shall be placed in an interest-bearing account, pursuant to Article XXXI, Section 9 of such Agreement, pending the resolution of the Grievance.
\end{enumerate}

\begin{enumerate}
\def\labelenumi{\arabic{enumi}.}
\setcounter{enumi}{6}
\tightlist
\item
  PHYSICAL CONDITION.
\end{enumerate}

\begin{enumerate}
\def\labelenumi{(\alph{enumi})}
\tightlist
\item
  The Player agrees to report at the time and place fixed by the Team in good physical condition and to keep himself throughout each NBA Season in good physical condition.
\item
  If the Player, in the judgment of the Team's physician, is not in good physical condition at the date of his first scheduled game for the Team, or if, at the beginning of or during any Season, he fails to remain in good physical condition (unless such condition results directly from an injury sustained by the Player as a direct result of participating in any basketball practice or game played for the Team during such Season), so as to render the Player, in the judgment of the Team's physician, unfit to play skilled basketball, the Team shall have the right to suspend such Player until such time as, in the judgment of the Team's physician, the Player is in sufficiently good physical condition to play skilled basketball. In the event of such suspension, the Compensation (excluding any signing bonus or Incentive Compensation) payable to the Player for any Season during such suspension shall be reduced in the same proportion as the length of the period during which, in the judgment of the Team's physician, the Player is unfit to play skilled basketball, bears to the length of such Season.
\item
  If, during the term of this Contract, the Player is injured as a direct result of participating in any basketball practice or game played for the Team, the Team will pay the Player's reasonable hospitalization and medical expenses (including doctor's bills), provided that the hospital and doctor are selected by the Team, and provided further that the Team shall be obligated to pay only those expenses incurred as a direct result of medical treatment caused solely by and relating directly to the injury sustained by the Player. Subject to the provisions set forth in Exhibit 3, if in the judgment of the Team's physician, the Player's injuries resulted directly from playing for the Team and render him unfit to play skilled basketball, then, so long as such unfitness continues, but in no event after the Player has received his full Compensation for the Season in which the injury was sustained, the Team shall pay to the Player the Compensation prescribed in Exhibit 1 to this Contract for such Season. The Team's obligations hereunder shall be reduced by (i) any workers' compensation benefits, which, to the extent permitted by law, the Player hereby assigns to the Team, and (ii) any insurance provided for by the Team whether paid or payable to the Player.
\item
  The Player agrees to provide to the Team's coach, trainer, or physician prompt notice of any injury, illness, or medical condition suffered by him that is likely to affect adversely the Player's ability to render the services required under this Contract, including the time, place, cause, and nature of such injury, illness, or condition.
\item
  Should the Player suffer an injury, illness, or medical condition as provided in this paragraph 7, he will submit himself to a medical examination and appropriate medical treatment by a physician designated by the Team. Such examination when made at the request of the Team shall be at its expense, unless made necessary by some act or conduct of the Player contrary to the terms of this Contract.
\end{enumerate}

\begin{enumerate}
\def\labelenumi{\arabic{enumi}.}
\setcounter{enumi}{7}
\item
  PROHIBITED SUBSTANCES.

  The Player acknowledges that this Contract may be terminated in accordance with the express provisions of Article XXXIII (Anti-Drug Program) of the NBA/NBPA Collective Bargaining Agreement, and that any such termination will result in the Player's immediate dismissal and disqualification from any employment by the NBA and any of its teams. Notwithstanding any terms or provisions of this Contract (including any amendments hereto), in the event of such termination, all obligations of the Team, including obligations to pay Compensation, shall cease, except the obligation of the Team to pay the Player's earned Compensation (whether Current or Deferred) to the date of termination.
\item
  UNIQUE SKILLS.

  The Player represents and agrees that he has extraordinary and unique skill and ability as a basketball player, that the services to be rendered by him hereunder cannot be replaced or the loss thereof adequately compensated for in money damages, and that any breach by the Player of this Contract will cause irreparable injury to the Team, and to its assignees. Therefore, it is agreed that in the event it is alleged by the Team that the Player is playing, attempting or threatening to play, or negotiating for the purpose of playing, during the term of this Contract, for any other person, firm, corporation, or organization, the Team and its assignees (in addition to any other remedies that may be available to them judicially or by way of arbitration) shall have the right to obtain from any court or arbitrator having jurisdiction such equitable relief as may be appropriate, including a decree enjoining the Player from any further such breach of this Contract, and enjoining the Player from playing basketball for any other person, firm, corporation, or organization during the term of this Contract. The Player agrees that the Team may at any time assign such right to the NBA for the enforcement thereof. In any suit, action, or arbitration proceeding brought to obtain such equitable relief, the Player does hereby waive his right, if any, to trial by jury, and does hereby waive his right, if any, to interpose any counterclaim or set-off for any cause whatever.
\item
  ASSIGNMENT.
\end{enumerate}

\begin{enumerate}
\def\labelenumi{(\alph{enumi})}
\tightlist
\item
  The Team shall have the right to assign this Contract to any other NBA team and the Player agrees to accept such assignment and to faithfully perform and carry out this contract with the same force and effect as if it had been entered into by the Player with the assignee team instead of with the Team. The Player further agrees that, should the Team contemplate the assignment of this Contract to one or more NBA teams, the Team's physician may furnish to the physicians and officials of such other team or teams all relevant medical information relating to the Player.
\item
  In the event that this Contract is assigned to any other NBA team, all reasonable expenses incurred by the Player in moving himself and his family to the home territory of the team to which such assignment is made, as a result thereof, shall be paid by the assignee team. Such assignee team hereby agrees that its acceptance of the assignment of this Contract constitutes agreement on its part to make such payment.
\item
  In the event that this Contract is assigned to another NBA team, the Player shall forthwith be provided notice orally or in writing, delivered to the Player personally or delivered or mailed to his last known address, and the Player shall report to the assignee team within forty-eight (48) hours after said notice has been received (if the assignment is made during a Season), within one (1) week after said notice has been received (if the assignment is made between Seasons), or within such longer time for reporting as may be specified in said notice. The NBA shall also promptly notify the Players Association of any such assignment. The Player further agrees that, immediately upon reporting to the assignee team, he will submit upon request to a physical examination conducted by a physician designated by the assignee team.
\item
  If the Player, without a reasonable excuse, does not report to the team to which this Contract has been assigned within the time provided in subsection (c) above, then, upon consummation of the assignment, the player may be suspended by the assignee team or, if the assignment is not consummated or is voided as a result of the Player's failure to so report, by the assignor Team. In either case, the Player's Compensation may be reduced by the NBA by the imposition of a fine in an amount equal to the lesser of (i) ten (10) percent of the Player's full Compensation for the then-current Season, or (ii) \$50,000.
\end{enumerate}

\begin{enumerate}
\def\labelenumi{\arabic{enumi}.}
\setcounter{enumi}{10}
\tightlist
\item
  VALIDITY AND FILING.
\end{enumerate}

\begin{enumerate}
\def\labelenumi{(\alph{enumi})}
\tightlist
\item
  This Contract shall be valid and binding upon the Team and the Player immediately upon its execution.
\item
  The Team agrees to file a copy of this Contract, and/or any amendment(s) thereto, with the Commissioner of the NBA as soon as practicable by facsimile and overnight mail, but in no event may such filing be made more than forty-eight (48) hours after the execution of this Contract and/or amendment(s).
\item
  If pursuant to the NBA Constitution and By-Laws or the NBA/NBPA Collective Bargaining Agreement, the Commissioner disapproves this Contract (or amendment) within ten (10) days after the receipt thereof in his office by overnight mail, this Contract (or amendment) shall thereupon terminate and be of no further force or effect and the Team and the Player shall thereupon be relieved of their respective rights and liabilities thereunder. If the Commissioner's disapproval is subsequently overturned in any proceeding brought under the arbitration provisions of the NBA/NBPA Collective Bargaining Agreement (including any appeals), the Contract shall again be valid and binding upon the Team and the Player, and the Commissioner shall be afforded another ten-day period to disapprove the Contract (based on the Team's Room at the time the Commissioner's disapproval is overturned) as set forth in the foregoing sentence. The NBA will promptly inform the Players Association if the Commissioner disapproves this Contract.
\end{enumerate}

\begin{enumerate}
\def\labelenumi{\arabic{enumi}.}
\setcounter{enumi}{11}
\item
  OTHER ATHLETIC ACTIVITIES.

  The Player and the Team acknowledge and agree that (i) the Player's participation in other sports may impair or destroy his ability and skill as a basketball player, and (ii) the Player's participation in basketball out of season may result in injury to him. Accordingly, the Player agrees that he will not, without the written consent of the Team, engage in (x) sports endangering his health or safety (including,but not limited to, professional boxing or wrestling, motorcycling, moped-riding, auto racing, sky-diving, and hang gliding), or (y) any game or exhibition of basketball, football, baseball, hockey, lacrosse, or other athletic sport, under penalty of such fine and/or suspension as may be imposed by the Team and/or the Commissioner of the NBA. Nothing contained herein shall be intended to require the Player to obtain the written consent of the Team in order to enable the Player to participate in, as an amateur, the sport of golf, tennis, handball, swimming, hiking, softball, or volleyball.
\item
  PROMOTIONAL ACTIVITIES.
\end{enumerate}

\begin{enumerate}
\def\labelenumi{(\alph{enumi})}
\tightlist
\item
  The Player agrees to allow the Team or the League to take pictures of the Player, alone or together with others, for still photographs, motion pictures, or television, at such times as the Team or the League may designate. No matter by whom taken, such pictures may be used in any manner desired by either the Team or the League for publicity or promotional purposes. The rights in any such pictures taken by the Team or by the League shall belong to the Team or to the League, as their interests may appear.
\item
  The Player agrees that, during any year of this Contract, he will not make public appearances, participate in radio or television programs, permit his picture to be taken, write or sponsor newspaper or magazine articles, or sponsor commercial products without the written consent of the Team, which shall not be withheld except in the reasonable interests of the Team or the NBA.
\item
  Upon request, the Player shall consent to and make himself available for interviews by representatives of the media conducted at reasonable times.
\item
  In addition to the foregoing, and subject to the conditions and limitations set forth in Article II, Section 8 of the NBA/NBPA Collective Bargaining Agreement, the Player agrees to participate, upon request, in all other reasonable promotional activities of the Team and the NBA. For each such promotional appearance made on behalf of a commercial sponsor of the Team, the Team agrees to pay the Player \$1,000 or, if the Team agrees, such higher amount that is consistent with the Team's past practice and not otherwise unreasonable.
\end{enumerate}

\begin{enumerate}
\def\labelenumi{\arabic{enumi}.}
\setcounter{enumi}{13}
\tightlist
\item
  GROUP LICENSE.
\end{enumerate}

\begin{enumerate}
\def\labelenumi{(\alph{enumi})}
\tightlist
\item
  The Player hereby grants to NBA Properties, Inc.~the exclusive rights to use the Player's Player Attributes as such term is defined and for such group licensing purposes as are set forth in the Agreement between NBA Properties, Inc.~and the National Basketball Players Association, made as of September 18, 1995 and amended January 20, 1999 (the ``Group License''), a copy of which will, upon his request, be furnished the Player; and the Player agrees to make the appearances called for by such Agreement.
\item
  Notwithstanding anything to the contrary contained in the Group License or this Contract, NBA Properties may use, in connection with League Promotions, the Player's (i) name or nickname and/or (ii) the Player's Player Attributes (as defined in the Group License) as such Player Attributes may be captured in game action footage or photographs. NBA Properties shall be entitled to use the Player's Player Attributes individually pursuant to the preceding sentence and shall not be required to use the Player's Player Attributes in a group or as one of multiple players. As used herein, League Promotion shall mean any advertising, marketing, or collateral materials or marketing programs conducted by the NBA, NBA Properties (or any subsidiary of NBA Properties) or any NBA team that is intended to promote (x) any game in which an NBA team participates or game telecast or broadcast (including Pre-Season, Exhibition, Regular Season, and Playoff games), (y) the NBA, its teams, or its players, or (z) the sport of basketball.
\end{enumerate}

\begin{enumerate}
\def\labelenumi{\arabic{enumi}.}
\setcounter{enumi}{14}
\item
  TEAM DEFAULT.

  In the event of an alleged default by the Team in the payments to the Player provided for by this Contract, or in the event of an alleged failure by the Team to perform any other material obligation that it has agreed to perform hereunder, the Player shall notify both the Team and the League in writing of the facts constituting such alleged default or alleged failure. If neither the Team nor the League shall cause such alleged default or alleged failure to be remedied within five (5) days after receipt of such written notice, the National Basketball Players Association shall, on behalf of the Player, have the right to request that the dispute concerning such alleged default or alleged failure be referred immediately to the Grievance Arbitrator in accordance with the provisions of the NBA/NBPA Collective Bargaining Agreement. If, as a result of such arbitration, an award issues in favor of the Player, and if neither the Team nor the League complies with such award within ten (10) days after the service thereof, the Player shall have the right, by a further written notice to the Team and the League, to terminate this Contract.
\item
  TERMINATION.
\end{enumerate}

\begin{enumerate}
\def\labelenumi{(\alph{enumi})}
\tightlist
\item
  The Team may terminate this Contract upon written notice to the Player if the Player shall:

  \begin{enumerate}
  \def\labelenumii{(\roman{enumii})}
  \tightlist
  \item
    at any time, fail, refuse, or neglect to conform his personal conduct to standards of good citizenship, good moral character (defined here to mean not engaging in acts of moral turpitude, whether or not such acts would constitute a crime), and good sportsmanship, to keep himself in first class physical condition, or to obey the Team's training rules; or
  \item
    at any time commit a significant and inexcusable physical attack against any official or employee of the Team or the NBA (other than another player), or any person in attendance at any NBA game or event, considering the totality of the circumstances, including (but not limited to) the degree of provocation (if any) that may have led to the attack, the nature and scope of the attack, the player's state of mind at the time of the attack, and the extent of any injury resulting from the attack; or
  \item
    at any time, fail, in the sole opinion of the Team's management, to exhibit sufficient skill or competitive ability to qualify to continue as a member of the Team; provided, however, (x) that if this Contract is terminated by the Team, in accordance with the provisions of this subparagraph, prior to January 10 of any Regular Season, and the Player, at the time of such termination, is unfit to play skilled basketball as the result of an injury resulting directly from his playing for the Team, the Player shall (subject to the provisions set forth in Exhibit 3) continue to receive his full Compensation, less all workers' compensation benefits (which, to the extent permitted by law, and if not deducted from the Player's Compensation by the Team, the Player hereby assigns to the Team) and any insurance provided for by the Team paid or payable to the Player by reason of said injury, until such time as the Player is fit to play skilled basketball, but not beyond the Season during which such termination occurred; and provided, further, (y) that if this Contract is terminated by the Team, in accordance with the provisions of this subparagraph, during the period from the January 10 of any Regular Season through the end of such Regular Season, the Player shall be entitled to receive his full Compensation for said Season; or
  \item
    at any time, fail, refuse, or neglect to render his services hereunder or in any other manner materially breach this Contract.
  \end{enumerate}
\item
  If this Contract is terminated by the Team by reason of the Player's failure to render his services hereunder due to disability caused by an injury to the Player resulting directly from his playing for the Team and rendering him unfit to play skilled basketball, and notice of such injury is given by the Player as provided herein, the Player shall (subject to the provisions set forth in Exhibit 3) be entitled to receive his full Compensation for the Season in which the injury was sustained, less all workers' compensation benefits (which, to the extent permitted by law, and if not deducted from the Player's Compensation by the Team, the Player hereby assigns to the Team) and any insurance provided for by the Team paid or payable to the Player by reason of said injury.
\item
  Notwithstanding the provisions of subparagraph 16(b) above, if this Contract is terminated by the Team prior to the first game of a Regular Season by reason of the Player's failure to render his services hereunder due to an injury or condition sustained or suffered during a preceding Season, or after such Season but prior to the Player's participation in any basketball practice or game played for the Team, payment by the Team of any Compensation earned through the date of termination under paragraph 3(b) above, payment of the Player's board, lodging, and expense allowance during the training camp period, payment of the reasonable traveling expenses of the Player to his home city, and the expert training and coaching provided by the Team to the Player during the training season shall be full payment to the Player.
\item
  If this Contract is terminated by the Team during the period designated by the Team for attendance at training camp, payment by the Team of any Compensation earned through the date of termination under paragraph 3(b) above, payment of the Player's board, lodging, and expense allowance during such period to the date of termination, payment of the reasonable traveling expenses of the Player to his home city, and the expert training and coaching provided by the Team to the Player during the training season shall be full payment to the Player.
\item
  If this Contract is terminated by the Team after the first game of a Regular Season, except in the case provided for in subparagraphs (a)(iii) and (b) of this paragraph 16, the Player shall be entitled to receive as full payment hereunder a sum of money which, when added to the salary which he has already received during such Season, will represent the same proportionate amount of the annual sum set forth in Exhibit 1 hereto as the number of days of such Regular Season then past bears to the total number of days of such Regular Season, plus the reasonable traveling expenses of the Player to his home.
\item
  If the Team proposes to terminate this Contract in accordance with subparagraph (a) of this paragraph 16, it must first comply with the following waiver procedure:

  \begin{enumerate}
  \def\labelenumii{(\roman{enumii})}
  \tightlist
  \item
    The Team shall request the NBA Commissioner to request waivers from all other clubs. Such waiver request may not be withdrawn.
  \item
    Upon receipt of the waiver request, any other team may claim assignment of this Contract at such waiver price as may be fixed by the League, the priority of claims to be determined in accordance with the NBA Constitution and By-Laws.
  \item
    If this Contract is so claimed, the Team agrees that it shall, upon the assignment of this Contract to the claiming team, notify the Player of such assignment as provided in paragraph 10(c) hereof, and the Player agrees he shall report to the assignee team as provided in said paragraph 10(c).
  \item
    If the Contract is not claimed, the Team shall promptly deliver written notice of termination to the Player at the expiration of the waiver period.
  \item
    The NBA shall promptly notify the Players Association of the disposition of any waiver request.
  \item
    To the extent not inconsistent with the foregoing provisions of this subparagraph (f), the waiver procedures set forth in the NBA Constitution and By-Laws, a copy of which, as in effect on the date of this Contract, is attached hereto, shall govern.
  \end{enumerate}
\item
  Upon any termination of this Contract by the Player, all obligations of the Team to pay Compensation shall cease on the date of termination, except the obligation of the Team to pay the Player's Compensation to said date.
\end{enumerate}

\begin{enumerate}
\def\labelenumi{\arabic{enumi}.}
\setcounter{enumi}{16}
\item
  DISPUTES.

  In the event of any dispute arising between the Player and the Team relating to any matter arising under this Contract, or concerning the performance or interpretation thereof (except for a dispute arising under paragraph 9 hereof), such dispute shall be resolved in accordance with the Grievance and Arbitration Procedure set forth in the NBA/NBPA Collective Bargaining Agreement.
\item
  PLAYER NOT A MEMBER.

  Nothing contained in this Contract or in any provision of the NBA Constitution and By-Laws shall be construed to constitute the Player a member of the NBA or to confer upon him any of the rights or privileges of a member thereof.
\item
  RELEASE.

  The Player hereby releases and waives every claim he may have against the NBA and its related entities and every member of the NBA, and against every director, officer, owner, stockholder, trustee, partner, and employee of the NBA and its related entities and/or any member of the NBA and their related entities (excluding persons employed as players by any such member), and against any person retained by the NBA and/or the Players Association in connection with the NBA/NBPA Anti-Drug Program, the Grievance Arbitrator, the System Arbitrator, and any other arbitrator or expert retained by the NBA and/or the Players Association under the terms of the NBA/NBPA Collective Bargaining Agreement, arising out of or in connection with (i) any injury that is subject to the provisions of paragraph 7, (ii) any fighting or other form of violent and/or unsportsmanlike conduct occurring during the course of any practice and/or any Exhibition, Regular Season, and/or Playoff game (on or adjacent to the playing floor or in or adjacent to any facility used for practices or games), (iii) the testing procedures or the imposition of any penalties set forth in paragraph 8 hereof and in the NBA/NBPA Anti-Drug Program, or (iv) any injury suffered in the course of his employment as to which he has or would have a claim for workers' compensation benefits. The foregoing shall not apply to any claim of medical malpractice against a Team-affiliated physician or other medical personnel.
\item
  ENTIRE AGREEMENT.

  This Contract (including any Exhibits hereto) contains the entire agreement between the parties and sets forth all components of the Player's Compensation from the Team or any Team Affiliate, and there are no undisclosed agreements of any kind, express or implied, oral or written, promises, undertakings, representations, commitments, inducements, assurances of intent, or understandings of any kind that have not been disclosed to the NBA (a) involving consideration of any kind to be paid, furnished, or made available to the Player, or any person or entity controlled by or related to the Player, by the Team or any Team Affiliate, either during the term of this Contract or thereafter, or (b) concerning any future Renegotiation, Extension, or other amendment of this Contract or the entry into any new Player Contract.
\end{enumerate}

\newpage

EXAMINE THIS CONTRACT CAREFULLY BEFORE SIGNING IT.

THIS CONTRACT INCLUDES EXHIBITS \_\_\_\_\_\_\_\_\_\_, WHICH ARE ATTACHED HERETO AND MADE A PART HEREOF.

IN WITNESS WHEREOF the Player has hereunto signed his name and the Team has caused this Contract to be executed by its duly authorized officer.

WITNESSES:

\begin{longtable}[]{@{}ll@{}}
\toprule()
\endhead
Dated: \_\_\_\_\_\_\_\_\_\_\_\_\_\_\_\_\_\_\_\_\_ & By: \_\_\_\_\_\_\_\_\_\_\_\_\_\_\_\_\_\_\_\_\_\_\_\_\_\_\_\_ \\
& Title: \_\_\_\_\_\_\_\_\_\_\_\_\_\_\_\_\_\_\_\_\_\_\_\_\_\_\_\_ \\
& Team: \_\_\_\_\_\_\_\_\_\_\_\_\_\_\_\_\_\_\_\_\_\_\_\_\_\_\_\_ \\
& \\
Dated: \_\_\_\_\_\_\_\_\_\_\_\_\_\_\_\_\_\_\_\_\_ & By: \_\_\_\_\_\_\_\_\_\_\_\_\_\_\_\_\_\_\_\_\_\_\_\_\_\_\_\_ \\
& Player: \_\_\_\_\_\_\_\_\_\_\_\_\_\_\_\_\_\_\_\_\_\_\_\_\_\_\_\_ \\
& Player's Address: \\
& \_\_\_\_\_\_\_\_\_\_\_\_\_\_\_\_\_\_\_\_\_\_\_\_\_\_\_\_\_\_\_\_\_\_\_\_ \\
& \_\_\_\_\_\_\_\_\_\_\_\_\_\_\_\_\_\_\_\_\_\_\_\_\_\_\_\_\_\_\_\_\_\_\_\_ \\
\bottomrule()
\end{longtable}

\newpage

\hypertarget{excerpt-from-nba-constitution}{%
\subsection{EXCERPT FROM NBA CONSTITUTION}\label{excerpt-from-nba-constitution}}

\hypertarget{misconduct}{%
\subsubsection{MISCONDUCT}\label{misconduct}}

\begin{enumerate}
\def\labelenumi{\arabic{enumi}.}
\setcounter{enumi}{34}
\tightlist
\item
  The provisions of this Article 35 shall govern all Players in the Association, hereinafter referred to as ``Players.''

  \begin{enumerate}
  \def\labelenumii{(\alph{enumii})}
  \tightlist
  \item
    Each Member shall provide and require in every contract with any of its Players that they shall be bound and governed by the provisions of this Article. Each Member, at the direction of the Board of Governors or the Commissioner, as the case may be, shall take such action as the Board or the Commissioner may direct in order to effectuate the purposes of this Article.
  \item
    The Commissioner shall direct the dismissal and perpetual disqualification from any further association with the Association or any of its Members, of any Player found by the Commissioner after a hearing to have been guilty of offering, agreeing, conspiring, aiding or attempting to cause any game of basketball to result otherwise than on its merits.
  \item
    Any Player who gives, makes, issues, authorizes or endorses any statement having, or designed to have, an effect prejudicial or detrimental to the best interests of basketball or of the Association or of a Member or its Team, shall be liable to a fine not exceeding \$25,000, to be imposed by the Commissioner. The Member whose Player has been so fined shall pay the amount of the fine should such Player fail to do so within ten (10) days of its imposition.
  \item
    If in the opinion of the Commissioner any other act or conduct of a Player at or during an Exhibition, Regular Season, or Playoff game has been prejudicial to or against the best interests of the Association or the game of basketball, the Commissioner shall impose upon such Player a fine not exceeding \$35,000, or may order for a time the suspension of any such Player from any connection or duties with Exhibition, Regular Season, or Playoff games, or he may order both such fine and suspension.
  \item
    The Commissioner shall have the power to suspend for a definite or indefinite period, or to impose a fine not exceeding \$35,000, or inflict both such suspension and fine upon any Player who, in his opinion, shall have been guilty of conduct that does not conform to standards of morality or fair play, that does not comply at all times with all federal, state, and local laws, or that is prejudicial or detrimental to the Association.
  \item
    Any Player who, directly or indirectly, entices, induces, persuades or attempts to entice, induce, or persuade any Player, Coach, Trainer, General Manager or any other person who is under contract to any other Member of the Association to enter into negotiations for or relating to his services or negotiates or contracts for such services shall, on being charged with such tampering, be given an opportunity to answer such charges after due notice and the Commissioner shall have the power to decide whether or not the charges have been sustained; in the event his decision is that the charges have been sustained, then the Commissioner shall have the power to suspend such Player for a definite or indefinite period, or to impose a fine not exceeding \$35,000, or inflict both such suspension and fine upon any such Player.
  \item
    Any Player who, directly or indirectly, wagers money or anything of value on the outcome of any game played by a Team in the league operated by the Association shall, on being charged with such wagering, be given an opportunity to answer such charges after due notice, and the decision of the Commissioner shall be final, binding and conclusive and unappealable. The penalty for such offense shall be within the absolute and sole discretion of the Commissioner and may include a fine, suspension, expulsion and/or perpetual disqualification from further association with the Association or any of its Members.
  \item
    Except for a penalty imposed under Paragraph (g) of this Article 35, any challenge by a Team to the decisions and acts of the Commissioner pursuant to Article 35 shall be appealable to the Board of Governors who shall determine such appeals in accordance with such rules and regulations as may be adopted by the Board in its absolute and sole discretion. Any such challenge by a player shall be resolved by the Grievance Arbitrator in accordance with the grievance and arbitration procedures of the collective bargaining agreement then in effect.
  \end{enumerate}
\end{enumerate}

\newpage

\hypertarget{excerpt-from-nbaby-laws}{%
\subsection{EXCERPT FROM NBABY-LAWS}\label{excerpt-from-nbaby-laws}}

5.01. \emph{Waiver Right.} Except for sales and trading between Members in accordance with these By-Laws, no Member shall sell, option, or otherwise assign the contract with, right to the services of, or right to negotiate with, a Player without complying with the waiver procedure prescribed by this Constitution and By-Laws.
5.02. \emph{Waiver Price.} The waiver price shall be \$1,000 per Player.
5.03. \emph{Waiver Procedure.} A Member desiring to secure waivers on a Player shall notify the Commissioner or the Commissioner's designee, who shall, on behalf of such Member, immediately notify all other Members of the waiver request. Such Player shall be assumed to have been waived unless a Member shall notify the Commissioner or the Commissioner's designee in accordance with Section 5.04 of a claim to the rights to such Player. Once a Member has notified the Commissioner or the Commissioner's designee of its desire to secure waivers on a Player, such notice may not be withdrawn. A Player remains the financial responsibility of the Member placing him on waivers until the waiver period set by the Commissioner or the Commissioner's designee has expired.
5.04. \emph{Waiver Period.} If the Commissioner or the Commissioner's designee distributes notice of request for waiver at any time between August 15 and the end of the next Season, any Members wishing to claim rights to the Player shall do so by giving notice by telephone and in a Writing of such claim to the Commissioner or the Commissioner's designee within forty-eight (48) hours after the time of such notice. If the Commissioner or the Commissioner's designee distributes notice of request for waiver at any other time, any Member wishing to claim rights to the Player shall do so by providing Written Notice of such claim to the Commissioner or the Commissioner's designee within ten (10) days after the date of such notice. A Team may not withdraw a claim to the rights to a Player on waivers.
5.05. \emph{Waiver Preferences.}
(a) In the event that more than one (1) Member shall have claimed the rights to a Player placed on waivers, the claiming Member with the lowest team standing at the time the waiver was requested shall be entitled to acquire the rights to such Player. If the request for waiver shall occur after the last day of the Season and before 11:59 p.m. eastern time on the following November 30, the standings at the close of the previous Season shall govern.
(b) If the winning percentage of two (2) claiming Teams are the same, then the tie shall be determined, if possible, on the basis of the Regular Season Games between the two (2) Teams during the Season or during the preceding Season, as the case may be. If still tied, a toss of a coin shall determine priority. For the purpose of determining standings, both Conferences of the Association shall be deemed merged and a consolidated standing shall control.
5.06. \emph{Players Acquired Through Waivers.} A Member who has acquired the rights and title to the contract of a Player through the waiver procedure may not sell or trade such rights for a period of thirty (30) days after the acquisition thereof; provided, however, that if the rights to such Player were acquired between Seasons, the 30-day period described herein shall begin on the first day of the next succeeding Season.5.07. Additional Waiver Rules. The Commissioner or the Board of Governors may from time to time adopt additional rules (supplementary to those set forth in this Section 5) with respect to the operation of the waiver procedure. Such rules shall not be inconsistent with the provisions of this Section 5 and shall apply to but shall not be limited to the mechanics of notice, inadvertent omission of notification to a Member, and rules of construction as to time.

\newpage

\hypertarget{agent-certification}{%
\subsection{AGENT CERTIFICATION}\label{agent-certification}}

(To be completed only if Player was represented by an agent who negotiated the terms of this Contract.)

I, the undersigned, having negotiated this Contract on behalf of \_\_\_\_\_\_\_\_\_\_\_\_\_\_\_\_\_, do hereby swear and certify, under penalties of perjury, that the terms of Paragraph 20 of this Contract (``Entire Agreement'') are true and correct to the best of my knowledge and belief.

\begin{longtable}[]{@{}l@{}}
\toprule()
\endhead
Player Representative: \_\_\_\_\_\_\_\_\_\_\_\_\_\_\_\_\_ \\
State of \_\_\_\_\_\_\_\_\_\_\_\_\_\_\_\_\_\_\_\_\_\_\_\_\_\_\_\_\_\_\_ \\
County of \_\_\_\_\_\_\_\_\_\_\_\_\_\_\_\_\_\_\_\_\_\_\_\_\_\_\_\_\_\_ \\
\bottomrule()
\end{longtable}

On \_\_\_\_\_\_\_\_\_\_\_\_\_\_\_\_\_\_\_\_\_, before me personally came \_\_\_\_\_\_\_\_\_\_\_\_\_\_\_\_\_\_\_\_\_ and acknowledged to me that he/she had executed the foregoing Agent Certification.

\begin{longtable}[]{@{}l@{}}
\toprule()
\endhead
Notary Public: \_\_\_\_\_\_\_\_\_\_\_\_\_\_\_\_\_\_\_\_\_\_\_\_ \\
\bottomrule()
\end{longtable}

\newpage

\hypertarget{uniform-player-contract-1}{%
\section{UNIFORM PLAYER CONTRACT}\label{uniform-player-contract-1}}

\hypertarget{exhibit-1-compensation}{%
\subsection{Exhibit 1 --- Compensation}\label{exhibit-1-compensation}}

\begin{longtable}[]{@{}l@{}}
\toprule()
\endhead
Player: \_\_\_\_\_\_\_\_\_\_\_\_\_\_\_\_\_\_\_\_\_\_\_\_\_\_\_\_\_\_\_\_ \\
Team: \_\_\_\_\_\_\_\_\_\_\_\_\_\_\_\_\_\_\_\_\_\_\_\_\_\_\_\_\_\_\_\_\_\_ \\
Date: \_\_\_\_\_\_\_\_\_\_\_\_\_\_\_\_\_\_\_\_\_\_\_\_\_\_\_\_\_\_\_\_\_\_ \\
\bottomrule()
\end{longtable}

\begin{longtable}[]{@{}
  >{\centering\arraybackslash}p{(\columnwidth - 4\tabcolsep) * \real{0.1071}}
  >{\centering\arraybackslash}p{(\columnwidth - 4\tabcolsep) * \real{0.4464}}
  >{\centering\arraybackslash}p{(\columnwidth - 4\tabcolsep) * \real{0.4464}}@{}}
\toprule()
\begin{minipage}[b]{\linewidth}\centering
Season
\end{minipage} & \begin{minipage}[b]{\linewidth}\centering
Current Base Compensation
\end{minipage} & \begin{minipage}[b]{\linewidth}\centering
Deferred Base Compensation
\end{minipage} \\
\midrule()
\endhead
\_\_\_\_\_\_\_\_ & \_\_\_\_\_\_\_\_\_\_\_\_\_\_\_\_\_\_\_\_\_\_\_ & \_\_\_\_\_\_\_\_\_\_\_\_\_\_\_\_\_\_\_\_\_\_\_\_\_ \\
\_\_\_\_\_\_\_\_ & \_\_\_\_\_\_\_\_\_\_\_\_\_\_\_\_\_\_\_\_\_\_\_ & \_\_\_\_\_\_\_\_\_\_\_\_\_\_\_\_\_\_\_\_\_\_\_\_\_ \\
\_\_\_\_\_\_\_\_ & \_\_\_\_\_\_\_\_\_\_\_\_\_\_\_\_\_\_\_\_\_\_\_ & \_\_\_\_\_\_\_\_\_\_\_\_\_\_\_\_\_\_\_\_\_\_\_\_\_ \\
\_\_\_\_\_\_\_\_ & \_\_\_\_\_\_\_\_\_\_\_\_\_\_\_\_\_\_\_\_\_\_\_ & \_\_\_\_\_\_\_\_\_\_\_\_\_\_\_\_\_\_\_\_\_\_\_\_\_ \\
\_\_\_\_\_\_\_\_ & \_\_\_\_\_\_\_\_\_\_\_\_\_\_\_\_\_\_\_\_\_\_\_ & \_\_\_\_\_\_\_\_\_\_\_\_\_\_\_\_\_\_\_\_\_\_\_\_\_ \\
\bottomrule()
\end{longtable}

\textbf{Payment Schedule} (if different from paragraph 3):

Current Base:

Deferred Base:

\textbf{Signing Bonus} (include dates of payment):

\textbf{Non-Cash Compensation} (include dates of payment):

\textbf{Incentive Compensation} (include dates of payment):

\textbf{Other Arrangements:}

\begin{longtable}[]{@{}ll@{}}
\toprule()
Initialed: & \\
\midrule()
\endhead
\_\_\_\_\_\_\_\_\_\_\_\_\_\_ & \_\_\_\_\_\_\_\_\_\_\_\_\_\_ \\
Player & Team \\
\bottomrule()
\end{longtable}

\newpage

\hypertarget{exhibit-1a-compensation-minimum-player-salary}{%
\subsubsection{Exhibit 1A --- Compensation: Minimum Player Salary}\label{exhibit-1a-compensation-minimum-player-salary}}

\begin{longtable}[]{@{}l@{}}
\toprule()
\endhead
Player: \_\_\_\_\_\_\_\_\_\_\_\_\_\_\_\_\_\_\_\_\_\_\_\_\_\_\_\_\_\_\_\_ \\
Team: \_\_\_\_\_\_\_\_\_\_\_\_\_\_\_\_\_\_\_\_\_\_\_\_\_\_\_\_\_\_\_\_\_\_ \\
Date: \_\_\_\_\_\_\_\_\_\_\_\_\_\_\_\_\_\_\_\_\_\_\_\_\_\_\_\_\_\_\_\_\_\_ \\
\bottomrule()
\end{longtable}

\begin{longtable}[]{@{}
  >{\centering\arraybackslash}p{(\columnwidth - 4\tabcolsep) * \real{0.1071}}
  >{\centering\arraybackslash}p{(\columnwidth - 4\tabcolsep) * \real{0.4464}}
  >{\centering\arraybackslash}p{(\columnwidth - 4\tabcolsep) * \real{0.4464}}@{}}
\toprule()
\begin{minipage}[b]{\linewidth}\centering
Season
\end{minipage} & \begin{minipage}[b]{\linewidth}\centering
Current Base Compensation
\end{minipage} & \begin{minipage}[b]{\linewidth}\centering
Deferred Base Compensation
\end{minipage} \\
\midrule()
\endhead
\_\_\_\_\_\_\_\_ & \_\_\_\_\_\_\_\_\_\_\_\_\_\_\_\_\_\_\_\_\_\_\_ & \_\_\_\_\_\_\_\_\_\_\_\_\_\_\_\_\_\_\_\_\_\_\_\_\_ \\
\_\_\_\_\_\_\_\_ & \_\_\_\_\_\_\_\_\_\_\_\_\_\_\_\_\_\_\_\_\_\_\_ & \_\_\_\_\_\_\_\_\_\_\_\_\_\_\_\_\_\_\_\_\_\_\_\_\_ \\
\_\_\_\_\_\_\_\_ & \_\_\_\_\_\_\_\_\_\_\_\_\_\_\_\_\_\_\_\_\_\_\_ & \_\_\_\_\_\_\_\_\_\_\_\_\_\_\_\_\_\_\_\_\_\_\_\_\_ \\
\_\_\_\_\_\_\_\_ & \_\_\_\_\_\_\_\_\_\_\_\_\_\_\_\_\_\_\_\_\_\_\_ & \_\_\_\_\_\_\_\_\_\_\_\_\_\_\_\_\_\_\_\_\_\_\_\_\_ \\
\_\_\_\_\_\_\_\_ & \_\_\_\_\_\_\_\_\_\_\_\_\_\_\_\_\_\_\_\_\_\_\_ & \_\_\_\_\_\_\_\_\_\_\_\_\_\_\_\_\_\_\_\_\_\_\_\_\_ \\
\bottomrule()
\end{longtable}

\textbf{This Contract is intended to provide for a Salary for the \_\_\_\_\_\_\_\_\_\_\_\_\_\_\_\_\_\_\_\_\_\_\_ Season(s) equal to the Minimum Player Salary for such Season(s) (with no Unlikely Bonuses) and shall be deemed amended to the extent necessary to so provide.}

\textbf{Payment Schedule} (if different from paragraph 3):

Current Base:

Deferred Base:

\textbf{Signing Bonus} (include dates of payment):

\textbf{Non-Cash Compensation} (include dates of payment):

\textbf{Other Arrangements:}

\begin{longtable}[]{@{}ll@{}}
\toprule()
Initialed: & \\
\midrule()
\endhead
\_\_\_\_\_\_\_\_\_\_\_\_\_\_ & \_\_\_\_\_\_\_\_\_\_\_\_\_\_ \\
Player & Team \\
\bottomrule()
\end{longtable}

\newpage

\hypertarget{exhibit-2-compensation-protection-or-insurance}{%
\subsection{Exhibit 2 --- Compensation Protection or Insurance}\label{exhibit-2-compensation-protection-or-insurance}}

\begin{longtable}[]{@{}l@{}}
\toprule()
\endhead
Player: \_\_\_\_\_\_\_\_\_\_\_\_\_\_\_\_\_\_\_\_\_\_\_\_\_\_\_\_\_\_\_\_ \\
Team: \_\_\_\_\_\_\_\_\_\_\_\_\_\_\_\_\_\_\_\_\_\_\_\_\_\_\_\_\_\_\_\_\_\_ \\
Date: \_\_\_\_\_\_\_\_\_\_\_\_\_\_\_\_\_\_\_\_\_\_\_\_\_\_\_\_\_\_\_\_\_\_ \\
\bottomrule()
\end{longtable}

\begin{longtable}[]{@{}
  >{\centering\arraybackslash}p{(\columnwidth - 6\tabcolsep) * \real{0.0806}}
  >{\centering\arraybackslash}p{(\columnwidth - 6\tabcolsep) * \real{0.2419}}
  >{\centering\arraybackslash}p{(\columnwidth - 6\tabcolsep) * \real{0.2742}}
  >{\centering\arraybackslash}p{(\columnwidth - 6\tabcolsep) * \real{0.4032}}@{}}
\toprule()
\begin{minipage}[b]{\linewidth}\centering
Season
\end{minipage} & \begin{minipage}[b]{\linewidth}\centering
Type of Protection
\end{minipage} & \begin{minipage}[b]{\linewidth}\centering
Amount of Protection or Insurance
\end{minipage} & \begin{minipage}[b]{\linewidth}\centering
Conditions or Limitations
\end{minipage} \\
\midrule()
\endhead
\_\_\_\_\_ & \_\_\_\_\_\_\_\_\_\_\_ & \_\_\_\_\_\_\_\_\_\_\_\_\_\_\_ & \_\_\_\_\_\_\_\_\_\_\_\_\_\_\_\_\_\_ \\
\_\_\_\_\_ & \_\_\_\_\_\_\_\_\_\_\_ & \_\_\_\_\_\_\_\_\_\_\_\_\_\_\_ & \_\_\_\_\_\_\_\_\_\_\_\_\_\_\_\_\_\_ \\
\_\_\_\_\_ & \_\_\_\_\_\_\_\_\_\_\_ & \_\_\_\_\_\_\_\_\_\_\_\_\_\_\_ & \_\_\_\_\_\_\_\_\_\_\_\_\_\_\_\_\_\_ \\
\_\_\_\_\_ & \_\_\_\_\_\_\_\_\_\_\_ & \_\_\_\_\_\_\_\_\_\_\_\_\_\_\_ & \_\_\_\_\_\_\_\_\_\_\_\_\_\_\_\_\_\_ \\
\_\_\_\_\_ & \_\_\_\_\_\_\_\_\_\_\_ & \_\_\_\_\_\_\_\_\_\_\_\_\_\_\_ & \_\_\_\_\_\_\_\_\_\_\_\_\_\_\_\_\_\_ \\
\bottomrule()
\end{longtable}

\begin{longtable}[]{@{}ll@{}}
\toprule()
Initialed: & \\
\midrule()
\endhead
\_\_\_\_\_\_\_\_\_\_\_\_\_\_ & \_\_\_\_\_\_\_\_\_\_\_\_\_\_ \\
Player & Team \\
\bottomrule()
\end{longtable}

\newpage

\hypertarget{exhibit-3-prior-injury-exclusion}{%
\subsection{Exhibit 3 --- Prior Injury Exclusion}\label{exhibit-3-prior-injury-exclusion}}

\begin{longtable}[]{@{}l@{}}
\toprule()
\endhead
Player: \_\_\_\_\_\_\_\_\_\_\_\_\_\_\_\_\_\_\_\_\_\_\_\_\_\_\_\_\_\_\_\_ \\
Team: \_\_\_\_\_\_\_\_\_\_\_\_\_\_\_\_\_\_\_\_\_\_\_\_\_\_\_\_\_\_\_\_\_\_ \\
Date: \_\_\_\_\_\_\_\_\_\_\_\_\_\_\_\_\_\_\_\_\_\_\_\_\_\_\_\_\_\_\_\_\_\_ \\
\bottomrule()
\end{longtable}

The Player's right to receive his Compensation as set forth in paragraphs 7(c), 16(a)(iii), 16(b) of this Contract, or otherwise is limited or eliminated with respect to the following reinjury of the injury or aggravation of the condition set forth below:

\begin{longtable}[]{@{}l@{}}
\toprule()
Describe injury or condition: \\
\midrule()
\endhead
\_\_\_\_\_\_\_\_\_\_\_\_\_\_\_\_\_\_\_\_\_\_\_\_\_\_\_\_\_\_\_\_\_\_\_\_\_\_\_\_\_\_\_\_\_\_\_\_\_\_\_\_\_\_\_\_\_\_\_\_\_ \\
\_\_\_\_\_\_\_\_\_\_\_\_\_\_\_\_\_\_\_\_\_\_\_\_\_\_\_\_\_\_\_\_\_\_\_\_\_\_\_\_\_\_\_\_\_\_\_\_\_\_\_\_\_\_\_\_\_\_\_\_\_ \\
\_\_\_\_\_\_\_\_\_\_\_\_\_\_\_\_\_\_\_\_\_\_\_\_\_\_\_\_\_\_\_\_\_\_\_\_\_\_\_\_\_\_\_\_\_\_\_\_\_\_\_\_\_\_\_\_\_\_\_\_\_ \\
\_\_\_\_\_\_\_\_\_\_\_\_\_\_\_\_\_\_\_\_\_\_\_\_\_\_\_\_\_\_\_\_\_\_\_\_\_\_\_\_\_\_\_\_\_\_\_\_\_\_\_\_\_\_\_\_\_\_\_\_\_ \\
\_\_\_\_\_\_\_\_\_\_\_\_\_\_\_\_\_\_\_\_\_\_\_\_\_\_\_\_\_\_\_\_\_\_\_\_\_\_\_\_\_\_\_\_\_\_\_\_\_\_\_\_\_\_\_\_\_\_\_\_\_ \\
\_\_\_\_\_\_\_\_\_\_\_\_\_\_\_\_\_\_\_\_\_\_\_\_\_\_\_\_\_\_\_\_\_\_\_\_\_\_\_\_\_\_\_\_\_\_\_\_\_\_\_\_\_\_\_\_\_\_\_\_\_ \\
\_\_\_\_\_\_\_\_\_\_\_\_\_\_\_\_\_\_\_\_\_\_\_\_\_\_\_\_\_\_\_\_\_\_\_\_\_\_\_\_\_\_\_\_\_\_\_\_\_\_\_\_\_\_\_\_\_\_\_\_\_ \\
\_\_\_\_\_\_\_\_\_\_\_\_\_\_\_\_\_\_\_\_\_\_\_\_\_\_\_\_\_\_\_\_\_\_\_\_\_\_\_\_\_\_\_\_\_\_\_\_\_\_\_\_\_\_\_\_\_\_\_\_\_ \\
\_\_\_\_\_\_\_\_\_\_\_\_\_\_\_\_\_\_\_\_\_\_\_\_\_\_\_\_\_\_\_\_\_\_\_\_\_\_\_\_\_\_\_\_\_\_\_\_\_\_\_\_\_\_\_\_\_\_\_\_\_ \\
\_\_\_\_\_\_\_\_\_\_\_\_\_\_\_\_\_\_\_\_\_\_\_\_\_\_\_\_\_\_\_\_\_\_\_\_\_\_\_\_\_\_\_\_\_\_\_\_\_\_\_\_\_\_\_\_\_\_\_\_\_ \\
\_\_\_\_\_\_\_\_\_\_\_\_\_\_\_\_\_\_\_\_\_\_\_\_\_\_\_\_\_\_\_\_\_\_\_\_\_\_\_\_\_\_\_\_\_\_\_\_\_\_\_\_\_\_\_\_\_\_\_\_\_ \\
\bottomrule()
\end{longtable}

\begin{longtable}[]{@{}
  >{\raggedright\arraybackslash}p{(\columnwidth - 0\tabcolsep) * \real{1.0000}}@{}}
\toprule()
\begin{minipage}[b]{\linewidth}\raggedright
Describe the extent to which liability for Compensation is limited or eliminated:
\end{minipage} \\
\midrule()
\endhead
\_\_\_\_\_\_\_\_\_\_\_\_\_\_\_\_\_\_\_\_\_\_\_\_\_\_\_\_\_\_\_\_\_\_\_\_\_\_\_\_\_\_\_\_\_\_\_\_\_\_\_\_\_\_\_\_\_\_\_\_\_ \\
\_\_\_\_\_\_\_\_\_\_\_\_\_\_\_\_\_\_\_\_\_\_\_\_\_\_\_\_\_\_\_\_\_\_\_\_\_\_\_\_\_\_\_\_\_\_\_\_\_\_\_\_\_\_\_\_\_\_\_\_\_ \\
\_\_\_\_\_\_\_\_\_\_\_\_\_\_\_\_\_\_\_\_\_\_\_\_\_\_\_\_\_\_\_\_\_\_\_\_\_\_\_\_\_\_\_\_\_\_\_\_\_\_\_\_\_\_\_\_\_\_\_\_\_ \\
\_\_\_\_\_\_\_\_\_\_\_\_\_\_\_\_\_\_\_\_\_\_\_\_\_\_\_\_\_\_\_\_\_\_\_\_\_\_\_\_\_\_\_\_\_\_\_\_\_\_\_\_\_\_\_\_\_\_\_\_\_ \\
\_\_\_\_\_\_\_\_\_\_\_\_\_\_\_\_\_\_\_\_\_\_\_\_\_\_\_\_\_\_\_\_\_\_\_\_\_\_\_\_\_\_\_\_\_\_\_\_\_\_\_\_\_\_\_\_\_\_\_\_\_ \\
\_\_\_\_\_\_\_\_\_\_\_\_\_\_\_\_\_\_\_\_\_\_\_\_\_\_\_\_\_\_\_\_\_\_\_\_\_\_\_\_\_\_\_\_\_\_\_\_\_\_\_\_\_\_\_\_\_\_\_\_\_ \\
\_\_\_\_\_\_\_\_\_\_\_\_\_\_\_\_\_\_\_\_\_\_\_\_\_\_\_\_\_\_\_\_\_\_\_\_\_\_\_\_\_\_\_\_\_\_\_\_\_\_\_\_\_\_\_\_\_\_\_\_\_ \\
\_\_\_\_\_\_\_\_\_\_\_\_\_\_\_\_\_\_\_\_\_\_\_\_\_\_\_\_\_\_\_\_\_\_\_\_\_\_\_\_\_\_\_\_\_\_\_\_\_\_\_\_\_\_\_\_\_\_\_\_\_ \\
\_\_\_\_\_\_\_\_\_\_\_\_\_\_\_\_\_\_\_\_\_\_\_\_\_\_\_\_\_\_\_\_\_\_\_\_\_\_\_\_\_\_\_\_\_\_\_\_\_\_\_\_\_\_\_\_\_\_\_\_\_ \\
\_\_\_\_\_\_\_\_\_\_\_\_\_\_\_\_\_\_\_\_\_\_\_\_\_\_\_\_\_\_\_\_\_\_\_\_\_\_\_\_\_\_\_\_\_\_\_\_\_\_\_\_\_\_\_\_\_\_\_\_\_ \\
\_\_\_\_\_\_\_\_\_\_\_\_\_\_\_\_\_\_\_\_\_\_\_\_\_\_\_\_\_\_\_\_\_\_\_\_\_\_\_\_\_\_\_\_\_\_\_\_\_\_\_\_\_\_\_\_\_\_\_\_\_ \\
\bottomrule()
\end{longtable}

\begin{longtable}[]{@{}ll@{}}
\toprule()
Initialed: & \\
\midrule()
\endhead
\_\_\_\_\_\_\_\_\_\_\_\_\_\_ & \_\_\_\_\_\_\_\_\_\_\_\_\_\_ \\
Player & Team \\
\bottomrule()
\end{longtable}

\newpage

\hypertarget{exhibit-4-assignment-payments}{%
\subsection{Exhibit 4 --- Assignment Payments}\label{exhibit-4-assignment-payments}}

\begin{longtable}[]{@{}l@{}}
\toprule()
\endhead
Player: \_\_\_\_\_\_\_\_\_\_\_\_\_\_\_\_\_\_\_\_\_\_\_\_\_\_\_\_\_\_\_\_ \\
Team: \_\_\_\_\_\_\_\_\_\_\_\_\_\_\_\_\_\_\_\_\_\_\_\_\_\_\_\_\_\_\_\_\_\_ \\
Date: \_\_\_\_\_\_\_\_\_\_\_\_\_\_\_\_\_\_\_\_\_\_\_\_\_\_\_\_\_\_\_\_\_\_ \\
\bottomrule()
\end{longtable}

In the event this Contract is traded by the Team executing the Contract to another NBA Team, the Player shall be entitled to receive from the assignor Team, within thirty (30) days of the date of such trade, the following payment:

\begin{longtable}[]{@{}l@{}}
\toprule()
\endhead
\_\_\_\_\_\_\_\_\_\_\_\_\_\_\_\_\_\_\_\_\_\_\_\_\_\_\_\_\_\_\_\_\_\_\_\_\_\_\_\_\_\_\_\_\_\_\_\_\_\_\_\_\_\_\_\_\_\_\_\_\_ \\
\_\_\_\_\_\_\_\_\_\_\_\_\_\_\_\_\_\_\_\_\_\_\_\_\_\_\_\_\_\_\_\_\_\_\_\_\_\_\_\_\_\_\_\_\_\_\_\_\_\_\_\_\_\_\_\_\_\_\_\_\_ \\
\_\_\_\_\_\_\_\_\_\_\_\_\_\_\_\_\_\_\_\_\_\_\_\_\_\_\_\_\_\_\_\_\_\_\_\_\_\_\_\_\_\_\_\_\_\_\_\_\_\_\_\_\_\_\_\_\_\_\_\_\_ \\
\_\_\_\_\_\_\_\_\_\_\_\_\_\_\_\_\_\_\_\_\_\_\_\_\_\_\_\_\_\_\_\_\_\_\_\_\_\_\_\_\_\_\_\_\_\_\_\_\_\_\_\_\_\_\_\_\_\_\_\_\_ \\
\_\_\_\_\_\_\_\_\_\_\_\_\_\_\_\_\_\_\_\_\_\_\_\_\_\_\_\_\_\_\_\_\_\_\_\_\_\_\_\_\_\_\_\_\_\_\_\_\_\_\_\_\_\_\_\_\_\_\_\_\_ \\
\_\_\_\_\_\_\_\_\_\_\_\_\_\_\_\_\_\_\_\_\_\_\_\_\_\_\_\_\_\_\_\_\_\_\_\_\_\_\_\_\_\_\_\_\_\_\_\_\_\_\_\_\_\_\_\_\_\_\_\_\_ \\
\_\_\_\_\_\_\_\_\_\_\_\_\_\_\_\_\_\_\_\_\_\_\_\_\_\_\_\_\_\_\_\_\_\_\_\_\_\_\_\_\_\_\_\_\_\_\_\_\_\_\_\_\_\_\_\_\_\_\_\_\_ \\
\_\_\_\_\_\_\_\_\_\_\_\_\_\_\_\_\_\_\_\_\_\_\_\_\_\_\_\_\_\_\_\_\_\_\_\_\_\_\_\_\_\_\_\_\_\_\_\_\_\_\_\_\_\_\_\_\_\_\_\_\_ \\
\_\_\_\_\_\_\_\_\_\_\_\_\_\_\_\_\_\_\_\_\_\_\_\_\_\_\_\_\_\_\_\_\_\_\_\_\_\_\_\_\_\_\_\_\_\_\_\_\_\_\_\_\_\_\_\_\_\_\_\_\_ \\
\_\_\_\_\_\_\_\_\_\_\_\_\_\_\_\_\_\_\_\_\_\_\_\_\_\_\_\_\_\_\_\_\_\_\_\_\_\_\_\_\_\_\_\_\_\_\_\_\_\_\_\_\_\_\_\_\_\_\_\_\_ \\
\_\_\_\_\_\_\_\_\_\_\_\_\_\_\_\_\_\_\_\_\_\_\_\_\_\_\_\_\_\_\_\_\_\_\_\_\_\_\_\_\_\_\_\_\_\_\_\_\_\_\_\_\_\_\_\_\_\_\_\_\_ \\
\_\_\_\_\_\_\_\_\_\_\_\_\_\_\_\_\_\_\_\_\_\_\_\_\_\_\_\_\_\_\_\_\_\_\_\_\_\_\_\_\_\_\_\_\_\_\_\_\_\_\_\_\_\_\_\_\_\_\_\_\_ \\
\_\_\_\_\_\_\_\_\_\_\_\_\_\_\_\_\_\_\_\_\_\_\_\_\_\_\_\_\_\_\_\_\_\_\_\_\_\_\_\_\_\_\_\_\_\_\_\_\_\_\_\_\_\_\_\_\_\_\_\_\_ \\
\_\_\_\_\_\_\_\_\_\_\_\_\_\_\_\_\_\_\_\_\_\_\_\_\_\_\_\_\_\_\_\_\_\_\_\_\_\_\_\_\_\_\_\_\_\_\_\_\_\_\_\_\_\_\_\_\_\_\_\_\_ \\
\_\_\_\_\_\_\_\_\_\_\_\_\_\_\_\_\_\_\_\_\_\_\_\_\_\_\_\_\_\_\_\_\_\_\_\_\_\_\_\_\_\_\_\_\_\_\_\_\_\_\_\_\_\_\_\_\_\_\_\_\_ \\
\bottomrule()
\end{longtable}

\begin{longtable}[]{@{}ll@{}}
\toprule()
Initialed: & \\
\midrule()
\endhead
\_\_\_\_\_\_\_\_\_\_\_\_\_\_ & \_\_\_\_\_\_\_\_\_\_\_\_\_\_ \\
Player & Team \\
\bottomrule()
\end{longtable}

\newpage

\hypertarget{exhibit-5-other-athletic-activities}{%
\subsection{Exhibit 5 --- Other Athletic Activities}\label{exhibit-5-other-athletic-activities}}

\begin{longtable}[]{@{}l@{}}
\toprule()
\endhead
Player: \_\_\_\_\_\_\_\_\_\_\_\_\_\_\_\_\_\_\_\_\_\_\_\_\_\_\_\_\_\_\_\_ \\
Team: \_\_\_\_\_\_\_\_\_\_\_\_\_\_\_\_\_\_\_\_\_\_\_\_\_\_\_\_\_\_\_\_\_\_ \\
Date: \_\_\_\_\_\_\_\_\_\_\_\_\_\_\_\_\_\_\_\_\_\_\_\_\_\_\_\_\_\_\_\_\_\_ \\
\bottomrule()
\end{longtable}

Notwithstanding the provisions of paragraph 12 of this Contract, the Player and the Team agree that the Player need not obtain the consent of the Team in order to engage in the activities set forth below:

\begin{longtable}[]{@{}l@{}}
\toprule()
\endhead
\_\_\_\_\_\_\_\_\_\_\_\_\_\_\_\_\_\_\_\_\_\_\_\_\_\_\_\_\_\_\_\_\_\_\_\_\_\_\_\_\_\_\_\_\_\_\_\_\_\_\_\_\_\_\_\_\_\_\_\_\_ \\
\_\_\_\_\_\_\_\_\_\_\_\_\_\_\_\_\_\_\_\_\_\_\_\_\_\_\_\_\_\_\_\_\_\_\_\_\_\_\_\_\_\_\_\_\_\_\_\_\_\_\_\_\_\_\_\_\_\_\_\_\_ \\
\_\_\_\_\_\_\_\_\_\_\_\_\_\_\_\_\_\_\_\_\_\_\_\_\_\_\_\_\_\_\_\_\_\_\_\_\_\_\_\_\_\_\_\_\_\_\_\_\_\_\_\_\_\_\_\_\_\_\_\_\_ \\
\_\_\_\_\_\_\_\_\_\_\_\_\_\_\_\_\_\_\_\_\_\_\_\_\_\_\_\_\_\_\_\_\_\_\_\_\_\_\_\_\_\_\_\_\_\_\_\_\_\_\_\_\_\_\_\_\_\_\_\_\_ \\
\_\_\_\_\_\_\_\_\_\_\_\_\_\_\_\_\_\_\_\_\_\_\_\_\_\_\_\_\_\_\_\_\_\_\_\_\_\_\_\_\_\_\_\_\_\_\_\_\_\_\_\_\_\_\_\_\_\_\_\_\_ \\
\_\_\_\_\_\_\_\_\_\_\_\_\_\_\_\_\_\_\_\_\_\_\_\_\_\_\_\_\_\_\_\_\_\_\_\_\_\_\_\_\_\_\_\_\_\_\_\_\_\_\_\_\_\_\_\_\_\_\_\_\_ \\
\_\_\_\_\_\_\_\_\_\_\_\_\_\_\_\_\_\_\_\_\_\_\_\_\_\_\_\_\_\_\_\_\_\_\_\_\_\_\_\_\_\_\_\_\_\_\_\_\_\_\_\_\_\_\_\_\_\_\_\_\_ \\
\_\_\_\_\_\_\_\_\_\_\_\_\_\_\_\_\_\_\_\_\_\_\_\_\_\_\_\_\_\_\_\_\_\_\_\_\_\_\_\_\_\_\_\_\_\_\_\_\_\_\_\_\_\_\_\_\_\_\_\_\_ \\
\_\_\_\_\_\_\_\_\_\_\_\_\_\_\_\_\_\_\_\_\_\_\_\_\_\_\_\_\_\_\_\_\_\_\_\_\_\_\_\_\_\_\_\_\_\_\_\_\_\_\_\_\_\_\_\_\_\_\_\_\_ \\
\_\_\_\_\_\_\_\_\_\_\_\_\_\_\_\_\_\_\_\_\_\_\_\_\_\_\_\_\_\_\_\_\_\_\_\_\_\_\_\_\_\_\_\_\_\_\_\_\_\_\_\_\_\_\_\_\_\_\_\_\_ \\
\_\_\_\_\_\_\_\_\_\_\_\_\_\_\_\_\_\_\_\_\_\_\_\_\_\_\_\_\_\_\_\_\_\_\_\_\_\_\_\_\_\_\_\_\_\_\_\_\_\_\_\_\_\_\_\_\_\_\_\_\_ \\
\_\_\_\_\_\_\_\_\_\_\_\_\_\_\_\_\_\_\_\_\_\_\_\_\_\_\_\_\_\_\_\_\_\_\_\_\_\_\_\_\_\_\_\_\_\_\_\_\_\_\_\_\_\_\_\_\_\_\_\_\_ \\
\_\_\_\_\_\_\_\_\_\_\_\_\_\_\_\_\_\_\_\_\_\_\_\_\_\_\_\_\_\_\_\_\_\_\_\_\_\_\_\_\_\_\_\_\_\_\_\_\_\_\_\_\_\_\_\_\_\_\_\_\_ \\
\_\_\_\_\_\_\_\_\_\_\_\_\_\_\_\_\_\_\_\_\_\_\_\_\_\_\_\_\_\_\_\_\_\_\_\_\_\_\_\_\_\_\_\_\_\_\_\_\_\_\_\_\_\_\_\_\_\_\_\_\_ \\
\_\_\_\_\_\_\_\_\_\_\_\_\_\_\_\_\_\_\_\_\_\_\_\_\_\_\_\_\_\_\_\_\_\_\_\_\_\_\_\_\_\_\_\_\_\_\_\_\_\_\_\_\_\_\_\_\_\_\_\_\_ \\
\bottomrule()
\end{longtable}

\begin{longtable}[]{@{}ll@{}}
\toprule()
Initialed: & \\
\midrule()
\endhead
\_\_\_\_\_\_\_\_\_\_\_\_\_\_ & \_\_\_\_\_\_\_\_\_\_\_\_\_\_ \\
Player & Team \\
\bottomrule()
\end{longtable}

\newpage

\hypertarget{exhibit-6-physical-exam}{%
\subsection{Exhibit 6 --- Physical Exam}\label{exhibit-6-physical-exam}}

\begin{longtable}[]{@{}l@{}}
\toprule()
\endhead
Player: \_\_\_\_\_\_\_\_\_\_\_\_\_\_\_\_\_\_\_\_\_\_\_\_\_\_\_\_\_\_\_\_ \\
Team: \_\_\_\_\_\_\_\_\_\_\_\_\_\_\_\_\_\_\_\_\_\_\_\_\_\_\_\_\_\_\_\_\_\_ \\
Date: \_\_\_\_\_\_\_\_\_\_\_\_\_\_\_\_\_\_\_\_\_\_\_\_\_\_\_\_\_\_\_\_\_\_ \\
\bottomrule()
\end{longtable}

The Player and the Team agree that this Contract will be invalid and of no further force and effect unless the Player passes, in the sole discretion of a physician designated by the Team, a physical examination conducted within seventy-two (72) hours of the execution of this Contract.

\begin{longtable}[]{@{}ll@{}}
\toprule()
Initialed: & \\
\midrule()
\endhead
\_\_\_\_\_\_\_\_\_\_\_\_\_\_ & \_\_\_\_\_\_\_\_\_\_\_\_\_\_ \\
Player & Team \\
\bottomrule()
\end{longtable}

\newpage

\hypertarget{exhibit-7-substitute-for-upc-paragraph-7b}{%
\subsection{Exhibit 7 --- Substitute for UPC Paragraph 7(b)}\label{exhibit-7-substitute-for-upc-paragraph-7b}}

\begin{longtable}[]{@{}l@{}}
\toprule()
\endhead
Player: \_\_\_\_\_\_\_\_\_\_\_\_\_\_\_\_\_\_\_\_\_\_\_\_\_\_\_\_\_\_\_\_ \\
Team: \_\_\_\_\_\_\_\_\_\_\_\_\_\_\_\_\_\_\_\_\_\_\_\_\_\_\_\_\_\_\_\_\_\_ \\
Date: \_\_\_\_\_\_\_\_\_\_\_\_\_\_\_\_\_\_\_\_\_\_\_\_\_\_\_\_\_\_\_\_\_\_ \\
\bottomrule()
\end{longtable}

Paragraph 7(b) is hereby deleted and the following shall be substituted in place and instead thereof:

``7. (b) The Player agrees, notwithstanding any other provision of this Contract, that he will to the best of his ability maintain himself in physical condition sufficient to play skilled basketball at all times. If the Player, in the reasonable judgment of the physician designated for that purpose by the Team, is not in good physical condition at the date of his first scheduled game for the Team, or if, at the beginning of or during any Season, he fails to remain in good physical condition, in either event so as to render the Player unfit in the reasonable judgment of said physician to play skilled basketball, the Team shall have the right to suspend the Player for a period of one (1) week. At the end of such one-week period, should the Team notify the Player, orally or in writing, that in its reasonable judgment it believes the Player is still not in good physical condition, the Team shall have the right to suspend the Player for successive one-week periods until the Player, in the reasonable judgment of the Team's physician, is in good physical condition; provided, however, that at the end of each such one-week period of suspension, the Team provides the notice specified above, and provided further that, at the end of any such one-week period and if the player so requests, then the Player shall be examined by a physician or physicians designated for such purpose by the President, or any Vice President if the President is not available, of the American Society of Orthopedic Physicians, or equivalent organization (the''Reviewing Physician''), whose sole judgment concerning the physical condition of the Player to play skilled basketball shall be binding upon the Team and the Player for purposes of this paragraph. The suspension of the Player shall be terminated promptly upon the failure of the Team to give the Player the notice required at the end of the one-week period or upon the finding of said Reviewing Physician that the Player is in physical condition sufficient to play skilled basketball. In the event of a suspension permitted hereunder, the Compensation (excluding any signing bonus or Incentive Compensation) payable to the Player for any Season during such suspension shall be reduced in the same proportion as the length of the period of disability so determined bears to the length of the Season.''

\begin{longtable}[]{@{}ll@{}}
\toprule()
Initialed: & \\
\midrule()
\endhead
\_\_\_\_\_\_\_\_\_\_\_\_\_\_ & \_\_\_\_\_\_\_\_\_\_\_\_\_\_ \\
Player & Team \\
\bottomrule()
\end{longtable}

\newpage

\hypertarget{exhibit-8-sign-and-trade}{%
\subsection{Exhibit 8 --- Sign and Trade}\label{exhibit-8-sign-and-trade}}

\begin{longtable}[]{@{}l@{}}
\toprule()
\endhead
Player: \_\_\_\_\_\_\_\_\_\_\_\_\_\_\_\_\_\_\_\_\_\_\_\_\_\_\_\_\_\_\_\_ \\
Team: \_\_\_\_\_\_\_\_\_\_\_\_\_\_\_\_\_\_\_\_\_\_\_\_\_\_\_\_\_\_\_\_\_\_ \\
Date: \_\_\_\_\_\_\_\_\_\_\_\_\_\_\_\_\_\_\_\_\_\_\_\_\_\_\_\_\_\_\_\_\_\_ \\
\bottomrule()
\end{longtable}

The Player and the Team agree that this Contract will be invalid and of no further force and effect unless the Contract is assigned to the {[}assignee team{]} within forty-eight (48) hours of its execution, and such assignment is ultimately consummated.

\begin{longtable}[]{@{}ll@{}}
\toprule()
Initialed: & \\
\midrule()
\endhead
\_\_\_\_\_\_\_\_\_\_\_\_\_\_ & \_\_\_\_\_\_\_\_\_\_\_\_\_\_ \\
Player & Team \\
\bottomrule()
\end{longtable}

\hypertarget{rookie-scales}{%
\chapter{ROOKIE SCALES}\label{rookie-scales}}

\hypertarget{nba-rookie-scale-000s}{%
\section{1998-99 NBA Rookie Scale (\$000's)}\label{nba-rookie-scale-000s}}

\begin{longtable}[]{@{}
  >{\centering\arraybackslash}p{(\columnwidth - 10\tabcolsep) * \real{0.0806}}
  >{\raggedright\arraybackslash}p{(\columnwidth - 10\tabcolsep) * \real{0.1452}}
  >{\raggedright\arraybackslash}p{(\columnwidth - 10\tabcolsep) * \real{0.1613}}
  >{\raggedright\arraybackslash}p{(\columnwidth - 10\tabcolsep) * \real{0.1613}}
  >{\centering\arraybackslash}p{(\columnwidth - 10\tabcolsep) * \real{0.2258}}
  >{\centering\arraybackslash}p{(\columnwidth - 10\tabcolsep) * \real{0.2258}}@{}}
\toprule()
\begin{minipage}[b]{\linewidth}\centering
Pick
\end{minipage} & \begin{minipage}[b]{\linewidth}\raggedright
1st Year Salary
\end{minipage} & \begin{minipage}[b]{\linewidth}\raggedright
2nd Year Salary
\end{minipage} & \begin{minipage}[b]{\linewidth}\raggedright
3rd Year Salary
\end{minipage} & \begin{minipage}[b]{\linewidth}\centering
4th Year Option: Percentage Increase Over 3rd Year Salary
\end{minipage} & \begin{minipage}[b]{\linewidth}\centering
Qualifying Offer: Percentage Increase Over 4th Year Salary
\end{minipage} \\
\midrule()
\endhead
1 & 2,679.3 & 2,880.2 & 3,081.2 & 26.1\% & 30.0\% \\
2 & 2,397.2 & 2,577.0 & 2,756.8 & 26.2\% & 30.5\% \\
3 & 2,152.8 & 2,314.3 & 2,475.7 & 26.4\% & 31.2\% \\
4 & 1,940.9 & 2,086.5 & 2,232.0 & 26.5\% & 31.9\% \\
5 & 1,757.6 & 1,889.4 & 2,021.2 & 26.7\% & 32.6\% \\
6 & 1,596.4 & 1,716.1 & 1,835.9 & 26.8\% & 33.4\% \\
7 & 1,457.3 & 1,566.6 & 1,675.9 & 27.0\% & 34.1\% \\
8 & 1,335.1 & 1,435.2 & 1,535.4 & 27.2\% & 34.8\% \\
9 & 1,227.2 & 1,319.2 & 1,411.3 & 27.4\% & 35.5\% \\
10 & 1,165.8 & 1,253.3 & 1,340.7 & 27.5\% & 36.2\% \\
11 & 1,107.5 & 1,190.6 & 1,273.7 & 32.7\% & 36.9\% \\
12 & 1,052.2 & 1,131.1 & 1,210.0 & 37.8\% & 37.6\% \\
13 & 999.6 & 1,074.5 & 1,149.5 & 42.9\% & 38.3\% \\
14 & 949.6 & 1,020.8 & 1,092.0 & 48.1\% & 39.1\% \\
15 & 902.1 & 969.8 & 1,037.4 & 53.3\% & 39.8\% \\
16 & 857.0 & 921.3 & 985.5 & 53.4\% & 40.5\% \\
17 & 814.1 & 875.2 & 936.3 & 53.6\% & 41.2\% \\
18 & 773.4 & 831.5 & 889.5 & 53.8\% & 41.9\% \\
19 & 738.6 & 794.0 & 849.4 & 54.0\% & 42.6\% \\
20 & 709.1 & 762.3 & 815.5 & 54.2\% & 43.3\% \\
21 & 680.7 & 731.8 & 782.8 & 59.3\% & 44.1\% \\
22 & 653.5 & 702.5 & 751.5 & 64.5\% & 44.8\% \\
23 & 627.4 & 674.4 & 721.5 & 69.7\% & 45.5\% \\
24 & 602.3 & 647.4 & 692.6 & 74.9\% & 46.2\% \\
25 & 578.2 & 621.5 & 664.9 & 80.1\% & 46.9\% \\
26 & 559.0 & 600.9 & 642.9 & 80.3\% & 47.6\% \\
27 & 542.9 & 583.6 & 624.3 & 80.4\% & 48.3\% \\
28 & 539.5 & 580.0 & 620.4 & 80.5\% & 49.0\% \\
29 & 535.6 & 575.8 & 615.9 & 80.5\% & 50.0\% \\
\bottomrule()
\end{longtable}

\newpage

\hypertarget{nba-rookie-scale-000s-1}{%
\section{1999-2000 NBA Rookie Scale (\$000's)}\label{nba-rookie-scale-000s-1}}

\begin{longtable}[]{@{}
  >{\centering\arraybackslash}p{(\columnwidth - 10\tabcolsep) * \real{0.0806}}
  >{\raggedright\arraybackslash}p{(\columnwidth - 10\tabcolsep) * \real{0.1452}}
  >{\raggedright\arraybackslash}p{(\columnwidth - 10\tabcolsep) * \real{0.1613}}
  >{\raggedright\arraybackslash}p{(\columnwidth - 10\tabcolsep) * \real{0.1613}}
  >{\centering\arraybackslash}p{(\columnwidth - 10\tabcolsep) * \real{0.2258}}
  >{\centering\arraybackslash}p{(\columnwidth - 10\tabcolsep) * \real{0.2258}}@{}}
\toprule()
\begin{minipage}[b]{\linewidth}\centering
Pick
\end{minipage} & \begin{minipage}[b]{\linewidth}\raggedright
1st Year Salary
\end{minipage} & \begin{minipage}[b]{\linewidth}\raggedright
2nd Year Salary
\end{minipage} & \begin{minipage}[b]{\linewidth}\raggedright
3rd Year Salary
\end{minipage} & \begin{minipage}[b]{\linewidth}\centering
4th Year Option: Percentage Increase Over 3rd Year Salary
\end{minipage} & \begin{minipage}[b]{\linewidth}\centering
Qualifying Offer: Percentage Increase Over 4th Year Salary
\end{minipage} \\
\midrule()
\endhead
1 & 2,813.3 & 3,024.3 & 3,235.3 & 26.1\% & 30.0\% \\
2 & 2,517.1 & 2,705.8 & 2,894.6 & 26.2\% & 30.5\% \\
3 & 2,260.4 & 2,430.0 & 2,599.5 & 26.4\% & 31.2\% \\
4 & 2,037.9 & 2,190.8 & 2,343.6 & 26.5\% & 31.9\% \\
5 & 1,845.5 & 1,983.9 & 2,122.3 & 26.7\% & 32.6\% \\
6 & 1,676.2 & 1,801.9 & 1,927.7 & 26.8\% & 33.4\% \\
7 & 1,530.2 & 1,644.9 & 1,759.7 & 27.0\% & 34.1\% \\
8 & 1,401.9 & 1,507.0 & 1,612.1 & 27.2\% & 34.8\% \\
9 & 1,288.6 & 1,385.2 & 1,481.8 & 27.4\% & 35.5\% \\
10 & 1,224.1 & 1,315.9 & 1,407.8 & 27.5\% & 36.2\% \\
11 & 1,162.9 & 1,250.1 & 1,337.4 & 32.7\% & 36.9\% \\
12 & 1,104.8 & 1,187.6 & 1,270.5 & 37.8\% & 37.6\% \\
13 & 1,049.5 & 1,128.3 & 1,207.0 & 42.9\% & 38.3\% \\
14 & 997.1 & 1,071.8 & 1,146.6 & 48.1\% & 39.1\% \\
15 & 947.2 & 1,018.3 & 1,089.3 & 53.3\% & 39.8\% \\
16 & 899.8 & 967.3 & 1,034.8 & 53.4\% & 40.5\% \\
17 & 854.9 & 919.0 & 983.1 & 53.6\% & 41.2\% \\
18 & 812.1 & 873.0 & 933.9 & 53.8\% & 41.9\% \\
19 & 775.6 & 833.7 & 891.9 & 54.0\% & 42.6\% \\
20 & 744.5 & 800.4 & 856.2 & 54.2\% & 43.3\% \\
21 & 714.8 & 768.4 & 822.0 & 59.3\% & 44.1\% \\
22 & 686.2 & 737.6 & 789.1 & 64.5\% & 44.8\% \\
23 & 658.7 & 708.1 & 757.5 & 69.7\% & 45.5\% \\
24 & 632.4 & 679.8 & 727.2 & 74.9\% & 46.2\% \\
25 & 607.1 & 652.6 & 698.1 & 80.1\% & 46.9\% \\
26 & 587.0 & 631.0 & 675.0 & 80.3\% & 47.6\% \\
27 & 570.0 & 612.8 & 655.5 & 80.4\% & 48.3\% \\
28 & 566.5 & 609.0 & 651.4 & 80.5\% & 49.0\% \\
29 & 562.4 & 604.6 & 646.7 & 80.5\% & 50.0\% \\
\bottomrule()
\end{longtable}

\newpage

\hypertarget{nba-rookie-scale-000s-2}{%
\section{2000-2001 NBA Rookie Scale (\$000's)}\label{nba-rookie-scale-000s-2}}

\begin{longtable}[]{@{}
  >{\centering\arraybackslash}p{(\columnwidth - 10\tabcolsep) * \real{0.0806}}
  >{\raggedright\arraybackslash}p{(\columnwidth - 10\tabcolsep) * \real{0.1452}}
  >{\raggedright\arraybackslash}p{(\columnwidth - 10\tabcolsep) * \real{0.1613}}
  >{\raggedright\arraybackslash}p{(\columnwidth - 10\tabcolsep) * \real{0.1613}}
  >{\centering\arraybackslash}p{(\columnwidth - 10\tabcolsep) * \real{0.2258}}
  >{\centering\arraybackslash}p{(\columnwidth - 10\tabcolsep) * \real{0.2258}}@{}}
\toprule()
\begin{minipage}[b]{\linewidth}\centering
Pick
\end{minipage} & \begin{minipage}[b]{\linewidth}\raggedright
1st Year Salary
\end{minipage} & \begin{minipage}[b]{\linewidth}\raggedright
2nd Year Salary
\end{minipage} & \begin{minipage}[b]{\linewidth}\raggedright
3rd Year Salary
\end{minipage} & \begin{minipage}[b]{\linewidth}\centering
4th Year Option: Percentage Increase Over 3rd Year Salary
\end{minipage} & \begin{minipage}[b]{\linewidth}\centering
Qualifying Offer: Percentage Increase Over 4th Year Salary
\end{minipage} \\
\midrule()
\endhead
1 & 2,947.2 & 3,168.3 & 3,389.3 & 26.1\% & 30.0\% \\
2 & 2,636.9 & 2,834.7 & 3,032.5 & 26.2\% & 30.5\% \\
3 & 2,368.1 & 2,545.7 & 2,723.3 & 26.4\% & 31.2\% \\
4 & 2,135.0 & 2,295.1 & 2,455.2 & 26.5\% & 31.9\% \\
5 & 1,933.4 & 2,078.4 & 2,223.4 & 26.7\% & 32.6\% \\
6 & 1,756.0 & 1,887.7 & 2,019.4 & 26.8\% & 33.4\% \\
7 & 1,603.0 & 1,723.3 & 1,843.5 & 27.0\% & 34.1\% \\
8 & 1,468.6 & 1,578.8 & 1,688.9 & 27.2\% & 34.8\% \\
9 & 1,349.9 & 1,451.2 & 1,552.4 & 27.4\% & 35.5\% \\
10 & 1,282.4 & 1,378.6 & 1,474.8 & 27.5\% & 36.2\% \\
11 & 1,218.3 & 1,309.7 & 1,401.0 & 32.7\% & 36.9\% \\
12 & 1,157.4 & 1,244.2 & 1,331.0 & 37.8\% & 37.6\% \\
13 & 1,099.5 & 1,182.0 & 1,264.4 & 42.9\% & 38.3\% \\
14 & 1,044.5 & 1,122.9 & 1,201.2 & 48.1\% & 39.1\% \\
15 & 992.3 & 1,066.7 & 1,141.2 & 53.3\% & 39.8\% \\
16 & 942.7 & 1,013.4 & 1,084.1 & 53.4\% & 40.5\% \\
17 & 895.6 & 962.7 & 1,029.9 & 53.6\% & 41.2\% \\
18 & 850.8 & 914.6 & 978.4 & 53.8\% & 41.9\% \\
19 & 812.5 & 873.4 & 934.4 & 54.0\% & 42.6\% \\
20 & 780.0 & 838.5 & 897.0 & 54.2\% & 43.3\% \\
21 & 748.8 & 805.0 & 861.1 & 59.3\% & 44.1\% \\
22 & 718.8 & 772.8 & 826.7 & 64.5\% & 44.8\% \\
23 & 690.1 & 741.9 & 793.6 & 69.7\% & 45.5\% \\
24 & 662.5 & 712.2 & 761.9 & 74.9\% & 46.2\% \\
25 & 636.0 & 683.7 & 731.4 & 80.1\% & 46.9\% \\
26 & 614.9 & 661.0 & 707.1 & 80.3\% & 47.6\% \\
27 & 597.2 & 642.0 & 686.7 & 80.4\% & 48.3\% \\
28 & 593.5 & 638.0 & 682.5 & 80.5\% & 49.0\% \\
29 & 589.2 & 633.3 & 677.5 & 80.5\% & 50.0\% \\
\bottomrule()
\end{longtable}

\newpage

\hypertarget{nba-rookie-scale-000s-3}{%
\section{2001-2002 NBA Rookie Scale (\$000's)}\label{nba-rookie-scale-000s-3}}

\begin{longtable}[]{@{}
  >{\centering\arraybackslash}p{(\columnwidth - 10\tabcolsep) * \real{0.0806}}
  >{\raggedright\arraybackslash}p{(\columnwidth - 10\tabcolsep) * \real{0.1452}}
  >{\raggedright\arraybackslash}p{(\columnwidth - 10\tabcolsep) * \real{0.1613}}
  >{\raggedright\arraybackslash}p{(\columnwidth - 10\tabcolsep) * \real{0.1613}}
  >{\centering\arraybackslash}p{(\columnwidth - 10\tabcolsep) * \real{0.2258}}
  >{\centering\arraybackslash}p{(\columnwidth - 10\tabcolsep) * \real{0.2258}}@{}}
\toprule()
\begin{minipage}[b]{\linewidth}\centering
Pick
\end{minipage} & \begin{minipage}[b]{\linewidth}\raggedright
1st Year Salary
\end{minipage} & \begin{minipage}[b]{\linewidth}\raggedright
2nd Year Salary
\end{minipage} & \begin{minipage}[b]{\linewidth}\raggedright
3rd Year Salary
\end{minipage} & \begin{minipage}[b]{\linewidth}\centering
4th Year Option: Percentage Increase Over 3rd Year Salary
\end{minipage} & \begin{minipage}[b]{\linewidth}\centering
Qualifying Offer: Percentage Increase Over 4th Year Salary
\end{minipage} \\
\midrule()
\endhead
1 & 3,081.2 & 3,312.3 & 3,543.4 & 26.1\% & 30.0\% \\
2 & 2,756.8 & 2,963.5 & 3,170.3 & 26.2\% & 30.5\% \\
3 & 2,475.7 & 2,661.4 & 2,847.1 & 26.4\% & 31.2\% \\
4 & 2,232.0 & 2,399.4 & 2,566.8 & 26.5\% & 31.9\% \\
5 & 2,021.2 & 2,172.8 & 2,324.4 & 26.7\% & 32.6\% \\
6 & 1,835.9 & 1,973.5 & 2,111.2 & 26.8\% & 33.4\% \\
7 & 1,675.9 & 1,801.6 & 1,927.3 & 27.0\% & 34.1\% \\
8 & 1,535.4 & 1,650.5 & 1,765.7 & 27.2\% & 34.8\% \\
9 & 1,411.3 & 1,517.1 & 1,623.0 & 27.4\% & 35.5\% \\
10 & 1,340.7 & 1,441.3 & 1,541.8 & 27.5\% & 36.2\% \\
11 & 1,273.7 & 1,369.2 & 1,464.7 & 32.7\% & 36.9\% \\
12 & 1,210.0 & 1,300.7 & 1,391.5 & 37.8\% & 37.6\% \\
13 & 1,149.5 & 1,235.7 & 1,321.9 & 42.9\% & 38.3\% \\
14 & 1,092.0 & 1,173.9 & 1,255.8 & 48.1\% & 39.1\% \\
15 & 1,037.4 & 1,115.2 & 1,193.0 & 53.3\% & 39.8\% \\
16 & 985.5 & 1,059.5 & 1,133.4 & 53.4\% & 40.5\% \\
17 & 936.3 & 1,006.5 & 1,076.7 & 53.6\% & 41.2\% \\
18 & 889.5 & 956.2 & 1,022.9 & 53.8\% & 41.9\% \\
19 & 849.4 & 913.1 & 976.8 & 54.0\% & 42.6\% \\
20 & 815.5 & 876.6 & 937.8 & 54.2\% & 43.3\% \\
21 & 782.8 & 841.6 & 900.3 & 59.3\% & 44.1\% \\
22 & 751.5 & 807.9 & 864.3 & 64.5\% & 44.8\% \\
23 & 721.5 & 775.6 & 829.7 & 69.7\% & 45.5\% \\
24 & 692.6 & 744.5 & 796.5 & 74.9\% & 46.2\% \\
25 & 664.9 & 714.8 & 764.6 & 80.1\% & 46.9\% \\
26 & 642.9 & 691.1 & 739.3 & 80.3\% & 47.6\% \\
27 & 624.3 & 671.1 & 718.0 & 80.4\% & 48.3\% \\
28 & 620.4 & 667.0 & 713.5 & 80.5\% & 49.0\% \\
29 & 615.9 & 662.1 & 708.3 & 80.5\% & 50.0\% \\
\bottomrule()
\end{longtable}

\newpage

\hypertarget{nba-rookie-scale-000s-4}{%
\section{2002-2003 NBA Rookie Scale (\$000's)}\label{nba-rookie-scale-000s-4}}

\begin{longtable}[]{@{}
  >{\centering\arraybackslash}p{(\columnwidth - 10\tabcolsep) * \real{0.0806}}
  >{\raggedright\arraybackslash}p{(\columnwidth - 10\tabcolsep) * \real{0.1452}}
  >{\raggedright\arraybackslash}p{(\columnwidth - 10\tabcolsep) * \real{0.1613}}
  >{\raggedright\arraybackslash}p{(\columnwidth - 10\tabcolsep) * \real{0.1613}}
  >{\centering\arraybackslash}p{(\columnwidth - 10\tabcolsep) * \real{0.2258}}
  >{\centering\arraybackslash}p{(\columnwidth - 10\tabcolsep) * \real{0.2258}}@{}}
\toprule()
\begin{minipage}[b]{\linewidth}\centering
Pick
\end{minipage} & \begin{minipage}[b]{\linewidth}\raggedright
1st Year Salary
\end{minipage} & \begin{minipage}[b]{\linewidth}\raggedright
2nd Year Salary
\end{minipage} & \begin{minipage}[b]{\linewidth}\raggedright
3rd Year Salary
\end{minipage} & \begin{minipage}[b]{\linewidth}\centering
4th Year Option: Percentage Increase Over 3rd Year Salary
\end{minipage} & \begin{minipage}[b]{\linewidth}\centering
Qualifying Offer: Percentage Increase Over 4th Year Salary
\end{minipage} \\
\midrule()
\endhead
1 & 3,215.2 & 3,456.3 & 3,697.4 & 26.1\% & 30.0\% \\
2 & 2,876.6 & 3,092.4 & 3,308.1 & 26.2\% & 30.5\% \\
3 & 2,583.4 & 2,777.1 & 2,970.9 & 26.4\% & 31.2\% \\
4 & 2,329.1 & 2,503.8 & 2,678.4 & 26.5\% & 31.9\% \\
5 & 2,109.1 & 2,267.3 & 2,425.5 & 26.7\% & 32.6\% \\
6 & 1,915.7 & 2,059.4 & 2,203.0 & 26.8\% & 33.4\% \\
7 & 1,748.8 & 1,879.9 & 2,011.1 & 27.0\% & 34.1\% \\
8 & 1,602.1 & 1,722.3 & 1,842.4 & 27.2\% & 34.8\% \\
9 & 1,472.6 & 1,583.1 & 1,693.5 & 27.4\% & 35.5\% \\
10 & 1,399.0 & 1,503.9 & 1,608.9 & 27.5\% & 36.2\% \\
11 & 1,329.1 & 1,428.7 & 1,528.4 & 32.7\% & 36.9\% \\
12 & 1,262.6 & 1,357.3 & 1,452.0 & 37.8\% & 37.6\% \\
13 & 1,199.5 & 1,289.4 & 1,379.4 & 42.9\% & 38.3\% \\
14 & 1,139.5 & 1,255.0 & 1,310.4 & 48.1\% & 39.1\% \\
15 & 1,082.5 & 1,163.7 & 1,244.9 & 53.3\% & 39.8\% \\
16 & 1,028.4 & 1,105.5 & 1,182.7 & 53.4\% & 40.5\% \\
17 & 977.0 & 1,050.3 & 1,123.5 & 53.6\% & 41.2\% \\
18 & 928.1 & 997.7 & 1,067.4 & 53.8\% & 41.9\% \\
19 & 886.4 & 952.8 & 1,019.3 & 54.0\% & 42.6\% \\
20 & 850.9 & 914.7 & 978.5 & 54.2\% & 43.3\% \\
21 & 816.9 & 878.1 & 939.4 & 59.3\% & 44.1\% \\
22 & 784.2 & 843.0 & 901.8 & 64.5\% & 44.8\% \\
23 & 752.8 & 809.3 & 865.8 & 69.7\% & 45.5\% \\
24 & 722.7 & 776.9 & 831.1 & 74.9\% & 46.2\% \\
25 & 693.8 & 745.8 & 797.9 & 80.1\% & 46.9\% \\
26 & 670.8 & 721.1 & 771.4 & 80.3\% & 47.6\% \\
27 & 651.5 & 700.3 & 749.2 & 80.4\% & 48.3\% \\
28 & 647.4 & 696.0 & 744.5 & 80.5\% & 49.0\% \\
29 & 642.7 & 690.9 & 739.1 & 80.5\% & 50.0\% \\
\bottomrule()
\end{longtable}

\newpage

\hypertarget{nba-rookie-scale-000s-5}{%
\section{2003-2004 NBA Rookie Scale (\$000's)}\label{nba-rookie-scale-000s-5}}

\begin{longtable}[]{@{}
  >{\centering\arraybackslash}p{(\columnwidth - 10\tabcolsep) * \real{0.0806}}
  >{\raggedright\arraybackslash}p{(\columnwidth - 10\tabcolsep) * \real{0.1452}}
  >{\raggedright\arraybackslash}p{(\columnwidth - 10\tabcolsep) * \real{0.1613}}
  >{\raggedright\arraybackslash}p{(\columnwidth - 10\tabcolsep) * \real{0.1613}}
  >{\centering\arraybackslash}p{(\columnwidth - 10\tabcolsep) * \real{0.2258}}
  >{\centering\arraybackslash}p{(\columnwidth - 10\tabcolsep) * \real{0.2258}}@{}}
\toprule()
\begin{minipage}[b]{\linewidth}\centering
Pick
\end{minipage} & \begin{minipage}[b]{\linewidth}\raggedright
1st Year Salary
\end{minipage} & \begin{minipage}[b]{\linewidth}\raggedright
2nd Year Salary
\end{minipage} & \begin{minipage}[b]{\linewidth}\raggedright
3rd Year Salary
\end{minipage} & \begin{minipage}[b]{\linewidth}\centering
4th Year Option: Percentage Increase Over 3rd Year Salary
\end{minipage} & \begin{minipage}[b]{\linewidth}\centering
Qualifying Offer: Percentage Increase Over 4th Year Salary
\end{minipage} \\
\midrule()
\endhead
1 & 3,349.1 & 3,600.3 & 3,851.5 & 26.1\% & 30.0\% \\
2 & 2,996.5 & 3,221.2 & 3,446.0 & 26.2\% & 30.5\% \\
3 & 2,691.0 & 2,892.8 & 3,094.7 & 26.4\% & 31.2\% \\
4 & 2,426.1 & 2,608.1 & 2,790.0 & 26.5\% & 31.9\% \\
5 & 2,197.0 & 2,361.8 & 2,526.6 & 26.7\% & 32.6\% \\
6 & 1,995.5 & 2,145.2 & 2,294.8 & 26.8\% & 33.4\% \\
7 & 1,821.6 & 1,958.2 & 2,094.9 & 27.0\% & 34.1\% \\
8 & 1,668.9 & 1,794.0 & 1,919.2 & 27.2\% & 34.8\% \\
9 & 1,534.0 & 1,649.1 & 1,764.1 & 27.4\% & 35.5\% \\
10 & 1,457.3 & 1,566.6 & 1,675.9 & 27.5\% & 36.2\% \\
11 & 1,384.4 & 1,488.3 & 1,592.1 & 32.7\% & 36.9\% \\
12 & 1,315.2 & 1,413.9 & 1,512.5 & 37.8\% & 37.6\% \\
13 & 1,249.5 & 1,343.2 & 1,436.9 & 42.9\% & 38.3\% \\
14 & 1,187.0 & 1,276.0 & 1,365.0 & 48.1\% & 39.1\% \\
15 & 1,127.6 & 1,212.2 & 1,296.8 & 53.3\% & 39.8\% \\
16 & 1,071.2 & 1,151.6 & 1,231.9 & 53.4\% & 40.5\% \\
17 & 1,017.7 & 1,094.0 & 1,170.3 & 53.6\% & 41.2\% \\
18 & 966.8 & 1,039.3 & 1,111.8 & 53.8\% & 41.9\% \\
19 & 923.3 & 992.5 & 1,061.8 & 54.0\% & 42.6\% \\
20 & 886.4 & 952.8 & 1,019.3 & 54.2\% & 43.3\% \\
21 & 850.9 & 914.7 & 978.5 & 59.3\% & 44.1\% \\
22 & 816.9 & 878.1 & 939.4 & 64.5\% & 44.8\% \\
23 & 784.2 & 843.0 & 901.8 & 69.7\% & 45.5\% \\
24 & 752.8 & 809.3 & 865.8 & 74.9\% & 46.2\% \\
25 & 722.7 & 776.9 & 831.1 & 80.1\% & 46.9\% \\
26 & 698.8 & 751.2 & 803.6 & 80.3\% & 47.6\% \\
27 & 678.6 & 729.5 & 780.4 & 80.4\% & 48.3\% \\
28 & 674.4 & 725.0 & 775.5 & 80.5\% & 49.0\% \\
29 & 669.5 & 719.7 & 769.9 & 80.5\% & 50.0\% \\
\bottomrule()
\end{longtable}

\newpage

\hypertarget{nba-rookie-scale-000s-6}{%
\section{2004-2005 NBA Rookie Scale (\$000's)}\label{nba-rookie-scale-000s-6}}

\begin{longtable}[]{@{}
  >{\centering\arraybackslash}p{(\columnwidth - 10\tabcolsep) * \real{0.0806}}
  >{\raggedright\arraybackslash}p{(\columnwidth - 10\tabcolsep) * \real{0.1452}}
  >{\raggedright\arraybackslash}p{(\columnwidth - 10\tabcolsep) * \real{0.1613}}
  >{\raggedright\arraybackslash}p{(\columnwidth - 10\tabcolsep) * \real{0.1613}}
  >{\centering\arraybackslash}p{(\columnwidth - 10\tabcolsep) * \real{0.2258}}
  >{\centering\arraybackslash}p{(\columnwidth - 10\tabcolsep) * \real{0.2258}}@{}}
\toprule()
\begin{minipage}[b]{\linewidth}\centering
Pick
\end{minipage} & \begin{minipage}[b]{\linewidth}\raggedright
1st Year Salary
\end{minipage} & \begin{minipage}[b]{\linewidth}\raggedright
2nd Year Salary
\end{minipage} & \begin{minipage}[b]{\linewidth}\raggedright
3rd Year Salary
\end{minipage} & \begin{minipage}[b]{\linewidth}\centering
4th Year Option: Percentage Increase Over 3rd Year Salary
\end{minipage} & \begin{minipage}[b]{\linewidth}\centering
Qualifying Offer: Percentage Increase Over 4th Year Salary
\end{minipage} \\
\midrule()
\endhead
1 & 3,483.1 & 3,744.3 & 4,005.6 & 26.1\% & 30.0\% \\
2 & 3,116.4 & 3,350.1 & 3,583.8 & 26.2\% & 30.5\% \\
3 & 2,798.6 & 3,008.5 & 3,218.4 & 26.4\% & 31.2\% \\
4 & 2,523.2 & 2,712.4 & 2,901.6 & 26.5\% & 31.9\% \\
5 & 2,284.9 & 2,456.2 & 2,627.6 & 26.7\% & 32.6\% \\
6 & 2,075.3 & 2,231.0 & 2,386.6 & 26.8\% & 33.4\% \\
7 & 1,894.5 & 2,036.6 & 2,178.7 & 27.0\% & 34.1\% \\
8 & 1,735.6 & 1,865.8 & 1,996.0 & 27.2\% & 34.8\% \\
9 & 1,595.4 & 1,715.0 & 1,834.7 & 27.4\% & 35.5\% \\
10 & 1,515.6 & 1,629.3 & 1,742.9 & 27.5\% & 36.2\% \\
11 & 1,439.8 & 1,547.8 & 1,655.8 & 32.7\% & 36.9\% \\
12 & 1,367.8 & 1,470.4 & 1,573.0 & 37.8\% & 37.6\% \\
13 & 1,299.4 & 1,396.9 & 1,494.3 & 42.9\% & 38.3\% \\
14 & 1,234.5 & 1,327.0 & 1,419.6 & 48.1\% & 39.1\% \\
15 & 1,172.7 & 1,260.7 & 1,348.6 & 53.3\% & 39.8\% \\
16 & 1,114.1 & 1,197.7 & 1,281.2 & 53.4\% & 40.5\% \\
17 & 1,058.4 & 1,137.8 & 1,217.2 & 53.6\% & 41.2\% \\
18 & 1,005.5 & 1,080.9 & 1,156.3 & 53.8\% & 41.9\% \\
19 & 960.2 & 1,032.2 & 1,104.3 & 54.0\% & 42.6\% \\
20 & 921.8 & 991.0 & 1,060.1 & 54.2\% & 43.3\% \\
21 & 884.9 & 951.3 & 1,017.7 & 59.3\% & 44.1\% \\
22 & 849.5 & 913.3 & 977.0 & 64.5\% & 44.8\% \\
23 & 815.6 & 876.7 & 937.9 & 69.7\% & 45.5\% \\
24 & 782.9 & 841.7 & 900.4 & 74.9\% & 46.2\% \\
25 & 751.6 & 808.0 & 864.4 & 80.1\% & 46.9\% \\
26 & 726.7 & 781.2 & 835.7 & 80.3\% & 47.6\% \\
27 & 705.7 & 758.7 & 811.6 & 80.4\% & 48.3\% \\
28 & 701.4 & 754.0 & 806.6 & 80.5\% & 49.0\% \\
29 & 696.3 & 748.5 & 800.7 & 80.5\% & 50.0\% \\
\bottomrule()
\end{longtable}

\hypertarget{minimum-annual-salary-scale}{%
\chapter{MINIMUM ANNUAL SALARY SCALE}\label{minimum-annual-salary-scale}}

\begin{longtable}[]{@{}
  >{\raggedright\arraybackslash}p{(\columnwidth - 14\tabcolsep) * \real{0.1895}}
  >{\raggedright\arraybackslash}p{(\columnwidth - 14\tabcolsep) * \real{0.1158}}
  >{\raggedright\arraybackslash}p{(\columnwidth - 14\tabcolsep) * \real{0.1158}}
  >{\raggedright\arraybackslash}p{(\columnwidth - 14\tabcolsep) * \real{0.1158}}
  >{\raggedright\arraybackslash}p{(\columnwidth - 14\tabcolsep) * \real{0.1158}}
  >{\raggedright\arraybackslash}p{(\columnwidth - 14\tabcolsep) * \real{0.1158}}
  >{\raggedright\arraybackslash}p{(\columnwidth - 14\tabcolsep) * \real{0.1158}}
  >{\raggedright\arraybackslash}p{(\columnwidth - 14\tabcolsep) * \real{0.1158}}@{}}
\toprule()
\begin{minipage}[b]{\linewidth}\raggedright
Years of Service
\end{minipage} & \begin{minipage}[b]{\linewidth}\raggedright
1998-99
\end{minipage} & \begin{minipage}[b]{\linewidth}\raggedright
1999-00
\end{minipage} & \begin{minipage}[b]{\linewidth}\raggedright
2000-01
\end{minipage} & \begin{minipage}[b]{\linewidth}\raggedright
2001-02
\end{minipage} & \begin{minipage}[b]{\linewidth}\raggedright
2002-03
\end{minipage} & \begin{minipage}[b]{\linewidth}\raggedright
2003-04
\end{minipage} & \begin{minipage}[b]{\linewidth}\raggedright
2004-05
\end{minipage} \\
\midrule()
\endhead
0 & 287,500 & 301,875 & 316,969 & 332,817 & 349,458 & 366,931 & 385,277 \\
1 & 350,000 & 385,000 & 423,500 & 465,850 & 512,435 & 563,679 & 620,046 \\
2 & 425,000 & 460,000 & 498,500 & 540,850 & 587,435 & 638,679 & 695,046 \\
3 & 450,000 & 485,000 & 523,500 & 565,850 & 612,435 & 663,679 & 720,046 \\
4 & 475,000 & 510,000 & 548,500 & 590,850 & 637,435 & 688,679 & 745,046 \\
5 & 537,500 & 572,500 & 611,000 & 653,350 & 699,935 & 751,179 & 807,546 \\
6 & 600,000 & 635,000 & 673,500 & 715,850 & 762,435 & 813,679 & 870,046 \\
7 & 662,500 & 697,500 & 736,000 & 778,350 & 824,935 & 876,179 & 932,546 \\
8 & 725,000 & 760,000 & 798,500 & 840,850 & 887,435 & 938,679 & 995,046 \\
9 & 850,000 & 885,000 & 923,500 & 965,850 & 1,000,000 & 1,000,000 & 1,000,000 \\
10+ & 1,000,000 & 1,000,000 & 1,000,000 & 1,000,000 & 1,030,000 & 1,070,000 & 1,100,000 \\
\bottomrule()
\end{longtable}

\hypertarget{bri-expense-ratios}{%
\chapter{BRI EXPENSE RATIOS}\label{bri-expense-ratios}}

Article VII, Section 1(a)(1)(v), (vi)

\begin{longtable}[]{@{}lc@{}}
\toprule()
Category & Ratio of Expenses to Revenues \\
\midrule()
\endhead
Novelties and Concessions & 50\% \\
Game Parking & Accountants to determine \\
Game Programs & 25\% \\
Team Sponsorships and Promotions & 34\% \\
In-arena signage & Accountants to determine \\
In-arena Club & Accountants to determine \\
\bottomrule()
\end{longtable}

Article VII, Section 1(a)(viii)

\begin{longtable}[]{@{}lc@{}}
\toprule()
Category & Ratio of Expenses to Revenues \\
\midrule()
\endhead
Sponsorships & 19\% \\
NBA Entertainment & 35\% \\
International Television & 22\% \\
Special Events & 100\% \\
\bottomrule()
\end{longtable}

\hypertarget{notice-to-veteran-players-concerning-summer-leagues}{%
\chapter{NOTICE TO VETERAN PLAYERS CONCERNING SUMMER LEAGUES}\label{notice-to-veteran-players-concerning-summer-leagues}}

An arbitration award issued on June 13, 1977, stated that, in order for an NBA Team to enroll veteran players in a summer pro league, it is necessary for the Team to have such players sign a form signifying that they have chosen to participate in the league on a voluntary basis, and to inform such players of certain provisions. Players participating in any summer league shall be informed as to the following:

\begin{enumerate}
\def\labelenumi{\arabic{enumi}.}
\tightlist
\item
  Under the Uniform Player Contract and the Collective Bargaining Agreement between the NBA and the Players Association, the Team cannot require players to participate in any summer league.
\item
  The failure of a player to sign such a form to participate in any summer league will not, by itself, prejudice or disadvantage such player in his Team standing or relationship.
\item
  The Team reserves the right to determine how many and which players it may enroll in any summer league.
\end{enumerate}

We would appreciate your signing and returning the attached form to:

\begin{longtable}[]{@{}l@{}}
\toprule()
\endhead
\_\_\_\_\_\_\_\_\_\_\_\_\_\_\_\_\_\_\_\_\_ \\
\_\_\_\_\_\_\_\_\_\_\_\_\_\_\_\_\_\_\_\_\_ \\
Name of Team \\
\bottomrule()
\end{longtable}

\hypertarget{form-regarding-summer-league-participation}{%
\chapter{FORM REGARDING SUMMER LEAGUE PARTICIPATION}\label{form-regarding-summer-league-participation}}

This is to acknowledge that I have freely chosen to participate in the (name of appropriate summer league) on a voluntary basis during the summer of ( ).

\begin{longtable}[]{@{}l@{}}
\toprule()
\endhead
\_\_\_\_\_\_\_\_\_\_\_\_\_\_\_\_\_\_\_\_\_ \\
(Signature of Player) \\
\_\_\_\_\_\_\_\_\_\_\_\_\_\_\_\_\_\_\_\_\_ \\
(Printed Name of Player) \\
\bottomrule()
\end{longtable}

\hypertarget{offer-sheet}{%
\chapter{OFFER SHEET}\label{offer-sheet}}

\begin{longtable}[]{@{}
  >{\raggedright\arraybackslash}p{(\columnwidth - 2\tabcolsep) * \real{0.5000}}
  >{\raggedright\arraybackslash}p{(\columnwidth - 2\tabcolsep) * \real{0.5000}}@{}}
\toprule()
\endhead
Name of Player: & Date: \\
\_\_\_\_\_\_\_\_\_\_\_\_\_\_\_\_\_\_\_\_\_\_\_\_\_ & \_\_\_\_\_\_\_\_\_\_\_\_\_\_\_\_\_\_\_\_\_\_\_\_\_ \\
Address of Player and Email Address of Player: & Name of New Team: \\
\_\_\_\_\_\_\_\_\_\_\_\_\_\_\_\_\_\_\_\_\_\_\_\_\_ & \\
\_\_\_\_\_\_\_\_\_\_\_\_\_\_\_\_\_\_\_\_\_\_\_\_\_ & \\
\_\_\_\_\_\_\_\_\_\_\_\_\_\_\_\_\_\_\_\_\_\_\_\_\_ & \_\_\_\_\_\_\_\_\_\_\_\_\_\_\_\_\_\_\_\_\_\_\_\_\_ \\
Name, Address and Email Address of Player's Representative Authorized to Act for Player: & Name of ROFR Team: \\
\_\_\_\_\_\_\_\_\_\_\_\_\_\_\_\_\_\_\_\_\_\_\_\_\_ & \_\_\_\_\_\_\_\_\_\_\_\_\_\_\_\_\_\_\_\_\_\_\_\_\_ \\
\_\_\_\_\_\_\_\_\_\_\_\_\_\_\_\_\_\_\_\_\_\_\_\_\_ & Address of ROFR Team: \\
\_\_\_\_\_\_\_\_\_\_\_\_\_\_\_\_\_\_\_\_\_\_\_\_\_ & \_\_\_\_\_\_\_\_\_\_\_\_\_\_\_\_\_\_\_\_\_\_\_\_\_ \\
\_\_\_\_\_\_\_\_\_\_\_\_\_\_\_\_\_\_\_\_\_\_\_\_\_ & \_\_\_\_\_\_\_\_\_\_\_\_\_\_\_\_\_\_\_\_\_\_\_\_\_ \\
\bottomrule()
\end{longtable}

Attached hereto is an unsigned Player Contract that the New Team has offered to the player and that the player desires to accept. The attached Player Contract separately specifies in its exhibits those Principal Terms that will be included in the Player Contract with the ROFR Team if that Team gives the player a timely First Refusal Exercise Notice.

\begin{longtable}[]{@{}ll@{}}
\toprule()
\endhead
Player: & New Team: \\
& \\
By \_\_\_\_\_\_\_\_\_\_\_\_\_\_\_\_\_\_\_\_\_\_\_\_\_ & By \_\_\_\_\_\_\_\_\_\_\_\_\_\_\_\_\_\_\_\_\_\_\_\_\_ \\
\bottomrule()
\end{longtable}

\hypertarget{first-refusal-exercise-notice}{%
\chapter{FIRST REFUSAL EXERCISE NOTICE}\label{first-refusal-exercise-notice}}

\begin{longtable}[]{@{}
  >{\raggedright\arraybackslash}p{(\columnwidth - 2\tabcolsep) * \real{0.5000}}
  >{\raggedright\arraybackslash}p{(\columnwidth - 2\tabcolsep) * \real{0.5000}}@{}}
\toprule()
\endhead
Name of Player: & Date: \\
\_\_\_\_\_\_\_\_\_\_\_\_\_\_\_\_\_\_\_\_\_\_\_\_\_ & \_\_\_\_\_\_\_\_\_\_\_\_\_\_\_\_\_\_\_\_\_\_\_\_\_ \\
Address of Player: & Name of New Team: \\
\_\_\_\_\_\_\_\_\_\_\_\_\_\_\_\_\_\_\_\_\_\_\_\_\_ & \\
\_\_\_\_\_\_\_\_\_\_\_\_\_\_\_\_\_\_\_\_\_\_\_\_\_ & \\
\_\_\_\_\_\_\_\_\_\_\_\_\_\_\_\_\_\_\_\_\_\_\_\_\_ & \_\_\_\_\_\_\_\_\_\_\_\_\_\_\_\_\_\_\_\_\_\_\_\_\_ \\
Name and Address of Player's Representative Authorized to Act for Player & Name of ROFR Team: \\
\_\_\_\_\_\_\_\_\_\_\_\_\_\_\_\_\_\_\_\_\_\_\_\_\_ & \_\_\_\_\_\_\_\_\_\_\_\_\_\_\_\_\_\_\_\_\_\_\_\_\_ \\
\_\_\_\_\_\_\_\_\_\_\_\_\_\_\_\_\_\_\_\_\_\_\_\_\_ & Address of ROFR Team: \\
\_\_\_\_\_\_\_\_\_\_\_\_\_\_\_\_\_\_\_\_\_\_\_\_\_ & \_\_\_\_\_\_\_\_\_\_\_\_\_\_\_\_\_\_\_\_\_\_\_\_\_ \\
\_\_\_\_\_\_\_\_\_\_\_\_\_\_\_\_\_\_\_\_\_\_\_\_\_ & \_\_\_\_\_\_\_\_\_\_\_\_\_\_\_\_\_\_\_\_\_\_\_\_\_ \\
\bottomrule()
\end{longtable}

The undersigned member of the NBA hereby exercises its Right of First Refusal so as to create a binding agreement with the player containing the Principal Terms set forth in the Player Contract annexed to the player's Offer Sheet (a copy of which is attached hereto).

\begin{longtable}[]{@{}l@{}}
\toprule()
\endhead
ROFR Team: \\
By \_\_\_\_\_\_\_\_\_\_\_\_\_\_\_\_\_\_\_\_\_ \\
\bottomrule()
\end{longtable}

\hypertarget{section-1}{%
\chapter{}\label{section-1}}

\hypertarget{authorization-for-testing}{%
\section{AUTHORIZATION FOR TESTING}\label{authorization-for-testing}}

\begin{longtable}[]{@{}lc@{}}
\toprule()
\endhead
To: & \_\_\_\_\_\_\_\_\_\_\_\_\_\_\_\_\_\_\_ \\
& \\
Player & \_\_\_\_\_\_\_\_\_\_\_\_\_\_\_\_\_\_\_ \\
\bottomrule()
\end{longtable}

Please be advised that on \_\_\_\_\_\_\_\_\_\_\_\_\_\_\_\_\_\_\_\_\_\_\_\_\_\_, you were the subject of a meeting or conference call held pursuant to the Anti-Drug Program set forth in Article XXXIII of the Collective Bargaining Agreement between the NBA and the National Basketball Players Association, dated January 20, 1999 (the ``Agreement''). Following the meeting or conference call, I authorized the NBA to conduct the testing procedures set forth in the Agreement, and you are hereby directed to submit to those testing procedures, on demand, no more than four times during the next six weeks.

Please be advised that your failure to submit to these procedures may result in the imposition of penalties under Article XXXIII of the Agreement.

\begin{longtable}[]{@{}l@{}}
\toprule()
\endhead
\_\_\_\_\_\_\_\_\_\_\_\_\_\_\_\_\_\_\_\_\_\_\_\_ \\
Independent Expert \\
Dated: \\
\_\_\_\_\_\_\_\_\_\_\_\_\_\_\_\_\_\_\_\_\_\_\_\_ \\
\bottomrule()
\end{longtable}

\hypertarget{prohibited-substances}{%
\section{PROHIBITED SUBSTANCES}\label{prohibited-substances}}

\begin{enumerate}
\def\labelenumi{(\alph{enumi})}
\item
  Drugs of Abuse

  \begin{itemize}
  \tightlist
  \item
    Amphetamine and its analogs (including but not limited to methamphetamine and MDMA)
  \item
    Cocaine
  \item
    LSD
  \item
    Opiates (Heroin, Codeine, Morphine)
  \item
    Phencyclidine (``PCP'')
  \end{itemize}
\item
  Marijuana and its by-products
\item
  Steroids

  For purposes of the foregoing, ``Steroids'' includes the following:

  \begin{itemize}
  \tightlist
  \item
    Bolasterone
  \item
    Bolderone
  \item
    Clostebol
  \item
    Dehydrochlormethyltestosterone
  \item
    Dromostanolone
  \item
    Ethylestrenol
  \item
    Furarebol
  \item
    Mesterolone
  \item
    Methandienone
  \item
    Methandriol
  \item
    Methenolone
  \item
    Mibolerone
  \item
    Oxymcaterone
  \item
    Trenbolone
  \item
    Clenbuterol
  \end{itemize}
\end{enumerate}

\hypertarget{analytic-testing-techniques}{%
\section{ANALYTIC TESTING TECHNIQUES}\label{analytic-testing-techniques}}

All specimens procured pursuant to the testing procedures set forth in this Agreement will be screened and tested through scientifically accepted analytical techniques. For breath, blood and other testing (conducted only in circumstances where the Medical Director deems appropriate), testing techniques will be determined by the Medical Director.

\hypertarget{collection-procedures}{%
\section{COLLECTION PROCEDURES}\label{collection-procedures}}

When the player arrives at the collection site, the collector will ensure that the player is positively identified through presentation of photo ID or identification by a team representative. If the player's identity cannot be established, the collectors shall not proceed with the collection.

The player will be asked to select a sealed urine specimen cup. The player will then provide his urine specimen under the direct observation of the collector.

The collector shall ensure that the player has provided a urine specimen of sufficient volume for accurate testing. If such a sample cannot immediately be provided by the player, he shall be instructed to remain at the testing site for a reasonable period of time until he can provide such a specimen. Once the specimen has been obtained, the player will select a sealed specimen kit, which contains two bottles. The collector, in the presence of the player, will pour the specimen into two bottles. One bottle will be used as the primary or ``A'' specimen and the other will be used as the split or ``B'' specimen. The specimen bottles will be sealed with tamper-proof seals in the presence of the player. The seals will contain a unique identification number that corresponds to the number on the chain of custody form.

The player and collector will complete the chain of custody form that documents the handling of the specimen. The collector will note any irregularities concerning the specimen on the chain of custody form. Both the player and collector will sign the chain of custody form. The chain of custody form along with the two specimen bottles will be placed back into the kit. The kit will be sealed and sent via overnight courier to the laboratory for testing.

Once the specimen arrives at the laboratory, the primary specimen will be analyzed. If the primary specimen tests positive, the split sample will be placed in frozen storage and will be available for testing by a different laboratory, if requested by the player.

\hypertarget{section-2}{%
\chapter{}\label{section-2}}

\hypertarget{exhibit-j-1}{%
\section{EXHIBIT J-1}\label{exhibit-j-1}}

January 20, 1999\\
Mr.~G. William Hunter\\
Executive Director\\
National Basketball Players Association\\
1700 Broadway, Suite 1400\\
New York, New York 10019

Dear Billy:

This will confirm our agreement that a team's termination of a Uniform Player Contract by reason of the player's ``lack of skill'' (under paragraph 16(a)(iii) thereof) shall be interpreted to include a termination based on the team's determination that, in view of the player's level of skill (in the sole opinion of the Team), the Compensation paid (or to be paid) to the player is no longer commensurate with the team's financial plans or needs. This agreement shall not affect any post-termination obligation to pay Compensation that may result from Compensation protection provisions included in a Uniform Player Contract.

If the foregoing coincides with your understanding of our agreement, please sign this letter in the space provided below.

Sincerely,\\
/s/ /s/ JEFFREY A. MISHKIN\\
Jeffrey A. Mishkin

AGREED TO AND ACCEPTED:

NATIONAL BASKETBALL PLAYERS ASSOCIATION\\
By: /s/ G. WILLIAM HUNTER\\
G. William Hunter\\
Executive Director

\hypertarget{exhibit-j-2}{%
\section{EXHIBIT J-2}\label{exhibit-j-2}}

January 20, 1999\\
Mr.~G. William Hunter\\
Executive Director\\
National Basketball Players Association\\
1700 Broadway---Suite 1400\\
New York, New York 10019

Dear Billy:

This will confirm our agreement that the attached accounting procedures are the procedures that will be in effect for purposes of Article VII, Section 10 of the Collective Bargaining Agreement entered into on January 20, 1999, unless such procedures shall be modified by agreement of the parties.

If the foregoing coincides with your understanding of our agreement, please sign this letter in the space provided below.

Sincerely,\\
/s/ /s/ JEFFREY A. MISHKIN\\
Jeffrey A. Mishkin

AGREED TO AND ACCEPTED:

NATIONAL BASKETBALL PLAYERS ASSOCIATION\\
By: /s/ G. WILLIAM HUNTER\\
G. William Hunter\\
Executive Director

\newpage

\hypertarget{minimum-procedures-to-be-provided-by-the-accountants}{%
\subsection{Minimum Procedures To Be Provided By The Accountants}\label{minimum-procedures-to-be-provided-by-the-accountants}}

\textbf{General}

\begin{itemize}
\tightlist
\item
  The Audit Report (and any Interim Audit Report or Interim Escrow Audit Report) must be prepared in accordance with the terms of the Collective Bargaining Agreement (``CBA''), which should be reviewed and understood by all auditors.
\item
  The Basketball Related Income Reporting Package and instructions should be reviewed and understood by all auditors.
\item
  All audit workpapers should be made available for review by representatives of the NBA and Players Association prior to issuance of the report.
\item
  A summary of all audit findings (including any unusual or non-recurring transactions) and proposed adjustments must be jointly reviewed with representatives of the NBA and Players Association prior to issuance of the report.
\item
  Any problems or questions raised during the audit should be resolved jointly with representatives of the NBA and Players Association.
\item
  All estimates should be reviewed in accordance with the CBA. Estimates are to be reviewed based upon the previous year's actual results and current year activity. All estimates should be confirmed with third parties when possible.
\item
  Revenue and expense amounts that have been estimated should be reconfirmed with the controller or other team representatives prior to the issuance of the Audit Report on or before July 31.
\item
  Where possible, team and NBA revenues and expenses should be reconciled to audited financial statements.
\item
  All reporting packages and supporting schedules are to be completed in U.S. dollars.
\item
  The Auditors may consider, but are not bound by, the value attributed to or treatment of revenue or expense items in prior years.
\item
  Auditors should be aware of revenues excluded from BRI. The Teams should be instructed to make available to the Auditors all information necessary to determine categories of revenues they have excluded from BRI. All revenues excluded by the Teams or the NBA should be reviewed with both parties to determine proper exclusion. Auditor should perform a review for revenues improperly excluded from, or included in, BRI.
\end{itemize}

\textbf{Team Salaries}

\begin{itemize}
\tightlist
\item
  Trace amounts to the team's general ledger or other supporting documentation for agreement.
\item
  Foot all schedules and perform other clerical tests.
\item
  Examine an appropriate sample of player contracts, noting agreement of all salary amounts, in accordance with the definition of Salary in the CBA.
\item
  Compare player names with all player lists for the season in question.
\item
  Discuss method used to value non-cash compensation with the controller or other representative of each team and conclude as to its reasonableness.
\item
  Examine trade arrangements to verify that each team has properly recorded its pro rata portion of the players' entire salary based upon roster days, and that any bonuses or salary increases payable to players have been properly accounted for.
\item
  Inquire of controller or other representative of each team if any additional compensation was paid to players and not included on the schedule, whether or not paid for basketball services. Also inquire if any business arrangements were entered into by the team or team affiliate with players or their affiliates, including with retired players who played for the team within the past five (5) years.
\item
  Review performance bonuses to determine whether such bonuses were actually earned for such season.
\item
  Review signing bonuses to determine if they have been allocated over the Compensation Protected years of the contract.
\item
  Confirm that, where provided in the CBA, certain contracts have been averaged.
\end{itemize}

\textbf{Benefits}

\begin{itemize}
\tightlist
\item
  Trace amounts to the team's general ledger or other supporting documentation for agreement.
\item
  Foot all schedules and perform other clerical tests.
\item
  Investigate variations in amounts from the prior year through discussion with the controller or other representative of the team.
\item
  Review each team's insurance expenses for premium credits (refunds) received from Planet Insurance Ltd.~(owned by Teams) and Prudential Insurance (amounts can be obtained from League Office).
\item
  Review League Office supporting documentation with respect to Benefits.
\end{itemize}

\textbf{Basketball Related Income}

\begin{itemize}
\tightlist
\item
  Trace amounts to team's general ledger or other supporting documentation for agreement.
\item
  Foot all schedules and perform other clerical tests.
\item
  Trace gate receipts to general ledger and test supporting documentation where appropriate.
\item
  Gate receipts should be reviewed and reconciled to League Office gate receipts summary.
\item
  Verify amounts reported as luxury suite revenues with supporting documentation from the entity that sold, leased or licensed such luxury suites.
\item
  Verify amounts reported as complimentary tickets and tickets traded for goods or services with supporting documentation from the team.
\item
  Trace amounts reported for novelties and concessions, game parking, game programs, Team sponsorships and promotions, arena signage and arena club sales to general ledgers and test supporting documentation where appropriate.
\item
  Where reported amounts include proceeds received by a Related Party, verify the amounts reported with supporting documentation from the Related Party.
\item
  Examine the National Television and Cable contracts at the League Office, and agree to amounts reported.
\item
  Review, at League Office, expenses deducted from the National contracts in accordance with the terms of the CBA. Review supporting documentation and test where applicable.
\item
  Examine local television, local cable and local radio contracts. Verify to amounts reported by teams.
\item
  When local broadcast revenues are not verifiable by reviewing a contract, detailed supporting documentation should be reviewed and tested.
\item
  All loans, advances, bonuses, etc. received by the League Office or its teams should be noted in the report and included in BRI where appropriate.
\item
  Schedules of international broadcast, market extension, copyright royalty revenues and expenses should be obtained from the NBA. Schedules should be verified by agreeing to general ledgers and examining supporting documentation where applicable.
\item
  Schedules of revenues and expenses reported by Properties for sponsorship, NBA related revenues from NBA Entertainment, and NBA Special Events should be obtained from the NBA. Schedules should be verified by agreeing to general ledgers and examining supporting documentation where applicable.
\item
  Net exhibition revenues and expenses should be verified to supporting documentation where appropriate.
\item
  All amounts of other revenues should be reviewed for proper inclusion/exclusion in BRI. Test appropriateness of balances where appropriate.
\item
  Determine the ratio of expenses to revenues for those categories of proceeds that come within the provisions of Article VII of the CBA and determine the extent to which expenses should be disallowed, if at all, pursuant to the provisions of that Section.
\end{itemize}

\textbf{Playoff Revenues}

\begin{itemize}
\tightlist
\item
  All sources of playoff revenues and expenses should be verified per the procedure outlined for Basketball Related Income.
\item
  Because of the late timing of the Playoffs, special attention should be given to revenue and expense estimates.
\item
  Playoff gate receipts should be recorded net of admission taxes. Payments made to the Playoff Pool should not be deducted. Odd game payments should not be either deducted by the paying team or recorded by the receiving team.
\item
  Other playoff expenses should be reviewed in accordance with the terms of the CBA.
\item
  Team expenses paid by the League Playoff Pool, including travel expenses, should not be deducted by teams.
\item
  Review League Office supporting documentation as to expenses deducted from the Playoff Pool.
\end{itemize}

\textbf{Questions Concerning Related Party Transactions}

\begin{itemize}
\tightlist
\item
  Review with controller or other representatives of the team the answers to all questions on this schedule.
\item
  Review that appropriate details are provided where requested.
\item
  Prepare summary of all changes.
\end{itemize}

\textbf{List of Related Parties}

\begin{itemize}
\tightlist
\item
  Review with controller or other representatives of the team all information included on the schedule of related entities.
\item
  Prepare a summary of any changes, corrections or additions to the schedule.
\item
  Review supporting details of any changes.
\end{itemize}

\end{document}
